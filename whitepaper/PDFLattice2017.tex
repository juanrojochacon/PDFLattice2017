%stylefile for "Progress in Particle and Nuclear Physics" from 20. March 2003
\documentclass[twoside,12pt]{article}
\usepackage[utf8]{inputenc} 
\usepackage{morefloats}
\usepackage{lineno}
\usepackage{graphicx}
\usepackage{booktabs}
\usepackage{threeparttable}
\usepackage{afterpage}
\usepackage{cite}
\usepackage{amssymb}
\usepackage{amsmath}
\usepackage{dsfont}
\usepackage{multirow}
\usepackage{url}
\usepackage{color}
\usepackage{verbatim}
\usepackage{rotating}
\usepackage{subfig}
\usepackage{colortbl}
\usepackage[]{xcolor}
\usepackage{siunitx} % For correctly displayed SI units.
\sisetup{range-phrase = \text{--}} % For correctly displayed number ranges
\usepackage{url}
\usepackage[colorlinks, citecolor=blue,anchorcolor=blue,menucolor=blue, 
linkcolor=blue,filecolor=blue,runcolor=blue,urlcolor=blue,frenchlinks=blue]{hyperref}

\def\Journal#1#2#3#4{{#1} {#2} (#4) #3 }
\def\NCA{{\em Nuovo Cimento} A}
\def\PHYS{{\em Physica}}
\def\NPA{{\em Nucl. Phys.} A}
\def\MATH{{\em J. Math. Phys.}}
\def\PRO{{\em Prog. Theor. Phys.}}
\def\NPB{{\em Nucl. Phys.} B}
\def\PLA{{\em Phys. Lett.} A}
\def\PLB{{\em Phys. Lett.} B}
\def\PLD{{\em Phys. Lett.} D}
\def\PL{{\em Phys. Lett.}}
\def\PRL{\em Phys. Rev. Lett.}
\def\PREV{\em Phys. Rev.}
\def\PREP{\em Phys. Rep.}
\def\PRA{{\em Phys. Rev.} A}
\def\PRD{{\em Phys. Rev.} D}
\def\PRC{{\em Phys. Rev.} C}
\def\PRB{{\em Phys. Rev.} B}
\def\ZPC{{\em Z. Phys.} C}
\def\ZPA{{\em Z. Phys.} A}
\def\ANNP{\em Ann. Phys. (N.Y.)}
\def\RMP{{\em Rev. Mod. Phys.}}
\def\CHEM{{\em J. Chem. Phys.}}
\def\INT{{\em Int. J. Mod. Phys.} E}
\def\r{\vec r}
\def\R{\vec R}
\def\p{\vec p}
\def\P{\vec P}
\def\q{\vec q}
\def\ss{\mbox{\boldmath $\sigma$}}
\newcommand{\nn}{\nonumber}
\topmargin-2.8cm
\oddsidemargin-1cm
\evensidemargin-1cm
\textwidth18.5cm
\textheight25.0cm

\bibliographystyle{JHEP}

\def\smallfrac#1#2{\hbox{$\frac{#1}{#2}$}}
\newcommand{\be}{\begin{equation}}
\newcommand{\ee}{\end{equation}}
\newcommand{\bea}{\begin{eqnarray}}
\newcommand{\eea}{\end{eqnarray}}
\newcommand{\bi}{\begin{itemize}}
\newcommand{\ei}{\end{itemize}}
\newcommand{\ben}{\begin{enumerate}}
\newcommand{\een}{\end{enumerate}}
\newcommand{\la}{\left\langle}
\newcommand{\ra}{\right\rangle}
\newcommand{\lc}{\left[}
\newcommand{\rc}{\right]}
\newcommand{\lp}{\left(}
\newcommand{\rp}{\right)}
\newcommand{\as}{\alpha_s}
\newcommand{\aq}{\alpha_s\left( Q^2 \right)}
\newcommand{\amz}{\alpha_s\left( M_Z^2 \right)}
\newcommand{\aqq}{\alpha_s \left( Q^2_0 \right)}
\newcommand{\aqz}{\alpha_s \left( Q^2_0 \right)}
\def\toinf#1{\mathrel{\mathop{\sim}\limits_{\scriptscriptstyle
{#1\rightarrow\infty }}}}
\def\tozero#1{\mathrel{\mathop{\sim}\limits_{\scriptscriptstyle
{#1\rightarrow0 }}}}
\def\toone#1{\mathrel{\mathop{\sim}\limits_{\scriptscriptstyle
{#1\rightarrow1 }}}}
\def\frac#1#2{{{#1}\over {#2}}}
\def\gsim{\mathrel{\rlap{\lower4pt\hbox{\hskip1pt$\sim$}}
    \raise1pt\hbox{$>$}}}         %greater than or approx. symbol
\def\lsim{\mathrel{\rlap{\lower4pt\hbox{\hskip1pt$\sim$}}
    \raise1pt\hbox{$<$}}}         %less than or approx. symbol
\newcommand{\mrexp}{\mathrm{exp}}
\newcommand{\dat}{\mathrm{dat}}
\newcommand{\one}{\mathrm{(1)}}
\newcommand{\two}{\mathrm{(2)}}
\newcommand{\art}{\mathrm{art}} 
\newcommand{\rep}{\mathrm{rep}}
\newcommand{\net}{\mathrm{net}}
\newcommand{\stopp}{\mathrm{stop}}
\newcommand{\sys}{\mathrm{sys}}
\newcommand{\stat}{\mathrm{stat}}
\newcommand{\diag}{\mathrm{diag}}
\newcommand{\pdf}{\mathrm{pdf}}
\newcommand{\tot}{\mathrm{tot}}
\newcommand{\minn}{\mathrm{min}}
\newcommand{\mut}{\mathrm{mut}}
\newcommand{\partt}{\mathrm{part}}
\newcommand{\dof}{\mathrm{dof}}
\newcommand{\NS}{\mathrm{NS}}
\newcommand{\cov}{\mathrm{cov}}
\newcommand{\gen}{\mathrm{gen}}
\newcommand{\cut}{\mathrm{cut}}
\newcommand{\parr}{\mathrm{par}}
\newcommand{\val}{\mathrm{val}}
\newcommand{\tr}{\mathrm{tr}}
\newcommand{\checkk}{\mathrm{check}}
\newcommand{\reff}{\mathrm{ref}}
\newcommand{\Mll}{M_{ll}}
\newcommand{\extra}{\mathrm{extra}}
\newcommand{\draft}[1]{}
%\newcommand{\comment}[1]{{\bf \it  #1}}
\def\beq{\begin{equation}}  
\def\eeq{\end{equation}}  

% Added by MU for the fast evolution section
\def\bgamma{\boldsymbol{\gamma}}
\def\nn{\nonumber}
\def \so{\sigma_I^{DIS}(x_I,Q^2_I)}
\def \sh{\frac{d\sigma^{hh}}{dX}}
\def\sdy{\frac{d\sigma^{\mathrm{DY}}}{dQ_I^2dY_I}}
\def \npdf{N_{\mathrm{pdf}}}
\def \gtilda{\tilde\Gamma_J^{\mathrm{OBS}}}
\def \n0{N_j^{(0)}}
\def \a{\alpha}
\def \b{\beta}
\def \g{\gamma}
\def \c{\xi}
\def \z{\zeta}
% Added by JR
\def\lapprox{\lower .7ex\hbox{$\;\stackrel{\textstyle <}{\sim}\;$}}
\def\gapprox{\lower .7ex\hbox{$\;\stackrel{\textstyle >}{\sim}\;$}}
\def\half{\smallfrac{1}{2}}
\def\GeV{{\rm GeV}}
\def\TeV{{\rm TeV}}
\def\ap{{a'}}
\def\vp{{v'}}
\def\e{\epsilon}
\def\d{{\rm d}}
\def\calN{{\cal N}}
\def\shat{\hat{s}}
\def\barq{\bar{q}}
\def\qq{q \bar q}
\def\uu{u \bar u}
\def\dd{d \bar d}
\def\pp{p \bar p}
\def\xa{x_{1}}
\def\xb{x_{2}}
\def\xaa{x_{1}^{0}}
\def\xbb{x_{2}^{0}}
\def\smx{\stackrel{x\to 0}{\longrightarrow}}
\def\Li{{\rm Li}}
% Added by CJM [benchmarking] 
% Please do not change spacing and sizing without good reason.
\def\bstar{{\color{blue}$\bigstar$}}
\def\bcirc{\raisebox{-1pt}{\scalebox{1.5}{\color{blue}$\circ$}}}
\def\rsquare{{\raisebox{-1pt}{\color{red}$\blacksquare$}}}

\newcommand{\Red}[1]{{\color{black}#1}}
\newcommand{\Blue}[1]{{\color{black}#1}}

\newcommand{\tmop}[1]{\ensuremath{\operatorname{#1}}}
\newcommand{\tmtextit}[1]{{\itshape{#1}}}
\newcommand{\tmtextrm}[1]{{\rmfamily{#1}}}
\newcommand{\tmtexttt}[1]{{\ttfamily{#1}}}

\numberwithin{equation}{section}
\numberwithin{figure}{section}
\numberwithin{table}{section}

\makeatletter
\g@addto@macro\bfseries{\boldmath}
\makeatother

%%%%%%%%%%%%%%%%%%%%%%%%%%%%%%%%%%%%%%%%%%%%%%%%%%%%%%%%%%%%%%%%%%%%%%%%%%%%%%%%
\begin{document}

\linenumbers
\vspace{.3cm}

\begin{center}
{\Large \bf Parton distributions and lattice QCD calculations:
\\[0.2cm] a community white paper}
\vspace{.4cm}

{\small 
  Huey-Wen~Lin$^{1,2}$,
  Emanuele~R.~Nocera$^{3,4}$,
  Fred~Olness$^5$,
  Kostas~Orginos$^{6,7}$,
  Juan~Rojo$^{8,9}$ (editors),
  Alberto~Accardi$^{7,10}$, 
  Constantia~Alexandrou$^{11,12}$, 
  Alessandro~Bacchetta$^{13}$, 
  Giuseppe~Bozzi$^{13}$, 
  Jiunn-Wei~Chen$^{14}$,
  Sara~Collins$^{15}$, 	
  Amanda~Cooper-Sarkar$^{16}$,
  Martha~Constantinou$^{17}$, 
  Luigi~Del~Debbio$^{4}$, 
  Michael~Engelhardt$^{18}$, 
  Jeremy~Green$^{19}$, 
  Rajan~Gupta$^{20}$, 
  Lucian~A.~Harland-Lang$^{3,21}$, 
  Tomomi~Ishikawa$^{22}$, 
  Aleksander~Kusina$^{24}$, 
  Keh-Fei~Liu$^{25}$, 	
  Simonetta~Liuti$^{26,27}$, 		
  Christopher~Monahan$^{28}$, 		
  Pavel~Nadolsky$^{5}$,
  Jian-Wei Qiu$^{7}$,
  Ingo~Schienbein$^{23}$, 	
  Gerrit~Schierholz$^{29}$,
  Robert~S.~Thorne$^{21}$,
  Werner~Vogelsang$^{30}$,\\
  Hartmut Wittig$^{31}$, 
  C.-P. Yuan$^{1}$, and
  James Zanotti$^{32}$
}

\vspace{.2cm}
{\it \footnotesize
~$^{1}$Department of Physics and Astronomy, 
Michigan State University, East Lansing, MI 48824, USA\\
~$^{2}$Department of Computational Mathematics, Science and Engineering,\\
Michigan State University, East Lansing, MI 48824, USA\\
~$^{3}$Rudolf Peierls Centre for Theoretical Physics, 1 Keble Road,\\ 
University of Oxford, OX1 3NP Oxford, United Kingdom\\
~$^{4}$Higgs Centre for Theoretical Physics, School of Physics and Astronomy,\\ 
University of Edinburgh, EH9 3FD, UK\\
~$^{5}$Department of Physics, 
Southern Methodist University, Dallas, TX 75275, USA\\
~$^{6}$Physics Department, 
College of William and Mary, Williamsburg, VA 23187, USA\\
~$^{7}$Thomas Jefferson National Accelerator Facility, 
Newport News, VA 23606, USA\\
~$^{8}$Department of Physics and Astronomy, VU University Amsterdam,\\
De Boelelaan 1081, NL-1081, HV Amsterdam, The Netherlands\\
~$^{9}$Nikhef, Science Park 105, NL-1098 XG Amsterdam, The Netherlands\\
~$^{10}$Hampton University, Hampton, VA 23668, USA\\
~$^{11}$Department of Physics, 
University of Cyprus, P.O. Box 20537, 1678 Nicosia, Cyprus\\
~$^{12}$Computation-based Science and Technology Research Research,\\
The Cyprus Institute, 20 Kavafi Str., Nicosia 2121, Cyprus\\
~$^{13}$Dipartimento di Fisica, Universit\`{a} degli Studi di Pavia, and INFN, 
Sezione di Pavia, 27100 Pavia, Italy\\
~$^{14}$Department of Physics, Center for Theoretical Sciences,
and Leung Center for Cosmology and Particle Astrophysics,
National Taiwan University, Taipei, Taiwan 106\\
~$^{15}$Institute for Theoretical Physics, 
Universit\"at Regensburg, D-93040 Regensburg, Germany\\
~$^{16}$Department of Physics, University of Oxford,\\
Denys Wilkinson Building, 1 Keble Road, OX1 3RH Oxford, United Kingdom\\
~$^{17}$ Department of Physics, 
Temple University, Philadelphia, PA 19122, USA\\
~$^{18}$Department of Physics, 
New Mexico State University, Las Cruces, NM 88003-8001, USA\\
~$^{19}$NIC, Deutsches Elektronen-Synchrotron, 15738 Zeuthen, Germany\\
~$^{20}$Los Alamos National Laboratory, Theoretical Division, 
T-2, Los Alamos, NM 87545, USA\\
~$^{21}$ Department of Physics and Astronomy, 
University College London, WC1E 6BT, United Kingdom\\
~$^{22}$T.~D.~Lee Institute, 
Shanghai Jiao Tong University, Shanghai, 200240, P.~R.~China\\
~$^{23}$Laboratoire de Physique Subatomique et de Cosmologie, 
Universit\`e Grenoble-Alpes,\\ 
CNRS/IN2P3, 53 avenue des Martyrs,  38026 Grenoble, France. \\
~$^{24}$Institute of Nuclear Physics Polish Academy of Sciences, 
PL-31342 Krakow, Poland\\
~$^{25}$Department of Physics and Astronomy, University of Kentucky, 
Lexington, KY 40506, USA\\
~$^{26}$Physics Department, University of Virginia, 382 MCormick Road,
Charlottesville, VA 22904, USA\\
~$^{27}$Laboratori Nazionali di Frascati, INFN, Frascati, Italy\\
~$^{28}$Institute for Nuclear Theory, 
University of Washington, Seattle, WA 98195, USA\\
~$^{29}$Deutsches Elektronen-Synchrotron DESY, 22603 Hamburg, Germany\\
~$^{30}$Institute for Theoretical Physics, 
T\"ubingen University, D-72076 T\"ubingen, Germany\\
~$^{31}$PRISMA Cluster of Excellence and Institute for Nuclear Physics 
Johannes Gutenberg University of Mainz,\\ 
Johann-Joachim-Becher-Weg 45, 55128 Mainz, Germany\\
~$^{32}$CSSM, Department of Physics, 
The University of Adelaide, Adelaide, SA, Australia 5005\\
}
\vspace{0.3cm}
       {\it Preprint numbers:} \\
DESY 17-185,
IFJPAN-IV-2017-19,       
INT-PUB-17-042, 
MSUHEP-17-017, 
Nikhef-2017-047,
OUTP-17-15P,
SMU-HEP-17-08.

\clearpage

{\bf \large Abstract}

\end{center}

\vspace{-0.4cm}
In the framework of quantum chromodynamics (QCD), parton distribution 
functions (PDFs) quantify how the momentum and spin of a hadron are divided 
among its quark and gluon constituents.
%
Two main approaches exist to determine PDFs.
%
The first approach, based on QCD factorization theorems, realizes a QCD 
analysis of a suitable set of hard-scattering measurements, often using a 
variety of hadronic observables.
%
The second approach, based on first-principle operator definitions of PDFs,
uses lattice QCD to compute directly some PDF-related quantities, such 
as their moments.
%
Motivated by recent progress in both approaches, in this document we present
an overview of lattice-QCD and global-analysis techniques used to 
determine unpolarized and polarized proton PDFs and their moments. 
%
We provide benchmark numbers to validate present and future lattice-QCD 
calculations and we illustrate how they could be used to reduce
the PDF uncertainties in current unpolarized and polarized global analyses.
%
This document represents a first step towards establishing a common
language between the two communities, to foster dialogue and
to further improve our knowledge of PDFs.
\vspace{-0.5cm}

\tableofcontents

%%%%%%%%%%%%%%%%%%%%%%%%%%%%%%%%%%%%%%%%%%%%%%%%%%%%%%%%%%%%%%%%%%%%%%%%%%%%%%%%
\section{Introduction and motivation}
%%%%%%%%%%%%%%%%%%%%%%%%%%%%%%%%%%%%%%%%%%%%%%%%%%%%%%%%%%%%%%%%%%%%%%%%%%%%%%%%

The detailed understanding of the inner structure of nucleons is an 
active research field with relevant phenomenological implications in 
high-energy, hadron, nuclear and astroparticle physics.
%
Within Quantum Chromodynamics (QCD), information on this structure --
specifically on how nucleon's momentum and spin are divided among quarks and 
gluons -- is encoded in Parton Distribution Functions (PDFs).

There exist two main methods to determine PDFs.

The first method is the {\it global QCD analysis}~\cite{Perez:2012um,
DeRoeck:2011na,Alekhin:2011sk,Ball:2012wy,Forte:2013wc,Jimenez-Delgado:2013sma,
Rojo:2015acz,Butterworth:2015oua,Accardi:2016ndt,Gao:2017yyd}.
%
It is based on QCD factorization of physical observables, {\it i.e.}
the fact that a class of hard-scattering cross-sections can be expressed as a 
convolution between short-distance, perturbative, matrix 
elements and long-distance, nonperturbative, PDFs.
%
By combining a variety of available hard-scattering experimental data with 
state-of-the-art perturbative calculations, complete PDF sets including 
the gluon and various combinations of quark flavors, are currently determined
for protons, in both the unpolarized~\cite{Ball:2017nwa,Harland-Lang:2014zoa,
Dulat:2015mca,Alekhin:2017kpj,Accardi:2016qay} and 
polarized~\cite{Nocera:2014gqa,deFlorian:2009vb,Sato:2016tuz,Hirai:2008aj} case.

Recent progress in global QCD analyses has been driven, on the one hand, 
by the increasing availability of a wealth of high-precision measurements from 
Jefferson Lab, HERA, RHIC, the Tevatron and the LHC and, on the other hand, 
by the advancement in perturbative calculations of QCD and 
electroweak (EW) higher-order corrections.
%
Parton distributions are now determined with unprecedented precision, 
in many cases at the few-percent level.
%
A paradigmatic illustration of this progress is provided by both the 
unpolarized and polarized gluon PDFs, which were affected by rather large 
uncertainties until recently, due to the limited experimental information 
available.
%
In the unpolarized case, the gluon PDF is now constrained quite accurately from 
small to large $x$ thanks to the inclusion of processes such as 
inclusive deep-inelatic scattering (DIS)~\cite{Abramowicz:2015mha}, 
$D$-meson production~\cite{Zenaiev:2015rfa,Gauld:2016kpd},
the transverse momentum of $Z$ bosons~\cite{Boughezal:2017nla},
inclusive jet production~\cite{Currie:2016bfm}, and top-quark pair
distributions~\cite{Czakon:2016olj,Guzzi:2014wia}.
%
In the polarized case, the gluon PDF is now constrained from double 
spin-asymmetries for high-$p_T$ jet and pion production in proton-proton 
collisions~\cite{deFlorian:2014yva,Nocera:2014gqa}, 
although only in the medium-to-large $x$ region.

The second method is {\it lattice QCD}~\cite{Olive:2016xmw,Gupta:1997nd}.
%
It is based on the direct computation of the QCD path integral in a 
discretized finite-volume Euclidean space-time, provided a suitable ultraviolet
cut-off.
%
In order to connect with experimental measurements, extrapolations to the 
continuum and infinite-volume limits are necessary in order to remove the  
cut-off dependence and finite-volume effects, respectively.
%
Lattice-QCD calculations require minimal external input: one needs only to 
set the hadronic scale $\Lambda_\text{QCD}$ and the values of the quark masses.
%
For calculations relevant to low-energy hadron structure, this means
setting the up, down and strange quark masses,
which is usually done using the pion and kaon masses as external inputs.
%
The overall hadronic scale can be set using well-determined baryon masses 
such as that of the $\Omega$ baryon.
%
A variety of QCD quantities can then be computed using lattice QCD, including
moments of PDFs or of certain quark flavor PDF combinations.

Early lattice-QCD attempts to determine the proton PDFs were limited by the 
available computational resources and various technical challenges, with most 
results restricted to the first few moments of non-singlet PDFs at relatively 
large (unphysical) quark masses.
%
Overcoming these limitations, recent progress has been mostly
driven by advances in two main areas. 
%
First, by improved systematic control (physical pion mass, excited-state 
contamination, large volumes) for quantities such as the nucleon matrix 
elements corresponding to the low moments of PDFs.
%
Second, by the  development of novel strategies
for the computation of the first few 
moments~\cite{Constantinou:2014tga,Syritsyn:2014saa,Lin:2012ev},
the determination of more challenging quantities 
such as gluon and flavor-singlet matrix elements, and
for the direct calculation of the 
Bjorken-$x$ dependence of PDFs~\cite{Lin:2014zya,Alexandrou:2015rja,
Chen:2016utp,Alexandrou:2016jqi}.

These developments have pushed lattice-QCD calculations to the point where, 
for the first time, it is possible to provide information on the PDF shape
of specific flavor combinations, both for quarks and for antiquarks, 
and where meaningful comparison with global fits can be made.
%
Indeed, one of the main motivations for these lattice-QCD efforts is to 
achieve a sufficient accuracy to constrain the PDFs obtained from global 
analyses.

Despite these developments in both the global QCD analysis and lattice-QCD 
methods, interplay between the two -- and communication between the 
respective communities of physicists -- have been rather limited so far.
%
This situation led some of us to organize the first workshop on
{\it Parton Distributions and Lattice Calculations in the LHC Era}
(PDFLattice2017), which took place in the Balliol College, University of 
Oxford, in March 2017.\footnote{\url{http://www.physics.ox.ac.uk/confs/PDFlattice2017/index.asp}}
%
The main goal of this workshop was to establish a common ground 
and languange for discussions between the two communities.
%
In addition, we aimed to carry out a first quantitative exploration of how PDF 
fits can be exploited to benchmark existing and future lattice calculations,
and of how lattice-QCD calculations could be used to improve global PDF fits.
%
In this context, some of the questions that were addressed during this workshop
included the following.
\begin{itemize}
\item What information from PDF fits is relevant to constrain, 
  test, or validate lattice calculations?

\item What PDF-related quantities are most compelling
  to compute in lattice QCD in terms of phenomenological relevance?

\item What accuracy do we need from lattice quantities 
  in order to have a significant impact on global PDF fits?

\item What information does lattice QCD provide on the
  shape (Bjorken-$x$ dependence) of the PDFs? Which specific
  PDF moments can be computed?
  
\item How do we consistently quantify the systematic uncertainties 
  in lattice-QCD calculations?

\item To what extent do available lattice results agree with the results of
  global PDF fits? Is there a tension between global PDF fits, PDF
  fits based on reduced datasets, and PDF calculations from the lattice?

\item What is the accuracy that can be expected from lattice-QCD
  calculations in the near and medium future? What will be their
  constraining power on PDFs?

\end{itemize}

This white paper summarizes the joint effort between the two communities to 
address some of these questions, and follows up on the very fruitful 
discussions and interactions that took place both during 
the workshop and in the subsequent months.
%
While this document does not represent the final word on this topic, it 
provides a solid starting point for subsequent collaborative efforts, and 
should facilitate smooth interactions between the two communities in the future.

The outline of this white paper is the following.
%
In Sect.~\ref{sec:theoryoverview} we review the global QCD analysis and 
lattice-QCD methods for the determination of polarized and unpolarized PDFs.
%
In Sect.~\ref{sec:benchmarking} we present state-of-the-art benchmarks 
for selected PDF moments between the most recent lattice-QCD calculations and 
global QCD analyses.
%
In Sect.~\ref{sec:projections} we quantitatively assess the impact that
lattice calculations of PDF-related quantities could have on unpolarized
and polarized global analyses, assuming different scenarios for the 
uncertainties in the lattice-QCD calculations.
%
In Sect.~\ref{sec:outlook} we conclude
and discuss future interactions between
the global QCD analysis and lattice-QCD communities.
%
In Appendix~\ref{app:notation} we summarize the conventional notation
adopted in this document for the definition of the PDF moments; 
in Appendix~\ref{sec:LQCDtables} we compile bibliographical tables for
existing lattice-QCD calculations of PDF moments;
and in Appendix~\ref{app:Hmoms} we collect some
additional results of PDF moments from global QCD analyses.




\section{Theory overview}
\label{sec:theoryoverview}

In this first section of the white paper we present a succinct
summary of the theoretical background that underlies
first of all lattice QCD calculations of PDF-related
quantities, on the one hand, and global fits
of PDFs, on the other hand.
%
Special attention has been devoted to ensure a unified
consistent notation among the various parts of the whole section.

Lattice QCD is QCD formulated on a finite Euclidean lattice and is generally
studied by numerical computation of QCD correlation functions in the
path-integral formalism, using methods adapted from statistical
mechanics. To make contact with the physical world and experimental
data, the numerical results are extrapolated to the continuum 
and infinite-volume limits. The past decade has seen significant progress in
the development of efficient algorithms for the generation of
ensembles of gauge field configurations, which represent the QCD
vacuum, and tools for extracting the relevant information from lattice-QCD
correlation functions. Lattice-QCD calculations have reached a level where
they not only complement, but also guide current and forthcoming
experimental programs.

\subsubsection{Systematic uncertainties}
Lattice-QCD calculations must demonstrate control over all sources of
systematic uncertainty introduced by the discretisation of QCD on the
lattice to make meaningful contact with experimental data. These
include discretisation effects that vanish in the continuum limit;
extrapolation from unphysically heavy pion masses; finite volume
effects; and renormalisation of composite operators. To take the 
continuum limit requires accurate determinations of the lattice-spacing. 
We briefly review these main sources of systematic
uncertainty here; for a fuller account see, for
example, \cite{Aoki:2016frl}.

\begin{itemize}

\item {\bfseries Discretisation effects and the continuum limit.} There is 
a fair degree of flexibility in discretising the QCD action. This has
led to a variety of formulations, which differ mainly in the choice of
the action for quarks. In the continuum limit, which corresponds to taking
the lattice spacing $a$ to zero with all physical quantities fixed,
the simplest discretisations differ from continuum QCD at ${\cal
O}(a)$. In practice, one cannot afford to perform numerical
simulations at arbitrarily small lattice spacings, because the cost of
computation increases with a large inverse power of the lattice
spacing, therefore ${\cal O}(a)$ effects can be significant even
with current lattice spacings ranging from $0.15 \,\mbox{fm}$ to
$0.05 \,\mbox{fm}$. To accelerate the convergence to the continuum
limit, improved quark and gluon actions are widely used, which include
higher-dimension operators to reduce the discretisation errors to
$O(a^2)$ or better.

\item {\bfseries Unphysical pion masses.} 
The computational cost of the fermion contribution to the path
integral increases with a large inverse power of the bare quark mass
(or, equivalently, the pion mass). Lattice-QCD calculations are therefore
often performed at unphysically heavy pion masses, although results calculated
directly with physical pion masses have become increasingly common, albeit with large
errors. To obtain results at the physical pion mass, lattice data are
generated at a sequence of pion masses and then extrapolated to the
physical pion mass. To control the associated systematic
uncertainties, these extrapolations are guided by effective
theories. In particular, the pion-mass dependence can be parameterised
using chiral perturbation theory ($\chi$PT), which accounts for the
Nambu-Goldstone nature of the lowest excitations that occur in the
presence of light quarks. 

\item {\bfseries Finite volume effects.} Numerical Lattice-QCD 
calculations are necessarily restricted to a finite space-time
volume. For most simple quantities, these effects decay exponentially
with the size of the lattice, and therefore the easiest way to
minimise or eliminate finite volume effects is to choose the volume
sufficiently large in physical units. Unfortunately, this can be
prohibitively expensive as one approaches the continuum limit, requiring the
number of lattice sites to grow as $L/a$ in all four directions. Finite volume $\chi$PT is the preferred
tool to develop systematic expansions that provide quantitative
information on finite-volume effects. In general, finite volume
effects of hadrons are dominated by their interactions with pions,
which can travel around the (periodic) lattice many times. Numerical
evidence suggests that lattice sizes of $m_\pi L \geq 4$, where
$m_\pi$ is the pion mass, are generally sufficiently large that finite
volume effects are negligible, within the current precision of Lattice-QCD
calculations.

\item {\bfseries Excited state contamination.} At small Euclidean times, a lattice-QCD correlation function
is a sum over a tower of states that behave as $e^{-m_it}$, where $m_i$ is the 
mass of the state and $t$ is the Euclidean time. Thus, at large Euclidean times,
ground-state quantities can be extracted by fitting to the dominant exponential behaviour.
Unfortunately, the signal-to-noise ratio
is exponentially suppressed by the difference in the nucleon and pion masses in this limit. Thus, lattice-QCD results
are extracted from an intermediate region in which excited state contributions are 
either small or well-controlled and the signal-to-noise ratio is sufficiently large that
the signal can be reliably extracted. This is a particular challenge for baryons and,
until recently, was one of the largest sources of systematic uncertainties for
nucleon matrix elements.

\item {\bfseries Renormalisation.} The matrix elements extracted from a 
Lattice-QCD calculation at a given lattice spacing are bare matrix elements,
rendered finite by the presence of the lattice spacing, which serves
as a gauge-invariant UV regulator. To take the continuum limit,
i.e. remove the regulator, one must renormalise the corresponding
operators and fields and match them to some common scheme and scale used 
by phenomenologists. Although renormalisation is traditionally
discussed in the framework of perturbation theory, at hadronic energy
scales the renormalisation constants should be computed
nonperturbatively to avoid uncontrolled uncertainties due to 
truncated perturbative results. In QCD with only light quarks it is technically
advantageous to employ so-called mass-independent renormalisation
schemes. This requires a renormalisation condition that can be
implemented on the lattice as well as in continuum perturbation
theory. A common choice is the RI-MOM scheme~\cite{Martinelli:1994ty}.

In addition, on a hypercubic lattice, the orthogonal group $O(4)$ of
continuum Euclidean space-time is reduced to the hyper-cubic group
$H(4)$. Thus, operators are classified according to irreducible
representations of $H(4)$~\cite{Gockeler:1996mu}. Different
irreducible representations belonging to the same $O(4)$ multiplet
will, in general, give different answers at finite lattice spacing, an
effect that can be reduced by improving the
operators~\cite{Gockeler:2004wp}. Conversely, operators that lie in
different irreducible representations of $O(4)$, but the same irreducible
representations of $H(4)$, will mix at finite lattice spacing but not
in the continuum. When these operators have differing mass dimension,
the mixing coefficients scale with the inverse lattice spacing to some
power, and diverge in the continuum limit. This power-divergent mixing
must be removed nonperturbatively, and is a particular challenge for
lattice calculations of the Mellin moments of PDFs (see
Section \ref{Sec:MomentsLQCD}).

Finally, it is worth noting that factorisation, the key assumption of
the operator product expansion (OPE), demands that the
nonperturbatively renormalised hadron matrix elements are matched to the
perturbatively renormalised Wilson coefficients at a scale where the perturbative 
expressions show convergence. This appears to be
the case for scales $\mu^2 \gtrsim 10 \, \mbox{GeV}^2$ at
least~\cite{Gockeler:2010yr}. This, however, is a fundamental aspect
of QCD, and is not restricted to lattice QCD. The DGLAP evolution equations,
for example, work best for $q^2_{\rm min} \approx
15 \, \mbox{GeV}^2$~\cite{Abramowicz:2015mha}, which should be kept in
mind when comparing lattice results with phenomenology.

\item {\bfseries Lattice-spacing determination.} Numerical lattice-QCD calculations 
naturally determine all dimensionful quantities in units of the
lattice spacing. Thus, extracting physical values requires the
determination of the lattice scale. This is achieved by matching a
quantity with mass-dimension to its experimental value or through a
well-defined theoretical procedure, that is referred to as
``scale-setting''. Popular reference scales include light decay
constants, hadron masses, scales defined in terms of the heavy quark
potential or, most recently, the Wilson flow time
$\sqrt{t_0}$~\cite{Luscher:2010iy}. The Wilson flow scale has become
increasingly popular, because it can be computed rather cheaply and
with high precision, unlike hadron masses, for example.

\end{itemize}

These sources of systematic uncertainty all need to be under control
when confronting experimental data with lattice results, or vice
versa. For a coherent assessment of the present state of lattice-QCD
calculations of various quantities, the degree to which each
systematic has been controlled in a given calculation is an important
consideration. Therefore, in the following sections, we indicate the
quality of the lattice calculations, based on criteria inspired by the
FLAG analysis of flavour physics on the lattice~\cite{Aoki:2016frl}.


\subsubsection{Mellin moments of PDFs from lattice QCD}
\label{Sec:MomentsLQCD}

PDFs cannot be directly determined in Euclidean lattice QCD, because their 
field-theoretic definition involves fields at light-like separations. Instead, 
the traditional approach for lattice-QCD calculations has been to determine the matrix elements of local twist-two operators, where twist is the dimension minus the spin, that can be related to the Mellin moments of PDFs. In principle, given a sufficient number of Mellin moments, PDFs can be reconstructed from the inverse Mellin transform. In practice, however, the calculation is limited to the lowest three moments, because power-divergent mixing occurs between twist-two operators on the lattice. Three moments is insufficient to reconstruct the momentum dependence of the PDFs without significant model dependence~\cite{Detmold:2003rq}. The lowest three moments do provide, however, useful information, both as benchmarks of lattice-QCD calculations and as constraints in global extractions of PDFs. Here we briefly review the determination of Mellin moments of PDFs from lattice QCD. 

Using the operator product expansion (OPE), the Mellin moments of structure functions, and the corresponding PDFs, can be expressed in terms of matrix elements of local operators:
\begin{align}
\!\!\!2 \int_0^1 dx\, x^{n-1} F_1(x,Q^2) &= \sum_a C_{1,a}^n(\mu^2)\, v_a^n(\mu^2)|_{\mu^2=Q^2} = \sum_a c_{1,a}^n(\mu^2)\, \int_0^1 dx\, x^{n-1} f_a(x,Q^2),\\
4 \int_0^1 dx\, x^n g_1(x,Q^2) &= \sum_a e_{1,a}^n(\mu^2)\, a_a^n(\mu^2)|_{\mu^2=Q^2} = \sum_a e_{1,a}^n(\mu^2)\, \int_0^1 dx\, x^n\, 2 \Delta f_a(x,Q^2),
\end{align}
where $v_i^n(\mu^2)$ and $a_i^n(\mu^2)$ are reduced matrix elements of the appropriate twist-two operators~\cite{Gockeler:1995wg},
\begin{align}
\frac{1}{2} \sum_s \langle p,s|\mathcal{O}^i_{\{\mu_1,\cdots,\mu_n\}}|p,s\rangle = {} & 2 v_i^n\, [p_{\mu_1}\cdots p_{\mu_n} - {\rm traces}] , \label{eq:twist2me}\\
\langle p,s|\mathcal{O}^{5\,i}_{\{\sigma \mu_1,\cdots,\mu_n\}}|p,s\rangle = {} & \frac{1}{n+1} a_i^n\, [s_\sigma p_{\mu_1}\cdots p_{\mu_n} - {\rm traces}]
\end{align}
and $C_{1,i}^n(\mu^2)$ and $e_{1,i}^n(\mu^2)$ are the Mellin moments of the corresponding Wilson coefficients
\begin{equation}
C_{1,i}^n(\mu^2) = \int_0^1 dy\, y^{n-1} c_{1,i}(y,\mu^2)\,, \quad
e_{1,i}^n(\mu^2) = \int_0^1 dy\, y^n e_{1,i}(y,\mu^2)\,.
\end{equation}
The trace terms include operators with at least one factor of the metric tensor $g^{\mu_i \mu_j}$ multiplied by
operators of dimension $(n+2)$ with $n-2$ Lorentz indices. The operators relevant for the lowest two moments are listed in Table \ref{Tab:twist2}. The operator $\mathcal{O}^q_{\mu_1\mu_2}$ decomposes into two different representations of $H(4)$~\cite{Gockeler:1996mu}, each with different lattice artifacts and renormalisation factors. In the continuum limit, however, both operators should lead to the same result. In contrast, the operator $\mathcal{O}^q_{\mu_1\mu_2\mu_3}$ splits into several representations transforming identically under $H(4)$, causing the corresponding operators to mix under renormalisation on the lattice.
\begin{table}
%\begin{ruledtabular}%Note this requires RevTeX, but makes the spacing look more professionally typeset.
\renewcommand{\arraystretch}{1.6} 
\centering
\begin{tabular}{@{}ccc@{}}
\hline 
%\rule[-3 ex]{0pt}{7 ex}  %% add some extra space %% Not necessary with the spacing required to fit operators properly.
Matrix element & Operator & Observable \\ 
\hline
$v_q^2$\,, $v_{\bar{q}}^2$  & $\displaystyle \left({\rm i}/2\right) \bar{q}(x)\gamma_{\mu_1} \overleftrightarrow{D}_{\mu_2} q(x)$ & $\langle x \rangle_q$\,, $\langle x \rangle_{\bar{q}}$   \\
$v_q^3$\,, $v_{\bar{q}}^3$  & $\displaystyle \left({\rm i}/2\right)^2 \bar{q}(x)\gamma_{\mu_1} \overleftrightarrow{D}_{\mu_2} \overleftrightarrow{D}_{\mu_3} q(x)$ & $\langle x^2 \rangle_q$\,, $\langle x^2 \rangle_{\bar{q}}$ \\
$a_q^0$ & $\displaystyle \bar{q}(x)\gamma_{\sigma} \gamma_5 q(x)$ & $2\, \langle 1 \rangle_{\Delta q}$ \\
$a_q^1$ & $\displaystyle \left({\rm i}/2\right) \bar{q}(x)\gamma_{\sigma} \gamma_5 \overleftrightarrow{D}_{\mu_1} q(x)$ & $2\, \langle x \rangle_{\Delta q}$ \\
$v_g^2$ & $\displaystyle - {\rm Tr}\, F_{\mu_1\alpha}F_{\mu_2\alpha}$ & $\langle x \rangle_g$ \\
\hline
\end{tabular}
%\end{ruledtabular}
\caption{\label{Tab:twist2}
Operators relevant to the lowest two Mellin moments of polarised and unpolarised PDFs.
}
\end{table}

\paragraph*{Higher-twist contributions} The discussion so far has focussed on the limit in which higher twist contributions, suppressed by powers of the momentum-transfer, have been ignored. In fact, higher twist contributions to the lowest moment of the structure function $F_1(x,Q^2)$ are found to be of ${\cal O}(1\, \mbox{GeV}^2/Q^2)$ \cite{Blumlein:2008kz}. For lattice QCD, typically $Q^2 \simeq 1/a^2$, and at present lattice spacings this corresponds to $Q^2 = O(10\,\mbox{GeV}^2)$ or a higher-twist contribution of $5 - 10\, \%$. With contributions of higher-twist included, the OPE reads
\begin{equation}
2 \int_0^1 dx\, x F_1^q(x,Q^2) = C_{1,q}^2(\mu^2)\, v_q^2(\mu^2)|_{\mu^2=Q^2} + \frac{\bar{C}_{1,q}^2(\mu^2)}{Q^2}\, \bar{v}_q^2(\mu^2)|_{\mu^2=Q^2} + \cdots \,,
\label{tex}
\end{equation}
where $\bar{C}_{1,q}^2$ and $\bar{v}_q^2(\mu^2)$ are the Wilson coefficient and reduced matrix element of a generic twist-four operator. Both twist-two and four contributions mix under renormalisation, to the extent that the perturbative series for the Wilson coefficients $C_{1,q}^2(\mu^2)$ diverges due to the presence of IR renormalon singularities. This ambiguity is canceled by that in the twist-four matrix element $\bar{v}_q^2(\mu^2)$ that arises as a result of an UV renormalon singularity~\cite{Martinelli:1996pk}. If mixing effects are ignored, the uncertainties will be, at least, comparable to the power corrections themselves. Power corrections can be assessed most efficiently, and the twist expansion tested, by a direct lattice-QCD evaluation of the Compton amplitude, which we discuss in Section \ref{Sec:InversionMethod}.

\paragraph*{Beyond the first three moments} Moving beyond the lowest three moments requires overcoming the challenge of power-divergent mixing for lattice-QCD twist-two operators. One novel approach to this problem~\cite{Davoudi:2012ya} builds upon the physical intuition that as long as the scale associated with the operator (for the twist-two operators, this is the renormalisation scale $\mu$) is taken to be much smaller than the hadronic scale but much larger than the inverse lattice spacing, no singularity necessarily arises as one takes the continuum limit. The operator can still probe the correct hadron structure at the scale $\mu$, but should be insensitive to the details of the discretisation of the operator at shorter distances. A simple way to incorporate an intrinsic ``smearing” scale for an operator is to sum over bilinears of quark fields that are displaced over many lattice sites in a small (compared to the scale $1/\mu$) region of Euclidean space-time (an alternative approach appears in~\cite{Monahan:2015lha}). To ensure that the correct $SO(4)$ transformation properties of the matrix elements are recovered in the continuum limit, one must project the sum using hyper-spherical harmonics. The properties of these operators, such as their mixing patterns and scaling properties, are discussed in detail in
Ref.~\cite{Davoudi:2012ya}. In particular, while the classical mixing with lower and higher spin operators are both suppressed by $\sim a^2$ for spatially improved operators, the mixing at one-loop in lattice perturbation theory is suppressed by ${\cal O}(\ alpha_s a)$ or ${\cal O}(\ alpha_s a^2)$, depending on the lattice action used and provided that the gauge links used in constructing the gauge invariant bilinears are tadpole-improved and smeared over a region whose physical size is held fixed as the continuum limit is taken. In principle, this allows higher moments of PDFs to be obtained from lattice QCD, without power-divergences. Numerical investigations of this approach,
which requires gauge configurations with very fine lattice spacings, are underway.

\subsubsection{The $x$-dependence of PDFs from lattice QCD}

While the lowest three moments of PDFs provide important benchmarks for lattice-QCD calculations of nucleon structure, and useful constraints in global extractions of PDFs, they are not in themselves sufficient to determine the $x$-dependence of PDFs, particularly at small $x$. In the following section we summarise recent approaches to determining the $x$-dependence of PDFs directly from lattice QCD.

\paragraph*{Inversion method} 
\label{Sec:InversionMethod}

The Compton amplitude $T_{\mu\nu}(p.q)$ can be directly obtained in lattice QCD, including disconnected contributions,  by a simple extension~\cite{Chambers:2017dov} of existing implementations of the Feynman-Hellmann technique to lattice QCD~\cite{Horsley:2012pz,Chambers:2014qaa,Chambers:2015bka}. Provided one works at sufficiently large $Q^2$, the Compton amplitude will be dominated by twist-two contributions. Varying $Q^2$ allows one to test the twist expansion and, in particular, isolate twist-four contributions. Moreover, one can distinguish between contributions from up, down and strange quarks, connected and disconnected, by appropriate insertions of the electromagnetic current.

To compute the Compton amplitude from the Feynman-Hellmann relation, a perturbation to the QCD Lagrangian is introduced, for example,
\begin{equation}
\mathcal{L}(x) \rightarrow \mathcal{L}(x) + \lambda \mathcal{J}_3(x)\,, \quad \mathcal{J}_3(x)=Z_V\cos(\vec{q}\vec{x})\; e_q \,\bar{q}(x)\gamma_3 q(x) 
\label{in}
\end{equation}
where $q$ is the quark field to which the photon is attached, and $e_q$ its electric charge. For simplicity, we consider the local vector current only, so that the renormalisation factor $Z_V$ is known and no further renormalisation is needed. Taking the second derivative of the nucleon two-point function 
\begin{equation}
\langle N(\vec{p},t) \bar{N}(\vec{p},0)\rangle_\lambda \simeq C_\lambda\, {\rm e}^{-E_\lambda(p,q)\,t}
\end{equation}
with respect to $\lambda$ on both sides, gives
\begin{equation}
-2 E_\lambda(p,q)\, \frac{\partial^2}{\partial\lambda^2}  E_\lambda(p,q)\,\big|_{\lambda=0} = T_{33}(p,q) \,.
\end{equation}
For $p_3=q_3=q_4=0$ this leaves us with
\begin{equation}
T_{33}(p,q) = 4 \omega^2 \int_0^1 dx\,  \frac{xF_1(x,Q^2)}{1-(\omega x)^2} \,.
\label{ff}
\end{equation}
Note that to extract the polarised structure functions requires insertions of two different currents with $\mu\neq \nu$. The idea is then to solve \eqref{ff} for $F_1(x,Q^2)$ numerically.
%
%\textbf{This section is clearly too technical and long.}
%Starting from a finite number of sampled points $T_\alpha=T_{33}(\omega_\alpha) \,,\; \alpha=1, \cdots, N$ and approximating the integral and structure function by a discrete set of $M$ points, $0 < x_1 < %x_2 < \cdots < x_M < 1$, 
%\begin{equation}
%K_{\alpha\beta} = \frac{4\,\omega_\alpha^2x_\beta}{1-(\omega_\alpha x_\beta)^2} \,, \quad F_\beta = F_1(x_\beta)\,,
%\end{equation}
%the integral equation (\ref{ff}) reduces to the set of equations 
%\begin{equation}
%T_\alpha = \epsilon \sum_{\beta=1}^M K_{\alpha\beta}\, F_\beta \,,\; \alpha=1, \cdots, N \,,
%\label{ieq}
%\end{equation}
%where, in general, $N < M$ and the dependence on $Q^2$ has been dropped. The simplest procedure uses equidistant step sizes, $\epsilon$, but the generalisation to adaptive step sizes is straightforward. %The $N \times M$ matrix $K$ can be written as the product of a $N \times N$ orthogonal matrix $U$, a $N \times N$ diagonal, singular matrix $W$ with positive or zero eigenvalues $w_1 < w_2 < \cdots < w_N$, %and the transpose of a row-orthogonal $M \times N$ matrix $V$,
%\begin{equation}
%K = U \,\left[{\rm diag}(w_1, \cdots, w_N)\right]\,V^T \,.
%\end{equation}
%Since the matrix $W$ is singular, singular value decomposition (SVD) is the method of choice for solving \eqref{ieq} under such conditions. The solution is
%\begin{equation}
%F_\beta = \sum_{\alpha=1}^N K^{-1}_{\beta\alpha}\epsilon^{-1}\, T_\alpha \,, 
%\label{svd}
%\end{equation}
%where $K^{-1}$ is the pseudo-inverse
%\begin{equation}
%K^{-1} = V \,\left[{\rm diag}(1/w_1, \cdots, 1/w_K, 0, \cdots, 0)\right]\, U^T 
%\end{equation}
%with $1/w_\gamma$ being replaced by zero if $w_\gamma=0$, which is assumed for $K < \gamma\leq N$. One has to exercise some discretion at deciding at
%what threshold to set $1/w_\gamma$ to zero. Several routines, such as {\tt Pseudo-Inverse} of {\it Mathematica}, solve this problem
%automatically. 
In~\cite{Chambers:2017dov} it was shown that the unpolarised structure function $F_1(x,Q^2)$ can be computed from a lattice calculation of the Compton amplitude with unprecedented accuracy, devoid of any renormalisation and mixing issues. Furthermore, by extending the calculation to values $\omega > 1$ it becomes possible
to compute the structure functions down to fractional momenta $x = {\cal O}(0.001)$. With the same method, the PDFs can be computed directly without the need to go through the structure functions, provided $Q^2$ is sufficiently large that power corrections can be neglected. 
%In this case
%\begin{equation}
%T_\alpha^q = \epsilon \sum_{j=1}^M K_{ab}\, q_i(x_\beta,Q^2) 
%\end{equation}
%where the index $i$ denotes the struck quark. The kernel now reads
%\begin{equation}
%K_{\alpha\beta} =2 \omega_\alpha^2 \int_0^1 dy\,y\,x_\beta\, \frac{c_{1,q}(y,\mu^2)|_{\mu^2=Q^2}\, }{1-(y\,\omega_\alpha\, x_\beta)^2}
%\end{equation}
%with $c_{1,q}(y,\mu^2)$ a perturbative Wilson coefficient.

%\paragraph{RQCD Approach}

%{\bf To be completed.}

%Gunnar Bali (0.5 page)

% At present this section does not describe a lattice method, so should probably be put in the global extraction of PDFs part of the theory review.
%\paragraph{PDFs from the Hadronic Tensor}
%\label{sec:HadronicTensorMethod}
%\input{HadronicTensorMethod}

\paragraph{Quasi PDFs}
%\label{Sec:QuasiPDFMethod}
Quasi PDFs provide an alternative approach to determining the $x$-dependence of PDFs directly from lattice QCD \cite{Ji:2013dva,Ji:2014gla}. In the following discussion, we focus on the flavor-nonsinglet quasi PDF, for which we can ignore mixing with the gluon quasi PDF. The unpolarized quasi quark PDF is defined as the momentum-dependent
nonlocal static matrix element
\begin{align}\label{eq:qPDF}
\widetilde{q}(x,\Lambda,p_z)  = \int \frac{dz}{4\pi} e^{-i x z p_z} 
\frac{1}{2}\sum_{s=1}^2\left\langle p,s\right\vert \bar{\psi}(z)\gamma_\alpha e^{ig\int_0^z
A_z(z^\prime) dz^\prime} \psi(0) \left\vert p,s\right\rangle ,
\end{align}
where $\Lambda$ is an ultraviolet (UV) cut-off scale, such as the inverse lattice spacing $1/a$. The Lorentz index $\alpha$ of the matrix $\gamma_\alpha$ has generally be chosen to be spatial, $\alpha = z$, but the alternative choice $\alpha = 4$ is also possible and removes the leading order twist-4 contamination \cite{Radyushkin:2016hsy}. Note that, because $p$ is finite, the momentum fraction $x$ can be larger than unity.

The quasi PDF is defined for nucleon states at finite momentum and must be related to the corresponding light-front PDF, for which the nucleon momentum is taken to infinity.
In the  large-momentum  effective field theory (LaMET) approach, the quasi PDF $\widetilde{q}(x,\Lambda,p_z)$ can be related to the $p_z$-independent
light-front PDF $q(x,Q^2)$ through~\cite{Ji:2013dva,Ji:2014gla}
\begin{equation} \label{eq:qPDFmatching}
\widetilde{q}(x,\Lambda ,p_z) = 
  \int_{-1}^1 \frac{dy}{\left\vert y\right\vert} 
    Z\left( \frac{x}{y}, \frac{\mu}{p_z}, \frac{\Lambda}{p_z}\right)_{\mu^2 = Q^2} q(y,Q^2) +
  \mathcal{O}\left( \frac{\Lambda_\text{QCD}^2}{p_z^2},\frac{m^2}{p_z^2}\right), 
\end{equation}
where $\mu$ is the renormalisation scale;
$Z$ is a matching kernel; and $m$ is the nucleon mass.
Here the $\mathcal{O}\left(m^2/p_z^2\right)$ terms are target-mass corrections and the $ \mathcal{O}\left(\Lambda_\text{QCD}^2/p_z^2\right)$ terms are higher twist effects, both of which are suppressed at large nucleon momentum. A complementary approach to LaMET views the quasi PDF as a ``lattice cross-section'' from which the light-front PDF can be factorized \cite{Ma:2014jla, Ma:2014jga}. 

To understand the origin of the power-suppressed corrections, we apply the operator product expansion (OPE) to the matrix element that defines the PDF, which becomes a linear combination of local twist-2 operators, with matrix elements in the proton state given by Eq.~\eqref{eq:twist2me}. From this it follows that the light-cone correlation function is {\it kinematically} connected with the +-components of the nucleon four-momentum. To eliminate the time dependence, we consider matrix elements of twist-two operators with $\mu_1=\mu_2=...=\mu_n=z$ and nucleon states with large $p_z$, to give
\begin{equation}
\frac{1}{2} \sum_s \langle p,s|\mathcal{O}^i_{\{z,\cdots,z\}}|p,s\rangle = 2v_i^n(\mu^2)\left[p_z^n-\lambda M^2 p_z^{n-2}-...\right], 
\end{equation}
where $\lambda$ is a number of order unity, and the ellipsis represents terms
with still lesser powers of $p_z$. Lorentz symmetry guarantees that the matrix elements of the trace terms are at most
$p_z^{n-2}$ times $\Lambda^2_{\rm QCD}$. Thus, we conclude that
\begin{equation}
      \langle p| {\overline \psi}\gamma_ziD_z ... iD_z \psi|p\rangle
       = 2v^n(\mu^2) p_z^n\times \left[ 1 + {\cal O}\left(\frac{\Lambda^2_{\rm QCD}}{p_z^2},  \frac{m^2}{p_z^2}\right)\right],
\end{equation}
where the non-leading terms are power-suppressed in the large $p_z$ limit. Equating the renormalized moments on the right hand side of this equation with those that appear in Eq.~\eqref{aaa}
%\eqref{eq:ope}
leads to the relation between the quasi PDF and the light-front PDF expressed in Eq.~\eqref{eq:qPDFmatching}.

Preliminary results from lattice calculations of quasi PDFs have been encouraging \cite{Lin:2014zya,Alexandrou:2015rja,Chen:2016utp,Alexandrou:2016jqi}, as we illustrate in Fig.~\ref{fig:quasipdfs}. There are a number of remaining challenges, however, that must be overcome for an {\it ab initio} determination of the $x$-dependence of PDFs directly from lattice QCD that incorporates complete control over systematic uncertainties. Lattice calculations of quasi PDFs are subject to the same sources of systematic uncertainty that plague all lattice calculations and are highlighted in Section \ref{Sec:IntroLQCD}, but here we focus on systematic uncertainties that are more specific to quasi PDFs. These are: uncertainties associated with the finite nucleon momentum of the lattice calculations and with the renormalisation of quasi PDFs.

\begin{itemize}
 \item Preliminary nonperturbative studies of the quasi PDF used nucleon momenta in the range $p_z = 2\pi/L$ to $6\pi/L$, where $L$ is the physical extent of the lattice, corresponding to $p_z = 0.5$ to $1.3$ GeV  \cite{Lin:2014zya,Alexandrou:2015rja,Chen:2016utp}. At such low momenta, higher-twist and target mass corrections are likely to be considerable.

Target mass corrections can be removed to all orders \cite{Chen:2016utp}, and twist-4 contributions can be removed in principle \cite{Radyushkin:2016hsy,Chen:2016utp}, leaving higher-twist contamination. To reduce these remaining effects, the authors of \cite{Lin:2014zya,Chen:2016utp} extrapolated to infinite nucleon momentum using the fit ansatz $a + b/p_z$ for each value of $x$, but do not include a complete estimate of the corresponding systematic uncertainty. Although the effects of finite nucleon momentum can be mitigated, it is likely that reducing systematic uncertainties to less than 20\% at moderate values of $x$ require significantly larger values of nucleon momentum \cite{Gamberg:2014zwa}, and larger values of $x\simeq 1$ may require nucleon momentum as large as $p_z > 4$ GeV.

Presently, the size of the nucleon momentum is restricted by the decreasing signal-to-noise ratio at large momenta, which requires very high statistics to extract a signal. New approaches to high-momentum nucleons are being investigated, with the most promising an approach that employs momentum smearing \cite{Bali:2016lva}. This method has been applied to quasi-PDFs in \cite{Alexandrou:2016jqi}, demonstrating a large improvement in the signal-to-noise ratio by reaching momenta of $\sim 2.5$ GeV, with small statistics.

\item The leading-twist quasi PDFs and light-front PDFs are connected through the matching (or ``factorization'') relation of Eq.~\eqref{eq:qPDFmatching}. Provided the quasi and light-front PDFs share the same infrared (IR) behavior, the matching kernel can be determined in perturbation theory~\cite{Xiong:2013bka}. The factorization of the IR structure of quasi PDFs into light-front PDFs and an IR-safe matching kernel was claimed to hold to all orders in \cite{Ma:2014jla, Ma:2014jga}. However, Ref.~\cite{Li:2016amo} asserted that there might be subtleties beyond leading order in perturbation theory. A distinct, but similar, issue is the IR structure of extended operators in Euclidean and Minkowski spacetime. There are again subtleties in perturbation theory \cite{Carlson:2017gpk}, but arguments based on general field-theoretic grounds demonstrate that the quasi PDF extracted from a Euclidean correlation function is exactly the same matrix element as that determined from the LSZ reduction formula in Minkowski spacetime \cite{Briceno:2017cpo}.

In contrast to the IR structure, the ultraviolet (UV) structure of the quasi PDF is quite different from the UV structure of the light-front PDF: the former has both linear and logarithmic divergences, while the latter contains only logarithmic divergences. Although there are no power-divergences in dimensional regularization, quasi PDFs determined on the lattice are regulated by the inverse lattice spacing. In the continuum limit (for which $a\to 0$, with all physical quantities held fixed) there is a divergence, associated with the length of the Wilson line $z$, that scales as $z/a$. This divergence must be removed nonperturbatively.

For a general non-local bilinear operator with Lorentz structure $\Gamma$, the renormalised operator $O_{\Gamma}^{\rm (ren)}(z,\mu)$ is
related to its bare operator $O^{(0)}_{\Gamma}(z)$ by \cite{Dotsenko:1979wb, Arefeva:1980zd, Craigie:1980qs,Dorn:1986dt}
\begin{eqnarray}\label{eq:renorm_non-local}
O_{\Gamma}^{\rm (ren)}(z,\mu)=e^{\delta m(\mu)|z|}Z_{\psi, z}(\mu,z)O^{(0)}_{\Gamma}(z),
\end{eqnarray}
where $\delta m$ is the mass renormalisation of a test particle moving along the Wilson line of length $z$ and $Z_{\psi, z}(\mu,z)$ removes the remaining logarithmic divergences associated with the Wilson line endpoints (the quark fields). This result holds to all orders in perturbation theory: the exponentiated counterterm $\delta m(\mu)$ completely removes the linear divergence. The multiplicative renormalizability of the remaining logarithmic UV divergence, however, has not been proven~\cite{Ji:2015jwa}. The exponentiated counterterm can be determined using a static heavy quark potential, which shares the same power-law divergence as the non-local quark bilinear \cite{Musch:2010ka,Ishikawa:2016znu, Chen:2016fxx}.

Once the linear divergence has been removed nonperturbatively, lattice perturbation theory can be used to renormalize the remaining logarithmic divergences in the quasi PDF \cite{Ishikawa:2016znu, Carlson:2017gpk}. A delicate point regarding the renormalisation is the mixing among certain subsets of these non-local operators. Such mixing has been identified at one-loop in perturbation theory in \cite{Constantiou:2017soon} for a variety of fermion/gluon actions. The mixing coefficients are necessary to disentangle the individual matrix elements for each quasi-PDF from lattice calculation data. Of particular interest is the case of the unpolarized quasi-PDF, which mixes with the scalar quasi-PDF if the Lorentz index of Eq.~\eqref{eq:qPDF} is in the same direction as the Wilson line. In contrast, the axial and transversity PDFs with a Lorentz index in the Wilson line direction do not exhibit any mixing (to one-loop in perturbation theory). 

In addition, nonperturbative schemes, such as the regularization-independent RI/MOM scheme~\cite{Martinelli:1994ty}, can be used to renormalize matrix elements determined on the lattice. Nonperturbative schemes avoid the use of lattice perturbation theory at low energy scales (usually chosen to be $\mu = \pi/a$), although perturbative matching between renormalisation schemes is still necessary for PDFs expressed in the $\overline{\rm MS}$ scheme. Combining a nonperturbative renormalisation scheme with a step-scaling procedure \cite{Luscher:1991wu} significantly reduces perturbative truncation uncertainties by providing a nonperturbative method for reaching high energy scales. Two such nonperturbative renormalisation methods for quasi PDFs have recently been constructed \cite{Alexandrou:2017huk,Chen:2017mzz}. 

An alternative approach to removing both the linear and logarithmic divergences is provided by the gradient flow, a deterministic evolution of the quark and gluon degrees of freedom in a new parameter, the flow time, that renders all correlation functions finite \cite{Narayanan:2006rf,Luscher:2011bx,Luscher:2013cpa}. By fixing the flow time in the continuum limit, finite continuum quasi PDFs can be extracted from lattice calculations and then directly matched to light-front PDFs using perturbation theory~\cite{Monahan:2016bvm}.

Ultimately, perturbative matching must be carried out at a sufficiently high scale that truncation uncertainties can be safely neglected. Until a nonperturbative step-scaling scheme has been devised for quasi PDFs, perturbative truncation uncertainties are likely to be, in conjunction with finite nucleon momentum effects, the dominant source of systematic uncertainty in lattice determinations of quasi PDFs.
\end{itemize}

\paragraph{Status} We review the current status of lattice calculations of Mellin moments of PDFs in Sec.~\ref{sec:benchmarking}. The study of the $x$-dependence of PDFs directly from lattice QCD is still in its infancy and at this stage it would be premature to attempt a global analysis, as we undertake for the moments of PDFs. Here, however, we illustrate the progress that has been made with a snapshot of the most recent results.

Two groups have carried out lattice calculations of quasi-PDFs, and example results are shown in Fig.~\ref{fig:quasipdfs}.

%%%%%%%%%%%%%%%%%%%%%%%%%%%%%%%%%%%%%%%%%%%%%%%%%%%%%%%%%%%%%%%%%%%%%
\begin{figure}[t]
\begin{center}
  \includegraphics[scale=0.75]{plots/nnpdf31nnlo-10.pdf}
  \includegraphics[scale=0.75]{plots/nnpdf31nnlo-1e4.pdf}
  \caption{\small Illustrative plots of quasi PDFs from \cite{XXX} (left) and \cite{YYY} (right).
    \label{fig:quasidfs}
  }
\end{center}
\end{figure}
%%%%%%%%%%%%%%%%%%%%%%%%%%%%%%%%%%%%%%%%%%%%%%%%%%%%%%%%%%%%%%%%%%%%%%



\subsection{Global PDF fits}

In this section we describe the theoretical and methodological framework that underpins global
determinations of parton distribution functions from hard-scattering data.
%
First we describe the general approach and its applications to unpolarized PDFs, and then
move to succinctly review the determination of polarized PDFs.
%
The discussion here is restricted to the essential information that is required to connect
with the lattice QCD discussion earlier in this document, more details about the
global PDF fitting framework can be found in the reviews of
Refs.~\cite{Ball:2012wy,Forte:2013wc,Rojo:2015acz,Butterworth:2015oua}.

%%%%%%%%%%%%%%%%%%%%%%%%%%%%%%%%%%%%%%%%%%%%%%%%%%%%%%%
\subsubsection{Unpolarized PDFs}

\subsubsection*{General framework}

We express the collinear unpolarized PDFs as $f_{i}(x,\mu)$
where the index $i$ represents the parton flavor $i=\{g,u,\bar{u},d,\bar{d},s,\bar{s},...\}$,
$x$ is the fractional momentum carried by the parton, and $\mu$
is the factorization  scale.\footnote{We could add an additional index to specify the particular hadron
(proton, neutron, pion, nuclei, ...); as we mainly refer to the proton
  in this work, we will omit such a designation unless necessary.}
%
The value of $\mu$ is usually chosen to be the typical scale of the hard scattering
process involving the PDFs.
%
The PDF is a scheme-dependent quantity, and we typically work in
the $\overline{\rm MS}$ scheme; when this is convoluted with an appropriate
hard cross section (Wilson coefficient), we obtain a scheme-independent
physical observable (up to subleading terms in the perturbative expansion). 


The unpolarized PDFs appear in the factorization formulae for both inclusive DIS and hadroproduction processes. Namely, the DIS structure functions are given by
\begin{equation}
F_i(x,Q^2)=x\sum_a \int_x^1 \frac{{\rm d}z}{z}\,C_{i,a}\left(\frac{x}{z},\alpha_S(Q^2)\right)f_a(z,Q^2)\, ,
\end{equation}
where $x=\frac{Q^2}{2p\cdot q}$ is the standard Bjorken variable, given in terms of the momenta $q$ of the virtual photon ($q^2=-Q^2$), and $p$ of the proton. 
%
The hard coefficient functions $C_{i,a}$ can be computed perturbatively as an expansion in $\alpha_S$.
%
For the hadroproduction of an object $X$ in $pp$ collisions we have
\begin{equation}
  \label{eq:LHCxsec}
\sigma_{pp\to X}=\sum_{a,b}\int {\rm d}x_1 {\rm d}x_2\, f_a(x_1,\mu_F^2)f_b(x_2,\mu_F^2)\,\hat{\sigma}_{ab\to X}(x_1,x_2,s;\mu_{F},\mu_R)\;,
\end{equation}
where the hard cross section $\hat{\sigma}$ can again be calculated perturbatively. The $x_i$ are related to the kinematics of the final state.
%
In particular, at leading order it can be shown that
\be
x_{1,2}=\frac{M_X}{\sqrt{s}}e^{\pm y_X} \, ,
\ee
where $M_X$, $y_X$ are the invariant mass and rapidity of the produced system respectively.
%
The factorization and renormalization scales, $\mu_F$ and $\mu_R$, are taken to be of order the hard scale, $\mu_F,\mu_R
\sim Q$, of the process.

While the PDFs themselves cannot be calculated using perturbative methods, their dependence on the scale $\mu$ can be, and is given by the  
DGLAP evolution equations~\cite{Dokshitzer:1977sg,Gribov:1972ri,Altarelli:1977zs},
a set of integro-differential coupled equations of the form
\begin{equation}
  \label{eq:dglap}
\frac{\partial f_i(x,\mu^2)}{\partial \ln \mu^2}=\sum_{j=g,q,\bar{q}}\int_x^1 \frac{{\rm d}z}{z}P_{ij}(x/z)f_j(z,\mu^2)\;,
\end{equation}
Here, the logarithmic derivative of the PDF is determined by a convolution
of the PDFs with the DGLAP kernel $P_{ij}$ which can be computed
perturbatively in powers of $\alpha_{s}(\mu)$; $P_{ij}$ is known
to NNLO.\footnote{The DGLAP (Dokshitzer\textendash Gribov\textendash Lipatov\textendash Altarelli\textendash Parisi)
evolution can be modified by $\ln(1/x)$ terms at small-$x$, and
this is characterized by the BFLK (Balitsky-FadinKuraev-Lipatov) equations~\cite{Kuraev:1976ge,Kuraev:1977fs,Balitsky:1978ic}.
%
Additionally, at large-$x$ and small $\mu$ scale the above framework
can receive corrections from non-factorizeable higher-twist (HT) corrections.} 
%When performing the global fit to the data, we use the DGLAP equations to combine data from different $\mu$ scales when constraining the PDFs. 

Thus, when performing a global fit, the PDFs can be parameterised at one arbitrary input scale $Q_0\sim 1$ GeV, and these are connected directly via (\ref{eq:dglap}) to the higher scales $\mu$ that are probed by the data.
This parameterisation takes the generic form
\begin{equation}\label{eq:pdffunc}
f_{i}(x,Q_0)\sim x^{a}(1-x)^{b}\:C(x)\quad.
\end{equation}
The $(1-x)^b$ term, with $b_{f}>0$, ensures that the PDFs vanish in the elastic $x\to 1$ limit, as we would expect on basic physical grounds. 
%Such a form is also expected from the quark counting rules~\cite{Brodsky:1973kr}. 
The $x^a$ form, which dominates at low $x$, is predicted from general Regge theory considerations, although in modern fits the value of the power itself is left free. The interpolating function $C(x)$ determines the behaviour of the PDFs away from the $x\to 0$ and 1 limits, where it tends to a constant value. This is assumed to be a smoothly varying function of $x$, for which a variety of choices have been made in PDF fits.

There are a few constraints we
can impose on the $x$-dependence of the PDFs at this point. Since
the proton has the quantum numbers of two up quarks and one down quark,
we have the following quark number sum rules given in terms of first
moments: 
%
\begin{eqnarray}
\int_{0}^{1}dx\ \left[u(x,\mu)-\bar{u}(x,\mu)\right] & =\left\langle 1\right\rangle _{u^{-}}= & 2\\
\int_{0}^{1}dx\ \left[d(x,\mu)-\bar{d}(x,\mu)\right] & =\left\langle 1\right\rangle _{d^{-}}= & 1\\
\int_{0}^{1}dx\ \left[s(x,\mu)-\bar{s}(x,\mu)\right] & =\left\langle 1\right\rangle _{s^{-}}= & 0
\end{eqnarray}
with similar results for the heavy quarks: $\left\langle 1\right\rangle _{c^{-}}=\left\langle 1\right\rangle _{b^{-}}=\left\langle 1\right\rangle _{t^{-}}=0$. Thus for these valence distributions we must have $a>-1$ for the exponents in
Eq.~(\ref{eq:pdffunc}) or these integrals will diverge.

The fractional momentum carried
by each parton flavor is given by the second moments: 
%
\begin{eqnarray}
\int_{0}^{1}dx\ x\ \left[u(x,\mu)\right] & = & \left\langle x\right\rangle _{u}\\
\int_{0}^{1}dx\ x\ \left[u(x,\mu)+\bar{u}(x,\mu)\right] & = & \left\langle x\right\rangle _{u^{+}}\ .
\end{eqnarray}
%
Here, $\left\langle x\right\rangle _{u}$ is the fractional momentum
carried by the up-quark, $\left\langle x\right\rangle _{u^{+}}$ is
the fractional momentum carried by the up-quark and anti-up-quark.
Since the total momentum of the proton must equal the momentum of
its constituents, we have the momentum sum rule constraint: 
%
\begin{eqnarray}\label{eq:mom}
1 & = & \left\langle x\right\rangle _{g}+\left\langle x\right\rangle _{u^{+}}+\left\langle x\right\rangle _{d^{+}}+\left\langle x\right\rangle _{s^{+}}+\left\langle x\right\rangle _{c^{+}}+\left\langle x\right\rangle _{b^{+}}+\left\langle x\right\rangle _{t^{+}}+...
\end{eqnarray}
%
where the ``...'' represents any other partonic components (such
as a photon). Thus for non--valence distributions, we must have $a>-2$ to avoid a divergent contribution to (\ref{eq:mom}); typically we have $-2<a<-1$ for such distributions, and therefore for small $x$ the number of soft partons is infinite, although the momentum fraction carried by them is of course not.

\subsubsection*{Fitting PDFs from hard-scattering data}

The global PDF fitting framework is based on a combination of three basic components:
\begin{itemize}
\item A broad set of input hard-scattering cross-sections from deep-inelastic
  scattering and proton-proton collisions, providing information on the PDFs
  over a wide range of $x$ and for different flavour combinations.
  %
  While traditional PDF fits were based mostly on DIS structure function and Drell-Yan
  and inclusive jet
  cross-sections, in the recent years many other processes have demonstrated
  their usefulness for constraining PDFs, from top-quark pair production~\cite{Czakon:2016olj}
  to the $p_T$ of the $Z$ bosons~\cite{Boughezal:2017nla}
  and $D$ meson production in the forward region~\cite{Gauld:2016kpd}.
  %
  In Fig.~\ref{fig:kinplot-report} we show the representative kinematical coverage in the
    $(x,Q^2)$ of the DIS and proton-proton hard-scattering measurements that are
    used as input in a global unpolarized PDF fit, in this case NNPDF3.1.
    %
    In order to facilitate visualization, different
    datasets have been clustered together into families of
    related processes.

  \item Theoretical calculations of DIS and hadronic cross-sections
    for the highest perturbative order available.
    %
    Currently this means NNLO for the QCD corrections and NLO
    for the electroweak and photon-induced effects.
    %
    Moreover, these calculations should be provided in
    a format such as the evaluation of the hadronic
    cross-sections Eq.~(\ref{eq:LHCxsec}) is not too burdensome
    from the computational point of view.
    %
    To bypass these limitations, a number of fast interfaces have
    been developed that allow the efficient calculation
    of NLO (and NNLO) fully differential hadronic cross-sections.
    %
    The exploitation of fast interfaces such as {\tt APPLgrid},
    {\tt FastNLO} and {\tt aMCfast} is of utmost importance
    for modern PDF fits given the wealth of collider data used.

  \item A fitting methodology that determines the best-fit
    PDF parameters and their uncertainty from the minimization
    of a suitable statistical estimator, typically the $\chi^2$
    or a related estimator.
    %
    There are different alternative definitions of the $\chi^2$
    to be used in the global fit, for instance one frequently
    used definition is
    \begin{equation}
\chi^2 = \sum_{i,j}^{N_{\rm dat}} (T_i - D_i) ({\rm cov^{-1}})_{ij} (T_j - D_j),
\label{eq:chi2}
    \end{equation}
    where $N_{\rm dat}$ is the number of data points of a given experiment,
    and $T_i$ and $D_i$ are the corresponding theoretical calculations
    and the central values of the experimental data, respectively.
    %
    The theoretical predictions $D_i$ depend on the input
    PDF parametrization and thus on the fitting parameters.
    %
    Eq~(\ref{eq:chi2}) is therefore used as a  figure of merit to
    assess the agreement between theory
    and data.
    %
    The covariance matrix $({\rm cov})_{ij}$
    accounts for the various sources of experimental
    systematic uncertainties and
    also accepts several
    different definitions.
    %
    One example is the so-called
 $t_{0}$-prescription~\cite{Ball:2009qv}, 
where a fixed theory prediction $T_{i}^{(0)}$
is used to define the  contribution to the $\chi^2$
from the multiplicative systematic uncertainties, namely
\be
\label{eq:covmat_t00}
({\rm cov})_{ij}=
\delta_{ij} \sigma_{\rm stat}^2 + 
\sum_{\alpha=1}^{N_c}\sigma^{(c)}_{i,\alpha}\sigma^{(c)}_{j,\alpha}D_{i} D_{j}
+ \sum_{\beta=1}^{N_{\cal L}} \sigma_{i,\beta}^{({\cal L})}\sigma_{j,\beta}^{({\cal L})}
T^{(0)}_{i} T^{(0)}_{j}\, ,
\ee
where $\sigma_{\rm stat}$ is the uncorrelated uncertainty,
and $\sigma^{(c)}_{i,\alpha}$ ($\sigma^{(\cal L)}_{i,\beta}$)
are the additive (multiplicative) systematic uncertainties.
    
\end{itemize}
The goal of a global PDF fit is therefore to combine together
the best data and theory available in order to determine
the best-fit shape of the PDFs and the associated
uncertainties by means of the minimization
of the $\chi^2$ estimator Eq.~(\ref{eq:chi2}).

%%%%%%%%%%%%%%%%%%%%%%%%%%%%%%%%%%%%%%%%%%%%%%%%%%%%%%%%%%%%%%%%%%%%%
\begin{figure}[t]
\begin{center}
  \includegraphics[scale=0.60]{plots/kinplot-report.pdf}
  \caption{\small Representative kinematical coverage in the
    $(x,Q^2)$ of the DIS and proton-proton hard-scattering measurements that are
    used as input in a global unpolarized PDF fit, in this case NNPDF3.1.
    %
    In order to facilitate visualization, different
    datasets have been clustered together into families of
    related processes.
    \label{fig:kinplot-report} 
  }
\end{center}
\end{figure}
%%%%%%%%%%%%%%%%%%%%%%%%%%%%%%%%%%%%%%%%%%%%%%%%%%%%%%%%%%%%%%%%%%%%%

A central ingredient of the global PDF fitting framework is the determination
of the various sources of experimental, theoretical and
methodological uncertainties that affect the best-fit PDFs.
%
In this respect,
there are two main methods to determine PDF uncertainties, the {\it
  Hessian} and the {\it Monte Carlo} methods, which we briefly
review now:
\begin{itemize}
\item The {\it Hessian method} is based on the parabolic
expansion of the $\chi^2$ in the vicinity of the minimum, parametrized
by a number of orthogonal eigenvectors within some fixed tolerance.
%
The basic idea of this method is that
in the vicinity of this minimum, the $\chi^2$ can
be approximated in terms of a quadratic expansion,
\be
\label{eq:hessianexpansion}
\Delta\chi^2 \equiv \chi^2- \chi^2_{\rm min}
=\sum_{i,j=1}^{n_{\rm par}}H_{ij}\lp a_i-a_i^0\rp
\lp a_j-a_j^0\rp \, ,
\ee
where the $n_{\rm par}$ PDF fit parameters are denoted by $\{a_1,\ldots,
a_{n_{\rm par}}\}$, and the best-fit values that minimize the
$\chi^2$ are indicated by
$\{a_1^0,\ldots,
a^0_{n_{\rm par}}\}$,
and the Hessian matrix is defined as
\be
H_{ij}\equiv \frac{1}{2} \frac{\partial^2\chi^2}{\partial a_i
\partial a_j}\Bigg|_{\{\vec{a}\}=
\{\vec{a^0}\}} \, .
\ee
By diagonalizing this Hessian matrix,
it is possible
to represent
PDF uncertainties in terms of orthogonal eigenvectors,
which then can be used to estimate the PDF uncertainty
for arbitrary cross-sections, using the master formula
of Hessian PDF sets for the uncertainty of the cross-section
$\mathcal{F}$, namely
\be
\label{eq:hessianmaster2}
\sigma_{\mathcal{F}}=\frac{1}{2}\lp \sum_{i,j}^{n_{\rm par}}
\lc \mathcal{F}(S_i^+)-\mathcal{F}(S_i^-) \rc \rp^{1/2} \, ,
\ee
where $S_i^{\pm}$ correspond to the $i$-th eigenvector
associated to positive and negative variations with respect
to the best fit value.

\item The {\it Monte Carlo method} is based instead
  on constructing a representation
  of the probability distribution of the experimental data in terms
  of a large number  $N_{\rm rep}$ of {\it replicas}, and then
PDF fits are then performed separately on each of these Monte Carlo replicas.
%
The resulting ensemble of PDFs represents the probability density in the space
of parton distributions.
%
This requires generating a large number of artificial replicas
of the original data which encode the same information on
central values, variances and correlations as that provided by the experiments.
%
In particular, given an experimental measurement of a hard-scattering
cross-section denoted generically by $F_{I}^{\rm (exp)}$ with
total uncorrelated uncertainty $\sigma_{I}^{\rm (stat)}$, $N_{\rm sys}$ fully
correlated systematic uncertainties $\sigma^{\rm (corr)}_{I,c}$ and
$N_a$ ($N_r$) absolute (relative) normalization uncertainties
$\sigma^{\rm (norm)}_{I,n}$, the artificial
MC replicas are constructed using the following expression
\be
\label{eq:replicas}
F_{I}^{(\art)(k)}=S_{I,N}^{(k)} F_{I}^{\rm (\mrexp)}\lp 1+
 \sum_{c=1}^{N_{\rm sys}}r_{I,c}^{(k)}\sigma^{\rm (corr)}_{I,c}+r_{I}^{(k)}\sigma_{I}^{\rm (stat)}\rp
 \ , \quad k=1,\ldots,N_{\rep} \ ,
\ee
where $S_{I,N}^{(k)}$ is the the normalization prefactor.
%
Here the variables $r_{I,c}^{(k)},r_{I}^{(k)},r_{p,n}^{(k)}$ are
 univariate gaussian random numbers.
 %
 For each replica the random fluctuations
 associated to a given fully correlated systematic
 uncertainty will be the same
 for all data points, $r^{(k)}_{I,c}=r^{(k)}_{I',c}$.
 %
 In the Monte Carlo method, the expectation function of a generic
cross-section $ \mathcal{F} [ \{  q \}]$
is evaluated as an average over the replica sample,
\be
\label{masterave}
\la \mathcal{F} [ \{  q \}] \ra
= \frac{1}{N_{\rm rep}} \sum_{k=1}^{N_{\rm set}}
\mathcal{F} [ \{  q^{(k)} \}] \, ,
\ee
and the corresponding uncertainty is then determined as the variance of the
Monte Carlo sample,
\be
\sigma_{\mathcal{F}} =
\left( \frac{1}{N_{\rm rep}-1}
\sum_{k=1}^{N_{\rm rep}}   
\lp \mathcal{F} [ \{  q^{(k)} \}] 
-   \la \mathcal{F} [ \{  q \}] \ra\rp^2 
 \right)^{1/2}.
\label{mastersig}
\ee

\end{itemize}

In the rest of this document, when comparing the results of global PDF fits
with the results of lattice QCD calculations we will use Eqns.~(\ref{eq:hessianmaster2}) and~(\ref{mastersig})
to evaluate the PDF uncertainties of Hessian and Monte Carlo PDF sets, respectively.

\subsubsection{State-of-the-art global PDF fits}

Various collaborations provide regular updates of their global unpolarized
PDF fits.
%
The latest fits from the three major global fitting collaborations, CT14~\cite{Dulat:2015mca}, MMHT14~\cite{Harland-Lang:2014zoa} and NNPDF3.1 are performed up to NNLO in the strong coupling, and include data from the HERA $e^{\pm} p$ collider, fixed (nuclear and proton) target experiments, the Tevatron $p\overline{p}$ collider and the LHC. 
%
The ABMP16~\cite{Alekhin:2017kpj} set fits to a similar global data set, but differs in its treatment of errors and heavy flavours (see below). The HERAPDF2.0~\cite{Abramowicz:2015mha} set fits to the final combined HERA Run I + II data set only, with the aim of determining the PDFs from a completely consistent DIS data sample; in $x$ regions which are less constrained by HERA data, the uncertainties can be quite large. The CJ15~\cite{Accardi:2016qay} NLO set focuses on constraining the PDFs at higher $x$ by lowering $Q^2$ and $W^2$ cuts in DIS. This greatly increases the available data, but requires additional modelling of power--like $\sim 1/Q^2$ corrections.

As described above, to perform a fit, some form for the interpolating function $C(x)$ in (\ref{eq:pdffunc}) must be assumed. The simplest ansatz, which has been very widely used, is to take a basic polynomial form in $x$ (or $\sqrt{x}$), such as
\begin{equation}\label{eq:lpower}
C(x)=1+c\sqrt{x}+d x+...\;.
\end{equation}
Forms of this type are for example taken by CJ, HERAPDF and earlier MMHT and CT sets. More recently, the CT and MMHT collaborations instead expand in terms of a basis of  Bernstein and Chebyshev polynomials, respectively.
%
While formally equivalent to the simply polynomial expansion (\ref{eq:lpower}), these are much more convenient for fitting as the number of free parameters $n$ is increased. In the latest sets, there are $O(20-40)$ free parameters in total.
%
An alternative approach is taken by the NNPDF collaboration. Here, the interpolating function is modeled with a multi--layer feed forward neural network. In practice, this allows for a greatly increased number of free parameters, typically $\sim$ an order of magnitude higher than other sets. The form of (\ref{eq:pdffunc}) is still assumed, but these are pre--processing factors that speed up the minimisation procedure but which do not in principle have to be explicitly included. 



\subsubsection{Polarized PDFs}
\label{sec:polPDFs}

Like their unpolarised (spin-averaged) counterparts described in the previous 
subsection, helicity PDFs, along with estimates of their uncertainties, can be 
determined from a comprehensive global analysis of the available spin-dependent 
data. 
%
Here we delineate the aspects of the framework specific to the polarised
case, and give a brief review of current global polarised PDF fits.

\paragraph{General framework.}
%
%
The dependence on the momentum fraction $x$, fixed by non-perturbative QCD 
dynamics, should satisfy some theoretical constraints.
%
First, PDFs must lead to positive cross sections.
At leading order (LO), this implies that polarised 
PDFs are bounded by their unpolarised counterparts\footnote{Beyond LO, more 
complicate relations hold~\cite{Altarelli:1998gn}; however they have little
effect on PDFs.}, $|\Delta f(x,\mu^2)|\leq f(x,\mu^2)$.
%
Second, PDFs must be integrable: this corresponds to the assumption 
that the nucleon matrix element of the axial current for each flavour is finite.
%
Third, SU(2) and SU(3) flavour symmetry, if assumed to be exact, imply that 
the first moments of the nonsinglet $\mathcal{C}$-even PDF combinations,
$\Delta T_3=\Delta u^+ -\Delta d^+$ and 
$\Delta T_8 = \Delta u^+ +\Delta d^+ -2\Delta s^+$ 
(where $\Delta q^+=\Delta q+\Delta\bar{q}$, $q=u,d,s$), are respectively
related to the baryon octet $\beta$-decay constants, whose 
measured values values are~\cite{Olive:2016xmw}
\begin{align}
 a_3
 & =
 \int_0^1 dx \Delta T_3 (x,\mu^2)
 = \langle 1\rangle_{\Delta u^+} - \langle 1\rangle_{\Delta d^+}  = 1.2701 \pm 0.0025\\
 a_8
 & =
 \int_0^1 dx \Delta T_8 (x,\mu^2)
 = \langle 1 \rangle_{\Delta u^+} + \langle 1 \rangle_{\Delta d^+} -2\,\langle 1 \rangle_{\Delta s^+} 
 =0.585  \pm 0.025
 \,\mbox{.}
\label{eq:decayconst}
\end{align}
%
Fairly significant violations of SU(3) symmetry are advocated
in the literature (see {\it e.g.} Ref.~\cite{Cabibbo:2003cu} for a review). 
%
In this case, an uncertainty on the octet axial charge, larger by up to $30\%$ 
than its experimental value in Eq.~\eqref{eq:decayconst}, 
is found~\cite{FloresMendieta:1998ii}. 

The bulk of the experimental information on polarised PDFs comes from 
neutral-current (photon exchange) inclusive and semi-inclusive deep-inelastic scattering 
(DIS and SIDIS) with charged lepton beams and nuclear targets. 
%
As the photon scattering does not distinguish quarks and antiquarks, inclusive DIS 
data constrain only the total quark combinations $\Delta q^+$, 
while SIDIS data with identified pions or kaons in the final state 
constrain individual quark and antiquark flavours. 
%
In principle, both DIS and SIDIS are also sensitive to the gluon 
distribution $\Delta g$, as it directly enters the factorised expressions of
the corresponding structure functions beyond LO, and indirectly via DGLAP 
evolution.
%
In practice the constraining power of DIS and SIDIS data on $\Delta g$ is 
rather weak, because of the limited $Q^2$ range covered by the data. 

Note that, in the case of SIDIS, a reliable knowledge of fragmentation 
functions (FFs) is required in the factorised expressions of the 
corresponding observables. 
%
Since FFs are non-perturbative objects on the same footing as PDFs, they are 
an additional source of uncertainty in the PDF determination, and can  
even become a bias.
%
For this reason, a significant experimental and theoretical effort has been
put in improving the independent determination of 
FFs~\cite{deFlorian:2014xna,deFlorian:2017lwf,
Hirai:2016loo,Sato:2016tuz,Nocera:2017qgb,Bertone:2017xsf,Ethier:2017zbq}.

Besides DIS and SIDIS fixed-target data, a significant amount of data from
longitudinally polarised proton-proton ($pp$) collisions at the Relativistic 
Heavy Ion Collider (RHIC) have become available recently (see {\it e.g.} 
Ref.~\cite{Aschenauer:2015eha} for an overview), although in a limited range 
of momentum fractions, $0.05\lesssim x \lesssim 0.4$.
%
On the one hand, longitudinal (parity-violating) single-spin and 
(parity-conserving) double-spin asymmetries for $W^\pm$ boson production are 
sensitive to the flavour decomposition of polarised quark and antiquark 
distributions, because of the chiral nature of the weak 
interactions~\cite{Bourrely:1993dd}. 
%
On the other hand, double-spin asymmetries for jet, di-jet and $\pi^0$ 
production are directly sensitive to the gluon polarization in 
the proton, because of the dominance of gluon-gluon and quark-gluon initiated 
subprocesses in the kinematic range accessed by RHIC~\cite{Bourrely:1990pz}.

The kinematic coverage of the data which can be used to constrain polarised 
PDFs is displayed in Fig.~\ref{fig:kinEIC}.
%
A comparison with Fig.~\ref{fig:kinplot-report} makes it apparent that the
quantity of data points, their kinematic coverage and the variety of 
available hard-scattering processes are presently much more limited in the polarised case
than in the unpolarised case.
%
Therefore polarised PDFs can currently be determined with much less 
precision than their unpolarised counterparts and also only over an $x$-range limited
to $x\gtrsim 0.005$.
%
The kinematic coverage is expected to be significantly extended in the future,
with DIS and SIDIS data from JLab-12~\cite{Dudek:2012vr} and a polarised 
high-energy Electron-Ion Collider (EIC)~\cite{Accardi:2012qut}.
%
Such an extended kinematic coverage is also displayed in Fig.\ref{fig:kinEIC}.
%
The eRHIC realisation of an EIC~\cite{Aschenauer:2014cki} has been considered.

%-------------------------------------------------------------------------------
\begin{figure}[!t]
\centering
\includegraphics[width=0.9\textwidth]{plots/kinEIC}\\
\caption{\small Representative kinematic coverage, in the $(x,Q^2)$ plane,
of the (neutral current) DIS, SIDIS and $pp$ hard-scattering measurements 
that are used as input in a global polarised PDF fit.
%
The extended kinematic coverage achieved by 
JLab-12~\cite{Dudek:2012vr} and by the eRHIC~\cite{Aschenauer:2014cki} 
realisation of an EIC~\cite{Accardi:2012qut}
(including projected charged-current (CC) DIS data) is also shown.
%
Figure taken from Ref.~\cite{Aschenauer:2014cki}.}
\label{fig:kinEIC}
\end{figure}
%-------------------------------------------------------------------------------

A representative illustration of polarised PDFs obtained from a global
QCD analysis, namely NNDPFpol1.1~\cite{Nocera:2014gqa}, is provided in Fig.~\ref{fig:qPDFpol}.
%
The format is the same as for the unpolarised case, Fig.~\ref{fig:nnlopdfs},
in order to ease any comparison between the two.
%
In particular, note the suppression of all polarised PDFs at small values of 
$x$, including polarised sea quark PDFs, with respect to their unpolarised 
counterparts.

%-------------------------------------------------------------------------------
\begin{figure}[!t]
\centering
\includegraphics[scale=0.4]{plots/NNPDFpol}\\
\caption{\small Same as Fig.~\ref{fig:nnlopdfs}, 
but for the polarised NNPDFpol1.1 NLO PDFs~\cite{Nocera:2014gqa}.}
\label{fig:qPDFpol}
\end{figure}
%-------------------------------------------------------------------------------

\paragraph{State-of-the-art global polarised PDF fits.}

Several modern determinations of polarised PDFs of the proton (up to 
NLO\footnote{A NNLO QCD analysis of polarised PDFs based on inclusive DIS
data only was performed in Refs.~\cite{Shahri:2016uzl,Khanpour:2017cha}.
Inclusive DIS is the only polarised process for which coefficient functions
are known up to NNLO (all others are known only up to NLO).} 
and mostly in the $\overline{\rm MS}$ factorization scheme) are available in 
the literature~\cite{Nocera:2014gqa,Nocera:2016xhb,deFlorian:2014yva,deFlorian:2008mr,deFlorian:2009vb,Sato:2016tuz,Leader:2010rb,Blumlein:2010rn,Bourrely:2014uha,Hirai:2008aj}. 
%
A key goal of these is to also unveil the size (and uncertainty) of
$\Delta\Sigma$ and  $\Delta G$ in Eq.~\eqref{eq:moments}. 
%
The various determinations differ among each other in the data sets included 
in the analysis, in some details of the QCD analysis (like the treatment of 
higher-twist corrections) and in the procedure used to determine PDFs from the 
data (for details, see {\it e.g.} Chap.~3 in Refs.~\cite{Nocera:2014vla} 
and~\cite{Nocera:2016xhb}). 
%
The NNDPF procedure and the conventional/standard one adopted by DSSV have 
already been outlined in Sec.~\ref{sec:unpPDFs}. 
%
We note that DSSV has developed a method based on Mellin moments of the PDFs 
in order to efficiently incorporate NLO computations
of $pp$ cross sections in the fitting procedure. 
%
The JAM collaboration has implemented for their analysis a new approach called 
iterative Monte Carlo procedure~\cite{Sato:2016tuz}. 

Motivated by the interest in assessing the impact of RHIC $pp$ 
data, two new global analyses of polarised PDFs have been carried out in
2014, DSSV14~\cite{deFlorian:2014yva} and NNPDFpol1.1~\cite{Nocera:2014gqa}.
%
They upgrade the corresponding previous analyses, 
DSSV08~\cite{deFlorian:2008mr,deFlorian:2009vb} and 
NNPDFpol1.0~\cite{Ball:2013lla}, with data respectively on double-spin 
asymmetries for inclusive jet production~\cite{Adamczyk:2014ozi} 
and $\pi^0$ production~\cite{Adare:2014hsq}\footnote{Preliminary RHIC results 
included in Ref.~\cite{deFlorian:2008mr} were replaced in
Ref.~\cite{deFlorian:2014yva} with final results.}, 
and on double-spin asymmetries for high-$p_T$ inclusive jet 
production~\cite{Adamczyk:2014ozi,Adamczyk:2012qj,Adare:2010cc} and single-spin
asymmetries for $W^\pm$ production~\cite{Adamczyk:2014xyw}.
%
The new data have been included in NNPDFpol1.1 
by means of Bayesian reweighting~\cite{Ball:2010gb},
and in DSSV14 by means of a full refit.  

Overall, both the DSSV14 and NNPDFpol1.1 PDF determinations are 
state-of-the-art in the inclusion of the available experimental information. 
%
The data sets in the two analyses differ between each other only in
fixed-target SIDIS and RHIC $\pi^0$ production measurements, included in 
DSSV14, but not in NNPDFpol1.1. 
%
The information brought in by these data is complementary to that provided by 
RHIC $W^\pm$ production and inclusive jet production data respectively,
although fraught with larger theoretical uncertainties related to fragmentation.

%------------------------------------------------------------------------------
\begin{figure}[!t]
%\centering

\hspace*{6mm}
\includegraphics[scale=0.325]{plots/gluoncomp}

\vspace*{-15.1cm}
\hspace*{8cm}
\includegraphics[scale=0.555]{plots/correlation_getot.pdf}

\caption{(Left) The polarised gluon momentum distribution  $x\Delta g$ from the 
DSSV14 (with $90\%$ C.L. uncertainty band)
and NNPDFpol1.1 PDF sets at $Q^2=10$ GeV$^2$. The NNPDF3.1 positivity
bound is also shown.
(Right) $90\%$ C.L.\ areas in the plane spanned by the truncated moments of
$\Delta g$ computed for $0.05\leq x\leq 1$ and $0.001\leq x\leq 0.05$ at $Q^2=10\,\mathrm{GeV}^2$~\cite{deFlorian:2014yva}.}
\label{fig:RHICpdfs}
\end{figure}
%------------------------------------------------------------------------------

The effect of RHIC data on the polarised PDFs of the proton is twofold:
\begin{itemize}

\item The 2009 STAR and PHENIX data sets on jet and $\pi^0$ 
production~\cite{Adamczyk:2014ozi,Adare:2014hsq}, included in DSSV14
and NNPDFpol1.1, provide the first evidence
of a sizable positive gluon polarization in the proton. 
%
A comparison of the gluon PDF in the two PDF sets is displayed in 
Fig.~\ref{fig:RHICpdfs} (left panel). 
%
Comparable results, both central values and uncertainties, are found in the 
$x$ region covered by RHIC data. 
%
The agreement between the two analyses is optimal in the
range $0.08\leq x \leq 0.2$, where the dominant experimental information comes
from jet data; a slightly smaller central value is found in the DSSV14 
analysis, in comparison to the NNPDFpol1.1, in the range 
$0.05\leq x \leq 0.08$, where the dominant experimental information comes from 
$\pi^0$ production data. 
%
Indeed, these are included in DSSV14 but are not
in NNPDFpol1.1. 
%
Nevertheless, best fits lie well within each other error
bands, though NNPDFpol1.1 uncertainties tend to be larger than DSSV14
uncertainties outside the region covered by RHIC data.
%
Very well consistent values of the integral of $\Delta g$, 
Eq.~\eqref{eq:moments}, truncated over the interval $0.05\leq x \leq 1$, are 
found: at $Q^2=10$ GeV$^2$, this is $0.20^{+0.06}_{-0.07}$ for 
DSSV14~\cite{deFlorian:2014yva}, and $0.23\pm 0.06$ for 
NNPDFpol1.1~\cite{Nocera:2014gqa}. The right plot in Fig.~\ref{fig:RHICpdfs} 
shows the corresponding DSSV14 result as an example; the impact of the RHIC
data is clearly visible. 

\item The 2012 STAR data sets on $W$ production~\cite{Adamczyk:2014xyw}, 
included in NNPDFpol1.1, provide evidence of a positive 
$\Delta\bar{u}$ distribution 
and a negative $\Delta\bar{d}$ distribution, with 
$|\Delta\bar{d}|>|\Delta\bar{u}|$~\cite{Nocera:2014gqa},
as shown in Fig.~\ref{fig:RHICpdfs1}.
% 
The size of the flavour symmetry breaking for polarised sea quarks is 
quantified by the asymmetry $\Delta\bar{u}-\Delta\bar{d}$, which,
in the NNPDFpol1.1 analysis, turned out to be roughly as large as its 
unpolarised counterpart (in absolute value), 
though much more uncertain~\cite{Nocera:2014rea}. 
%
Even within this uncertainty, polarised and unpolarised light sea quark 
asymmetries show opposite signs,
with the polarised ones being clearly positive.  This trend is also found
from analysis of the polarized SIDIS data, as revealed by the DSSV analyses. 
%
This result may discriminate among models of nucleon structure; 
see Fig.~\ref{fig:RHICpdfs1}: 
specifically, some meson-cloud (MC) models are disfavored, while a more 
accurate experimental information is needed to establish whether 
chiral quark-soliton (CQS), Pauli-blocking (PB) or statistical (ST)
models are preferred (all these models are described in 
Ref.~\cite{Chang:2014jba}).

\end{itemize}

%------------------------------------------------------------------------------
\begin{figure}[!h]
\centering
\includegraphics[scale=0.35]{plots/asysea_2}\\
\caption{The polarised light sea quark asymmetry 
$x(\Delta\bar{u}-\Delta\bar{d})$ from the NNPDFpol1.1 and 
DSSV08 PDF sets at $Q^2=10$ GeV$^2$, compared to expectations from 
various models of nucleon structure~\cite{Chang:2014jba}.}
\label{fig:RHICpdfs1}
\end{figure}
%------------------------------------------------------------------------------
\paragraph{Open issues.}

Despite the achievements described above, the polarised PDFs presently cannot 
be determined in a global QCD analysis with the same accuracy as their 
unpolarised counterparts.
%
The experimental data are so far confined to a relatively narrow range of 
$x$ and $Q^2$.
%
As a consequence, the sizes of the contribution of quarks, antiquarks and 
gluons to the nucleon spin, as quantified by their first moments, 
Eq.~\eqref{eq:moments}, are still affected by large uncertainties. 
%
These come predominantly from the extrapolation into the small-$x$ region 
($x\lesssim 10^{-3}$). 
%
Here potential modifications in the PDF shape induced by small-$x$ evolution 
could arise, which presently cannot be tested.
%
Significant uncertainties also affect the PDFs in the large-$x$ 
{\it valence} region ($x\gtrsim 0.7$). 
%
This regime is less relevant for determination of the first moments, but is 
important for comparisons to nonperturbative models of nucleon structure, 
especially in terms of ratios of light-quark polarised to unpolarised PDFs 
(for a comparison between large-$x$ PDFs 
and model predictions, see Ref.~\cite{Nocera:2014uea}).
%
Finally, the small lever arm of the data in $Q^2$ is a serious limiting factor 
in the determination of $\Delta g$ via evolution and the assessment of possible 
higher-twist contributions. 

The determination of the total polarised strange distribution $\Delta s^+$ is 
also particularly delicate.
%
Inclusive DIS data, together with nonsinglet axial couplings, 
Eq.~\eqref{eq:decayconst}, and kaon SIDIS data provide the sole available 
constraint on $\Delta s^+$.
%
A sizeable negative $\Delta s^+$ is found 
consistently in all analyses based on inclusive DIS data only, as a result 
of the constraint from hyperon decays that is usually adopted. 
%
However, the shape of $\Delta s^+$ may change significantly in analyses based
also on SIDIS data. Typically SIDIS data lead to a trend for $\Delta s^+$ to be
small or even slightly positive in the medium $x$-range, although this depends 
also on the set of kaon FFs used to compute
the corresponding observables~\cite{Leader:2011tm}.  
%
The recent study in Ref.~\cite{Ethier:2017zbq} sheds some light on this issue
by performing a simultaneous determination of polarised PDFs and unpolarised 
fragmentation functions using DIS, SIDIS and single-inclusive annihilation data.
%
In order to avoid biasing the determination of $\Delta s^+$ by 
assumptions on SU(3) symmetry, the octet axial charge in 
Eq.~(\ref{eq:decayconst}) has been allowed to be determined by the data alone.
%
As a consequence, a slightly positive $\Delta s^+$ distribution, but
compatible with the negative result found from inclusive DIS within its 
large uncertainties, has been obtained.
% 
An octet axial charge about $20\%$ smaller than its quoted experimental value, 
Eq.~(\ref{eq:decayconst}) appears to be preferred by the data.
%
However, we stress that the determination of $\Delta s^+$ from SIDIS data 
also relies on good knowledge of the {\it un}polarized strange distribution. 
%
Furthermore, unpolarized SIDIS data themselves set constraints on 
fragmentation functions and ultimately would need to be included as well
in order to obtain a reliable picture. 
%
In any case, further higher precision kaon SIDIS data will be needed in order 
to reduce the uncertainty on $\Delta s^+$ and further test the degree of 
SU(3) breaking. 

Ongoing and future experimental campaigns at current facilities are
expected to provide additional experimental information
useful to clarify some of the issues outlined above (for an 
assessment of the impact of very recent/forthcoming data, see {\it e.g.}
Refs.~\cite{Aschenauer:2015eha,Aschenauer:2015ata,Nocera:2015vva,
Nocera:2017wep}).
%
However, a future high-energy, polarised EIC~\cite{Accardi:2012qut} will 
likely be the only facility to be able to address all the above issues 
with the highest precision. 
% 
The extension of the kinematic reach down to $x\sim 10^{-4}$ and up to
$Q^2=10^4$ GeV$^2$ will allow for an accurate determination of $\Delta g$
via evolution in DIS/SIDIS, of $\Delta\bar{u}$ and 
$\Delta\bar{d}$ via inclusive DIS at high $Q^2$ mediated by electroweak bosons,
and of $\Delta s$ via kaon-tagged SIDIS. 
%
The potential impact of the longitudinally polarised program at an EIC
has been quantitatively assessed in several dedicated 
studies~\cite{Aschenauer:2012ve,Ball:2013tyh,Aschenauer:2013iia,
Aschenauer:2015ata}.



%%%%%%%%%%%%%%%%%%%%%%%%%%%%%%%%%%%%%%%%%%%%%%%%%%%%%%%%%%%%%%%%%%%%%%%%%%%%%%%%
\section{Benchmarking PDF moments}
\label{sec:benchmarking}
%%%%%%%%%%%%%%%%%%%%%%%%%%%%%%%%%%%%%%%%%%%%%%%%%%%%%%%%%%%%%%%%%%%%%%%%%%%%%%%%

In this section we provide a quantitative comparison between 
current lattice-QCD and global-fit results of the lowest
moments of unpolarized and polarized PDFs.
%
To this purpose, we identify benchmark quantities
and define the criteria to appraise the determinations
available in the literature.
%
For each benchmark quantity, we specify a prescription to 
select and combine lattice-QCD calculations and global-fit determinations.
%
We present our benchmark numbers from each side and compare them.

%%%%%%%%%%%%%%%%%%%%%%%%%%%%%%%%%%%%%%%%%%%%%%%%%%%%%%%%%%%%%%%%%%%%%%%%%%%%%%%%
\subsection{Benchmark criteria}
\label{subsec:BC}

We start by describing our benchmark criteria, which include the definition
of the benchmark quantities and the determination of their reference values,
based on a careful assessment of the lattice-QCD and global-fit results 
available in the literature.

\subsubsection{Benchmark quantities}
\label{subsubsec:BQ}

We identify our benchmark quantities with the following moments of unpolarized 
and polarized PDFs, or of PDF quark flavor combinations.
\begin{itemize}
  \item
$\langle x\rangle_{u^+-d^+}$, $\langle x \rangle_{u^+}$, $\langle x \rangle_{d^+}$, 
$\langle x \rangle_{s^+}$ and $\langle x \rangle_{g}$ in the unpolarized case; 
\item $g_A\equiv\langle 1 \rangle_{\Delta u^+ - \Delta d ^+}$, 
$\langle 1 \rangle_{\Delta u^+}$, $\langle 1 \rangle_{\Delta d^+}$,  
$\langle 1 \rangle_{\Delta s^+}$ and $\langle x \rangle_{\Delta u^- - \Delta d^-}$ 
  in the polarized case.
  \end{itemize}
%
We adopt the conventional notation described in Appendix~\ref{app:notation}.
%
We focus on the above quantities because current lattice 
calculations of higher moments and moments of other PDF 
combinations are not sufficiently controlled to allow for a meaningful 
comparison between lattice-QCD and global-fit results. 

\subsubsection{Appraising lattice-QCD calculations}
\label{subsubsec:BClQCD}

To accurately assess current lattice-QCD calculations
available in the literature, we follow a procedure inspired by the review of 
low-energy mesons undertaken by the Flavor Lattice Averaging Group 
(FLAG)~\cite{Aoki:2016frl}. 
%
For each lattice calculation, we characterize each source of 
uncertainty outlined in Sect.~\ref{Sec:IntroLQCD}. 
%
We use a rating system inspired by FLAG, awarding a blue star (\bstar) for 
sources of uncertainty that are well controlled or very conservatively 
estimated, a blue circle (\bcirc) for sources of uncertainty that have been 
controlled or estimated to some extent, and a red square (\rsquare) for 
uncertainties that have not met our criteria or for which no estimate is given.
%
Specifically, the rating system works as follows.

\begin{itemize}
\item {\bfseries Discretization effects and the continuum limit.}
We assume that the lattice actions are ${\cal O}(a)$-improved, {\it i.e.}, 
that the discretization errors vanish quadratically with the lattice spacing. 
%
For unimproved actions, an additional lattice spacing is required. 
%
These criteria must be satisfied in each case for at 
least one pion mass below 300~MeV.
%
\begin{itemize}
%
\item[\bstar] At least three lattice spacings with at least two lattice 
spacings below 0.1~fm and a range of lattice spacings that satisfies 
$[a_{\mathrm{max}}/a_{\mathrm{min}}]^2 \geq 2$.
%
\item[\bcirc] At least two lattice spacings with at least one point below 
0.1~fm and a range of lattice spacings that satisfy
$[a_{\mathrm{max}}/a_{\mathrm{min}}]^2 \geq 1.4$.
%
\end{itemize}
%
To receive a \bstar~or \bcirc~either a continuum extrapolation must be 
performed, or the results must demonstrate no significant discretization 
effects over the appropriate range of lattice spacings.

\item {\bfseries Unphysical pion masses.}
We define a physical pion mass ensemble to be one with $M_\pi=135\pm 10$~MeV
for the following criteria.
%
\begin{itemize}
\item[\bstar] One ensemble with a physical pion mass \emph{or} a chiral 
extrapolation with three or more pion masses, with at least two pion masses 
below 250~MeV and at least one below 200~MeV.
%
\item[\bcirc] A chiral extrapolation with three or more pion masses, two of 
which are below 300~MeV.
%
\end{itemize}

\item {\bfseries Finite-volume effects.}
%
For calculations that use a mixed-action approach, {\it i.e.},
with different lattice actions for the valence and sea quarks, 
we apply the following criteria to $M_\pi L$ for the valence quarks
($M_{\pi,\mathrm{min}}$ is the lightest pion mass employed in the calculation).
%
\begin{itemize}
%
\item[\bstar] Ensembles with $M_{\pi,\mathrm{min}}L\geq 4$, \emph{or} at least 
three volumes with spatial extent $L>2.5$~fm.
\item[\bcirc] Ensembles with $M_{\pi,\mathrm{min}}L \geq 3.4$, \emph{or} at least 
two volumes with spatial extent $L>2.5$~fm.
\end{itemize}

\item {\bfseries Excited-state contamination.}
%
\begin{itemize}
%
\item[\bstar] At least three source-sink separations or a variational method 
to optimize the operator derived from at least a $3\times 3$ correlator matrix, 
at every pion mass and lattice spacing.
% 
\item[\bcirc] Two source-sink separations at every pion mass and lattice 
spacing, or three or more source-sink separations at one pion mass below 
300~MeV.
%
For the variational method, an optimized operator derived from a $2\times 2$ 
correlator matrix at every pion mass and lattice spacing, or a $3\times 3$ 
correlator matrix for one pion mass below 300~MeV.
%
\end{itemize}

\item {\bfseries Renormalization.}
\begin{itemize}
%
\item[\bstar] Nonperturbative renormalization.
%
\item[\bcirc] Perturbative renormalization.
%
\end{itemize}
%
For $g_A$ we also award a \bstar~for calculations that use fermion actions 
for which $Z_A/Z_V=1$ or employ combinations of quantities for which the 
renormalization is unity by construction.

\item {\bfseries Lattice-spacing determination.}
For lattice-QCD calculations of nucleons, the lattice-spacing determination is 
generally sufficiently precise that it is a very small or negligible source
of systematic uncertainty. 
%
Therefore we do not include an assessment of the lattice-spacing
determination in our criteria.

\end{itemize}

Another important parameter in lattice-QCD calculations is the number of sea 
quark flavors, $N_f$. 
%
Following the approach used by FLAG, we prefer to avoid combining calculations 
with differing $N_f$; for more discussion of this issue, see the FLAG 
review~\cite{Aoki:2016frl}.

We now summarize the current status of lattice-QCD calculations of
our benchmark moments of unpolarized and polarized PDFs respectively.
%
Following FLAG, we consider only those results that are published in 
peer-reviewed journals or that have appeared as preprints. 
%
Where recent results are a clear update of previously published work, we do 
not include earlier results.
%
A bibliographical compilation of the results available in the literature 
is given for completeness in Appendix~\ref{sec:LQCDtables},
Tables~\ref{tab:latticebibfirst}-\ref{tab:latticebiblast}.
%
We characterize the results according to the criteria 
described above, and provide a prescription to combine those results that 
satisfy the criteria into a single benchmark value.

Our criteria and the corresponding ratings are chosen to provide not only as 
fair an assessment of the relative merits of various calculations as possible, 
but also a solid reference for future studies.
%
Where lattice-QCD results do not meet these standards, we hope that the lattice 
community will work towards improved calculations and greater precision.

\paragraph{Unpolarized parton distributions.}
We summarize the current status of lattice-QCD calculations of the benchmark 
moments of unpolarized PDFs listed in Sect.~\ref{subsubsec:BQ} in 
Table~\ref{tab:unpolLQCDstatus1}. 
%
We indicate: the computed moment in the first column; the collaboration who
performed the computation in the second column; the corresponding reference
in the third column; the number of sea quark flavors, $N_f$, in the fourth 
column.
%
We show whether the calculation has been published~(P) 
or has appeared as a preprint~(PreP) in the fifth column.
%
In the following five columns, we assess each source of systematic uncertainty
according to the criteria listed above. 
%
In the last column, we report the computed value at $\mu^2=4\mbox{ GeV}^2$
in the $\overline{{\rm MS}}$ scheme.
%
We refer the reader to the corresponding references for details on the 
meaning of the errors reported in parentheses.
%
We do not list results that have not been extrapolated to the physical pion 
mass, nor do we include quenched results in Table~\ref{tab:unpolLQCDstatus1}. 
%
For completeness, we report these results in  Appendix~\ref{sec:LQCDtables},
Table~\ref{tab:unpolLQCDstatus1B}.

%-------------------------------------------------------------------------------
\begin{table}[!t] 
\renewcommand{\arraystretch}{1.2} 
\centering 
\begin{threeparttable}
\begin{tabular}{llcllccccccl}
\toprule
Mom. & Collab. & Ref. & $N_f$ & Status & 
Disc &
QM &
FV &
Ren &
ES &
%
& Value\\
\midrule
$\langle x\rangle_{u^+-d^+}$ 
& LHPC\,14  
  & \cite{Green:2012ud} 
  & 2+1 
  & P  
  & \rsquare 
  & \bstar   
  & \bstar   
  & \bstar 
  & \bstar 
  & 
  & 0.140(21)\\
& ETMC 17  
  & \cite{Alexandrou:2017oeh} 
  & 2   
  & P
  & \rsquare 
  & \bstar   
  & \rsquare 
  & \bstar 
  & \bstar 
  & $^*$ 
  & 0.194(9)(11)\\
& RQCD 14  
  & \cite{Bali:2014gha} 
  & 2   
  & P  
  & \rsquare 
  & \rsquare 
  & \bcirc   
  & \bstar 
  & \bstar 
  & $^{**}$ 
  & 0.217(9)\\
\midrule
$\langle x\rangle_{u^+}$
&  ETMC 17  
  & \cite{Alexandrou:2017oeh} 
  & 2 
  & P
  & \rsquare 
  & \bstar   
  & \rsquare 
  & \bstar 
  & \bstar 
  & $^{*\triangleright}$ 
  & $0.453(57)(48)$\\
\midrule
$\langle x\rangle_{d^+}$
& ETMC 17  
  & \cite{Alexandrou:2017oeh} 
  & 2 
  & P
  & \rsquare 
  & \bstar   
  & \rsquare 
  & \bstar 
  & \bstar 
  & $^{*\triangleright}$ 
  & $0.259(57)(47)$\\
\midrule
$\langle x\rangle_{s^+}$
& ETMC 17  
  & \cite{Alexandrou:2017oeh} 
  & 2 
  & P
  & \rsquare  
  & \bstar   
  & \rsquare 
  & \bstar 
  & \bstar 
  & $^{*\triangleright}$ & $0.092(41)(0)$\\
\midrule
$\langle x\rangle_{g}$
& ETMC 17  
  & \cite{Alexandrou:2017oeh} 
  & 2 
  & P 
  & \rsquare 
  & \bstar   
  & \rsquare 
  & \bcirc 
  & \bstar 
  & $^*$ 
  & 0.267(22)(27)\\
\bottomrule
\end{tabular}
\begin{tablenotes}
\footnotesize
\item[$\ \,*$] Study employing a single physical pion mass ensemble.
\item[$**$] Study employing a single ensemble with $m_\pi=150$~MeV.
\item[$\ \,\triangleright$] Nonsinglet renormalization is applied.
%The mixing with $\langle x\rangle_{g}$ is computed.
\end{tablenotes}
\end{threeparttable}
\caption{\small Status of current lattice-QCD calculations of the benchmark 
first moments of unpolarized PDFs listed in Sect.~\ref{subsubsec:BQ}.
%
A detailed description of each entry, including the symbols used to 
characterize the various sources of systematics, is provided in the text.
%
Values are shown at $\mu^2=4\mbox{ GeV}^2$.
%
We refer the reader to the corresponding references for details on the 
errors reported in parentheses.
%
To denote the various sources of systematic uncertainty, 
we use the abbreviations Disc (discretization),
QM (quark mass), FV (finite volume),
Ren (renormalization) and ES (excited states).
%
}
\label{tab:unpolLQCDstatus1}
\end{table}
%-------------------------------------------------------------------------------

As is apparent from Table~\ref{tab:unpolLQCDstatus1}, there are no lattice 
calculations of the considered first moments for which all systematics 
have been fully explored and controlled.  
%
In the case of $\langle x\rangle_{u^+-d^+}$ three different results are available 
in the literature.
%
We present the lattice-QCD benchmark value for this quantity 
as a best-estimate band.
% 
This band extends from the mean of the smallest result minus its error 
to the mean of the largest result plus its error, and includes all results 
listed in Table~\ref{tab:unpolLQCDstatus1} with two or more sea 
quark flavors.
%
Current studies are not sufficiently precise to distinguish between 
results with different numbers of sea quark flavors.
%
In the case of $\langle x \rangle_{u^+}$, $\langle x \rangle_{d^+}$, 
$\langle x \rangle_{s^+}$ and $\langle x \rangle_g$, there is only one
lattice result available in the literature:
for these quantities, our lattice-QCD benchmark value is the single result; 
however, it should be noted that these results may underestimate some sources 
of uncertainty. 

The lattice-QCD benchmark numbers for $\langle x\rangle_{u^+-d^+}$,
$\langle x \rangle_{u^+}$, $\langle x \rangle_{d^+}$, 
$\langle x \rangle_{s^+}$ and $\langle x \rangle_g$ will be further
commented below, where they will be collected together with their 
global-fit counterparts in Table~\ref{tab:BMunp}.

Finally, we summarize the current status of lattice-QCD calculations of the 
second moment of the unpolarized valence-quark PDFs, 
$\langle x^2 \rangle_{u^-}$, $\langle x^2 \rangle_{d^-}$ and 
$\langle x^2\rangle_{u^--d^-}$ in Appendix~\ref{sec:LQCDtables},
Table~\ref{tab:unpolLQCDstatus2B}.
% 
The study of these moments is not sufficiently mature to provide benchmark 
values and we only list the results for completeness.

\paragraph{Polarized parton distributions.}
The zeroth moment of the isotriplet polarized PDF combination is related to the 
axial charge of the nucleon, $g_A\equiv \langle 1\rangle_{\Delta u^+-\Delta d^+}$.
%
This quantity is of central importance to nucleon physics and has long been 
considered an important benchmark for lattice calculations. 
%
Historically, lattice-QCD calculations of the axial charge have underestimated 
the experimental value $g_A^{\mathrm{exp}} = 1.2723(23)$~\cite{Olive:2016xmw}
(see also Eq.~\eqref{eq:a3}), 
which is most precisely determined from neutron weak decays. 
%
Thus, the axial charge has been the single most-studied moment in lattice QCD.
%
We summarize the current status of these calculations in 
Table~\ref{tab:gAstatus} using the same format as in 
Table~\ref{tab:unpolLQCDstatus1}.
%
All results are quoted at $\mu^2=4\mbox{ GeV}^2$.

%-------------------------------------------------------------------------------
\begin{table}[!t]
\renewcommand{\arraystretch}{1.2} 
\centering
\begin{threeparttable}
\begin{tabular}{llcllccccccl}
\toprule
Mom. & Collab. & Ref. & $N_f$ & Status &  
Disc &
QM &
FV &
Ren &
ES &
%
& Value \\
\midrule
$g_A$
& CalLat\,17 
  & \cite{Berkowitz:2017gql} 
  & 2+1+1 
  & PreP 
  & \rsquare 
  & \bstar  
  & \rsquare 
  & \bstar 
  & \bstar 
  & %$^\diamond$ 
  & 1.278(21)(26) \\
& PNDME\,16  
  & \cite{Bhattacharya:2016zcn} 
  & 2+1+1 
  & P    
  & \bcirc   
  & \bstar  
  & \bcirc   
  & \bstar 
  & \bstar 
  & 
  & 1.195(33)(20)\\
& LHPC\,14    
  & \cite{Green:2012ud} 
  & 2+1 
  & P 
  & \rsquare 
  & \bstar 
  & \bstar 
  & \bstar  
  & \bstar & & 0.97(8)\\
& Mainz\,17   
  & \cite{Capitani:2017qpc} 
  & 2 
  & PreP 
  & \bstar 
  & \bcirc 
  & \bstar 
  & \bstar  
  & \bstar 
  & 
  & $1.278(68)({}^{+0}_{-0.087})$\\
& ETMC\,17    
  & \cite{Alexandrou:2017hac} 
  & 2 
  & P
  & \rsquare  
  & \bstar 
  & \rsquare  
  & \bstar  
  & \bstar 
  & $^*$ 
  & 1.212(33)(22)\\
& RQCD\,15    
  & \cite{Bali:2014nma} 
  & 2 
  & P 
  & \bcirc 
  & \bcirc  
  & \bcirc  
  & \bstar   
  & \bcirc 
  & $^\ddag$
  & 1.280(44)(46) \\
  & QCDSF\,14   
  & \cite{Horsley:2013ayv} 
  & 2 
  & P 
  & \bcirc 
  & \bcirc  
  & \bcirc  
  & \bstar  
  & \rsquare 
  & $^\ddag$
  & 1.29(5)(3) \\
\bottomrule
\end{tabular}
\begin{tablenotes}
\footnotesize
\item[$*$] Study employing a single physical pion mass ensemble.
\item[$^\ddag$] $g_A$ is determined via the ratio $g_A/f_\pi$, employing the 
physical value for $f_\pi$.
%\item[$\diamond$] Approach inspired by the Feynman-Hellmann method.
\end{tablenotes}
\end{threeparttable}
\caption{\small Same as Table~\ref{tab:unpolLQCDstatus1}, but for the axial 
coupling, $g_A\equiv \langle 1\rangle_{\Delta u^+-\Delta d^+}$. 
%
Studies with three or more red squares are omitted from this table.
%
Values are shown at $\mu^2=4\mbox{ GeV}^2$.
%
}
\label{tab:gAstatus}
\end{table}
%-------------------------------------------------------------------------------

As is apparent from Table~\ref{tab:gAstatus}, we consider 
only three calculations of $g_A$ to have all systematics
sufficiently controlled to obtain a blue circle or star.
%
One of them~\cite{Bhattacharya:2016zcn} is for $N_f=2+1+1$, while two of 
them~\cite{Capitani:2017qpc,Bali:2014nma} are for $N_f=2$.
%
In the former case, our benchmark value corresponds to the single calculation;
in the latter case, our benchmark value corresponds to a weighted average 
of \cite{Capitani:2017qpc} and \cite{Bali:2014nma}, assuming correlations
between the results, and applying the procedure of \cite{Schmelling:1994pz}.
%
In summary, our benchmark values are
\begin{equation}\label{eq:gAcriteria}
g_A^{N_f=2+1+1} = 1.195(33)(20)
\,,\qquad \mathrm{and}\qquad 
g_A^{N_f=2} = 1.279(50)\,.
\end{equation}

We observe that the result of~\cite{Berkowitz:2017gql}, although it does
not fulfill all our requirements on systematic uncertainties, uses the same 
gauge configurations as those of \cite{Bhattacharya:2016zcn}.
%
Therefore, we also carry out a simultaneous fit to the two results for
completeness.
%
We use a fit function of the form
\begin{eqnarray}
g_A^{\mathrm{fit}}
&=&
c_0 +
f(a) +
c_3M_\pi^2 +
c_4M_\pi^2 \exp(-M_\pi L) +
c_5M_\pi^2 \log\left(\frac{M_\pi^2}{\Lambda_{\chi \mathrm{PT}}^2}\right)\,,
\\
f(a) &=&
\begin{cases}
  c_1a   & \qquad \text{Ref.~\cite{Bhattacharya:2016zcn}} \\
  c_2a^2 & \qquad \text{Ref.~\cite{Berkowitz:2017gql}}\\
\end{cases}
\qquad .
\end{eqnarray}
%
The coefficient $c_1$ captures ${\cal O}(a)$ effects present in the
valence-quark action of~\cite{Bhattacharya:2016zcn}, while~\cite{Berkowitz:2017gql} 
has discretization effects starting at ${\cal O}(a^2)$. 
%
The term proportional to $c_4$ captures the leading finite-volume effects, and 
$c_3$ and $c_5$ represent chiral-extrapolation terms. 
%
Modifications to this fit form, including setting $c_5=0$, have a negligible 
effect on the fit results within extrapolation uncertainties, and the final 
result is in very good agreement with a weighted average of the two 
calculations, assuming 100\% correlations, which is 
$g_A^{N_f=2+1+1,\mathrm{avg}} = 1.243(36)$. 
%
Based on this fit, we find the best-estimate band of
\begin{equation}\label{eq:gAfit}
g_A^{N_f=2+1+1,\mathrm{fit}} = \numrange{1.22}{1.28}\,.
\end{equation}

We plot all lattice results for the axial coupling, listed in 
Table~\ref{tab:gAstatus}, in Fig.~\ref{fig:gaLQCDstatus}. 
%
We show the world-average experimental value as a vertical black line. 
%
The light gray bands for $N_f=2+1+1$ and $N_f=2$ represent the benchmark 
results of Eq.~\eqref{eq:gAcriteria}, and the dashed gray band for
$N_f=2+1+1$ is the combined fit band given in Eq.~\eqref{eq:gAfit}. 

%-------------------------------------------------------------------------------
\begin{figure}[!t]
\centering
\includegraphics[scale=0.7]{plots/ga_summary.pdf}\\
\caption{\small Summary of the current status of lattice-QCD calculations of 
the axial charge, $g_A\equiv \langle 1\rangle_{\Delta u^+-\Delta d^+}$.
%
The vertical black line represents the current experimental world average 
$g_A^{\mathrm{exp}} = 1.2723(23)$~\cite{Olive:2016xmw}. 
%
The light gray bands for $N_f=2+1+1$ and $N_f=2$ represent the benchmark 
results of Eq.~\eqref{eq:gAcriteria}, and the dashed gray band for
$N_f=2+1+1$ is the fit band of Eq.~\eqref{eq:gAfit}.}    
\label{fig:gaLQCDstatus}
\end{figure}
%-------------------------------------------------------------------------------

In addition to the axial charge, we summarize the zeroth moments of the 
individual light-quark total polarized distributions in 
Table~\ref{tab:polLQCDstatus0}. 
%
We summarize the status of lattice-QCD calculations of the
first moments of the polarized PDF combination 
$\langle x \rangle_{\Delta u^- - \Delta d^-}$ in Table~\ref{tab:polLQCDstatus1}. 
%
We use the same format as in Table~\ref{tab:unpolLQCDstatus1}.
%
All values are at $\mu^2=4\mbox{ GeV}^2$.
%
Available results that have not been extrapolated to the physical pion mass
or quenched results are not reported here, but in Appendix~\ref{sec:LQCDtables},
Tables~\ref{tab:polLQCDstatus1B}-\ref{tab:polLQCDstatus2B}, for completeness.

In the case of $\langle 1 \rangle_{\Delta u^+}$ and $\langle 1 \rangle_{\Delta d^+}$,
there is only one result available in the literature for each quantity.
%
Therefore, although the corresponding systematic uncertainties are not 
completely under control and possibly underestimated, we take the individual 
results as our benchmark values.
%
In the case of $\langle 1 \rangle_{\Delta s^+}$ and 
$\langle x \rangle_{\Delta u^- - \Delta d^-}$, however, several results are available
in the literature, although without a full characterization of
their systematic uncertainties.
%
We present our lattice-QCD benchmark value for these quantities as
a best-estimate band extending from the mean minus the error of the 
smallest result to the mean plus the error of the largest. 
%
We include all results with two or more flavors of sea quarks listed in 
Tables~\ref{tab:polLQCDstatus0} and \ref{tab:polLQCDstatus1}, respectively.

The lattice-QCD benchmark numbers for $g_A$,
$\langle 1 \rangle_{\Delta u^+}$, $\langle 1 \rangle_{\Delta d^+}$,
$\langle 1 \rangle_{\Delta s^+}$ and $\langle x \rangle_{\Delta u^- - \Delta d^-}$
will be further commented below, where they will be collected together 
with their global-fit counterparts in Table~\ref{tab:BMpol}.

%-------------------------------------------------------------------------------
\begin{table}[!t]
\renewcommand{\arraystretch}{1.2} 
\centering
\begin{threeparttable}
\begin{tabular}{llcllccccccl}
\toprule
Mom. & Collab. & Ref. & $N_f$ & Status &
Disc &
QM &
FV &
Ren &
ES &
%
& Value \\
\midrule
$\langle 1\rangle_{\Delta u^+}$
& ETMC\,17 
  & \cite{Alexandrou:2017oeh} 
  & 2 
  & P
  & \rsquare 
  & \bstar 
  & \rsquare 
  & \bstar 
  & \bstar 
  & $^*$ 
  & $0.830(26)(4)$\\
\midrule
$\langle 1\rangle_{\Delta d^+}$
& ETMC\,17  
  & \cite{Alexandrou:2017oeh} 
  & 2 
  & P
  & \rsquare 
  & \bstar 
  & \rsquare  
  & \bstar 
  & \bstar 
  & $^*$ 
  & $-0.386(16)(6)$\\
\midrule
$\langle 1\rangle_{\Delta s^+}$
& $\chi$QCD\,17 
  & \cite{Gong:2015iir} 
  & 2+1 
  & P 
  & \rsquare  
  & \bcirc 
  & \bcirc  
  & \bstar 
  & \bstar
  & $^{\dagger,\triangleleft}$ 
  & -0.0403(44)(78)\\
& Engelhardt\,12 
  & \cite{Engelhardt:2012gd} 
  & 2+1 
  & P 
  & \rsquare  
  & \rsquare 
  & \bcirc  
  & \bstar  
  & \bstar  
  & $^\triangleleft$ 
  & -0.031(17)\\
& ETMC\,17 
  & \cite{Alexandrou:2017oeh} 
  & 2 
  & P
  & \rsquare  
  & \bstar 
  & \rsquare  
  & \bstar  
  & \bstar 
  & $^*$ 
  & -0.042(10)(2)\\
\bottomrule
\end{tabular}
\begin{tablenotes}
\footnotesize
\item[$*$] Study employing a single physical pion mass ensemble.
\item[$\dagger$] Partially quenched simulation with $m_\pi=330$~MeV. 
Criteria applied to the valence quarks. 
\item[$\triangleleft$] Some parts of the renormalization are estimated, 
see references for details.
\end{tablenotes}
\end{threeparttable}
\caption{\small Same as Table~\ref{tab:unpolLQCDstatus1}, but for the 
zeroth moments of the polarized total quark distributions.
%
Values are shown at $\mu^2=4\mbox{ GeV}^2$.
}
\label{tab:polLQCDstatus0}
\end{table}
%-------------------------------------------------------------------------------

%-------------------------------------------------------------------------------
\begin{table}[!t] 
\renewcommand{\arraystretch}{1.2}
\centering
\begin{threeparttable}
\begin{tabular}{llcllccccccl}
\toprule
Mom. & Collab. & Ref. & $N_f$ & Status &
Disc &
QM &
FV &
Ren &
ES &
& Value \\
\midrule
$\langle x\rangle_{\Delta u^--\Delta d^-}$
& RBC/ 
  & \multirow{2}{*}{\cite{Aoki:2010xg}} 
  & \multirow{2}{*}{2+1} 
  & \multirow{2}{*}{P} 
  & \multirow{2}{*}{\rsquare}  
  & \multirow{2}{*}{\rsquare} 
  & \multirow{2}{*}{\bstar}  
  & \multirow{2}{*}{\bstar}  
  & \multirow{2}{*}{\rsquare} 
  &  
  & 0.256(23)/\\
& UKQCD\,10 
  &  
  &  
  &  
  &   
  &  
  &   
  &   
  &  
  &  
  & 0.205(59)\\
& LHPC\,10 
  & \cite{Bratt:2010jn} 
  & 2+1 
  & P 
  & \rsquare  
  & \rsquare 
  & \bcirc  
  & \bcirc  
  & \rsquare 
  &  
  & 0.1972(55)\\
& ETMC\,15 
  & \cite{Abdel-Rehim:2015owa} 
  & 2 
  & P 
  & \rsquare  
  & \bstar 
  & \rsquare  
  & \bstar  
  & \bstar 
  & $^*$ 
  & 0.229(33)\\
\bottomrule
\end{tabular}
\begin{tablenotes}
\footnotesize
\item[$*$] Study employing a single physical pion mass ensemble.
\end{tablenotes}
\end{threeparttable}
\caption{\small Same as Table~\ref{tab:unpolLQCDstatus1}, but for the 
first moment of the polarized valence-quark distribution.
%
Values are shown at $\mu^2=4\mbox{ GeV}^2$.
}
\label{tab:polLQCDstatus1}
\end{table}
%-------------------------------------------------------------------------------

\subsubsection{Appraising global-fit results}
\label{subsubsec:GPDFfits}

The current status of global PDF fit determinations and their 
uncertainties has been carefully assessed in dedicated reviews
recently~\cite{Forte:2013wc,Jimenez-Delgado:2013sma}, and further 
summarized in Sect.~\ref{sec:unpPDFs}. 
%
It is now recognized that PDF uncertainties receive various contributions: 
the measurement uncertainty propagated from the data, uncertainties associated 
with incompatible data sets, procedural uncertainties such as those related to 
the choice of the PDF parametrization, 
and the handling of systematic errors, among others.
%
As outlined in Sect.~\ref{sec:unpPDFs}, in principle all of these uncertainties 
can be accounted for with suitable methods, both in the Hessian and the 
MC frameworks.
%
In practice, there is a significant spread in the sophistication 
of these methods between unpolarized and polarized PDF fits.

In Sect.~\ref{sec:unpPDFs}, we also emphasized that there are additional 
theoretical uncertainties on PDFs associated with uncertainty in
the input values of the physical parameters used in the fit (such as the 
reference value of the strong coupling) and with missing higher-order
uncertainties (given that fits are usually performed with fixed-order
perturbation theory).
%
The size of the former can be accounted for by studying the stability of the 
results upon variation of the input parameters; the size of the latter is
currently unknown, although it is supposed to be sub-dominant.
%
Therefore, theoretical uncertainties will not be considered in the following.

As far as full moments of PDFs are concerned, global-fit results involve
some degree of extrapolation to the region not covered by experimental data, 
that is not necessarily well accounted for in the PDF error estimates.
%
Extrapolation is particularly delicate to small $x$ values in the case of 
polarized PDFs: opposite to unpolarized PDFs, the kinematic coverage is 
fairly limited (see Sect.~\ref{sec:polPDFs} and in particular 
Fig.~\ref{fig:kinEIC}) and there is no analog of the momentum sum rule,
Eq.~\eqref{eq:mom}, to further constrain the PDFs.
%
Extrapolation uncertainties are difficult to quantify, unless
one naively extrapolates uncertainty bands from the measured region.

We now summarize the results for our benchmark moments listed in 
Sect.~\ref{subsubsec:BQ}, based on current global-fit determinations of
unpolarized and polarized PDFs.
%
We specify how the
available results are combined into a single benchmark value.

\paragraph{Unpolarized parton distributions.}

We summarize the current status of global-fit results of the benchmark
moments of unpolarized PDFs listed in Sect.~\ref{subsubsec:BQ} 
in Table~\ref{tab:unpPDFmoms}.
%
In the first column we indicate the computed moment, and in the subsequent 
six columns the moment's value, obtained from the most recent available PDF 
determinations: NNPDF3.1~\cite{Ball:2017nwa},
CT14~\cite{Dulat:2015mca}, MMHT2014~\cite{Harland-Lang:2014zoa},
ABMP16~\cite{Alekhin:2017kpj} (with $N_f=4$ flavors), 
CJ15~\cite{Accardi:2016qay} and 
HERAPDF2.0~\cite{Abramowicz:2015mha} respectively.
%
The most relevant features of these PDF sets have been presented in 
Sect.~\ref{sec:unpPDFs}.
%
All values in Table~\ref{tab:unpPDFmoms} are displayed
at $\mu^2=4\mbox{ GeV}^2$. 
%
They have been obtained from the default PDF sets at the highest available 
perturbative order, which is NNLO for all of them except CJ15
for which it is NLO.
%
The uncertainties for the CT14 PDF set has been rescaled by a factor $1/1.65$ 
to convert from  90\%-CL bands to  68\%-CL bands.
%
Note that tolerance of $\Delta \chi^2=1$ at 68\% CL is used in the CJ15 PDF 
set; hence, the smaller uncertainties of this set compared to all the other 
PDF sets.
%
Also, the CJ15 set does  not fit $\langle x \rangle_{s^+}$, therefore the 
corresponding number is not displayed in Table~\ref{tab:unpPDFmoms}. 
%
In the case of the HERAPDF2.0 set, the error band is the sum in quadrature 
of the statistical, model and parametrization uncertainties.
%
Taking the results of Table~\ref{tab:unpPDFmoms} at face value,
there are clear discrepancies arising from a variety of 
factors~\cite{Butterworth:2015oua,Accardi:2016ndt};
we examine some of these in the following. 

%-------------------------------------------------------------------------------
\begin{table}[!t]
\centering
\renewcommand{\arraystretch}{1.2}
\begin{tabular}{lcccccc}
\toprule
Mom. 
& NNPDF3.1 & CT14 & MMHT2014 & ABMP2016 & CJ15 & HERAPDF2.0 \\
\midrule
$\langle x \rangle_{u^+-d^+}$ 
& 0.152(3) & 0.158(4) & 0.151(4) & 0.167(4) & 0.152(2) & 0.188(3)\ \,\\
$\langle x \rangle_{u^+}$    
& 0.348(4) & 0.348(3) & 0.348(5) & 0.353(3) & 0.348(1) & 0.372(4)\ \,\\
$\langle x \rangle_{d^+}$    
& 0.196(3) & 0.190(3) & 0.197(5) & 0.186(3) & 0.196(1) & 0.185(7)\ \,\\
$\langle x \rangle_{s^+}$    
& 0.039(3) & 0.035(5) & 0.035(9) & 0.041(2) & ---   & 0.035(11)\\
$\langle x \rangle_{g}$     
& 0.410(4) & 0.416(5) & 0.411(9) & 0.412(4) & 0.416(1) & 0.401(10)\\
\bottomrule
\end{tabular}
\caption{\small Status of current global PDF fit determinations of the 
benchmark moments of unpolarized PDFs listed in Sect.~\ref{subsubsec:BQ}.
All values are shown at $\mu^2=4\mbox{ GeV}^2$.
%
See text for details about the calculation of PDF uncertainties in each case.
}
\label{tab:unpPDFmoms}
\end{table}
%-------------------------------------------------------------------------------

In order to provide a benchmark value for the first moments of unpolarized PDFs
listed in Table~\ref{tab:unpPDFmoms}, we follow the latest PDF4LHC 2015 
recommendations~\cite{Butterworth:2015oua}.
%
Even though the recommendations were primarily formulated for the usage of PDFs
in LHC-related physics, and alternative recommendations have been 
suggested~\cite{Accardi:2016ndt}, we find it useful to apply them here as well.
%
The reason is twofold.
%
First, this benchmark exercise aims at accuracy and precision,  
two of the guiding principles underlying the recommendations.
%
Second, they led to the release of a specific PDF set
that can be easily used to compute all the needed benchmark values.

While we refer the reader to \cite{Butterworth:2015oua} for details,
here we only mention that the PDF4LHC15 PDF set was constructed by means of
a statistical combination~\cite{Carrazza:2015hva,Gao:2013bia,Watt:2012tq,
Carrazza:2015aoa} (an unweighted average) of the 
NNPDF3.0~\cite{Ball:2014uwa}, CT14 and MMHT2014 PDF sets.\footnote{The 
NNPDF3.1 PDF set was not available when the recommendations were formulated.}
%
The three PDF sets were selected among all the publicly available PDF sets
based on four criteria~\cite{Butterworth:2015oua}.
%
\begin{itemize}
%
\item A global data set from a wide variety of observables and processes
should be included in the fit analysis.
%
\item Theoretical hard cross sections should be evaluated up to NNLO in a
general-mass variable-flavor number scheme with up to $N_f^\text{max}=5$ 
active quark flavors.
%
\item The central value of the strong coupling at the $Z$-boson mass,
$\alpha_s(M_Z^2)$ should be fixed at an agreed common value, consistent 
with the PDG world-average~\cite{Olive:2016xmw} ($\alpha_s(M_Z)=0.118$).
%
\item All known experimental and procedural sources of uncertainty should be 
properly accounted for.
%
\end{itemize}
%
The ABMP2016 set (as well as its previous versions) does not meet the second 
and third criteria; the CJ15 set does not meet the first, second and fourth
criteria, while the HERAPDF2.0 set does not meet the first criterion.
%
Hence, these sets were not included in the PDF4LHC2015 PDF set, although the 
possibility of including them in future versions of the recommendation 
remains open.

In order not to loose important information contained in the PDF sets excluded 
from the PDF4LHC recommendations, we also provide alternative benchmark numbers.
%
Specifically, we combined all the numbers quoted in Table~\ref{tab:unpPDFmoms}
so that the mean value is an unweighted average of the mean 
values and the error is half of the difference between the smallest and the 
largest result.
%
The rationale for this choice is that PDF sets entering the PDF4LHC 
recommendations are not benchmarked in the $x\gtrsim 0.1$ region, which can be 
relevant for the moment analysis.
%
The combination of all results in Table~\ref{tab:unpPDFmoms}, although 
sometimes less precise than the PDF4LHC combination, maximizes the amount of 
experimental information included in the benchmark numbers.
%
Specifically, it includes the information taken into account 
at large $x$ and small $Q^2$ in the CJ15 and ABMP16 PDF sets, 
which is otherwise excluded from the PDF4LHC set.

The global-fit benchmark numbers for $\langle x\rangle_{u^+-d^+}$,
$\langle x \rangle_{u^+}$, $\langle x \rangle_{d^+}$, 
$\langle x \rangle_{s^+}$ and $\langle x \rangle_g$ will be further
commented below, where they will be collected together with their 
lattice-QCD counterparts in Table~\ref{tab:BMunp}.

\paragraph{Polarized parton distributions.}
%
We summarize the current status of global-fit results of the benchmark
moments of polarized PDFs listed in Sect.~\ref{subsubsec:BQ} in 
Table~\ref{tab:polPDFmoms}.
%
In the first column, we indicate the computed moment, and in the subsequent 
three columns, its value as obtained from the most recent available PDF 
determinations: NNPDFpol1.1~\cite{Nocera:2014gqa}, 
DSSV08~\cite{deFlorian:2009vb}~\footnote{The DSSV08 analysis has been updated
by the DSSV14 analysis~\cite{deFlorian:2014yva} essentially 
only in the determination of the gluon PDF. 
The moments in Table~\ref{tab:polPDFmoms} therefore hardly differ
in the two analyses.}, JAM15~\cite{Sato:2016tuz} and 
JAM17~\cite{Ethier:2017zbq}.
%
The most relevant features of these PDF sets have been presented in
Sect.~\ref{sec:polPDFs}.
%
All values in Table~\ref{tab:unpPDFmoms} are displayed
at $\mu^2=4\mbox{ GeV}^2$ at NLO.
%
The uncertainties correspond to 68\%-CL bands with tolerance of 
$\Delta \chi^2=1$ for the DSSV08 PDF set.
%
In the case of the JAM15 set, we do not provide a value for 
$\langle x \rangle _{\Delta u^--\Delta d^-}$:
the fit is based on inclusive DIS data only, which are not sensitive to 
the valence distribution $\Delta u^- - \Delta d^-$.
%
We emphasize that, because of extrapolation uncertainties difficult to quantify,
the error estimates in Table~\ref{tab:polPDFmoms} should be interpreted
as a lower bound, especially for the DSSV08 and JAM sets based on 
conventional parametrizations.
%
In these cases, uncertainty bands are naively extrapolated from the measured 
kinematic region, therefore they are likely to underestimate the contribution 
coming from the small-$x$ region.

%-------------------------------------------------------------------------------
\begin{table}[!t]
\centering
\renewcommand{\arraystretch}{1.2}
\begin{tabular}{lcccc}
\toprule
Mom. 
& NNPDFpol1.1 & DSSV08 & JAM15 & JAM17 \\
\midrule
$\langle 1 \rangle_{\Delta u^+-\Delta d^+}$ &
\ 1.250(16) & \ 1.260(18) & 1.314(6)\,  & \ 1.240(41)\\
$\langle 1 \rangle_{\Delta u^+}$ &
\ 0.794(46) & \ 0.814(12) & \ 0.831(21) & \ 0.812(22)\\
$\langle 1 \rangle_{\Delta d^+}$ &  
-0.453(52)  &  -0.456(11) &  -0.476(22) &  -0.428(31)\\
$\langle 1 \rangle_{\Delta s^+}$ &  
-0.120(81)  &  -0.112(23) &  -0.109(20) &  -0.038(96)\\
$\langle x \rangle_{\Delta u^- - \Delta d^-}$ &     
\ 0.195(14) &  0.203(9)\, &  ---        & \ 0.241(26)\\
\bottomrule
\end{tabular}
\caption{\small Status of current global-fit determinations of the 
benchmark moments of polarized PDFs listed in Sect.~\ref{subsubsec:BQ}.
All values are shown at $\mu^2=4\mbox{ GeV}^2$.}
\label{tab:polPDFmoms}
\end{table}
%-------------------------------------------------------------------------------

As outlined in Sect.~\ref{sec:polPDFs}, polarized PDFs cannot be determined in 
a global QCD analysis with the same accuracy as their unpolarized counterparts.
%
Also, because polarized PDFs do not enter precision physics studies at the LHC, 
the selection and combination of different PDF sets has received much less
attention.
%
No recommendations analogous to those from the PDF4LHC working group
exist for polarized PDFs.

To provide a benchmark value for the relevant moments of 
polarized PDFs listed in Table~\ref{tab:polPDFmoms}, we apply an unweighted 
average of the NNPDFpol1.1, DSSV08 and JAM15 PDF sets.
%
The rationale for this choice is twofold.
%
On the one hand, we maximize the amount of experimental information 
that can determine the central value of our benchmark moments.
%
As explained in Sect.~\ref{sec:polPDFs}, the NNPDFpol1.1 and the DSSV08 PDF 
sets are based on a very similar set of inclusive DIS data, while the JAM15 
PDF set is based on a much wider inclusive DIS data set.
%
This wider set can help constrain the moments of the total quark 
distributions.
%
The NNPDFpol1.1 and the DSSV08 PDF sets are based respectively on $pp$ and 
SIDIS data to disentangle the quark and antiquark distributions.
%
This can help constrain the moments of the valence distributions.
%
On the other hand, we balance the rather different uncertainties among the 
three PDF sets, specifically the larger NNPDFpol1.1 estimate
against the smaller DSSV08 and JAM15 values.
%
This way, we avoid a possible underestimation of the procedural uncertainties 
induced for example by the choice of a simple PDF parametrization 
in the DSSV08 and JAM15 fits, or by the extrapolation to the small-$x$ region.
%
Because the JAM17 set is unique in fitting simultaneously polarized PDFs and 
FFs, we do not include it in our benchmark average, but quote it as a useful 
comparison.

The global-fit benchmark numbers for $g_A$,
$\langle 1 \rangle_{\Delta u^+}$, $\langle 1 \rangle_{\Delta d^+}$,
$\langle 1 \rangle_{\Delta s^+}$ and $\langle x \rangle_{\Delta u^- - \Delta d^-}$
will be further commented below, where they will be collected
with their lattice-QCD counterparts in Table~\ref{tab:BMpol}.

%%%%%%%%%%%%%%%%%%%%%%%%%%%%%%%%%%%%%%%%%%%%%%%%%%%%%%%%%%%%%%%%%%%%%%%%%%%%%%%%
\subsection{Comparing lattice-QCD and global-fit benchmark moments}
\label{subsec:BN}

We can now compare the lattice-QCD and global PDF fit results presented in 
Sects.~\ref{subsubsec:BClQCD}-\ref{subsubsec:GPDFfits} for the unpolarized
and polarized PDF moments respectively.

\paragraph{Unpolarized parton distributions.}
%
The benchmark values of the first moments of the unpolarized PDFs, obtained
as described in Sects.~\ref{subsubsec:BClQCD}-\ref{subsubsec:GPDFfits}, 
are summarized in Table~\ref{tab:BMunp}.
%
Both the PDF4LHC and the unweighted average (uw avg) are displayed in the case 
of global fits.
%
The results from a single lattice calculation, which might underestimate some 
sources of uncertainty, are denoted with a superscript~$\dagger$.
%
All values shown here are at $\mu^2=4\mbox{ GeV}^2$.
%
For ease of comparison, these benchmark results are also graphically
compared in Fig.~\ref{fig:Bmomsunp}, both in terms of absolute values 
(left panel) and of uncorrelated ratios to the lattice central values 
(right panel).

%-------------------------------------------------------------------------------
\begin{table}[!t]
\centering
\renewcommand{\arraystretch}{1.2}
\begin{tabular}{lccc}
\toprule
Moment & Lattice QCD & Global Fit (PDF4LHC) & Global fit (uw avg)\\
\midrule
$\langle x \rangle_{u^+ -d^+}$ 
& \numrange{0.119}{0.226} 
& 0.155(5)
& \, 0.161(18)\\
$\langle x \rangle_{u^+}$     
& 0.453(75)$^\dagger$ 
& 0.347(5)
& \, 0.352(12)\\
$\langle x \rangle_{d^+}$     
& 0.259(74)$^\dagger$ 
& 0.193(6)
& 0.192(6)\\
$\langle x \rangle_{s^+}$     
& 0.092(41)$^\dagger$ 
& 0.036(6)
& 0.037(3)\\
$\langle x\rangle_{g}$       
& 0.267(35)$^\dagger$ 
& 0.414(9)
& 0.411(8)\\
\bottomrule
\end{tabular}
\caption{\small Benchmark values for lattice-QCD calculations and global-fit 
determinations of the benchmark moments of unpolarized PDFs.
%
All values are shown at $\mu^2=4\mbox{ GeV}^2$.
%
Results with a superscript~$\dagger$ are from a single lattice 
calculation; they may underestimate some sources of uncertainty.}
\label{tab:BMunp}
\end{table}
%-------------------------------------------------------------------------------

%-------------------------------------------------------------------------------
\begin{figure}[!t]
\centering
\includegraphics[scale=0.44,angle=270]{plots/unpmoms}
\includegraphics[scale=0.44,angle=270]{plots/unpmomsratio}\\
\caption{\small A comparison of the unpolarized PDF benchmark moments 
between the lattice QCD computations and global fit determinations.
%
Results are displayed both in terms of absolute values (left) and ratios to
the lattice values (right) at $\mu^2=4$ GeV$^2$.}
\label{fig:Bmomsunp}
\end{figure} 
%-------------------------------------------------------------------------------

As is apparent from Table~\ref{tab:BMunp} and Fig.~\ref{fig:Bmomsunp}, there is 
a significant difference in the uncertainties between the lattice QCD and 
global fit results, with the latter always about one order of magnitude 
smaller than the former.
%
Moreover, even within their large uncertainties, the lattice-QCD results 
for the first moments of the total up and strange quark and the gluon PDFs
are not compatible with their global-fit counterparts.
%
In the case of quarks, the discrepancy is below $2\sigma$ (in units of the 
lattice-QCD uncertainty), while in the case of the gluon the discrepancy is
slightly larger than $3\sigma$.

On the lattice-QCD side, we note that in the flavor-singlet sector calculations
neglected part of the renormalization and computed some other parts only 
perturbatively.
%
Most of the discrepancies between lattice-QCD and global-fit results are 
observed in the flavor-singlet sector.
%
Progress in taking into account the renormalization properly
could shift lattice-QCD results significantly, and reconcile them 
with the global fits in the future.
%
We also note that the momentum sum rule, Eq.~\eqref{eq:mom}, usually is not 
imposed in lattice-QCD calculations.
%
In the ETMC\,17 analysis~\cite{Alexandrou:2017oeh}, it turns
out to be $1.071(93)(72)$, see Table~\ref{tab:unpolLQCDstatus1}, if 
uncertainties are assumed to be uncorrelated.
%
Although there is no evidence for a violation of the momentum sum rule 
based on this result, one must be careful combining results from different 
calculations to account for correlations and other sources of error. 
%
Also, note that the ETMC\,17 analysis is performed with $N_f=2$ flavors,
hence the strange quark should not participate in the sum rule.

On the global-fit side, we note that the amount of experimental information 
that constrains the total up-quark distribution is the largest among all 
distributions.
%
Therefore, it seems unlikely that its global-fit central value could vary 
significantly in the future, and become compatible with the current
lattice result.
%
Conversely, the amount of experimental information that constrains the
total strange distribution in a global fit is less abundant and less accurate.
%
A slight shift in its central value, towards the current lattice-QCD results,
might be observed in the future, as soon as new data sensitive to the strange 
quark becomes available.
%
Finally, in an attempt to reconcile the lattice-QCD and the global-fit results
of the first moment of the gluon PDF, one could assume a completely
different behavior of the gluon PDF below the HERA kinematic
coverage, $x\sim 10^{-5}$ (see Fig.~\ref{fig:kinplot-report}).
%
While such a kinematic region remains completely unexplored,
in general the contribution of this region to the moments is negligible
and thus unlikely to resolve the situation. 

All these remarks apply irrespective of the benchmark value used for 
global fits, either the PDF4LHC or the unweighted average.
%
They also still hold if individual lattice-QCD and/or global-fit
results in Tables~\ref{tab:unpolLQCDstatus1}-\ref{tab:unpPDFmoms} are 
compared instead of their benchmark values in Table~\ref{tab:BMunp}. 
%
These results suggest that both greater accuracy and greater precision are
required in lattice-QCD calculations to match the accuracy and 
precision of the first moments of unpolarized PDFs determined from a global
fit.

\paragraph{Polarized parton distributions.}
%
The benchmark values of the first moments of the unpolarized PDFs, obtained
as described in Sects.~\ref{subsubsec:BClQCD}-\ref{subsubsec:GPDFfits}, 
are summarized in Table~\ref{tab:BMpol}.
%
Results from a single lattice calculation, which might underestimate some 
sources of uncertainty, are denoted with a superscript~$\dagger$.
%
In the case of $g_A$, we report the two values with $N_f=2+1+1$ and
$N_f=2$ sea quarks from lattice QCD.
%
The value of $g_A$ is scale-independent, and we quote all other results at 
$\mu^2=4\mbox{ GeV}^2$.
%
For ease of comparison, these values are also displayed in 
Fig.~\ref{fig:Bmomspol} in the same format as in Fig.~\ref{fig:Bmomsunp}.
%
In the case of $g_A$, the result with $N_f=2+1+1$ is used as normalization
factor in the right panel of Fig.~\ref{fig:Bmomspol}.
%
Results from the JAM17 analysis~\cite{Ethier:2017zbq}, see 
Table~\ref{tab:polPDFmoms}, are displayed separately.
%
In contrast with the NNPDFpol1.1, DSSV08 and JAM15 fits, in the JAM17 fit 
the experimental value of $g_A$, Eq.~\eqref{eq:a3}, 
is not an input of the PDF fit.

%-------------------------------------------------------------------------------
\begin{table}[!t]
\centering
\renewcommand{\arraystretch}{1.2}
\begin{tabular}{lcc}
\toprule
Moment & Lattice QCD & Global Fit\\
\midrule
\multirow{2}{*}{$g_A\equiv\langle 1\rangle_{\Delta u^+ - \Delta d^+}$} 
& 1.195(39) ($N_f=2+1+1$) 
& \multirow{2}{*}{\ 1.275(12)} \\
& 1.279(50) ($N_f=2$) & \\
$\langle 1 \rangle_{\Delta u^+}$     
& 0.830(26)$^\dagger$ 
& \ 0.813(25)\\
$\langle 1 \rangle_{\Delta d^+}$     
& -0.386(17)$^\dagger$ 
& -0.462(29)\\
$\langle 1 \rangle_{\Delta s^+}$     
& -0.052\,--\,-0.014
& -0.114(43)\\
$\langle x\rangle_{\Delta u^- - \Delta d^-}$       
& \numrange{0.146}{0.279} 
& \ 0.199(16)\\
\bottomrule
\end{tabular}
\caption{\small Same as Table~\ref{tab:BMunp}, but for the polarized benchmark 
moments.}
\label{tab:BMpol}
\end{table}
%-------------------------------------------------------------------------------

%-------------------------------------------------------------------------------
\begin{figure}[!t]
\centering
\includegraphics[scale=0.44,angle=270]{plots/polmoms}
\includegraphics[scale=0.44,angle=270]{plots/polmomsratio}\\
\caption{\small Same as Fig.~\ref{fig:Bmomsunp}, but for the polarized
benchmark moments.}
\label{fig:Bmomspol}
\end{figure} 
%-------------------------------------------------------------------------------

As is apparent from Table~\ref{tab:BMpol} and Fig.~\ref{fig:Bmomspol}, the 
size of the uncertainties on the moments is in general comparable between the 
lattice-QCD and the global-fit results, opposite to the unpolarized case.
%
The corresponding central values are also in reasonable agreement within their
mutual uncertainties.

In the case of $g_A$, the global-fit result obtained from the unweighted 
average of the NNPDFpol1.1, DSSV08 and JAM15 fits shows a preference for the
lattice-QCD result obtained with $N_f=2$ sea quarks (compared to that with 
$N_f=2+1+1$ sea quarks).
%
Its uncertainty is, however, four times smaller than that of both lattice 
results.
%
This is not unexpected, since, in all the three fits that are combined, the 
experimental value of $g_A$ is imposed in the fits themselves.
%
The final uncertainty on the global-fit value of $g_A$ is thus reduced by 
the uncertainty of its experimental value $g_A^\text{exp}$, which is almost
one order of magnitude smaller than the uncertainty on the lattice-QCD results
(see Fig.~\ref{fig:gaLQCDstatus}).
%
If the experimental value of $g_A$ is not imposed as a boundary condition in 
the fit, as in the JAM17 analysis, the size of the uncertainty on $g_A$ is 
comparable to that of the lattice results, although it is not able to 
discriminate between the $N_f=2$ or the $N_f=2=1+1$ results.
%
Overall, this is a noteworthy confirmation of SU(2) symmetry in QCD to
almost 2\%.

In the case of the zeroth moments of the total polarized quark distributions,
the uncertainty on the lattice-QCD result is comparable to (in the case
of $\langle 1 \rangle_{\Delta u^+}$) or smaller than (in the case
of $\langle 1 \rangle_{\Delta d^+}$ and $\langle 1 \rangle_{\Delta s^+}$)
the uncertainty on the global-fit result.
%
However, in the case of the zeroth moments of the total down- and strange-quark 
distributions, the lattice-QCD and the global-fit results are discrepant
by about two $\sigma$ (in units of the
lattice QCD uncertainty).
%
On the one hand, we observe that the uncertainty on the lattice-QCD results 
might have been underestimated because of the lack of full control over
all systematics (see Sect.~\ref{subsubsec:BClQCD}).
%
On the other hand, we observe that the global-fit result has been obtained
by requiring SU(3) symmetry, {\it i.e.}, by imposing in the individual fits 
the experimental value (with a possibly inflated uncertainty) of the octet PDF 
combination, as explained in Sect.~\ref{sec:polPDFs}.
%
Relaxing this constraint can reconcile the discrepancy observed between 
the lattice-QCD and the global-fit result for the zeroth moments of the 
total down and strange PDFs.
%
This is demonstrated by comparison with the JAM17 result, whose uncertainty 
band nicely includes both the lattice-QCD and the global-fit benchmark values.

In the case of the first moment of the valence distribution 
$\Delta u^--\Delta d^-$, the lattice-QCD and the global-fit results are 
in excellent agreement, although the uncertainty of the former is five times
larger than that of the latter.

All these remarks still hold if individual lattice-QCD and/or global-fit
results in Tables~\ref{tab:gAstatus}-\ref{tab:polLQCDstatus0}-\ref{tab:polLQCDstatus1}-\ref{tab:polPDFmoms}
are compared instead of their benchmark values in Table~\ref{tab:BMpol}.
%
These results suggest that lattice-QCD calculations could provide a useful
input to global fits of polarized PDFs, especially in limiting the
extrapolation uncertainty into the completely unknown small-$x$ region.
%
This will become more and more useful as full control over all sources of
systematic uncertainties is achieved.


%%%%%%%%%%%%%%%%%%%%%%%%%%%%%%%%%%%%%%%%%%%%%%%%%%%%%%%%%%%%%%%%%%%%%%%%%%%%%%%%
\section{Improving PDF fits with lattice QCD calculations}
\label{sec:projections}
%%%%%%%%%%%%%%%%%%%%%%%%%%%%%%%%%%%%%%%%%%%%%%%%%%%%%%%%%%%%%%%%%%%%%%%%%%%%%%%%

In this section, we provide an estimate of the potential
impact of future lattice-QCD calculations
in global unpolarized and polarized PDF fits.
%
This study is carried out with two publicly available
tools: the Bayesian reweighting
method~\cite{Ball:2011gg,Ball:2010gb} applied to the
NNPDF3.1~\cite{Ball:2017nwa} and NNPDFpol1.1~\cite{Nocera:2014gqa} sets; 
and the Hessian profiling method~\cite{Camarda:2015zba} applied to
HERAPDF2.0 set~\cite{Abramowicz:2015mha}.
%
Both methods allow us to quantify the impact of new measurements
(or of future measurements, if pseudo-data are used) on PDFs without
repeating the global analysis.
%
The main limitation of these methods is that they are maximally reliable
if the amount of information carried in by the new (pseudo-)data is moderate
in comparison to that already included in the fit.

For simplicity, we limit our study to the impact of a subset of the moments 
that can be computed using lattice QCD, focusing on those that can be 
currently calculated with the highest precision.
%
Therefore, we restrict ourselves to the benchmark moments discussed in 
Sect.~\ref{sec:benchmarking}.
%
We also consider pseudo-data based on $x$-space
lattice-QCD calculations from the quasi-PDF approach
discussed in Sect.~\ref{sec:xdependence}.
%
As we show, particularly in the unpolarized case, the
constraining power of direct $x$-space calculations is
superior to that of PDF moments.

\subsection{Impact of lattice calculations of PDF moments}

We start by quantifying the constraining power of projected lattice-QCD 
calculations of PDF moments on both unpolarized and polarized global fits.
%
We define the settings for our study and present our results following
Bayesian reweighting and Hessian profiling, respectively.

% General projections for moments
\subsubsection{Settings}


Taking into account
these considerations, we will consider in this analysis the following
moments.
%
For the unpolarized case, we will use
\be
  \la x\ra_{u^+}\, , \quad
\la x\ra_{d^+}\, , \quad
\la x\ra_{s^+}\, , \quad
\la x\ra_{g}\, , \quad {\rm and} \quad
\la x\ra_{u^+-d^+} \, ,
\ee
while for the polarized side, we will include instead
the following five moments:
\be
\la 1\ra_{\Delta u^+}\, , \quad
\la 1\ra_{\Delta d^+}\, , \quad
\la 1\ra_{\Delta s^+}\, , \quad
\la x\ra_{\Delta u^--\Delta d^-}\, , \quad {\rm and} \quad
\la 1\ra_{\Delta u^+ - \Delta d^+} \, .
\ee
Recall that Appendix~\ref{app:notation} contains the
explicit definitions and conventions used for these moments.
%
Therefore, we see that for the unpolarized case we include
the second moments (momentum fractions) of $q^+$ (with $q=u,d,s$),
of the gluon, and of the isoscalar combination $u^+-d^+$.
%
In the polarized case instead, we include the first moments (which
contribute to the proton spin content) of $\Delta q^+$ (with $q=u,d,s$)
and of the isoscalar combination $\Delta u^+-\Delta d^+$, as well as
the second moment of $\Delta u^- - \Delta d^-$.

In the present exercise we will consider three
different scenarios, which we denote
as Scenario A, B, and C respectively, for the total systematic
uncertainty than we associate to lattice
QCD calculations of PDF moments.
%
In Table~\ref{tab:scenarios} we summarize the
values assumed for this total uncertainty
    of the lattice QCD calculation, denoted by $\delta_L$, for each
    of the various unpolarized and polarized PDF moments that enter
    this analysis.
    %
    We emphasize that here, while trying to be reasonably
    realistic, we do not aim to associate a given scenario
    within a specific time-scale for the calculation.
    %
    Our results  merely provide an illustrative guidance about the potential
    constraining power of existing and future lattice QCD calculations
    of PDF  moments in the context
    of a global analysis.
    
    The motivation for our choice of the scenarios
    in Table~\ref{tab:scenarios}
    is rather different from the unpolarized and polarized cases.
    %
    For the polarized fits,
    scenario A assumes that the uncertainties $\delta_L$
    for the lattice QCD calculations
    are on same the ball-park of the current ones, taking as
    representative values for the latter  those from the
    state-of-the-art lattice QCD calculations
    selected for the benchmarking exercise of Sect.~\ref{sec:benchmarking},
    and summarized in Table~\ref{tab:BMpol}.
    %
    Then scenarios B and C represent two possible optimistic scenarios for the
    future improvement of these systematic uncertainties, where these are decreased
    by roughly a factor 2 and a factor 4 with respect current values.
    
    On the other hand, for the unpolarized case scenario A is based on values
    of $\delta_L^{(i)}$ already rather smaller than the typical
    uncertainties that affect state-of-the-art calculations, see Table~\ref{tab:BMunp}.
    %
    The reason is that we have verified that including the pseudo-data $\mathcal{F}_i^{(\rm)}$
    assuming lattice-QCD uncertainties of similar size as those of Table~\ref{tab:BMunp}
    leaves the PDFs essentially unchanged, and only once the uncertainties
    $\delta_L^{(i)}$ are significantly reduced that we start to obtain a reduction
    of the uncertainties from the global fit.
    %
    The only connection with the uncertainties of the calculations in Table~\ref{tab:BMunp}
    is that we assume that $\delta_L^{(i)}$ is typically larger for $\la x\ra_{s^+}$
    and $\la x\ra_{u^+-d^+}$ as compared to the other moments.
    %
    A total systematic error of $\delta_L^{(i)}$ is probably the best that one can achieve
    within a lattice-QCD calculation even in principle, since at that level many other
    effects such as QED corrections become relevant and these are much more difficult
    to deal with.
    %
    For both
    the polarized and
    the unpolarized case,
    we emphasize that the generalization of these projections to other conceivable scenarios
    is straightforward and can be obtained from the authors upon request.
 
%%%%%%%%%%%%%%%%%%%%%%%%%%%%%%%%%%%%%%%
\begin{table}[t]
  \centering
  \renewcommand{\arraystretch}{1.3} 
  \begin{tabular}{c||ccccc}
    \hline
    Scenario &  \multicolumn{5}{c}{$\delta_L^{(i)}$ for unpolarized moments}   \\
&    $\la x\ra_{u^+}$  &   $\la x\ra_{d^+}$   &  $\la x\ra_{s^+}$  &
$\la x\ra_{g}$  &   $\la x\ra_{u^+-d^+}$  \\
    \hline
    Current  & $\sim 16\%$  &  $\sim 30\%$
    & $\sim 45\%$  & $\sim 13\%$  &  $\sim 60\%$ \\
    A   & 3\%  & 3\% &  5\% &  3\% &  5\% \\
 B   & 2\%  & 2\% &  4\% &  2\% &  4\%  \\
  C   & 1\%  & 1\% &  3\% &  1\% &  3\%  \\
    \hline
  \end{tabular}\vspace{0.7cm}
   \begin{tabular}{c||ccccc}
    \hline
    Scenario   &
    \multicolumn{5}{c}{$\delta_L^{(i)}$ for polarized moments} \\ 
& $\la 1\ra_{\Delta u^+}$  & $\la 1\ra_{\Delta d^+}$  & $\la 1\ra_{\Delta s^+}$
&  $\la x\ra_{\Delta u^--\Delta d^-}$  &  $\la 1\ra_{\Delta u^+ - \Delta d^+}$\\
    \hline
    Current  &
    $\sim 3\%$  & $\sim 5\%$ & $\sim 70\%$ & $\sim 65\%$ & $\sim 3\%$ \\
    \hline
    A   & 
    5\% &    10\%  &   100\% &    70\%  &    5\% \\
 B   &
 3\% &    5\%  &   50\% &    30\%  &    3\% \\
  C   & 1\% &    2\%  &   20\% &    15\%  &    1\% \\
    \hline
  \end{tabular}
   \caption{\small The three scenarios assumed here
     for the total percentage
     systematic uncertainty
    in future lattice QCD calculation $\delta_L$ for each
    of the unpolarized (upper) and polarized (lower table) PDF
    moments that are included
    in the present reweighting analysis.
    %
    In addition, the first line indicates the current systematic
    uncertainties of the state-of-the-art lattice QCD calculations
    selected for the benchmarking exercise of Sect.~\ref{sec:benchmarking},
    and summarized in Tables~\ref{tab:BMunp} and~\ref{tab:BMpol}
    for the unpolarized and polarized cases, respectively.
    %
    See text for more details.
\label{tab:scenarios}
  }
\end{table}
%%%%%%%%%%%%%%%%%%%%%%%%%%%%%%%%%%%%%%%




% PDF moment analysis using reweighting
\subsubsection{Bayesian reweighting analysis}

The procedure followed to quantify the impact of future
lattice QCD calculations
in PDF fits  (for the three scenarios of Table~\ref{tab:scenarios})
is common for unpolarized
and polarized global analyses.
%
We briefly describe this procedure here,
and refer to~\cite{Ball:2011gg,Ball:2010gb} for
additional details.
\begin{itemize}
\item First of all, we generate pseudo-data for the lattice QCD calculation
  of the PDF moments used in this exercise, namely $\la x\ra_{u^+}$,
$\la x\ra_{d^+}$,
$\la x\ra_{s^+}$,
$\la x\ra_{g}$, and
  $\la x\ra_{u^+-d^+}$ for the unpolarized case, and
  $\la 1\ra_{\Delta u^+}$,
$\la 1\ra_{\Delta d^+}$,
$\la 1\ra_{\Delta s^+}$,
$\la x\ra_{\Delta u^--\Delta d^-}$, and
  $\la 1\ra_{\Delta u^+ - \Delta d^+}$ for the polarized case.
  %
  We denote generically these moments by $\mathcal{F}_i$.
\item This pseudo-data, denoted by $\mathcal{F}_i^{\rm (exp)}$,
  is constructed by taking the central values from
  the corresponding NNPDF fits, NNPDF3.1 NNLO for the unpolarized case and NNPDFpol1.1 NLO
  for the polarized one.
  %
  That is, we {\it assume} for simplicity that the central value
  of such future lattice calculations would coincide with the current ones
  from the global fit.\footnote{Repeating the exercise with the actual lattice QCD
    central values would be straightforward, but
    lies beyond the scope of the present studies.}
  %
  As discussed in Sect.~\ref{sec:unpPDFs}, this corresponds to computing
  the mean over the Monte Carlo replica sample,
  \be
  \label{eq:pseudodatadef}
  \mathcal{F}_i^{\rm (exp)} \equiv \frac{1}{N_{\rm rep}}\sum_{k=1}^{N_{\rm rep}}
  \mathcal{F}_i^{\rm (k)} \, , \quad i=1,\ldots,N_{\rm mom} \, ,
  \ee
  where $N_{\rm mom}$ are the number of PDF moments that will be included
  in the reweighting, in this case $N_{\rm mom}=5$ both for the unpolarized
  and polarized cases.
  %
  To be consistent with the calculations in Sect.~\ref{sec:benchmarking},
  here the central values of the pseudo-data Eq.~(\ref{eq:pseudodatadef})
  are also evaluated at $Q^2=4$ GeV$^2$ (see Tables~\ref{tab:unpPDFmoms} and~\ref{tab:polPDFmoms}).
\item The uncertainty in the pseudo data, denoted by $\delta\mathcal{F}_i^{\rm (exp)} $,
  is taken for each moment to be the value indicated in
  Table~\ref{tab:scenarios} for each of the three scenarios.%
  That is, we have that the absolute uncertainty on the $i$-th moment
  will be given by $\delta\mathcal{F}_i^{\rm (exp)}=\delta_L^{(i)}\mathcal{F}_i^{\rm (exp)} $.
\item Using the pseudo-data (central values and total uncertainties)
  as defined above, we next need to compute
  the weights  $\omega_k$.
  %
  These weights
  quantify the agreement between each of $N_{\rm rep}$ replicas
  of the input PDF set and the corresponding lattice pseudo-data.
  %
  Specifically, first of all we compute the $\chi^2$ between each of the Monte Carlo
  replicas and the lattice pseudo-data as follows,
  \be
  \chi^{2(k)}= \sum_{i=1}^{\rm N_{\rm mom}} \frac{\lp
    \mathcal{F}_i^{\rm (k)} -\mathcal{F}_i^{\rm (exp)} \rp^2}{
    \lp \delta\mathcal{F}_i^{\rm (exp)}\rp^2} \, , \quad k=1,\ldots,N_{\rm rep} \, ,
  \ee
  assuming that there are no correlations between the different $N_{\rm mom}$ moments.
  %
  This assumption in general might not be a good approximation, since most lattice
  QCD systematic errors are correlated among the different moments, and should be
  avoided provided the full breakdown of systematic error of each quantity is available.

  
  Once the values of the $\chi^2$ have been evaluated,
  we can compute the corresponding weights for each replica.
  %
  The relation between the weights $w_k$  and the values of
  the $\chi^{2(k)}$ of each replica is the following
  \be
  \omega_k =\frac{\lp \chi^{2(k)} \rp^{(N_{\rm mom}-1)/2}\exp(-\chi^{2(k)}/2)}{
  \sum_{k=1}^{N_{\rm rep}} \lc \lp \chi^{2(k)} \rp^{(N_{\rm mom}-1)/2}\exp(-\chi^{2(k)}/2)\rc} \, ,
  \ee
  where the denominator ensures that the weight admit
  a probabilistic interpretation, that is, $\sum_k w_k=1$.
  %
  These weights represent a measure of the agreement of the individual replicas with the new pseudo-data.
  %
  For instance, replicas which have associated values
  of the moments far from the pseudo-data (within uncertainties) will
  have associated a very large weight, being thus effectively discarded.
\item These weights can be use to now recompute the PDFs, their moments,
  as well as any generic cross-section.
  %
  This procedure emulates the
  impact that adding these lattice-QCD pseudo-data in a complete PDF fit would have.
  %
  For instance, after the reweighting the mean value of
  the PDF moments should be computed as
   \be
  \label{eq:pseudodatadef1}
  \mathcal{F}_i^{\rm (rw)} \equiv \frac{1}{N_{\rm rep}}\sum_{k=1}^{N_{\rm rep}}\omega_k
  \mathcal{F}_i^{\rm (k)} \, , \quad i=1,\ldots,N_{\rm mom} \, ,
  \ee
  and same for the associated uncertainties.
\end{itemize}

One limitation of the reweighting procedure just
outlined is that it can be considered as fully 
 reliable provided only the 
  effective number of replicas $N_{\rm eff}$ that survive the reweighting
  procedure (which is a measure of the amount
  of information left) is not too small.
  %
  This effective number of replicas
    is quantified in terms of the Shannon entropy from information
    theory, namely
    \be
    \label{eq:effnrep}
    N_{\rm eff}\equiv \exp\lc \frac{1}{N_{\rm rep}}\omega_k
    \log \lp N_{\rm rep}/\omega_k\rp\rc \, .
    \ee
    Finding that $N_{\rm eff}\ll N_{\rm rep}$ means that the pseudo-data
    has a large impact on the fit, potentially leading to a large
    reduction of the PDF uncertainties.
    %
    But if the effective number of replicas becomes too
    small, say $N_{\rm eff}\lsim 25$, then the results
    become unreliable since they are affected by large
    statistical fluctuations.

    Therefore, before considering the effects
    of the lattice-QCD pseudo-data at the PDF
    level, we need to ensure that the
    three scenarios defined
    in Table~\ref{tab:scenarios} still lead
    to values of $N_{\rm eff}$ large enough for
    the reweighting procedure to be reliable.
  %
In Table~\ref{tab:neff} we indicate the effective number of replicas
    $N_{\rm eff}$, Eq.~(\ref{eq:effnrep}), remaining when the pseudo-data
    on the PDF moments is included in the global
    fit according to the 
    %
    For completeness, we also indicate here the original number
    of replicas $N_{\rm rep}$ for the original
    PDF sets, NNPDF3.1 NNLO and NNPDFpol1.1 respectively.
    %
    As we can see, there is a marked decrease of $N_{\rm rep}$
    for the three scenarios, indicating that adding the
    PDF moments leads to non-trivial constraints on the global
    fit.
    %
    For instance, in the most optimistic scenario,
    Scenario A, the effective number of replicas is around five (three)
    smaller than the starting number of replicas.

%%%%%%%%%%%%%%%%%%%%%%%%%%%%%%%%%%%%%%%%%%%%%%
\begin{table}[t]
  \centering
  \renewcommand{\arraystretch}{1.3} 
  \begin{tabular}{c|c|c}
    \hline
    &  NNPDF3.1  &  NNPDFpol1.1 \\
    \hline
    \hline
    $N_{\rm rep}$ original   &   1000 &  100   \\
    \hline
     $N_{\rm eff}$ Scenario A    &   740  &  72   \\
     $N_{\rm eff}$ Scenario B    &   750   &   59  \\
     $N_{\rm eff}$ Scenario C   &   510  &   20  \\
    \hline
  \end{tabular}
  \caption{\small The effective number of replicas
    $N_{\rm eff}$, Eq.~(\ref{eq:effnrep}), remaining when the pseudo-data
    on the PDF moments is included in the global
    fit according to the scenarios outlined
    in Table~\ref{tab:scenarios}.
    %
    For completeness, we also indicate the original number
    of replicas $N_{\rm rep}$ for the original
    PDF sets, NNPDF3.1 NNLO and NNPDFpol1.1 respectively.
    \label{tab:neff}
  }
\end{table}
%%%%%%%%%%%%%%%%%%%%%%%%%%%%%%%%%%%%%%%%%%%%%%

\subsubsection*{Impact on unpolarized global fits}
\label{subsec:upolfits}
%
We start by discussing the results of applying the reweighting procedure
outlined above to a representative unpolarized
global fit, in this case the NNPDF3.1 NNLO analysis.
%
To begin with, in Table~\ref{tab:unpolmomentsrw} we summarize
the values of the unpolarized PDF moments
used as pseudo-data $\mathcal{F}_i^{(\rm exp)}$,
as well as the corresponding results
  after the reweighting has been performed for the
three scenarios summarized in 
in Table~\ref{tab:scenarios}.
%
The PDF uncertainties quoted there correspond to 68\% confidence level intervals.
%
We recall that, as explained above, the three scenarios considered here exhibit
uncertainties $\delta_L^{(i)}$ on the lattice-QCD pseudo-data rather smaller
than those of current calculations (see Table~\ref{tab:BMunp}).

%%%%%%%%%%%%%%%%%%%%%%%%%%%%%%%%%%%%%%%%%%%%%%%%%%%%%%%%
\begin{table}[t]
  \centering
  \renewcommand{\arraystretch}{1.3} 
\begin{tabular}{c||c|c|c|c}
  \hline &  Original  & Scen A  &  Scen B  &  Scen C  \\
  \hline
  \hline
  $\la x\ra_{u^+}$     &   $0.348 \pm  0.005$    &  $ 0.349 \pm 0.004$     &
  $ 0.349 \pm 0.004$   &  $ 0.349 \pm 0.003$   \\
  $\la x\ra_{d^+}$     &   $0.196\pm  0.004$     & $0.196 \pm0.004$       &
  $0.196 \pm0.003$ &   $0.196 \pm0.002$ \\
  $\la x\ra_{s^+}$     &   $0.0393 \pm 0.0036$   &  $0.0389\pm 0.0030$   &
 $0.0389\pm 0.0024$   &   $0.0389\pm 0.0014$  \\
  $\la x\ra_{g}$       &   $0.4097\pm 0.0042$    &  $0.4097 \pm 0.0043$    &
   $0.4097 \pm 0.0040$  &    $0.4097 \pm 0.0029$  \\
  $\la x\ra_{u^+-d^+}$  &   $0.1522 \pm 0.0033$   &  $0.1521 \pm 0.0037$   &
   $0.1521 \pm 0.0035$ &    $0.1521 \pm 0.0029$ \\
  \hline
\end{tabular}
\caption{\small Values of the unpolarized PDF moments
  used as pseudo-data, as well as the corresponding results
  after the reweighting has been performed for the
three scenarios summarized in 
in Table~\ref{tab:scenarios}.
%
The PDF uncertainties quoted correspond in all cases to 68\%
CL intervals.
\label{tab:unpolmomentsrw}
}
\end{table}
%%%%%%%%%%%%%%%%%%%%%%%%%%%%%%%%%%%%%%%%%%%%%%%%%%%%%%%%

From Table~\ref{tab:unpolmomentsrw} we see that a significant
reduction of the uncertainties in the unpolarized PDF moments is challenging to achieve
unless we go to the most aggressive scenarios.
%
For instance, in scenario C, which is about the best precision that
can achieved from lattice-QCD, the PDF uncertainties on the second moments
(that is, the momentum fractions) for $u^+,d^+,s^+$ and $g$ decrease by around
between 30\% and 60\%.
%
The most marked decrease is for the strange momentum fraction, since this is the
one affected by the largest PDF errors to begin with.
%
On the other hand, the non-singlet combination $\la x\ra_{u^+-d^+}$ is essentially
unchanged in all three scenarios.
%
Note that the reason why in Table~\ref{tab:unpolmomentsrw} the central values
of the PDF moments are very stable is that by construction we assume that the
lattice-QCD central value corresponds to the PDF one.
%
Of course in a realistic situation, such as that summarized in the
comparisons of Fig.~\ref{fig:Bmoms}, this will not be necessarily the case
and there also the central value of the PDF moments expected to vary
upon reweighting.


%------------------------------------------------------
\begin{figure}[!t]
\centering
\includegraphics[scale=0.45]{plots/xg-unpol-lattice-relerr.pdf}
\includegraphics[scale=0.45]{plots/xup-unpol-lattice-relerr.pdf}
\includegraphics[scale=0.45]{plots/xdp-unpol-lattice-relerr.pdf}
\includegraphics[scale=0.45]{plots/xsp-unpol-lattice-relerr.pdf}
\caption{\small The percentage PDF uncertainty in NNPDF3.1 NNLO
  for the gluon and the $u^+$, $d^+$ and $s^+$ quark PDFs at
  $Q^2=4$ GeV$^2$,
  compared to the results of including the five lattice
  QCD moments as pseudo-data points in the fit using the three
  different scenarios in  Table~\ref{tab:scenarios}.
  %
See text for more details.
}    
\label{fig:impactUnpol}
\end{figure}
%----------------------------------------------------------

The result that, at least with the specific moments that we have used
in this exercise, it will be challenging to reduce the
PDF uncertainties is also illustrated by the comparisons of 
 Fig.~\ref{fig:impactUnpol}, where we show
percentage PDF uncertainty in NNPDF3.1 NNLO
for the gluon and the $u^+$, $d^+$ and $s^+$ quark PDFs in NNPDF3.1 NNLO
at $Q^2=4$ GeV$^2$,
compared to the corresponding results once
the lattice-QCD pseudo-data points of Table~\ref{tab:scenarios} have
been added by reweighting.
  %
In the case of the $u^+,d^+$ and $s^+$, we observe a slight reduction
of the PDF uncertainties, which is more marked as the move
from scenario A to C.
%
For instance, in the latter case the percentage PDF
uncertainty on $u^+$ ($d^+$ and $s^+$) at $x\simeq 0.1$
decreases from 1.8\% to 1.2\% (from 2.2\% to 1.7\% and from 13\% to 10\% respectively).
%
The PDF uncertainties of the gluon PDF, however,
are essentially unchanged even in the most optimistic scenario.

Focusing on the large-$x$ region, where the sensitivity to the
results of the second moments is more marked, in
Fig.~\ref{fig:impactUnpollargex} we show a similar comparison
as in Fig.~\ref{fig:impactUnpo} but now showing the ratio of the
  PDF uncertainties in the fits based on the three scenarios
  as a ratio of the original
  PDF uncertainty of the NNPDF3.1 NNLO set, for the $d^+$
  and $s^+$ total quark PDFs.
  %
  This comparison illustrates better that the relative reduction
  of the PDF uncertainties upon the addition of the lattice-QCD
  pseudo-data is not completely flat, and that it exhibits some
  non-trivial structure.

%------------------------------------------------------
\begin{figure}[!t]
\centering
\includegraphics[scale=0.45]{plots/xdp-unpol-lattice-relerr-largex.pdf}
\includegraphics[scale=0.45]{plots/xsp-unpol-lattice-relerr-largex.pdf}
\caption{\small Same as Fig.~\ref{fig:impactUnpol}, now focusing
  on the large-$x$ region, and showing the ratio of the
  PDF uncertainties in the fits based on the three scenarios
  as a ratio of the original
  PDF uncertainty of the NNPDF3.1 NNLO set, for the $d^+$
  and $s^+$ total quark PDFs.
}    
\label{fig:impactUnpollargex}
\end{figure}
%----------------------------------------------------------

\subsubsection*{Impact on polarized global fits}

Next, we move to discuss the results of applying the
reweighting procedure this time to a representative polarized
global fit, specifically the NNPDFpol1.1 NLO set.
%
First of all, in Table~\ref{tab:polmomentsrw}
we list the values of the polarized PDF moments
  used as pseudo-data, as well as the corresponding results
  after the reweighting has been performed for the
three scenarios summarized in 
in Table~\ref{tab:scenarios}.
%
The PDF uncertainties quoted correspond in all cases to 68\%
CL intervals.
%
As we can see from this comparison, already in the first scenario
(which assumes lattice-QCD pseudo-data with similar uncertainties
as existing calculations) there is a marked impact on the
polarized PDF moments.
%
Specifically, for both $\la 1\ra_{\Delta u^+}$ and $\la 1\ra_{\Delta d^+}$
the PDF uncertainties are roughly halved, with the same
trend but less marked for $\la 1\ra_{\Delta s^+}$.
%
At this level, there is no impact for the non-singlet
combinations $\la 1\ra_{\Delta u^+ - \Delta d^+}$
and $\la x\ra_{\Delta u^--\Delta d^-}$.

%%%%%%%%%%%%%%%%%%%%%%%%%%%%%%%%%%%%%%%%%%%%%%%%%%%%%%%%
\begin{table}[t]
  \centering
  \renewcommand{\arraystretch}{1.3} 
\begin{tabular}{c||c||c|c|c}
  \hline &  Original  & Scen A  &  Scen B  & Scen C  \\
  \hline
  $\la 1\ra_{\Delta u^+}$    &  $+0.788\pm  0.079$   & $+0.798\pm  0.039$     &
  $+0.797\pm  0.023$ &   $+0.790\pm  0.009$ \\
  $\la 1\ra_{\Delta d^+}$   &  $-0.450 \pm 0.083$  &  $-0.450 \pm 0.042$  &
  $-0.456 \pm 0.026$    &  $-0.465 \pm 0.012$   \\
  $\la 1\ra_{\Delta s^+}$    &  $-0.124\pm   0.108 $  & $-0.120\pm   0.070 $  &
  $-0.121\pm   0.076 $    &   $-0.111\pm   0.029 $  \\
  $\la 1\ra_{\Delta u^+ - \Delta d^+}$  & $+1.250 \pm 0.024$   & $+1.250 \pm 0.022$  &
  $+1.253 \pm 0.016$ &    $+1.256 \pm 0.012$  \\
  $\la x\ra_{\Delta u^--\Delta d^-}$     & $+0.196 \pm 0.014$    & $+0.195 \pm 0.014$
  & $+0.196 \pm 0.016$     &  $+0.198 \pm 0.012$    \\
  \hline
\end{tabular}
\caption{\small Same as Table~\ref{tab:unpolmomentsrw}, now for
  the polarized PDF moments computed with NNPDFpol1.1.
  %
  The corresponding impact at the PDF level is shown in
  Fig.~\ref{fig:impactPol}.
\label{tab:polmomentsrw}
}
\end{table}
%%%%%%%%%%%%%%%%%%%%%%%%%%%%%%%%%%%%%%%%%%%%%%%%%%%%%%%%

As we further decrease the assumed uncertainties in the lattice-QCD pseudo-data
for the PDF moments, we observe a consequent reduction of the uncertainties
in the global fit.
%
In the most optimistic scenario C, we find that for both
$\la 1\ra_{\Delta u^+}$ and $\la 1\ra_{\Delta d^+}$ there is an uncertainty
reduction by about an order of magnitude as compared to the current values,
and about a factor 5 for $\la 1\ra_{\Delta s^+}$.
%
Therefore, we demonstrate that future lattice-QCD calculations of
polarized PDF moments can potentially lead to a much more
precise understanding of the spin structure of the proton.
%
The other quark combinations exhibit less sensitivity to the inclusion
of the PDF moments in the global fit, given that to begin with
they are already quite well constrained by available experimental
data.
%
Specifically, the PDF uncertainties for  $\la 1\ra_{\Delta u^+ - \Delta d^+}$
are reduced by a factor 2 in this quite optimistic scenario, while
those of $\la x\ra_{\Delta u^--\Delta d^-}$ are essentially unaffected even
in this case.

Next we move to illustrate the impact of the lattice-QCD pseudo-data
on the polarized PDFs themselves, rather than on their
moments.
%
With this motivation,
in Fig.~\ref{fig:impactPol} we present a 
similar comparison as that of Fig.~\ref{fig:impactUnpol}, now
  showing the absolute PDF uncertainties of the NNPDFpol1.1 fit,
  compared to the corresponding results once the lattice pseudo-data
  on polarized moments in included in the analysis by means of the
  reweighting.
  %
  The reason to show absolute rather than relative uncertainties
  is that, unlike unpolarized PDFs, polarized PDFs often exhibit nodes
  (in particular for strangeness and the gluon) and in the nearby regions
  the concept of relative uncertainty becomes ill-defined.
  
%-------------------------------------------------------------------
\begin{figure}[!t]
\centering
\includegraphics[scale=0.45]{plots/xg-pol-lattice-relerr.pdf}
\includegraphics[scale=0.45]{plots/xup-pol-lattice-relerr.pdf}
\includegraphics[scale=0.45]{plots/xdp-pol-lattice-relerr.pdf}
\includegraphics[scale=0.45]{plots/xsp-pol-lattice-relerr.pdf}
\caption{\small Same as Fig.~\ref{fig:impactUnpol}, now
  showing the absolute PDF uncertainties of the NNPDFpol1.1 fit
   $Q^2=4$ GeV$^2$,
  compared to the corresponding results once the lattice pseudo-data
  on polarized moments in included in the analysis via
  reweighting.
}    
\label{fig:impactPol}
\end{figure}
%---------------------------------------------------------------------

From Fig.~\ref{fig:impactPol} we find that for scenarios
A and B there only a very moderate reduction (or even a slight increase)
of the PDF uncertainties, seemingly at odds with the results
for their moments in Table~\ref{tab:polmomentsrw}.
%
The reason is that the first PDF moments alone provide only limited
information on the shape of the PDFs itself, and therefore in some
cases one will find a large error reduction on the moments (since these
are the fitted quantified) than on the PDFs themselves (which are
only directly constrained).
%
On the other hand, once the lattice-QCD pseudo-data uncertainties
decrease beyond a certain level, the start to impact the PDF shape
as well, as we can see from the results of scenario C.
%
In that case we find that the PDF uncertainties can decreases by up to a factor
2 (3) for $\Delta d^+(x,Q)$ ($\Delta s^+(x,Q)$).
%
Interestingly, we see that the relative reduction of PDF uncertainties is more
or less constant along the whole range of $x$, consistent with the fact that
the lattice-QCD pseudo-data is only sensitive to the total integral
of the $x$ distribution.



% PDF moment analysis using Hessian profiling 
\subsubsection{Hessian profiling analysis}
\label{sec:hessianprofiling}

To complement the results obtained
with the Bayesian reweighting approach,
we use a profiling method, suitable
for Hessian PDF sets, to estimate the effect of including
lattice-QCD pseudo-data into the fit~\cite{Paukkunen:2014zia,Camarda:2015zba}.
%
We choose HERAPDF2.0~\cite{Abramowicz:2015mha}
as a representative set of Hessian PDFs.
%
As in the case of the Bayesian reweighting
exercise presented in the previous section
we consistently use the same lattice-QCD
pseudo-data on PDF moments to estimate the impact on HERAPDF2.0.
%
An additional advantage of the HERAPDF2.0 set is
the use
of a standard tolerance
$\Delta\chi^2=1$ for defining the 68\%-CL PDF
uncertainties,
%%%% RST , which provides a correspondence between the profiling and reweighting methods.
which enables a robust framework for applying the profiling method. 


For Hessian PDF sets, the Hessian profiling method
can be used to both check the compatibility of new data with a given PDF set,
and also  estimate the impact these data will have on the PDFs. 
In the following we describe the essential components of the profiling method, 
and assume  that the  Hessian PDF set uses a tolerance of $\Delta\chi^2=1$, 
which corresponds to 68\%~CL uncertainties,
as is the case with the HERAPDF2.0 set.\footnote{In this exercise
we consider only the {\it experimental} HERAPDF2.0
uncertainties, but not the {\it model} and {\it parametrization}
variations, which are not suited for profiling.}
%
The central values of the considered moments are obtained using the central PDFs and the corresponding
errors are calculated according to:
\begin{equation}
\delta\mathcal{F}_i = \frac{1}{2} \sqrt{\sum_{k}\left(\mathcal{F}_i(f_k^+)-\mathcal{F}_i(f_k^-)\right)^2}\, ,
\quad i=1,\ldots,N_\text{mom} \, ,
\end{equation}
where $k$ labels the number of error PDFs (Hessian eigenvectors)
which have both a positive and negative direction.
%
In the profiling method, one considers a $\chi^2$ function in which the $\chi^2$ of the new
data has been added to the initial $\chi^2_0$, namely
\begin{equation}
\label{eq:newchi2}
\chi^2_{\text{new}} = \chi_0^2 + \sum_{k}^{N_{\text{eig}}} z_k^2
                    + \sum_{i=1}^{N_{\text{data}}}
                      \frac{\lp \mathcal{F}_i - \mathcal{F}_i^{\rm(exp)}\rp^2}
                           {\lp\delta\mathcal{F}_i^{\rm(exp)}\rp^2}\,,
\end{equation}
where $\chi^2_0$ is the value of the $\chi^2$ function in the minimum of the initial PDF set,
$z_k$ are the parameters diagonalizing the Hessian matrix of the initial PDF set,
$N_{\text{eig}}$ is the dimension of the eigenvector space in which initial Hessian errors are defined
(half of the number of error PDFs), $\mathcal{F}_i^{\rm(exp)}$ is the new
\hbox{(pseudo-)data},
and $\mathcal{F}_i$ the corresponding theory prediction.

In the spirit of the Hessian method, the new theory predictions $\mathcal{F}_i$ can be expanded
using a linear approximation:
\begin{equation}
\mathcal{F}_i \simeq \mathcal{F}_i[S_0] + \sum_k \frac{\partial\mathcal{F}_i[S]}{\partial z_k}\bigg|_{S=S_0} z_k \quad
              \simeq \mathcal{F}_i[S_0] + \sum_k D_{ik} w_k \ ,
\end{equation}
where $S_0$ represents the central PDF and we have defined
\be
D_{ik}=\frac{1}{2}(\mathcal{F}_i[S_k^+]-\mathcal{F}_i[S_k^-]) \, ;
\ee
here the  derivative has been approximated by a finite difference of the 
Hessian PDF error sets $S_k^{\pm}$.
%
The new $\chi^2$ of Eq.~\eqref{eq:newchi2} can now be minimized with respect to the parameters $w_k$,
which results in:
\begin{equation}
%\boldsymbol{\vec{w}_{\text{{\bf min}}}} = \boldsymbol{-B^{-1}} \boldsymbol{\vec{a}},
%
w_k^{min}  = \sum_n \ -B_{kn}^{-1} \, a_n \quad ,
\end{equation}
where we have introduced
\begin{equation}
%\begin{split}
B_{kn} = \sum_i \frac{D_{ik}D_{in}}{\lp\delta\mathcal{F}_i^{\rm(exp)}\rp^2} + \delta_{kn},
%\\
\qquad
\qquad
a_k = \sum_i \frac{D_{ik}(\mathcal{F}_i[S_0] - \mathcal{F}_i^{\rm(exp)})}{\lp\delta\mathcal{F}_i^{\rm(exp)}\rp^2} \, . 
%\end{split}
\end{equation}

The key result of the Hessian profiling method
is that now the components of the solution 
%$\boldsymbol{\vec{w}_{\text{{\bf min}}}}$ 
$w_k^{min}$
define a new set
of PDFs representing a global minimum after including the new data:
\begin{equation}
f_{\text{new}} = f_{S_0} + \sum_{k=1}^{N_{\text{eig}}} \frac{f_{S_k^+}-f_{S_k^-}}{2} w_k^{\text{min}} \ .
\end{equation}
%At the same time 
%%$\boldsymbol{\vec{w}_{\text{{\bf min}}}}$ 
%$w_k^{min}$
%also  defines  a penalty term 
%\begin{equation}
%P = \sum_{k=1}^{N_{\text{eig}}} \lp w_k^{\text{min}} \rp^2 \, ,
%\end{equation}
%which can be used to estimate whether the new data is consistent with the initi%al set of PDFs.
%%
%Specifically, a value of
%the penalty term of $P\ll1$ means that the new data is consistent
%with that included in the original fit.
%%
A set of new error PDFs can be also defined; in this case the matrix $B_{kn}$ plays the role of
the Hessian matrix from which the PDF uncertainties
can be obtained. 

We performed this study using the xFitter program~\cite{Alekhin:2014irh}
assuming the same three scenarios for the lattice-QCD pseudo-data as 
in Table~\ref{tab:scenarios}. 
%
The results are shown in Table~\ref{tab:unpolmomentsProf}, where we tabulate 
the uncertainties of the input HERAPDF2.0 PDF in column two and the 
corresponding uncertainties for each scenario in columns three to five. 
%
The analogous results from the reweighting method, applied to the 
NNPDF3.1 data set, were listed in Table~\ref{tab:unpolmomentsrw}.

%-------------------------------------------------------------------------------
\begin{table}[!t]
\centering
\footnotesize
\renewcommand{\arraystretch}{1.3} 
\begin{tabular}{lcccc}
\toprule 
&  Original  & Scenario A  &  Scenario B  &  Scenario C  \\
\midrule
  $\la x\ra_{u^+}$     
&  $0.3720\pm 0.0036$  
&  $0.3720\pm 0.0030$  
&  $0.3720\pm 0.0027$  
&  $0.3720\pm 0.0020$ \\
  $\la x\ra_{d^+}$     
&  $0.1845\pm 0.0053$  
&  $0.1845\pm 0.0028$  
&  $0.1845\pm 0.0023$  
&  $0.1845\pm 0.0015$ \\
  $\la x\ra_{s^+}$     
&  $0.0346\pm 0.0037$  
&  $0.0346\pm 0.0015$  
&  $0.0346\pm 0.0012$  
&  $0.0346\pm 0.0009$ \\
  $\la x\ra_{g}$       
&  $0.4006\pm 0.0078$  
&  $0.4006\pm 0.0042$  
&  $0.4006\pm 0.0035$  
&  $0.4006\pm 0.0024$ \\
  $\la x\ra_{u^+-d^+}$ 
&  $0.1875\pm 0.0074$  
&  $0.1875\pm 0.0045$  
&  $0.1875\pm 0.0039$  
&  $0.1875\pm 0.0027$ \\
\bottomrule
\end{tabular}
\caption{\small Values of the unpolarized PDF moments
  used as lattice-QCD pseudo-data, as well as the corresponding results
  after the profiling  for the
three scenarios summarized in Table~\ref{tab:scenarios}.
%
The HERAPDF2.0 PDFs were used, and the PDF uncertainties quoted correspond in all cases to 68\%~CL intervals.
%
The corresponding results of applying the reweighting method
to NNPDF3.1 were listed in Table~\ref{tab:unpolmomentsrw}.
\label{tab:unpolmomentsProf}
}
\end{table}
%-------------------------------------------------------------------------------

From a comparison of the constraining power of the lattice-QCD pseudo-data  
displayed in Table~\ref{tab:unpolmomentsProf} to Table~\ref{tab:unpolmomentsrw},
we observe a consistent trend between Bayesian reweighting of NNDPF3.1 and 
Hessian profiling of HERAPDF2.0.
%
The PDF uncertainties for $\la x\ra_{d^+}$ ($\la x\ra_{s^+}$
and  $\la x\ra_g$) reduce by a factor of roughly
four (four and three, respectively) compared to the original
HERAPDF2.0 uncertainties.
%
When comparing with Sec.~\ref{sec:projections:rw},
the initial uncertainties of the HERAPDF2.0  analysis 
are affected by the choice of data (DIS data only), and 
the number and form of the parametrization (14 parameter HERAPDF form);
the final uncertainties are determined by the profiling procedure. 
%
In particular the profiling for the HERAPDF2.0 study assigns an effective 
uncertainty on the pseudodata corresponding to $\Delta\chi^2=1$, whereas the 
constraint in the NNPDF study is weaker, as it would be for a PDF set with 
eigenvectors, but which applied a tolerance criterion. 
%
While these initial studies are instructive, 
further comparisons of these analyses would be valuable. 

In Fig.~\ref{fig:pdfsProf} we present a comparison of the
$u^+$, $d^+$, $g$, and $s^+$ PDFs at the scale of $Q^2=4\text{ GeV}^2$
between the original  HERAPDF2.0 set and the results of the profiling
exercise for Scenarios~A, B and C.
%
Only the {\it experimental} PDF uncertainties are shown in this comparison,
but not the {\it model} and {\it parametrization} variations.
%
The corresponding results based on the reweighting
of NNPDF3.1 were shown in Figs.~\ref{fig:impactUnpol}
and~\ref{fig:impactUnpollargex}.

%-------------------------------------------------------------------------------
\begin{figure}[!t]
\centering
\includegraphics[width=0.45\textwidth]{plots/ratio_uPubar_Q2.pdf}
\includegraphics[width=0.45\textwidth]{plots/ratio_dPdbar_Q2.pdf}\\
\includegraphics[width=0.45\textwidth]{plots/ratio_g_Q2.pdf}
\includegraphics[width=0.45\textwidth]{plots/ratio_sPsbar_Q2.pdf}
\caption{\small Comparison
of the $u^+$, $d^+$, $g$, and $s^+$ PDFs at the scale of $Q^2=4\text{ GeV}^2$
between the original  HERAPDF2.0 set and the results of the profiling
exercise accounting for the constraints of
the lattice-QCD moments
pseudo-data in Scenarios~A, B and C.
%
Only the {\it experimental} PDF uncertainties are shown,
but not the {\it model} and {\it parametrization} variations.
}
\label{fig:pdfsProf}
\end{figure}
%-------------------------------------------------------------------------------

From Fig.~\ref{fig:pdfsProf} we see that, as expected, the impact of the 
lattice pseudo-data is greatest in the medium and large-$x$ regions.
%
The precise impact on the PDFs is rather
similar for the three scenarios, with the most optimistic
Scenario~C leading to the largest reduction in uncertainties.
%
The quark flavor combinations that are most affected by the
lattice-QCD pseudo-data are the $d^{+}$ and $s^{+}$ PDFs,
and, to a lesser extent, the gluon PDF.
%
The improvement in the PDF uncertainties for $d^{+}$ and $s^{+}$
occurs because the DIS data
used in HERAPDF2.0 include only limited constraints
on quark flavor separation, and, for these PDFs, the lattice-QCD 
pseudo-data add important new information.



% x-space analysis using reweighting
\subsection{Impact of $x$-space lattice QCD calculations of PDFs}
\label{sec:projectionsxspace}

In this section we perform an initial exploration of the
potential impact that future lattice QCD calculations
of $x$-space PDFs can have on the global analysis.
%
Specifically, we will focus on the isotriplet
combination $x u-x d$ ($x\Delta u - x\Delta d$), which is the
one for which more progress has been performed recently.
%
Following the same Bayesian reweighting procedure that
has been employed to quantify the impact of the PDF moments,
we have generated pseudo-data for the isotriplet
combinations
\be
\label{eq:isotriplet_unpol}
u(x_i,Q^2)-d(x_i,Q^2) \, \quad{\rm and} \, \quad
\bar{u}(x_i,Q^2)-\bar{d}(x_i,Q^2) \, , \quad i=1,\ldots,N_x \, ,
\ee
for the unpolarized case, and for
\be
\label{eq:isotriplet_pol}
\Delta u(x_i,Q^2)-\Delta d(x_i,Q^2) \, \quad{\rm and} \, \quad
\Delta\bar{u}(x_i,Q^2)-\Delta\bar{d}(x_i,Q^2) \, , \quad i=1,\ldots,N_x \, ,
\ee
for the polarized case, with $N_x$ being the number of points
in $x$-space that are being sampled and we
take $Q^2=4$ GeV$^2$, consistently with the choice used
in the case of the PDF moments.

We consider here three scenarios, denoted by scenario D, E, and F,
for the total uncertainty that will be assigned to
the lattice-QCD calculations of the specific quark
combinations listed in Eqns.~(\ref{eq:isotriplet_unpol})
and~(\ref{eq:isotriplet_pol}).
%
First of all we indicate the values of $\left\{ x_i \right\}$
that have been chosen for this exercise.
%
With the motivation that the lattice-QCD computation is expected to have the
smallest systematic uncertainties at large $x$,
we have chosen the $N_x=5$ points to be
\be
x_i = 0.70\, ,0.75,\, 0.80,\, 0.85, \, 0.90 \, .
\ee
For each scenario we assume that that the total error of the lattice calculation
is the same for each value of $\left\{ x_i \right\}$, and
moreover we neglect the possible correlations between neighboring $x$-points.
%
We have assumed then that we have $\delta_{L}=12\%, 6\%$ and 3\% for scenarios
D, E, and F, respectively.
%
Note that here we assume the same values of $\delta_{L}$ for the polarized
and unpolarized cases, as well as for both the quark
and the antiquark isotriplet combinations Eqns.~(\ref{eq:isotriplet_unpol})
and~(\ref{eq:isotriplet_pol}).

The results of this exercise are summarized
in Fig.~\ref{fig:impactxspace}, where we show the
ratio of PDF uncertainties to the original
  NNPDF3.1 (NNPDFpol1.1) in the fits where lattice-QCD pseudo-data
  on $x$-space PDFs has been added to the global unpolarized
  (polarized) analysis.
  %
  Specifically, we show the impact on the PDF uncertainties
  in $\bar{u}$ and $\bar{d}$ at large-$x$ in the upper
  plots, with the corresponding comparison for $\Delta\bar{u}$
  and $\Delta\bar{d}$ in the lower plots.
  %
  From this comparison, we find that
  indeed these lattice-QCD calculations can reduce significantly
  the errors of the large-$x$ unpolarized and polarized
  antiquarks.
  %
  Taking into account that the PDF uncertainties on the large-$x$
  antiquarks to begin with are rather large, and that they
  enter a number of important BSM search channels
  (such as for instance for new heavy gauge bosons $W'$ and $Z'$)
  is clear that such calculations would have direct
  phenomenological implications.

%-------------------------------------------------------------------
\begin{figure}[!t]
\centering
\includegraphics[scale=0.45]{plots/xubar-unpol-lattice-relerr-xdata-xspace.pdf}
\includegraphics[scale=0.45]{plots/xdbar-unpol-lattice-relerr-xdata-xspace.pdf}
\includegraphics[scale=0.45]{plots/xubar-pol-lattice-relerr-xdata-xspace.pdf}
\includegraphics[scale=0.45]{plots/xdbar-pol-lattice-relerr-xdata-xspace.pdf}
\caption{\small The ratio of PDF uncertainties to the original
  NNPDF3.1 (NNPDFpol1.1) in the fits where lattice-QCD pseudo-data
  on $x$-space PDFs has been added to the global unpolarized
  (polarized) analysis.
  %
  Specifically, we show the impact on the PDF uncertainties
  in $\bar{u}$ and $\bar{d}$ at large-$x$ in the upper
  plots, with the corresponding comparison for $\Delta\bar{u}$
  and $\Delta\bar{d}$ in the lower plots.
}    
\label{fig:impactxspace}
\end{figure}
%---------------------------------------------------------------------

Specifically, from Fig.~\ref{fig:impactxspace} we find that
in unpolarized case the large-$x$ PDF uncertainties can be reduced
down to $60\%$ of its original value.
%
We also find that there are no big differences between the three
scenarios, suggesting that a direct lattice-QCD calculation
of $x u-x d$ does not need to reach uncertainties
at the few-percent level in order to impact positively
the PDFs.
%
In the polarized case, we find a similar result but the reduction
of PDF uncertainties can be more marked.
%
For instance in the case of $\Delta \bar{d}$, at $x\simeq 0.8$
the resulting PDF uncertainty from scenario C is less than 50\%
of the original one.
%
Note that the in a Monte Carlo approach such as NNPDF, the
PDF uncertainties fluctuate themselves, specially at low-scales,
explaining the wiggles in these plots.

Of course the results of this exercise have to be interpreted
with some care.
%
First of all, the results depend sensitively on the specific values of
$\left\{ x_i \right\}$
that we have assumed for the lattice-QCD calculation,
as well as of the associated uncertainties.
%
Also, the quantitative results would vary if a different input PDF set
was used, for example the HERAPDF2.0 set that was used for
the Hessian profiling exercise of Sect.~\ref{sec:hessianprofiling}.
%
But even accounting for these caveats, is clear that a direct
computation of the isotriplet combination $x u-x d$ on the lattice
has the potential to constrain the large-$x$ PDFs in
a more significant way that the corresponding PDF moments calculation,
specially in the unpolarized case.
%
And given the importance of large-$x$ antiquarks for LHC phenomenology,
pursuing this approach is thus most promising.


% General discussion
\subsection{Discussion}

We conclude this section with a brief discussion of the main lessons that
can be learned from this exercise, which provides the first quantitative estimate
of the impact of present and future lattice-QCD calculations of PDF moments
and $x$-space PDFs, for both polarised and unpolarised PDFs.

First, we have demonstrated that in the polarised case,
even with current uncertainties, lattice-QCD calculations of
selected PDF moments can impose sizable constraints on several
important polarised quark combinations.
%
This suggests that global polarised PDF analyses should consider
including existing lattice-QCD calculations in their fits to constrain some
of the least known quark combinations, such as the total strangeness.
%
The situation is rather different in the unpolarised case,
where a reduction of the current lattice-QCD uncertainties by a factor of between five 
and ten seems to be required to influence global fits.
%
This difference arises because unpolarised PDFs are known with much higher precision than polarised
PDFs, thanks to the much wider amount of experimental data sensitive to unpolarised PDFs,
including the constraints from recent high-precision measurements at the
LHC.
%
Thus, in addition to the differences highlighted  in Fig.~\ref{fig:Bmomsunp},
much more precise lattice-QCD calculations than in the polarised case 
need to be used to be competitive with current PDF fits.


Second, lattice-QCD calculations of the quark isotriplet combinations
$xu-xd$ and $x\bar{u}-x\bar{d}$ would be instrumental in constraining
quark PDFs at large-$x$.
%
Even a calculation with $\delta_L\simeq 10\%$ uncertainties at large-$x$ would
start to provide useful constraints on global fits.
%
Moreover, we find that, in the unpolarised case, the information on the
PDFs that could be derived from a direct $x$-space calculation
from lattice-QCD is clearly superior to the information that can be obtained
from PDF moments alone, at least for the subset of PDFs and moments used in the present
exercise.

The profiling studies presented in this section could be extended in
a number of directions.
%
In the polarised case, one could include the current lattice-QCD
values of the moments listed in Tab.~\ref{tab:BMpol} in global analyses: 
indeed, we have demonstrated that at the current level of uncertainties one 
expects to find some non-trivial constraints.
%
In this respect, a crucial topic to investigate is the compatibility 
(or lack thereof) of the existing lattice-QCD numbers compared to constraints 
from experimental data.
%
For both unpolarised and polarised PDFs, it would be interesting to include the 
effects of other moments and flavour combinations.
%
Higher moments, in particular, typically probe regions of higher $x$, compared
to lower moments, and in the large-$x$ regions uncertainties in the phenomenological PDFs are
more marked.
%
One could also consider the effects of the quark combinations for which $x$-space
calculations might be available, for example those related to the proton strangeness.
%
Finally, a more refined analysis should include the theoretical correlations
expected in lattice-QCD calculations, for instance, in the case of $x$-space calculations,
one expects neighbouring points in $x$ to be highly correlated.

%%%%%%%%%%%%%%%%%%%%%%%%%%%%%%%%%%%%%%%%%%%%%%%%%%%




%%%%%%%%%%%%%%%%%%%%%%%%%%%%%%%%%%%%%%%%%%%%%%%%%%%%%%%%%%%%%%%%%%%%%%%%%%%%%%%%
\section{Outlook}
\label{sec:outlook}
%%%%%%%%%%%%%%%%%%%%%%%%%%%%%%%%%%%%%%%%%%%%%%%%%%%%%%%%%%%%%%%%%%%%%%%%%%%%%%%%

The quark and gluon structure of the proton is a fascinating topic
at the cross-roads between high-energy and nuclear physics, with important
applications also for astroparticle physics.
%
Motivated by the recent breakthroughs in lattice QCD calculations of
PDF-related quantities, in particular PDF moments and their $x$-space
shape, this white paper presents a detailed snapshot of what we have learned so far
about the internal proton structure both from the global PDF fitting and from
the lattice QCD frameworks.
%
One of the main goals of this document is to clearly establish a common language
between the two communities, in order to facilitate the cross-talk between them,
and ensure that information can flow effortlessly between global PDF fitters
and lattice QCD practitioners.

Perhaps the main outcome of this initial effort is the first ever systematic
comparison between state-of-the-art lattice QCD calculations of PDF moments with
the corresponding results from the global PDF fits, both
for the unpolarized and for the polarized case.
%
Moreover, we have also presented numbers for higher moments, which can be used
in the future as reference to validate upcoming lattice calculations.

The studies presented in this white-paper can be extended in a number of directions.
%
First of all, here we have restricted our benchmark comparison only to the
first moment of polarized and unpolarized
PDFs.
%
Future work should extend this comparison to also the second and third moments,
as well as perhaps to include as well other non-perturbative objects
such as the transversity and transverse-momentum dependent PDFs.

x-space PDF comparisons, for different values of x

The ultimate goal of this efforts is to reach a situation where
lattice QCD calculations can be used to provide novel genuine inputs
to unpolarized and polarized PDF fits.
%
Here we have estimated .... using a number of assumptions ...


%%%%%%%%%%%%%%%%%%%%%%%%%%%%%%%%%%%%%%%%%%%%%%%%%%%%%%%%%%%%%%%%%%%%%%%%%%%%%%%%
\subsection*{Acknowledgments}

We are very grateful to Jacqueline Gills for the flawless organization
of the workshop at Balliol College and to Michelle Bosher for
invaluable help and support with the workshop logistics.
%
This workshop was partly supported by the European Research Council via
the Starting Grant {\it ``PDF4BSM - Parton Distributions in the
  Higgs Boson Era}.
%
E.~R.~N. acknowledges financial support from the
UK STFC via the Rutherford Grant ST/M003787/1.
%
J.~R. is funded by the ERC via the Starting Grant ``PDF4BSM'' and by the
Dutch Organization for Scientific Research (NWO).


%%%%%%%%%%%%%%%%%%%%%%%%%%%%%%%%%%%%%%%%%%%%%%%%%%%%%%%%%%%%%%%%%%%%%%%%%%%%%%%%
\subsection*{Acknowledgments}

We are grateful to Jacqueline Gills and Michelle Bosher for their help in the
organization of the workshop.
%
The workshop was partly supported by the European Research Council (ERC) via 
the Starting Grant {\it PDF4BSM - Parton Distributions in theHiggs Boson Era}.
%
We also thank the Department of Energy's (DoE) Institute of Nuclear Theory 
(INT) at the University of Washington in Seattle for partial support during 
the completion of this work.
%
This work was also partially funded by the U.S. DoE contract 
No.~DE-AC05-06OR23177, under which Jefferson Science Associates, 
LLC operates Jefferson Lab, and by the DOE contract No.~DE-SC008791. 
% 
H.-W.L. is supported by the U.S. National Science Foundation (NSF) under grant 
PHY 1653405; E.R.N. by the U.K. Science and Technology Facilities Council 
(STFC) via the Rutherford Grant ST/M003787/1; J.~R. by the ERC via the Starting 
Grant {\it PDF4BSM - Parton Distributions in theHiggs Boson Era} and by the 
Dutch Organization for Scientific Research (NWO); F.I.O. by the U.S. DoE under 
Grant No.~DE-SC0010129; K.O. by Jefferson Science Associates, LLC under U.S. 
DoE Contract DE-AC05-06OR23177, by U.S. DoE grant DE-FG02-04ER41302, 
and  by STFC consolidated grant ST/P000681/1.
%
A.A. acknowledges support by the U.S. DoE under contract No.~DE-SC008791;
A.B. and G.B. by the ERC via the Consolidator Grant {\it 3DSPIN - Mapping the
proton in 3D};
J.-W.C. partly by the Ministry of Science and Technology, Taiwan,
under Grant No. 105-2112-M-002-017-MY3 and the Kenda Foundation;
M.C. by the U.S. DoE, Office of Science, Office of Nuclear Physics, within the 
framework of the TMD Topical Collaboration, as well as by the NSF under Grant 
No.~PHY-1714407;
L.H.L. and R.S.T. by the STFC via grant awards ST/L000377/1 and ST/P000274/1;
M.E. by the U.S. DoE, Office of Science, Office of Nuclear Physics through 
grant DE-FG02-96ER40965 as well as through the TMD Topical Collaboration;
P.M.N. by the U.S. DoE under Grant No.~DE-SC0010129; 
C.J.M. by the U.S. DoE through Grant No.~DE-FG02-00ER41132;
T.I. by Science and Technology Commission of Shanghai Municipality 
(Grants No.~16DZ2260200) and in part by the DoE, Laboratory Directed Research 
and Development (LDRD) funding of BNL, under contract No.~DE-EC0012704.





\appendix

\section{Definition of the PDF moments}
\label{app:notation}

In this appendix, we summarize the conventions adopted in this paper to denote 
the moments of relevant unpolarized and polarized PDF combinations.
%
We focus on the quantities which can be presently computed in lattice QCD,
although those used for benchmarks in Sect.~\ref{sec:benchmarking} are only
a subset of them.
%
In the equations below, we use the shorthand notation
\begin{equation}
q^\pm \equiv q\pm\bar{q}\, 
\quad\text{ and }\quad
\Delta q^\pm \equiv \Delta q\pm\Delta\bar{q}\, 
,\qquad q=u,d,s,c \,,
\end{equation}
%
for unpolarized and polarized PDFs respectively.
%
We identify $\mu$ with the QCD factorization scale and $Q$ with the 
characteristic scale of a given hard-scattering process.
%
The use of the following notation is strongly recommended for any comparison 
between lattice-QCD computations and global-fit determinations of 
PDF moments.

\begin{itemize}

\item Unpolarized moments.

\begin{enumerate}

\item The first moment of the total $u^+-d^+$ PDF combination
\begin{equation}
\left.\langle x\rangle_{u^+-d^+}(\mu^2)\right|_{\mu^2=Q^2}
=
\int_0^1 dx\, x\left\{u(x,Q^2)+\bar{u}(x,Q^2)-d(x,Q^2)-\bar{d}(x,Q^2)\right\} \, .
\label{eq:unpfmumdtot}
\end{equation}

\item The second moment of the valence $u^--d^-$ PDF combination
\begin{equation}
\left.\langle x^2\rangle_{u^--d^-}(\mu^2)\right|_{\mu^2=Q^2}
=
\int_0^1 dx\, x^2\left\{u(x,Q^2)-\bar{u}(x,Q^2)-d(x,Q^2)+\bar{d}(x,Q^2)\right\} \, .
\label{eq:unpsmumdval}  
\end{equation}

\item The first moment of the individual quark $q^+$ total PDF combination
\begin{equation}
\left.\langle x\rangle_{q^+=u^+,d^+,s^+,c^+}(\mu^2)\right|_{\mu^2=Q^2}
=
\int_0^1 dx\, x\left\{q(x,Q^2)+\bar{q}(x,Q^2)\right\} \, .
\label{eq:unpfmiqtot}
\end{equation}

\item The second moment of the individual quark $q^-$ valence PDF combination
\begin{equation}
\left.\langle x^2\rangle_{q^-=u^-,d^-,s^-,c^-}(\mu^2)\right|_{\mu^2=Q^2}
=
\int_0^1 dx\, x^2\left\{q(x,Q^2)-\bar{q}(x,Q^2)\right\} \, .
\label{eq:unpsmiqval}
\end{equation}

\item The first moment of the gluon PDF
\begin{equation}
\left.\langle x \rangle_g(\mu^2)\right|_{\mu^2=Q^2}
=
\int_0^1 dx\, x\, g(x,Q^2) \, .
\label{eq:unpfmg}
\end{equation}

\end{enumerate}

\item Polarized moments.

\begin{enumerate}

\item The zeroth moment of the total $u^+-d^+$ PDF combination
\begin{equation}
\left.\langle 1 \rangle_{\Delta u^+-\Delta d^+}(\mu^2)\right|_{\mu^2=Q^2}
=
\int_0^1 dx \left\{\Delta u(x,Q^2)+\Delta\bar{u}(x,Q^2)
-\Delta d(x,Q^2)-\Delta\bar{d}(x,Q^2)\right\} \, .
\label{eq:polzmumdtot}
\end{equation}

\item The first moment of the valence $u^--d^-$ PDF combination
\begin{equation}
\left.\langle x\rangle_{\Delta u^--\Delta d^-}(\mu^2)\right|_{\mu^2=Q^2}
=
\int_0^1 dx\, x\left\{\Delta u(x,Q^2)-\Delta\bar{u}(x,Q^2)-\Delta d(x,Q^2)+\Delta \bar{d}(x,Q^2)\right\}
\label{eq:polfmumdval}  
\end{equation}

\item The zeroth moment of the individual quark $q^+$ total PDF combination
\begin{equation}
\left.\langle 1\rangle_{q^+=\Delta u^+,\Delta d^+,\Delta s^+,\Delta c^+}(\mu^2)\right|_{\mu^2=Q^2}
=
\int_0^1 dx \left\{\Delta q(x,Q^2)+\Delta\bar{q}(x,Q^2)\right\} \, .
\label{eq:polzmiqtot}
\end{equation}

\item The first moment of the individual quark $q^-$ valence PDF combination
\begin{equation}
\left.\langle x\rangle_{\Delta q^-=\Delta u^-,\Delta d^-,\Delta s^-,\Delta c^-}(\mu^2)\right|_{\mu^2=Q^2}
=
\int_0^1 dx\, x\left\{\Delta q(x,Q^2)-\Delta\bar{q}(x,Q^2)\right\} \, .
\label{eq:polfmiqval}
\end{equation}

\end{enumerate}

\end{itemize}

Some of these moments have a direct physical interpretation, see 
Sect.~\ref{Sec:IntroPDFs}.
%
For instance, Eq.~\eqref{eq:unpfmiqtot} and Eq.~\eqref{eq:polzmiqtot}
correspond respectively to the proton's momentum and spin fractions carried
by a given quark flavor (and its corresponding antiquark) at the scale 
$\mu^2=Q^2$.
%
Higher moments and/or moments of other flavor combinations are readily
computable from any phenomenological PDF set.
%
We do not consider them though, as the corresponding lattice-QCD
computations are outside the current reach.


\section{PDF moments from lattice QCD}
\label{sec:LQCDtables}

In this appendix, we present an extensive summarise available results for the moments of both 
unpolarised and polarised PDFs from lattice QCD, including those results 
that do not include a chiral extrapolation to the physical pion mass and 
quenched results. 
%
We also provide bibliographic tables with full
details of the available lattice-QCD results.
%
The discussion on the various sources of systematic uncertainty which
affect each specific calculation can be found in Sects.~\ref{Sec:IntroLQCD}
and~\ref{subsubsec:BClQCD}.
%
See appendix~\ref{app:notation} for the definitions and conventions used
for the various PDF moments considered here.

The results for the lattice-QCD calculations of PDF moments that are summarized
in this appendix are the following:
\begin{itemize}

  \item In Table~\ref{tab:unpolLQCDstatus1B}
we summarize the status of current  of the
first moments of unpolarised PDFs $\la x\ra_{u^+-d^+}$, $\la x\ra_{q^+}$,
and $\la x\ra_g$.

\item In Table~\ref{tab:unpolLQCDstatus2B} we show
  the status of current lattice-QCD calculations of the second moments of unpolarised PDFs
  $\la x^2\ra_{u^--d^-}$, $\la x^2\ra_{u^-}$,
and $\la x^2\ra_{d^-}$.

\item In Table~\ref{tab:polLQCDstatus1B} we show
  a summary of the current status of lattice-QCD calculations of zeroth moments of
  longitudinally polarised PDFs,
  $\la 1\ra_{\Delta u^+,\Delta d^+}$ and $\la 1\ra_{\Delta s^+}$.

\item In Table~\ref{tab:polLQCDstatus2B} we display a summary of the current status of lattice-QCD calculations of the first moments of longitudinally polarised PDFs,
  $\la x\ra_{\Delta u^-,\Delta d^-}$, $\la x\ra_{\Delta u^-}$, and  $\la x\ra_{\Delta d^-}$.

\item In Table~\ref{tab:latticebibfirst} we show the full details of lattice-QCD calculations
  of the axial coupling $g_A\equiv\langle 1\rangle_{\Delta u^+-\Delta d^+}$.
  %
  In this comparison we do not include
quenched results, perturbatively renormalized results, and conference proceedings.

\item In Table~\ref{tablenonisovectorquarkspins} we show
  the full details of lattice-QCD calculations of the non-isovector quark spins.
  %
  Note that
  earlier results are summarized in Ref.~\cite{Liu:1995kb}.

\item In Table~\ref{tab:unpolarizedisotriplet} we provide the full details of
  lattice-QCD calculations of the $\langle x\rangle_{u^+-d^+}$.
  %
  In this comparison we
  omit quenched and non-renormalized results.

\item In Table~\ref{tab:nonisovectormomfrac}
  we summarize
  the full details of lattice-QCD calculations of the non-isovector momentum fractions.

\item In Table~\ref{tab:nonisopolcase}
  we show the full details of lattice-QCD calculations of $\langle x\rangle_{\Delta u^--\Delta d^-}$.

\item Finally, in  Table~\ref{tab:latticebiblast} we
  give full details of lattice-QCD calculations of higher moments of unpolarised
  and polarised PDFs.

\end{itemize}

A representative subset of the results contained in
Tables~\ref{tab:unpolLQCDstatus1B} to~\ref{tab:latticebiblast}
is compared graphically in  
Fig.~\ref{fig:latt_res}.
%
Specifically, in this figure we show from left to right and from top to bottom
an overview of 
  the results for $\la x\ra_{u^+-d^+}$, $\la x\ra_{q^+}$, $\la x\ra_{g}$,
  $\la 1 \ra_{\Delta s^+}$, $\la 1\ra_{\Delta q^+}$, and finally
  $\la x\ra_{\Delta u^- - \Delta d^-}$.
  %
  See the corresponding entries of each table for
  the details of the differences and similarities for each
  of the results that enter this comparison.

%%%%%%%%%%%%%%%%%%%%%%%%%%%%%%%%%%%%%%%%%%%%%%%%%%%%%%%%%%%%%%%%%%%%%
\begin{figure}[t]
\begin{center}
\centerline{
\subfloat[]{\includegraphics[width=0.49\textwidth]{plots/x_world.pdf}}
\subfloat[]{\includegraphics[width=0.47\textwidth]{plots/xq_world_ud.pdf}}
}
\centerline{
\subfloat[]{\includegraphics[width=0.49\textwidth]{plots/xg_world.pdf}}
\subfloat[]{\includegraphics[width=0.49\textwidth]{plots/ga_world_strange.pdf}}
}
\centerline{
\subfloat[]{\includegraphics[width=0.47\textwidth]{plots/ga_world_ud.pdf}}
\subfloat[]{\includegraphics[width=0.49\textwidth]{plots/xdeltaq_isovector.pdf}}
}
\end{center}
\caption{\small Comparison of lattice-QCD results for
  the  zeroth and first moments
  of unpolarised and polarised PDFs,
  see Tables~\ref{tab:unpolLQCDstatus1B} to~\ref{tab:latticebiblast}
  for more details.
  %
  From left to right and from top to bottom, we show
  the results for $\la x\ra_{u^+-d^+}$, $\la x\ra_{q^+}$, $\la x\ra_{g}$,
  $\la 1 \ra_{\Delta s^+}$, $\la 1\ra_{\Delta q^+}$, and finally
  $\la x\ra_{\Delta u^- - \Delta d^-}$.
}
\label{fig:latt_res}
\end{figure}
%%%%%%%%%%%%%%%%%%%%%%%%%%%%%%%%%%%%%%%%%%%%%%%%%%%%%%%%%%%%%%%%%%%%%


%-------------------------------------------------------------------------------
\begin{table}[t]
  \small
\renewcommand{\arraystretch}{1.2} 
\centering 
\makebox[\textwidth]{ %
  \footnotesize
\begin{tabular}{lllccccccl}\\[1cm]
  Ref. & $N_f$ & Status & 
\hspace{0.15cm}\begin{rotate}{70}{discretisation}\end{rotate}\hspace{-0.15cm} &
\hspace{0.15cm}\begin{rotate}{70}{quark mass}\end{rotate}\hspace{-0.15cm} &
\hspace{0.15cm}\begin{rotate}{70}{finite volume}\end{rotate}\hspace{-0.15cm} &
\hspace{0.15cm}\begin{rotate}{70}{renormalisation}\end{rotate}\hspace{-0.15cm} &
\hspace{0.15cm}\begin{rotate}{70}{excited states}\end{rotate}\hspace{-0.15cm}&
 &  \\
  \hline
\multicolumn{10}{c}{$\langle x\rangle_{u^+-d^+}$}\\\hline
  ETMC 15 \cite{Abdel-Rehim:2015owa} &
  2+1+1 & P & 0.06,0.08~fm  & ---  & \rsquare,\bstar & \bstar,\bstar & \rsquare,\bstar  &  & Fig.~\ref{fig:latt_res}~(a) \\
  ETMC 15 \cite{Abdel-Rehim:2015owa} &
  2 & P & 0.06-0.09~fm  & ---  & \bcirc & \bstar & \rsquare  &  & Fig.~\ref{fig:latt_res}~(a) \\
  RQCD 14 \cite{Bali:2014gha} &
  2 &  P & 0.06-0.08~fm & --- & \bcirc & \bstar  & \bcirc  &  & Fig.~\ref{fig:latt_res}~(a) \\
\hline
\multicolumn{10}{c}{$\langle x\rangle_{q^+}$}\\\hline
  ETMC 13 \cite{Abdel-Rehim:2013wlz} &
  2+1+1 & P &  0.08~fm  & --- &\bstar  & \bstar  &   \bstar  & $\&$ & Fig~\ref{fig:latt_res}~(b) \\
    $\chi$QCD 13 \cite{Deka:2013zha} &   0 & P &  \rsquare  & \rsquare &\rsquare  & \bcirc  &  \rsquare& $\dagger\ddag$ & $\langle x\rangle_{u^+}=0.451(37)$,\\
    $\chi$QCD 13 \cite{Deka:2013zha} &   0 & P &  \rsquare  & \rsquare &\rsquare  & \bcirc  &  \rsquare& $\dagger\ddag$ & $\langle x\rangle_{d^+}=0.188(20)$,\\
    $\chi$QCD 13 \cite{Deka:2013zha} & 
  0 & P & \rsquare  & \rsquare &\rsquare  & \bcirc  &  \rsquare & $\dagger\ddag$ & 
$\langle x\rangle_{s^+}=0.024(6)$\\
  \hline
\multicolumn{10}{c}{$\langle x\rangle_{g}$}\\\hline
  ETMC 13 \cite{Alexandrou:2016ekb} &
  2+1+1 & P &  0.08~fm  & --- &\bstar  & \bcirc  &   \bstar  &  & Fig.~\ref{fig:latt_res}~(c) \\
    $\chi$QCD 13 \cite{Deka:2013zha} &
  0 & P &\rsquare  & \rsquare &\rsquare  & \bcirc   &   \bstar & $\dagger\dagger\ddag$ & 0.334(55) \\
  QCDSF 12 \cite{Horsley:2012pz} &
  0 & P &\rsquare  & \rsquare & \bstar  & \bstar  &   - & $\dagger$ & 0.43(7)(5) \\\hline
\end{tabular}
} % End makebox
\begin{minipage}{\linewidth}
{\footnotesize 
\begin{itemize}
\item[$\&$] Non-singlet renormalisation is applied.
\item[$\dagger$] The lightest $m_\pi$ has $Lm_\pi\ge 4.0$, however, $L\sim 1.6$~fm.
\item[$\dagger\dagger$] The lightest $m_\pi$ has $Lm_\pi\ge 4.0$, however, $L\sim 1.6$~fm.
\item[$\ddag$] The connected contribution is only evaluated at one $t_{sep}$.
\end{itemize}
}
\end{minipage}
\caption{\small Status of current lattice-QCD calculations of the first moments of unpolarised PDFs.}
\label{tab:unpolLQCDstatus1B}
\end{table}
%%%%%%%%%%%%%%%%%%%%%%%%%%%%%%%%%%%%%%%%%%%%%%%%%%%%%%%%%%%%%%%%%%%%%


%%%%%%%%%%%%%%%%%%%%%%%%%%%%%%%%%%%%%%%%%%%%%%%%%%%%%%%%%%%%%%%%%%%%%
\begin{table}[t]
\renewcommand{\arraystretch}{1.2} 
\centering % See https://tex.stackexchange.com/questions/23650/when-should-we-use-begincenter-instead-of-centering
\makebox[\textwidth]{ % Centre table on page, even though it is a little wide
  \footnotesize
\begin{tabular}{lllccccccl}\\[1cm]
  Ref. & $N_f$ & Status & 
\hspace{0.15cm}\begin{rotate}{70}{discretisation}\end{rotate}\hspace{-0.15cm} &
\hspace{0.15cm}\begin{rotate}{70}{quark mass}\end{rotate}\hspace{-0.15cm} &
\hspace{0.15cm}\begin{rotate}{70}{finite volume}\end{rotate}\hspace{-0.15cm} &
\hspace{0.15cm}\begin{rotate}{70}{renormalisation}\end{rotate}\hspace{-0.15cm} &
\hspace{0.15cm}\begin{rotate}{70}{excited states}\end{rotate}\hspace{-0.15cm}&
 &  \\
  \hline
    \multicolumn{10}{c}{$\langle x^2\rangle_{u^--d^-}$}\\\hline
   LHPC and
  SESAM 02 \cite{Dolgov:2002zm} &  2 & P & \rsquare & \rsquare &  \rsquare & \bcirc & \rsquare &  &  0.145(69)\\
  QCDSF 05 \cite{Gockeler:2004wp} &
0 & P & \rsquare  & \rsquare &\rsquare  & \bstar  &   \rsquare &  & 0.083(17)\\
  LHPC and
  SESAM 02 \cite{Dolgov:2002zm} &
 0 & P & \rsquare & \rsquare &  \rsquare & \bcirc & \rsquare &  & 0.090(68)\\
\hline
  \multicolumn{10}{c}{$\langle x^2\rangle_{u^-}$}\\\hline
 $\chi$QCD 09 \cite{Deka:2008xr} &
  0 & P &\rsquare  & \rsquare &\rsquare  & \bcirc  &   \rsquare & $\ast$  & $0.117(18)$ \\
  \hline
  \multicolumn{10}{c}{$\langle x^2\rangle_{d^-}$}\\\hline
  $\chi$QCD 09 \cite{Deka:2008xr} &
  0 & P &\rsquare  & \rsquare &\rsquare  & \bcirc  &   \rsquare & $\ast$  & $0.052(9)$  \\
  \hline
\end{tabular}
} % End makebox
\begin{minipage}{0.94\linewidth}
{\footnotesize 
\begin{itemize}
\item[$\ast$] Only the connected contribution is included.
\end{itemize}
}
\end{minipage}
\caption{\small Status of current lattice-QCD calculations of the second moments of unpolarised PDFs.}
\label{tab:unpolLQCDstatus2B} 
\end{table}
%%%%%%%%%%%%%%%%%%%%%%%%%%%%%%%%%%%%%%%%%%%%%%%%%%%%%%%%%%%%%%%%%%%%%


%%%%%%%%%%%%%%%%%%%%%%%%%%%%%%%%%%%%%%%%%%%%%%%%%%%%%%%%%%%%%%%%%%%%%
\begin{table}[t]
\renewcommand{\arraystretch}{1.2} 
\centering
\makebox[\textwidth]{ % Centre table on page, even though it is a little wide
  \footnotesize
\begin{tabular}{lllccccccl}\\[1cm]
  Ref. & $N_f$ & Status & 
\hspace{0.15cm}\begin{rotate}{70}{discretisation}\end{rotate}\hspace{-0.15cm} &
\hspace{0.15cm}\begin{rotate}{70}{quark mass}\end{rotate}\hspace{-0.15cm} &
\hspace{0.15cm}\begin{rotate}{70}{finite volume}\end{rotate}\hspace{-0.15cm} &
\hspace{0.15cm}\begin{rotate}{70}{renormalisation}\end{rotate}\hspace{-0.15cm} &
\hspace{0.15cm}\begin{rotate}{70}{excited states}\end{rotate}\hspace{-0.15cm}&
 &  \\
\hline
\multicolumn{10}{c}{$\langle 1\rangle_{\Delta u^+, \Delta d^+}$}\\\hline
  ETMC 13 \cite{Abdel-Rehim:2013wlz} &
  2+1+1 & P &  0.08~fm  & --- &\bstar  & \bstar  &   \bstar  & $\&$ & Fig.~\ref{fig:latt_res}~(e)\\
  LHPC  17 \cite{Green:2017keo} &
  2+1 &  P & 0.11~fm & --- & \bstar  & \bstar  &  \bstar &  &  Fig.~\ref{fig:latt_res}~(e)\\
  QCDSF/CSSM 15 \cite{Chambers:2015bka}  &
  2+1 &  P & 0.07~fm  & --- & \bstar & \bstar  & \bstar   & $\diamond$  &  Fig.~\ref{fig:latt_res}~(e) \\
  QCDSF 11 \cite{QCDSF:2011aa}  &
  2 &  P & 0.07~fm  & --- & \bstar & \bstar  & \rsquare   & $\$$  &  Fig.~\ref{fig:latt_res}~(e)\\\hline
\multicolumn{10}{c}{$\langle 1\rangle_{\Delta s^+}$}\\\hline
  ETMC 13 \cite{Abdel-Rehim:2013wlz} &
  2+1+1 & P &  0.08~fm  & --- &\bstar  & \bstar  &   \bstar  & $\&$ & Fig.~\ref{fig:latt_res}~(d)\\
  LHPC  17 \cite{Green:2017keo} &
  2+1 &  P & 0.11~fm & --- & \bstar  & \bstar  &  \bstar &  & Fig.~\ref{fig:latt_res}~(d) \\
  QCDSF/CSSM 15 \cite{Chambers:2015bka}  &
  2+1 &  P & 0.07~fm  & --- & \bstar & \bstar  & \bstar   & $\diamond$  &  Fig.~\ref{fig:latt_res}~(d) \\
  QCDSF 11 \cite{QCDSF:2011aa}  &
  2 &  P & 0.07~fm  & --- & \bstar & \bstar  & \bstar   & $\$$  &  Fig.~\ref{fig:latt_res}~(d) \\
    \hline
\end{tabular}
} % End makebox
\begin{minipage}{\linewidth}
{\footnotesize 
\begin{itemize}
\item[$\&$] Non-singlet renormalisation is applied.
\item[$\diamond$] Feynman-Hellmann approach is employed.
\item[$\$$]
The mixing with $\langle x\rangle_{q^+}$ is small so only the excited state analysis for $\langle x\rangle_g$ is considered.
\end{itemize}
}
\end{minipage}
\caption{\small Summary of the current status of lattice-QCD calculations of zeroth moments of longitudinally polarised PDFs.}
\label{tab:polLQCDstatus1B}
\end{table}
%%%%%%%%%%%%%%%%%%%%%%%%%%%%%%%%%%%%%%%%%%%%%%%%%%%%%%%%%%%%%%%%%%%%%


%%%%%%%%%%%%%%%%%%%%%%%%%%%%%%%%%%%%%%%%%%%%%%%%%%%%%%%%%%%%%%%%%%%%%
\begin{table}[t]
\renewcommand{\arraystretch}{1.2} 
\centering
\makebox[\textwidth]{ % Centre table on page, even though it is a little wide
  \footnotesize
\begin{tabular}{lllccccccl}\\[1cm]
  Ref. & $N_f$ & Status & 
\hspace{0.15cm}\begin{rotate}{70}{discretisation}\end{rotate}\hspace{-0.15cm} &
\hspace{0.15cm}\begin{rotate}{70}{quark mass}\end{rotate}\hspace{-0.15cm} &
\hspace{0.15cm}\begin{rotate}{70}{finite volume}\end{rotate}\hspace{-0.15cm} &
\hspace{0.15cm}\begin{rotate}{70}{renormalisation}\end{rotate}\hspace{-0.15cm} &
\hspace{0.15cm}\begin{rotate}{70}{excited states}\end{rotate}\hspace{-0.15cm}&
 &  \\
\hline
\multicolumn{10}{c}{$\langle x\rangle_{\Delta u^--\Delta d^-}$}\\\hline
  ETMC 15 \cite{Abdel-Rehim:2015owa} &
  2+1+1 & P &  0.06,0.08~fm  & --- & \rsquare,\bstar & \bstar,\bstar & \rsquare,\bstar  &   & Fig.~\ref{fig:latt_res}~(f) \\
  ETMC 15 \cite{Abdel-Rehim:2015owa} &
  2 & P & 0.06-0.09~fm  & --- & \bcirc & \bstar & \rsquare &  & Fig.~\ref{fig:latt_res}~(f) \\
  \hline
\multicolumn{10}{c}{$\langle x\rangle_{\Delta u^-}$}\\\hline
  ETMC 13 \cite{Abdel-Rehim:2013wlz} &
  2+1+1 & P & 0.08~fm  & $373$~MeV &\bstar  & \bstar  &   \bstar 
& $\&$ &  $0.214(11)$\\
  \hline
\multicolumn{10}{c}{$\langle x\rangle_{\Delta d^-}$}\\\hline
  ETMC 13 \cite{Abdel-Rehim:2013wlz} &
  2+1+1 & P & 0.08~fm  & $373$~MeV &\bstar  & \bstar  &   \bstar 
& $\&$ &  $0.083(11)$\\
\hline
\end{tabular}
} % End makebox
\begin{minipage}{\linewidth}
{\footnotesize 
\begin{itemize}
\item[$\&$] Non-singlet renormalisation is applied.
\end{itemize}
}
\end{minipage}
\caption{\small Summary of the current status of lattice-QCD calculations of the first moments of longitudinally polarised PDFs.}
\label{tab:polLQCDstatus2B}
\end{table}
%%%%%%%%%%%%%%%%%%%%%%%%%%%%%%%%%%%%%%%%%%%%%%%%%%%%%%%%%%%%%%%%%%%%%




%%%%%%%%%%%%%%%%%%%%%%%%%%%%%%%%%%%%%%%%%%%%%%%%%%%%%%%%%%%%%%%%%%%%%
\begin{table}[t]
\renewcommand{\arraystretch}{1.2} 
\centering
\makebox[\textwidth]{ % Centre table on page, even though it is a little wide
  \footnotesize
\begin{tabular}{llllll}
  Ref. & Sea quarks & Valence quarks & Renormalisation & $N_{\Delta t}$ & $m_\pi$ (MeV)\\
\hline

  Mainz '17b* \cite{Capitani:2017qpc} &
  2 clover & clover & Schrödinger functional & 4--6 & 193--473\\

  ETMC '17* \cite{Alexandrou:2017hac} &
  2 clover-TM & clover-TM & Rome-Southampton & 3 & 131\\

  CalLat '17* \cite{Berkowitz:2017gql} &
  2+1+1 staggered & domain wall & Rome-Southampton & all & 131--313 \\

  LHPC '17 \cite{Green:2017keo} &
  2+1 clover & clover & Rome-Southampton & 5 & 317 \\

  NME '17 \cite{Yoon:2016jzj} &
  2+1 clover & clover & Rome-Southampton & 1**,4--5 & 172--285 \\

  Mainz '17a \cite{vonHippel:2016wid} &
  2 clover & clover & Schrödinger functional & 4--6 & 193--456\\

  Dragos et al.\ '16 \cite{Dragos:2016rtx} &
  3 clover & clover & Rome-Southampton & 1,2**,5 & 460 \\

  PNDME '16 \cite{Bhattacharya:2016zcn} &
  2+1+1 staggered & clover & Rome-Southampton & 3--5 & 128--319\\

  $\chi$QCD '16 \cite{Yang:2015zja} &
  2+1 domain wall & overlap & $Z_A/Z_V=1$ & 3 & 330 \\

  ETMC '15b \cite{Abdel-Rehim:2015owa} &
    2 clover-TM & \multicolumn{4}{l}{superseded by ETMC '17} \\
  & 2 twisted mass & twisted mass & Rome-Southampton & 1 & 262--470\\
  & 2+1+1 twisted mass & twisted mass & & 1, 4 & 213, 373\\

  RQCD '15 \cite{Bali:2014nma} &
  2 clover & clover & Rome-Southampton & 1--5 & 150--490\\

  PNDME '14 \cite{Bhattacharya:2013ehc} &
  \multicolumn{5}{l}{superseded by PNDME '16} \\

  QCDSF '14 \cite{Horsley:2013ayv} &
  2 clover & clover & $g_A/f_\pi \times f_\pi^\text{phys}$ & 1,5 & 157--1591 \\

  LHPC '14 \cite{Green:2012ud} &
  2+1 clover & clover & Rome-Southampton & 3 & 149--356\\

  ETMC '13 \cite{Alexandrou:2013joa} &
  \multicolumn{5}{l}{superseded by ETMC '15b} \\

  CSSM '13 \cite{Owen:2012ts} &
  2+1 clover & clover & Schrödinger functional & 1**$^\dagger$ & 290 \\

  Mainz '12 \cite{Capitani:2012gj} &
  \multicolumn{5}{l}{superseded by Mainz '17b} \\

  ETMC '11 \cite{Alexandrou:2011nr} &
  \multicolumn{5}{l}{superseded by ETMC '15b} \\

  LHPC '10 \cite{Bratt:2010jn} &
  2+1 staggered & domain wall & $A_\mu/\mathcal{A}_\mu$ ratio & 1--2 & 293--758 \\

  RBC+UKQCD '09 \cite{Yamazaki:2009zq} &
  2+1 domain wall & domain wall & $Z_A/Z_V=1$ & 1 & 329--668 \\

  RBC+UKQCD '08 \cite{Yamazaki:2008py} &
  \multicolumn{5}{l}{superseded by RBC+UKQCD '09} \\

  RBC '08 \cite{Lin:2008uz} &
  2 domain wall & domain wall & $Z_A/Z_V=1$ & 1--2 & 493--695 \\

  LHPC '08 \cite{Hagler:2007xi} &
  \multicolumn{5}{l}{superseded by LHPC '10} \\

  Alexandrou et al.\ '07 \cite{Alexandrou:2007xj} &
  2 Wilson & Wilson & Rome-Southampton & 1 & 384--691 \\

  LHPC '06 \cite{Edwards:2005ym} &
  \multicolumn{5}{l}{superseded by LHPC '10} \\

  QCDSF '06 \cite{Khan:2006de} &
  \multicolumn{5}{l}{superseded by QCDSF '14?} \\
\hline
\end{tabular}
} % End makebox
\begin{minipage}{\linewidth}
{\footnotesize 
\begin{itemize}
\item[$*$] Preprint.
\item[$**$] A variationally optimized interpolating operator is employed.
\item[$\dagger$] Carried out with a single fixed source-operator separation and all source-sink separations.
\end{itemize}
}
\end{minipage}
\caption{\small Full details of lattice-QCD calculations of the axial coupling $g_A\equiv\langle 1\rangle_{\Delta u^+-\Delta d^+}$.
We omit quenched results, perturbatively renormalized results, and conference proceedings.}
\label{tab:latticebibfirst}
\end{table}
%%%%%%%%%%%%%%%%%%%%%%%%%%%%%%%%%%%%%%%%%%%%%%%%%%%%%%%%%%%%%%%%%%%%%


%%%%%%%%%%%%%%%%%%%%%%%%%%%%%%%%%%%%%%%%%%%%%%%%%%%%%%%%%%%%%%%%%%%%%
\begin{table}[t]
\renewcommand{\arraystretch}{1.2} 
\centering
\makebox[\textwidth]{ % Centre table on page, even though it is a little wide
  \footnotesize
  \begin{tabular}{lllll}
  Ref. & Flavours & Sea quarks & Valence quarks & Renormalisation \\
\hline

  ETMC '17* \cite{Alexandrou:2017hac} &
  $u,d,s,c$ & 2 clover-TM & clover-TM & Rome-Southampton \\

  $\chi$QCD '17* \cite{Gong:2015iir} &
  $s,c$ & 2+1 domain wall & overlap & single-flavour anomalous WI \\

  LHPC '17 \cite{Green:2017keo} &
  $u,d,s$ & 2+1 clover & clover & Rome-Southampton \\

  CSSM and &
  $u+d+s$ &
  2+1, 3 clover & clover & Rome-Southampton \\
  QCDSF/UKQCD '15 \cite{Chambers:2015bka} & conn.\ / disc. & & & \\

  ETMC '14 \cite{Abdel-Rehim:2013wlz} &
  $u+d,s$ & 2+1+1 twisted mass & twisted mass & non-singlet Rome-Southampton\\

  Engelhardt '12 \cite{Engelhardt:2012gd} &
  $s$ & 2+1 staggered & domain wall & non-singlet $A_\mu/\mathcal{A}_\mu$ ratio \\

  QCDSF '12 \cite{QCDSF:2011aa} &
  $u,d,s$ & 2 clover & clover & non-singlet Rome-Southampton \\
  & & & &+ two-loop singlet-nonsinglet\\

  Babich et al.\ '10 \cite{Babich:2010at} &
  $s$ & 2 aniso-clover & aniso-clover & none \\

  SESAM '99 \cite{Gusken:1999as} &
  $u,d,s$ & 2 Wilson & Wilson & one loop \\

  $\chi$QCD '95 \cite{Dong:1995rx} &
  $u,d,s$ & quenched & Wilson & one loop \\

  Fukugita et al.\ '95 \cite{Fukugita:1994fh} &
  $u,d,s$ & quenched & Wilson & one loop \\

  Gupta and Mandula '94 \cite{Gupta:1994qw} &
  singlet** & quenched & Wilson & anomalous Ward identity \\

  Allés et al.\ '94 \cite{Alles:1994ss} &
  singlet** & quenched & Wilson & anomalous Ward identity \\

  Altmeyer et al.\ '94 \cite{Altmeyer:1992nt} &
  singlet & 4 staggered & staggered & anomalous Ward identity \\

  Mandula and Ogilvie '93 \cite{Mandula:1992bc} &
  $s$** & quenched & Wilson & none \\
  \hline
\end{tabular}
} % End makebox
\begin{minipage}{\linewidth}
{\footnotesize 
\begin{itemize}
\item[$*$] Preprint.
\item[$**$] No signal available.
\end{itemize}
}
\end{minipage}
\caption{\small Full details of lattice-QCD calculations of the non-isovector quark spins.
  Earlier results are summarized in Ref.~\cite{Liu:1995kb}.
\label{tablenonisovectorquarkspins}
}
\end{table}
%%%%%%%%%%%%%%%%%%%%%%%%%%%%%%%%%%%%%%%%%%%%%%%%%%%%%%%%%%%%%%%%%%%%%


%%%%%%%%%%%%%%%%%%%%%%%%%%%%%%%%%%%%%%%%%%%%%%%%%%%%%%%%%%%%%%%%%%%%%
\begin{table}[t]
\renewcommand{\arraystretch}{1.2} 
\centering
\makebox[\textwidth]{ % Centre table on page, even though it is a little wide
  \footnotesize
\begin{tabular}{llllll}
  Ref. & Sea quarks & Valence quarks & Renormalisation & $N_{\Delta t}$ & $m_\pi$ (MeV)\\
\hline

  $\chi$QCD '16 \cite{Yang:2015zja} &
  2+1 domain wall & overlap & one loop & 3 & 330 \\

  ETMC '15b \cite{Abdel-Rehim:2015owa} &
    2 clover-TM & clover-TM & Rome-Southampton & 3 & 131 \\
  & 2 twisted mass & twisted mass & & 1 & 262--470\\
  & 2+1+1 twisted mass & twisted mass & & 1, 5 & 213, 373\\

  ETMC '15a \cite{Alexandrou:2015qia} &
  2+1+1 twisted mass & twisted mass & Rome-Southampton & 1 & 302--466 \\

  RQCD '14 \cite{Bali:2014gha} &
  2 clover & clover & Rome-Southampton & 1--6 & 149--490 \\

  LHPC '14 \cite{Green:2012ud} &
  2+1 clover & clover & Rome-Southampton & 3 & 149--356\\

  ETMC '13 \cite{Alexandrou:2013joa} &
  \multicolumn{5}{l}{superseded by ETMC '15b} \\

  RQCD '12 \cite{Bali:2012av} &
  \multicolumn{5}{l}{superseded by RQCD '14} \\

  ETMC '11 \cite{Alexandrou:2011nr} &
  \multicolumn{5}{l}{superseded by ETMC '15b} \\

  QCDSF/UKQCD '11* \cite{Pleiter:2011gw} &
  2 clover & clover & Rome-Southampton & 1 & 170--670 \\

  LHPC '11* \cite{Syritsyn:2011vk} &
  2+1 domain wall & domain wall & Rome-Southampton & 1 & 297--403 \\

  LHPC '10 \cite{Bratt:2010jn} &
  2+1 staggered & domain wall & one-loop $Z_\mathcal{O}/Z_A$ & 1--2 & 293--758 \\

  RBC-UKQCD '10 \cite{Aoki:2010xg} &
  2+1 domain wall & domain wall & Rome-Southampton & 1 & 329--668 \\

  RBC '08 \cite{Lin:2008uz} &
  2 domain wall & domain wall & Rome-Southampton & 1--2 & 493--695 \\

  LHPC '08 \cite{Hagler:2007xi} &
  \multicolumn{5}{l}{superseded by LHPC '10} \\

  LHPC and &
  2 Wilson & Wilson & one loop & 1--2 & ?\\
  SESAM '02 \cite{Dolgov:2002zm} &
  and quenched & & & \\
\hline
\end{tabular}
} % End makebox
\begin{minipage}{\linewidth}
{\footnotesize 
\begin{itemize}
\item[$*$] Conference proceedings.
\end{itemize}
}
\end{minipage}
\caption{\small Full details of lattice-QCD calculations of the $\langle x\rangle_{u^+-d^+}$. We omit quenched and non-renormalized results.
\label{tab:unpolarizedisotriplet}
}
\end{table}
%%%%%%%%%%%%%%%%%%%%%%%%%%%%%%%%%%%%%%%%%%%%%%%%%%%%%%%%%%%%%%%%%%%%%


%%%%%%%%%%%%%%%%%%%%%%%%%%%%%%%%%%%%%%%%%%%%%%%%%%%%%%%%%%%%%%%%%%%%%
\begin{table}[t]
\renewcommand{\arraystretch}{1.2} 
\centering
\makebox[\textwidth]{ % Centre table on page, even though it is a little wide
  \footnotesize
\begin{tabular}{lllll}
  Ref. & Flavours & Sea quarks & Valence quarks & Renormalisation \\
\hline

  ETMC '16* \cite{Alexandrou:2016ekb} & $g$
    & 2+1+1 twisted mass & twisted mass & one loop \\
  & & 2 clover-TM & clover-TM & \\

  ETMC '15a \cite{Alexandrou:2015qia} &
  $u+d-2s$ &  2+1+1 twisted mass & twisted mass & Rome-Southampton \\

  ETMC '14 \cite{Abdel-Rehim:2013wlz} &
  $u+d$ & 2+1+1 twisted mass & twisted mass & non-singlet Rome-Southampton\\

  $\chi$QCD '15 \cite{Deka:2013zha} &
  $u,d,s,g$ & quenched & Wilson & sum rule + one-loop \\

  QCDSF-UKQCD '12 \cite{Horsley:2012pz} &
  $g$ & quenched & clover & non-perturbative \\
\hline
\end{tabular}
} % End makebox
\begin{minipage}{\linewidth}
{\footnotesize 
\begin{itemize}
\item[$*$] Preprint.
\end{itemize}
}
\end{minipage}
\caption{\small
  Full details of lattice-QCD calculations of the non-isovector momentum fractions.
\label{tab:nonisovectormomfrac}}
\end{table}
%%%%%%%%%%%%%%%%%%%%%%%%%%%%%%%%%%%%%%%%%%%%%%%%%%%%%%%%%%%%%%%%%%%%%



%%%%%%%%%%%%%%%%%%%%%%%%%%%%%%%%%%%%%%%%%%%%%%%%%%%%%%%%%%%%%%%%%%%%%
\begin{table}[t]
\renewcommand{\arraystretch}{1.2} 
\centering
\makebox[\textwidth]{ % Centre table on page, even though it is a little wide
  \footnotesize
\begin{tabular}{lllll}
  Ref. & Sea quarks & Valence quarks & Renormalisation & $N_{\Delta t}$ \\
  \hline

  ETMC '15b \cite{Abdel-Rehim:2015owa} &
    2 clover-TM & clover-TM & Rome-Southampton & 3 \\
  & 2 twisted mass & twisted mass & & 1 \\
  & 2+1+1 twisted mass & twisted mass & & 1 or 4 \\

  ETMC '13 \cite{Alexandrou:2013joa} &
  \multicolumn{4}{l}{superseded by ETMC '15b} \\

  ETMC '11 \cite{Alexandrou:2011nr} &
  \multicolumn{4}{l}{superseded by ETMC '15b} \\

  QCDSF/UKQCD '11* \cite{Pleiter:2011gw} &
  2 clover & clover & Rome-Southampton & 1 \\

  LHPC '10 \cite{Bratt:2010jn} &
  2+1 staggered & domain wall & one-loop $Z_\mathcal{O}/Z_A$ & 1--2 \\

  RBC-UKQCD '10 \cite{Aoki:2010xg} &
  2+1 domain wall & domain wall & Rome-Southampton & 1 \\

  RBC '08 \cite{Lin:2008uz} &
  2 domain wall & domain wall & Rome-Southampton & 1--2 \\

  LHPC '08 \cite{Hagler:2007xi} &
  \multicolumn{4}{l}{superseded by LHPC '10} \\

  LHPC and &
  2 Wilson & Wilson & one loop & 1--2 \\
  SESAM '02 \cite{Dolgov:2002zm} &
  and quenched & & & \\

  QCDSF '97 \cite{Gockeler:1997zr} &
  quenched & Wilson & one loop & 1 \\
\hline
\end{tabular}
} % End makebox
\begin{minipage}{\linewidth}
{\footnotesize 
\begin{itemize}
\item[$*$] Conference proceedings.
\end{itemize}
}
\end{minipage}
\caption{\small Full details of lattice-QCD calculations of $\langle x\rangle_{\Delta u^--\Delta d^-}$.
\label{tab:nonisopolcase}
}
\end{table}
%%%%%%%%%%%%%%%%%%%%%%%%%%%%%%%%%%%%%%%%%%%%%%%%%%%%%%%%%%%%%%%%%%%%%


%%%%%%%%%%%%%%%%%%%%%%%%%%%%%%%%%%%%%%%%%%%%%%%%%%%%%%%%%%%%%%%%%%%%%
\begin{table}[t]
\renewcommand{\arraystretch}{1.2} 
\centering
\makebox[\textwidth]{ % Centre table on page, even though it is a little wide
  \footnotesize
\begin{tabular}{lllll}
  Ref. & Observables & Sea quarks & Valence quarks & Renormalisation \\
\hline

  LHPC '10$^{\dagger}$
  \cite{Bratt:2010jn} &
  $\langle x\rangle_{u^+-d^+}$,
  $\langle x^2\rangle_{u^--d^-}$, &
  2+1 staggered &
  domain wall &
  one-loop $Z_\mathcal{O}/Z_A$ \\
  &   $g_A$,
  $\langle x\rangle_{\Delta u^--\Delta d^-}$,
  $\langle x^2\rangle_{\Delta u^+-\Delta d^+}$ & & & \\

  $\chi$QCD '09 \cite{Deka:2008xr} &
  $\langle x\rangle_{u^+,d^+,s^+}$ (superseded by $\chi$QCD '15), &
  quenched &
  Wilson &
  one loop \\
  & $\langle x^2 \rangle_{u^-,d^-,s^-}$ & & &\\

  LHPC '08 \cite{Hagler:2007xi} &
  superseded by LHPC '10 & & &\\

  QCDSF '05c \cite{Gockeler:2005vw} &
  $\langle x^2\rangle_{\Delta u^+-\Delta d^+}$ &
  2 clover & clover & Rome-Southampton \\

  QCDSF '05b \cite{Gockeler:2004wp} &
  $\langle x\rangle_{u^+-d^+}$,
  $\langle x^2\rangle_{u^--d^-}$,
  $\langle x^3\rangle_{u^+-d^+}$ &
  quenched &
  clover &
  Rome-Southampton \\

  QCDSF '05a* \cite{Gockeler:2004vx} &
  $\langle x\rangle_{u^+-d^+}$,
  $\langle x^2\rangle_{u^--d^-}$,
  $\langle x^3\rangle_{u^+-d^+}$ &
  2 clover & clover & one loop \\

  LHPC and &
  $\langle x\rangle_{u^+-d^+}$,
  $\langle x^2\rangle_{u^--d^-}$,
  $\langle x^3\rangle_{u^+-d^+}$, &
  2 Wilson & Wilson & one loop \\
  SESAM '02 \cite{Dolgov:2002zm} &
  $g_A$,
  $\langle x\rangle_{\Delta u^--\Delta d^-}$,
  $\langle x^2\rangle_{\Delta u^+-\Delta d^+}$ &
  and quenched & & \\

  QCDSF '01 \cite{Gockeler:2000ja} &
  $\langle x^2\rangle_{\Delta u^+-\Delta d^+}$ &
  quenched & clover & Rome-Southampton \\

  QCDSF '96 \cite{Gockeler:1995wg} &
  $\langle x\rangle_{u^+-d^+}$,
  $\langle x^2\rangle_{u^--d^-}$,
  $\langle x^3\rangle_{u^+-d^+}$, &
  quenched & Wilson & one loop \\
  & $g_A$,
  $\langle x^2\rangle_{\Delta u^+-\Delta d^+}$ & & &\\
\hline
\end{tabular}
} % End makebox
\begin{minipage}{\linewidth}
{\footnotesize 
\begin{itemize}
\item[$*$] Conference proceedings.
\item[$\dagger$] The moment $\langle x^2\rangle_{u-d}=A_{30}^{u-d}(0)$ is plotted in the ratio of form factors $A_{30}(t)/A_{10}(t)$, where we can use $A_{10}^{u-d}(0)=1$. The moment $\langle x^2\rangle_{\Delta u-\Delta d}=\tilde A_{30}^{u-d}(0)$ is plotted in the ratio of form factors $\tilde A_{30}(t)/\tilde A_{10}(t)$ and we can use $\tilde A_{10}^{u-d}(0)=g_A$.
\end{itemize}
}
\end{minipage}
\caption{\small Full details of lattice-QCD calculations of higher moments of unpolarised and polarised PDFs. }
\label{tab:latticebiblast}
\end{table}
%%%%%%%%%%%%%%%%%%%%%%%%%%%%%%%%%%%%%%%%%%%%%%%%%%%%%%%%%%%%%%%%%%%%%

\clearpage

\clearpage % this \clearpage prints all the Appendix B tables  so the C text is near the C tables. 

\section{PDF fit results for higher moments }
\label{app:Hmoms}

In this appendix, we summarise the current values of the higher moments of unpolarised 
and polarised PDFs, not previously listed in Sect.~\ref{subsubsec:GPDFfits}.
%
Even though these moments were not selected for the benchmark exercise performed in
Sect.~\ref{subsec:BN}, we find it useful to collect them here for future reference.

\begin{itemize}

\item In Table~\ref{tab:unpHmoms} we display the values of the second moments of the 
unpolarised quark valence distributions $\langle x^2\rangle_{u^-}$, 
$\langle x^2\rangle_{d^-}$, $\langle x^2\rangle_{s^-}$ and $\langle x^2\rangle_{u^--d^-}$.

\item In Table~\ref{tab:polHmoms} we display the values of the first moments of the 
polarised quark valence distributions $\langle x^2\rangle_{u^-}$, 
$\langle x^2\rangle_{d^-}$ and $\langle x^2\rangle_{s^-}$.

\end{itemize}
%
All values in Tables~\ref{tab:unpHmoms}-\ref{tab:polHmoms} are computed at 
$\mu^2=Q^2=4$ GeV$^2$.
%
For the description of the correspnding PDF sets and their uncertainties, see
Sects.~\ref{sec:unpPDFs}-\ref{sec:polPDFs}-\ref{subsubsec:GPDFfits}.

%------------------------------------------------------------------------------------------
\begin{table}[!t]
\centering
\small
\begin{tabular}{lccccccc}
\toprule
Mom. & NNPDF3.1 & CT14 & MMHT14 & ABMP16 & CJ15 & HERAPDF2.0 & PDF4LHC15 \\
\midrule
$\langle x^2\rangle_{u^-}$ 
& 0.0851(27) & 0.0841(13) & 0.0831(14)    & 0.0845(8) & 0.0853(3) & 0.0886(29) & 0.0833(15) \\
$\langle x^2\rangle_{d^-}$
& 0.0284(27) & 0.0295(10) & 0.0305(11)    & 0.0267(7) & 0.0305(3) & 0.0334(18) & 0.0305(17) \\ 
$\langle x^2\rangle_{s^-}$
& 0.0010(31) & ---        & 0.0006(8)\ \, & ---       & ---       & ---        & 0.0011(11) \\
$\langle x^2\rangle_{u^--d^-}$
& 0.0571(27) & 0.0546(19) & 0.0526(19)    & 0.0578(9) & 0.0548(3) & 0.0553(17) & 0.0530(24) \\
\bottomrule
\end{tabular}
\caption{\small Second moments of unpolarised valence PDFs from phenomenological PDF fits
at $\mu=Q=2$ GeV.}
\label{tab:unpHmoms}
\end{table}
%------------------------------------------------------------------------------------------

%------------------------------------------------------------------------------------------$\langle x^2\rangle_{u^-}$
\begin{table}[!t]
\centering
\footnotesize
\begin{tabular}{lcc}
\toprule
Mom. & NNPDFpol1.1 & DSSV08 \\
\midrule
$\langle x\rangle_{\Delta u^-}$ 
& \ 0.1493(85) & \ 0.1624(56) \\
$\langle x\rangle_{\Delta d^-}$ 
&  -0.0468(79) &  -0.0410(55) \\
\bottomrule
\end{tabular}
\caption{\small Second moments of polarised valence PDFs from phenomenological PDF fits
at $\mu=Q=2$ GeV.}
\label{tab:polHmoms}
\end{table}

\clearpage % this prints out all tables so they are not mixed in with refs

\newpage % Start bib on new page
\bibliography{PDFLattice2017}

\end{document}
