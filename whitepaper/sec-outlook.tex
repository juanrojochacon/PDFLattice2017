%%%%%%%%%%%%%%%%%%%%%%%%%%%%%%%%%%%%%%%%%%%%%%%%%%%%%%%%%%%%%%%%%%%%%%%%%%%%%%%%
\section{Outlook}
\label{sec:outlook}
%%%%%%%%%%%%%%%%%%%%%%%%%%%%%%%%%%%%%%%%%%%%%%%%%%%%%%%%%%%%%%%%%%%%%%%%%%%%%%%%

Parton distributions lie at the crossroads of high-energy, 
hadron and nuclear physics, with applications also in astroparticle physics.
%
In this white paper, we have reviewed our current knowledge of PDFs, as they
are determined from global phenomenological fits and lattice-QCD computations. 

We have first established a common language
between the two communities, in order to facilitate interactions between them.
%
We have then presented the first systematic comparison between state-of-the-art 
lattice-QCD calculations of PDF moments and the corresponding results from 
global PDF fits, both for unpolarized and polarized PDFs.
%
We have also provided benchmark numbers from global fits for the higher moments 
not used in our benchmark comparison with lattice QCD. 
%
These higher moments can be used to validate future lattice-QCD calculations.
%
We have finally quantified the impact of lattice-QCD calculations in the 
global fits assuming the current and potential future scenarios.
%
In the case of unpolarized PDFs, we have demonstrated that a reduction 
of the uncertainties of current lattice-QCD calculations is needed 
prior to provide any impact on global-PDF fits.
%
In the case of polarized PDFs, we have shown that current lattice-QCD
calculation can already provide a useful input into global-PDF analyses.
%
The studies presented in this white paper can be extended in a number of 
directions.

First, we have restricted our benchmark comparison only to the lowest moments 
of polarized and unpolarized PDFs, whose various sources of systematic 
uncertainties have been computed with the greatest control.
%
Future work should extend this comparison to higher PDF moments.
%
These could have some impact on PDF fits, provided the precision and accuracy
of lattice-QCD calculations keeps improving.

Second, a similar benchmark exercise between global fit results and 
lattice-QCD calculations should be performed at the level of
$x-$space calculations.
%
It will be important to compare in detail the available lattice-QCD results 
with state-of-the-art global fits, to first validate the former and
thereby demonstrate to what extent lattice-QCD calculations of $x$-space PDFs 
can contribute to global fits.

Third, it should be possible to assess the impact of lattice-QCD 
calculations on other non-perturbative objects
determined from global analyses of experimental data.
%
Examples of these include the transversity, transverse-momentum dependent 
PDFs (TMDs), generalized PDFs (GPDs), or collinear PDFs for hadrons 
other than nucleons.
%
All these quantities are known with much less precision than unpolarized
and polarized PDFs, given that the corresponding experimental information
is rather scarce. 
%
In this case, lattice-QCD calculations could have the potential
to provide new information, without the need of too much precision.

In summary, the aim of this study has been to build a bridge between the 
lattice-QCD and global fitting communities.
%
The final goal would be to reach a situation where lattice-QCD calculations 
could be used to provide novel genuine inputs to unpolarized and polarized 
PDF fits, which is now limited to some polarized PDFs. 
%
Precise lattice-QCD results could actually reduce the uncertainties of
global PDF fits and/or discriminate between different sets.
%
We hope this white paper motivates the lattice-QCD and global-fit
communities to continue fruitful interactions to improve our knowledge of PDFs
and our understanding of the internal structure of hadrons.
