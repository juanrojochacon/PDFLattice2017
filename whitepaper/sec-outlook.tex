%%%%%%%%%%%%%%%%%%%%%%%%%%%%%%%%%%%%%%%%%%%%%%%%%%%%%%%%%%%%%%%%%%%%%%%%%%%%%%%%
\section{Outlook}
\label{sec:outlook}
%%%%%%%%%%%%%%%%%%%%%%%%%%%%%%%%%%%%%%%%%%%%%%%%%%%%%%%%%%%%%%%%%%%%%%%%%%%%%%%%

The quark and gluon structure of the proton lies
at the cross-roads between high-energy and nuclear physics, with important
applications for astroparticle physics.
%
Motivated by the recent breakthroughs in lattice QCD calculations of
PDF-related quantities, in particular in the calculation of their moments and 
of their $x$-space shape, this white paper presents a detailed snapshot of 
what we have learned so far about the internal structure of the proton, 
both from the global fitting and from the lattice QCD frameworks.
%
One of the main goals of this document was to establish a common language
between the two communities, to facilitate dialogue between them,
and to ensure that information can flow between global PDF fitters
and lattice-QCD practitioners.

We have presented the first systematic comparison between state-of-the-art 
lattice QCD calculations of PDF moments with the corresponding results from 
global PDF fits, both for unpolarised and polarised PDFs.
%
Moreover, we have presented a first quantitative estimate of the impact
of future lattice QCD calculations in the global fit, demonstrating that
in some scenarios there is a marked uncertainty reduction in the large-$x$ region.
%
Moreover, we have provided benchmark numbers
from global fits for the higher moments not used
in our benchmark comparison with lattice-QCD. These higher moments can be used
to validate future lattice-QCD calculations.

The studies presented in this document can be extended in a number of directions.

First of all, we have restricted our benchmark comparison only to the
first moment of polarised and unpolarised PDFs, which are the PDFs
that have been computed with greatest control over the various sources of
systematic uncertainties.
%
Future work should extend this comparison to the second and third PDF moments,
because, as we have demonstrated, provided the precision and accuracy
of lattice-QCD calculations keeps improving, these can provide genuinely
useful information for PDF fits.
%
The overview of such calculations collected in Appendix~\ref{sec:LQCDtables}
provides a first step towards such an extended study.

Moreover, a similar benchmarking exercise between global fit results and lattice-QCD calculations should
be performed at the level of
$x-$space calculations, such as those presented in Sect.~\ref{sec:xdependence}
from the quasi-PDF approach.
%
It will be important to compare in detail the available lattice-QCD results with
state-of-the-art global fits, to first validate the former and
thereby demonstrate that lattice-QCD calculations of $x$-space PDFs can contribute to
global fits.
%
An important region to be considered here is the small-$x$ region, where current lattice
calculations exhibit a behaviour that is qualitatively inconsistent with
the expectations from perturbative QCD and with DIS structure function measurements.

Another possible avenue to explore is to assess
the possible impact of lattice-QCD calculations on other non-perturbative objects
extracted from global analyses of experimental data.
%
Examples of these include the transversity, transverse-momentum dependent PDFs (TMD-PDFs),
or generalised PDFs (GPDFs).
%
A characteristic feature of these quantities is that, as opposed to unpolarised and
polarised PDFs that are known rather precisely, the experimental constraints on them
are rather scarce. Therefore, lattice-QCD calculations offer the potential
to provide new information, without having to reduce the systematic uncertainties 
in lattice-QCD calculations to the few percent level.

The goal of this study was to build a bridge between the lattice QCD
and global fitting communities, by developing a common language, ultimately
aiming to reach a situation where lattice QCD calculations can be used to 
provide novel genuine inputs to unpolarised and polarised PDF fits.

Although, as we have seen in the benchmark comparisons in 
Sect.~\ref{sec:benchmarking}, existing calculations are not yet able to 
provide meaningful constraints on unpolarised PDFs, we have also demonstrated 
in Sect.~\ref{sec:projections} that, in a number
of plausible scenarios, these constraints could be significant.
%
We hope this white paper motivates the lattice-QCD and global-fittting
communities to continue fruitful interactions and opens a novel
window onto the internal structure of hadrons.
