%%%%%%%%%%%%%%%%%%%%%%%%%%%%%%%%%%%%%%%%%%%%%%%%%%%%%%%%%%%%%%%%%%%%%%%%%%%%%%%%
\section{Outlook}
\label{sec:outlook}
%%%%%%%%%%%%%%%%%%%%%%%%%%%%%%%%%%%%%%%%%%%%%%%%%%%%%%%%%%%%%%%%%%%%%%%%%%%%%%%%

The study of the parton distributions
of the proton is an active interdisciplinary research field
lying
at the crossroads of high-energy, 
hadronic, and nuclear physics, with applications also in astroparticle physics.
%
In this white paper, we have reviewed our current knowledge of PDFs as
determined from both the global analysis framework
and from lattice-QCD computations. 
%
We have first established a common language
between the two communities, in order to facilitate the
interactions between them.
%
We have then presented a first systematic comparison between state-of-the-art 
lattice-QCD calculations of PDF moments and the corresponding results from the
global analyses both in the unpolarized and the polarized cases.
%
Moreover, we have also provided additional benchmark numbers from the global fits for the higher moments 
not used in this benchmark comparison, which should be useful
to validate future lattice-QCD calculations.

One of the main outcomes of this white paper is a first
quantitative study of the impact of lattice-QCD calculations
in the 
global fits, based on both PDF moments and  on Bjorken-$x$ dependence pseudo-data,
assuming a number of different scenarios for
the associated uncertainties.
%
In the case of unpolarized PDFs, we have demonstrated that a reduction 
of the uncertainties of current lattice-QCD calculations is needed 
prior to provide any impact on global-PDF fits.
%
In the case of polarized PDFs, we have shown that current lattice-QCD
calculation can already start providing a useful input into global-PDF analyses.
%
Despite the studies presented here are still in an initial exploratory phase, they provide
strong motivation for global fitters to begin to consider lattice-QCD
constraints into their global analyses.

The studies presented in this white paper can be extended in a number of 
directions.
%
First, we have restricted our benchmark comparison only to the lowest moments 
of polarized and unpolarized PDFs, whose various sources of systematic 
uncertainties have been computed with the greatest control.
%
Future work should extend this comparison to higher PDF moments.
%
These could have some impact on PDF fits, provided the precision and accuracy
of lattice-QCD calculations keeps improving.

Second, a similar benchmark exercise between global fit results and 
lattice-QCD calculations should be performed at the level of
$x-$space calculations.
%
It will be important to compare in detail the available lattice-QCD results 
with state-of-the-art global fits, to first validate the former and
thereby demonstrate to what extent lattice-QCD calculations of $x$-space PDFs 
can contribute to global fits.

Third, it should be possible to assess the impact of lattice-QCD 
calculations on other non-perturbative objects
determined from global analyses of experimental data.
%
Examples of these include the transversity, transverse-momentum dependent 
PDFs (TMDs), generalized PDFs (GPDs), or collinear PDFs for hadrons 
other than nucleons.
%
All these quantities are known with much less precision than unpolarized
and polarized PDFs, given that the corresponding experimental information
is rather scarce. 
%
In this case, lattice-QCD calculations could have the potential
to provide new information, without the need of too much precision.

In summary, the aim of this study has been to build a bridge between the 
lattice-QCD and global fitting communities.
%
The final goal would be to reach a situation where lattice-QCD calculations 
could be used to provide novel genuine inputs to unpolarized and polarized 
PDF fits, which is now limited to some polarized PDFs. 
%
Precise lattice-QCD results could actually reduce the uncertainties of
global PDF fits and/or discriminate between different sets.
%
We hope this white paper motivates the lattice-QCD and global-fit
communities to continue fruitful interactions to improve our knowledge of PDFs
and our understanding of the internal structure of hadrons.
