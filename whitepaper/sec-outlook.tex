%%%%%%%%%%%%%%%%%%%%%%%%%%%%%%%%%%%%%%%%%%%%%%%%%%%%%%%%%%%%%%%%%%%%%%%%%%%%%%%%
\section{Outlook}
\label{sec:outlook}
%%%%%%%%%%%%%%%%%%%%%%%%%%%%%%%%%%%%%%%%%%%%%%%%%%%%%%%%%%%%%%%%%%%%%%%%%%%%%%%%

The quark and gluon structure of the proton is a fascinating topic
at the cross-roads between high-energy and nuclear physics, with important
applications also for astroparticle physics.
%
Motivated by the recent breakthroughs in lattice QCD calculations of
PDF-related quantities, in particular in the calculation of their moments and of
their $x$-space shape, this white paper presents a detailed snapshot of what we have learned so far
about the internal proton structure, both from the global PDF fitting and from
the lattice QCD frameworks.
%
One of the main goals of this document was to clearly establish a common language
between the two communities, in order to facilitate the cross-talk between them,
and ensure that information can flow effortlessly between global PDF fitters
and lattice QCD practitioners.
%
As we hope the reader can appreciate, we have made our best to fulfill this requirement.

Perhaps the main outcome of this initial effort is the first ever systematic
comparison between state-of-the-art lattice QCD calculations of PDF moments with
the corresponding results from the global PDF fits, both
for the unpolarized and for the polarized case.
%
Moreover, we have presented a first quantitative estimate of the impact
of future lattice QCD calculations in the global fit, demonstrating that
in some scenarios there is a marked uncertainty reduction in the large-$x$ region.
%
Moreover, we have also presented benchmark numbers
from the PDF global fit side for the higher moments not used
for the benchmark comparison, which can be used
in the future as a fully documented
reference in order to validate upcoming lattice calculations.

The studies presented in this document can be extended in a number of directions.
%
First of all, here we have restricted our benchmark comparison only to the
first moment of polarized and unpolarized PDFs, which are those
that have been computed with better control over the various sources of
systematic uncertainties.
%
Future work should extend this comparison to also the second and third PDF moments,
since as we have demonstrated in this work, provided the precision and accuracy
of lattice QCD calculations keeps improving, these can provide genuinely
useful information for the PDF fit.
%
In this respect, the overview of such calculations collected in Appendix~\ref{sec:LQCDtables}
provide a first step towards such extended study.

Moreover, a similar benchmarking exercise between global fit results and lattice
QCD calculations should also be performed at the level of
$x-$space calculations, such as the ones presented in Sect.~\ref{sec:xdependence}
from the quasi-PDF approach.
%
In this respect, it will be important to compare in detail the available lattice
QCD results with modern global fits, to first of all validate the former and
this way demonstrate that they can be used to contribute to the latter.
%
An important region to be considered here is the small-$x$ region, where current lattice
calculations exhibit a behaviour which is qualitatively inconsistent with
the expectations from perturbative QCD as well as with DIS structure functions measurements.

Another possible avenue in which the comparisons presented in these report might assessing
the possible impact of lattice QCD calculations in other non-perturbative objects that
are also extracted from a global analysis to experimental data.
%
Examples of these include the transversity, transverse-momentum dependent PDFs (TMD-PDFs),
or generalized PDFs (GPDFs).
%
A characteristic feature of these quantities is that, as opposed to unpolarized and
polarized PDFs that are known rather precisely, the experimental constraints on them
are rather scarce, and therefore lattice QCD calculations offer the potential
to bring in new information without having to go to systematics at the few percent level.

The overarching goal of this study was to build a bridge between the lattice QCD
and global fitting communities, by developing a common language, ultimately
aiming to reach a situation where
lattice QCD calculations can be used to provide novel genuine inputs
to unpolarized and polarized PDF fits.
%
Although as we have seen in the benchmark comparisons in Sect.~\ref{sec:benchmarking} that
existing calculations are not yet able to provide meaningful constraints on the
PDFs, we have also demonstrated in Sect.~\ref{sec:projections} that in a number
of scenarios these constraints can be significant.
%
We hope that the studies presented in this white-paper motivate to continue the fruitful
interactions between the two communities and allow us to open a novel window
to peer into the details of the internal structure of hadrons.


%%%%%%%%%%%%%%%%%%%%%%%%%%%%%%%%%%%%%%%%%%%%%%%%%%%%%%%%%%%%%%%%%%%%%%%%%%%%%%%%
\subsection*{Acknowledgments}

We are very grateful to Jacqueline Gills for the flawless organization
of the workshop at Balliol College and to Michelle Bosher for
invaluable help and support with the workshop logistics.
%
This workshop was partly supported by the European Research Council via
the Starting Grant {\it ``PDF4BSM - Parton Distributions in the
  Higgs Boson Era},
and
the U.S. Department of Energy under Grant No. DE-SC0010129.
%
E.~R.~N. acknowledges financial support from the
UK STFC via the Rutherford Grant ST/M003787/1.
%
J.~R. is funded by the ERC via the Starting Grant ``PDF4BSM'' and by the
Dutch Organization for Scientific Research (NWO).
