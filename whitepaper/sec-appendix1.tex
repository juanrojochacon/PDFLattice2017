\section{Definition of the PDF moments}
\label{app:notation}

In this appendix, we summarise the convention adopted in this paper to denote the
moments of relevant unpolarised and longitudinally polarised PDF combinations.
%
Here we restrict ourselves to those
quantities which can be presently computed in lattice QCD,
although those used for benchmarks in Sect.~\ref{sec:benchmarking} are only
a subset of them.
%
In the equations below, we use the shorthand notation
\be
(\Delta)q^\pm \equiv (\Delta)q\pm(\Delta)\bar{q}\, ,\qquad q=u,d,s,c \, .
\ee
Moreover, we will identify $\mu$  with the QCD factorisation scale,
and $Q$ with the characteristic scale of a given 
hard-scattering process.
%
The use of these conventions is strongly recommended for any future
comparison between lattice QCD and global fit calculations of PDF moments.

\begin{itemize}

\item Unpolarised moments:

\begin{enumerate}

\item The first moment of the total $u^+-d^+$ PDFs
\begin{equation}
\left.\langle x\rangle_{u^+-d^+}(\mu^2)\right|_{\mu^2=Q^2}
=
\int_0^1 dx\, x\left\{u(x,Q^2)+\bar{u}(x,Q^2)-d(x,Q^2)-\bar{d}(x,Q^2)\right\} \, .
\label{eq:unpfmumdtot}
\end{equation}

\item The second moment of the valence $u^--d^-$ PDFs
\begin{equation}
\left.\langle x^2\rangle_{u^--d^-}(\mu^2)\right|_{\mu^2=Q^2}
=
\int_0^1 dx\, x^2\left\{u(x,Q^2)-\bar{u}(x,Q^2)-d(x,Q^2)+\bar{d}(x,Q^2)\right\} \, .
\label{eq:unpsmumdval}  
\end{equation}

\item The first moment of the individual quark $q^+$ total PDFs
\begin{equation}
\left.\langle x\rangle_{q^+=u^+,d^+,s^+,c^+}(\mu^2)\right|_{\mu^2=Q^2}
=
\int_0^1 dx\, x\left\{q(x,Q^2)+\bar{q}(x,Q^2)\right\} \, .
\label{eq:unpfmiqtot}
\end{equation}

\item The second moment of the individual quark $q^-$ valence PDFs
\begin{equation}
\left.\langle x^2\rangle_{q^-=u^-,d^-,s^-,c^-}(\mu^2)\right|_{\mu^2=Q^2}
=
\int_0^1 dx\, x^2\left\{q(x,Q^2)-\bar{q}(x,Q^2)\right\} \, .
\label{eq:unpsmiqval}
\end{equation}

\item The first moment of the gluon PDF
\begin{equation}
\left.\langle x \rangle_g(\mu^2)\right|_{\mu^2=Q^2}
=
\int_0^1 dx\, x\, g(x,Q^2) \, .
\label{eq:unpfmg}
\end{equation}

\end{enumerate}

\item Longitudinally polarised moments:

\begin{enumerate}

\item The zeroth moment of the total $u^+-d^+$ PDFs
\begin{equation}
\left.\langle 1 \rangle_{\Delta u^+-\Delta d^+}(\mu^2)\right|_{\mu^2=Q^2}
=
\int_0^1 dx \left\{\Delta u(x,Q^2)+\Delta\bar{u}(x,Q^2)-\Delta d(x,Q^2)-\Delta\bar{d}(x,Q^2)\right\} \, .
\label{eq:polzmumdtot}
\end{equation}

\item The first moment of the valence $u^--d^-$ PDFs
\begin{equation}
\left.\langle x\rangle_{\Delta u^--\Delta d^-}(\mu^2)\right|_{\mu^2=Q^2}
=
\int_0^1 dx\, x\left\{\Delta u(x,Q^2)-\Delta\bar{u}(x,Q^2)-\Delta d(x,Q^2)+\Delta \bar{d}(x,Q^2)\right\}
\label{eq:polfmumdval}  
\end{equation}

\item The zeroth moment of the individual quark $q^+$ total PDFs
\begin{equation}
\left.\langle 1\rangle_{q^+=\Delta u^+,\Delta d^+,\Delta s^+,\Delta c^+}(\mu^2)\right|_{\mu^2=Q^2}
=
\int_0^1 dx \left\{\Delta q(x,Q^2)+\Delta\bar{q}(x,Q^2)\right\} \, .
\label{eq:polzmiqtot}
\end{equation}

\item The first moment of the individual quark $q^-$ valence PDFs
\begin{equation}
\left.\langle x\rangle_{\Delta q^-=\Delta u^-,\Delta d^-,\Delta s^-,\Delta c^-}(\mu^2)\right|_{\mu^2=Q^2}
=
\int_0^1 dx\, x\left\{\Delta q(x,Q^2)-\Delta\bar{q}(x,Q^2)\right\} \, .
\label{eq:polfmiqval}
\end{equation}

\end{enumerate}

\end{itemize}

As discussed in Sect.~\ref{Sec:IntroPDFs}, some of these moments have a direct
physical interpretation, for instance Eq.~(\ref{eq:unpfmiqtot}) corresponds
to the proton's momentum fraction carried by a given quark flavour (and its
corresponding antiquark) at the scale $\mu$, while
Eq.~(\ref{eq:polzmiqtot}) is related to the contribution of the quark flavour
$q$ to the total proton spin.
%
Moreover, we should emphasize here that other flavour combinations as well as higher
moments are trivially computable within the global PDF fitting framework,
but we have not considered them since their evaluation within the lattice
QCD approach is outside the reach of present and future capabilities.












