\subsubsection{Polarized PDFs}
\label{sec:polPDFs}

The fitting framework outlined in Sec.~\ref{sec:unpPDFs} can be applied
to a determination of longitudinally polarised PDFs as well.
%
Here we delineate the aspect of this framework specific to the polarised
case, and give a review of current global polarised PDF fits.

\paragraph{General framework.}
%
Longitudinally-polarised PDFs are defined as the net amount of parton 
densities with spin aligned along ($\uparrow$) or opposite ($\downarrow$)
the direction of polarisation of the parent nucleon
\begin{equation}
\Delta f(x,\mu^2) \equiv f^{\uparrow}(x,\mu^2) - f^{\downarrow}(x,\mu^2)
\,\mbox{,}
\ \ \ \ \ \ \ \ \ \
f=u,\bar{u},d,\bar{d},s,\bar{s},g
\,\mbox{.}
\label{eq:polPDFs}
\end{equation}
%
The interest in their determination is specifically related to the
first moments of the singlet and gluon polarised PDFs
\begin{align}
\Delta\Sigma(\mu^2)
& =
\sum_{q}^{n_f}\int_0^1 dx 
\left[\Delta q(x, \mu^2) + \Delta\bar{q}(x, \mu^2)\right]
=
\sum_q^{n_f}\langle 1 \rangle_{q^+}
\\
\Delta G(\mu^2)
& =
\int_0^1 dx \Delta g(x,\mu^2)
=
\langle 1 \rangle_g
\,,
\label{eq:moments}
\end{align}
which can be interpreted~\cite{} as the nucleon spin fraction carried 
respectively by quarks/antiquarks and gluons.

As in the case of unpolarised PDFs, their dependence on the scale $\mu$ can 
be computed perturbatively with DGLAP evolution equations, 
Eq.~(\ref{eq:dglap}).
%
The unpolarised splitting kernels $P_{ij}$ should be replaced with their
polarised counterparts, $\Delta P_{ij}$, which have been computed up to 
NNLO~\cite{} in the $\overline{\rm MS}$ renormalisation scheme.

The dependence on the momentum fraction $x$, fixed by non-perturbative QCD 
dynamics, should however satisfy some theoretical constraints.
%
First, PDFs must lead to positive cross sections: 
at leading order (LO), this implies that polarized 
PDFs are bounded by their unpolarized counterparts\footnote{Beyond LO, more 
complicate relations hold, but they have negligible impact on  
PDFs~\cite{Altarelli:1998gn}.}, $|\Delta f(x,\mu^2)|\leq f(x,\mu^2)$.
%
Second, polarized PDFs must be integrable: this corresponds to the assumption 
that the nucleon matrix element of the axial current for each flavor is finite.
%
Third, it follows from SU(2) and SU(3) flavor symmetry that 
the first moments of the nonsinglet $\mathcal{C}$-even PDF combinations,
$\Delta T_3=\Delta u^+ -\Delta d^+$ and 
$\Delta T_8 = \Delta u^+ +\Delta d^+ -2\Delta s^+$ 
(where $\Delta q^+=\Delta q+\Delta\bar{q}$, $q=u,d,s$), are respectively
related to the baryon octet $\beta$-decay constants, whose values are 
well measured~\cite{Agashe:2014kda}:
\begin{align}
 a_3
 & =
 \int_0^1 dx \Delta T_3 
 = \langle 1\rangle_{u^+} - \langle 1\rangle_{d^+}  = 1.2701 \pm 0.0025\\
 a_8
 & =
 \int_0^1 dx \Delta T_8 
 = \langle 1 \rangle_{u^+} + \langle 1 \rangle_{d^+} -2\,\langle 1 \rangle_{s^+} 
 =0.585  \pm 0.025
 \,\mbox{.}
\label{eq:decayconst}
\end{align}

The bulk of the experimental information on polarized PDFs comes from 
neutral-current inclusive and semi-inclusive Deep-Inelastic Scattering 
(DIS and SIDIS) with charged lepton beams and nuclear targets. Because of the
way the corresponding observables factorize, inclusive DIS data constrain the 
total quark combinations $\Delta q^+$, 
while SIDIS data, with identified pions or kaons in the final state, 
constrain individual quark and antiquark flavors. In principle, both DIS and 
SIDIS data would constrain the gluon distribution $\Delta g$ via scaling 
violations, but in practice their effect is rather weak because of the small 
$Q^2$ range covered. 

Beside DIS and SIDIS fixed-target data, a significant amount of data from
longitudinally polarized proton-proton ($pp$) collisions at the Relativistic 
Heavy Ion Collider (RHIC) have become available 
recently~\cite{Aschenauer:2015eha}, though in a limited range of momentum 
fractions, $0.05\lesssim x \lesssim 0.4$.
%
On the one hand, longitudinal (parity-violating) single-spin and 
(parity-conserving) double-spin asymmetries for $W^\pm$ boson production are 
sensitive to the flavor decomposition of polarized quark and antiquark 
distributions, because of the chiral nature of the weak 
interactions~\cite{Bourrely:1993dd}. 
%
On the other hand, double-spin asymmetries for jet and $\pi^0$ 
production are directly sensitive to the gluon polarization in 
the proton, because of the dominance of gluon-gluon and quark-gluon initiated 
subprocesses in the kinematic range accessed by RHIC~\cite{Bourrely:1990pz}.

%Add here a plot with th ekinematic coverage of the data

\paragraph{State-of-the-art global polarised PDF fits.}

Several determinations of polarized PDFs of the proton (up to NLO and mostly 
in the $\overline{\rm MS}$ factorization scheme) are presently available, 
aiming at unveiling how much large and uncertain are $\Delta\Sigma$ and 
$\Delta G$, Eq.~\eqref{eq:moments}. Above all, they differ among each others 
for the procedure used to determine PDFs from data and for the included data 
sets (for details, see {\it e.g.} Chap.~3 in 
Ref.~\cite{Nocera:2014vla}). Both are a source of uncertainty, but the 
former may be elusive: in this respect, the {\tt NNPDF} collaboration has 
developed a methodology to reduce and keep it under control as much as possible.
This is based on cutting-edge statistical tools, including Monte Carlo sampling
for error propagation, neural networks for PDF parametrization, and closure 
tests for explicit characterization of procedural 
uncertainties~\cite{Ball:2014uwa}.

Motivated by the interest in studying the effects of this piece of experimental
information, two new global analyses of polarized PDFs have been carried out in
2014, {\tt DSSV14}~\cite{deFlorian:2014yva} and 
{\tt NNPDFpol1.1}~\cite{Nocera:2014gqa}. 
These upgrade the corresponding previous analyses, 
{\tt DSSV08}~\cite{deFlorian:2008mr} and 
{\tt NNPDFpol1.0}~\cite{Ball:2013lla}, with data respectively on double-spin 
asymmetries for inclusive jet production~\cite{Adamczyk:2014ozi} 
and $\pi^0$ production~\cite{Adare:2014hsq}\footnote{Preliminary RHIC results 
included in Ref.~\cite{deFlorian:2008mr} have been replaced in
Ref.~\cite{deFlorian:2014yva} with final results.}, 
and on double-spin asymmetries for high-$p_T$ inclusive jet 
production~\cite{Adamczyk:2014ozi,Adamczyk:2012qj,Adare:2010cc} and single-spin
asymmetries for $W^\pm$ production~\cite{Adamczyk:2014xyw}.
Some new data by the COMPASS experiment have also been included in 
{\tt DSSV14} and {\tt NNPDFpol1.1}, respectively new DIS and SIDIS 
data~\cite{Alekseev:2010hc,Alekseev:2010ub} and open-charm leptoproduction 
data~\cite{Adolph:2012ca}. The new data have been included in {\tt NNPDFpol1.1}
by means of Bayesian reweighting~\cite{Ball:2010gb},
and in {\tt DSSV14} by means of a full refit.  

Overall, both {\tt DSSV14} and {\tt NNPDFpol1.1} PDF determinations are 
state-of-the-art in the inclusion of the available experimental information. 
The data sets in the two analyses differ between each other only for 
fixed-target SIDIS and RHIC $\pi^0$ production measurements, included in 
{\tt DSSV14}, but not in {\tt NNPDFpol1.1}. The information brought in by 
these data is complementary to that provided by RHIC $W^\pm$ production and 
inclusive jet production data respectively, but this is less 
constraining~\cite{Nocera:2014gqa}. These data were not included in the 
{\tt NNPDFpol1.1} analysis because fragmentation functions (FFs) enter the 
factorized expression of the corresponding observables: since FFs are 
nonperturbative objects on the same footing as PDFs, they are likely to 
introduce an additional source of bias in the PDF determination. The
{\tt NNPDF} methodology aims at reducing this bias as much as possible, hence 
the inclusion of these data would require the consistent determination 
of FFs within the {\tt NNPDF} methodology, which is under consideration,
though not yet available~\cite{Bertone:2015cwa}. 

The effect of RHIC data on the polarized PDFs of the proton is twofold.  
\begin{itemize}
\item The 2012 STAR data sets on $W$ production~\cite{Adamczyk:2014xyw}, 
included in {\tt NNPDFpol1.1}, provide evidence of a positive 
$\Delta\bar{u}$ distribution 
and a negative $\Delta\bar{d}$ distribution, with 
$|\Delta\bar{d}|>|\Delta\bar{u}|$~\cite{Nocera:2014gqa}. 
The size of this flavor symmetry breaking for polarized sea quarks is 
quantified by the asymmetry $\Delta\bar{u}-\Delta\bar{d}$, which,
in the {\tt NNPDFpol1.1} analysis, turned out to be roughly as large as its 
unpolarized counterpart (in absolute value), 
though much more uncertain~\cite{Nocera:2014rea}. Even within this uncertainty,
polarized and unpolarized light sea quark asymmetries show opposite sings,
with the polarized being definitely positive.  
This result starts to discriminate between different 
models of nucleon structure, see left panel of Fig.~\ref{fig:RHICpdfs}: 
specifically, some meson-cloud (MC) models are disfavored, while a more 
accurate experimental information is needed to establish whether 
chiral quark-soliton (CQS), Pauli-blocking (PB) or statistical (ST)
models are preferred (these models are described in 
Ref.~\cite{Chang:2014jba}).

\item The 2009 STAR and PHENIX data sets on jet and $\pi^0$ 
production~\cite{Adamczyk:2014ozi,Adare:2014hsq}, included in both {\tt DSSV14}
and {\tt NNPDFpol1.1}, provide first evidence
of a sizable, positive gluon polarization in the proton. 
A comparison of the gluon PDF in the two PDF sets is displayed in 
Fig.~\ref{fig:RHICpdfs} (right panel). Comparable results, both central values 
and uncertainties, are found in the $x$ region covered by RHIC data. 
The agreement between the two analyses is optimal in the
range $0.08\leq x \leq 0.2$, where the dominant experimental information comes
from jet data; a slightly smaller central value is found in the {\tt DSSV14} 
analysis, in comparison to the {\tt NNPDFpol1.1}, in the range 
$0.05\leq x \leq 0.08$, where the dominant experimental information comes from 
$\pi^0$ production data. Indeed, these are included in {\tt DSSV14} but are not
in {\tt NNPDFpol1.1}. Nevertheless, best fits lie well within each other error
bands, though {\tt NNPDF} uncertainties tend to be larger than {\tt DSSV14}
uncertainties outside the region covered by RHIC data.
Very well compatible values of the integral of $\Delta g$, 
Eq.~\eqref{eq:moments}, truncated over the interval $0.05\leq x \leq 1$, are 
found: at $Q^2=10$ GeV$^2$, this is $0.20^{+0.06}_{-0.07}$ for 
{\tt DSSV14}~\cite{deFlorian:2014yva}, and $0.23\pm 0.06$ for 
{\tt NNPDFpol1.1}~\cite{Nocera:2014gqa}.
\end{itemize}

%briefly mention what are the open issues
