%%%%%%%%%%%%%%%%%%%%%%%%%%%%%%%%%%%%%%%%%%%%%%%%%%%%%%%%%%%%%%%%%%%%%%%%%%%%%%%%
\section{Introduction and motivation}
%%%%%%%%%%%%%%%%%%%%%%%%%%%%%%%%%%%%%%%%%%%%%%%%%%%%%%%%%%%%%%%%%%%%%%%%%%%%%%%%

The detailed understanding of the internal structure of protons is an active
research field with important phenomenological implications
at the cross-roads between high-energy, nuclear, and astroparticle physics.
%
There exist two main methods to quantify how the momentum and spin of
the nucleons
are divided among its constituents, the quarks and gluons.
%
The first one
is the {\it global QCD analysis}, which exploits the fact that hard-scattering
cross sections can be factorised into short-distance matrix elements,
computable with perturbation theory, and long-distance
non-perturbative dynamics encoded in the
so-called Parton Distribution Functions (PDFs).
%
By combining a wide variety of experimental data on hard-scattering processes
with state-of-the-art perturbative calculations through a robust statistical
methodology, it is possible to achieve a reasonably precise determination
of these non-perturbative PDFs for both unpolarized and polarized nucleons.
%
In this context, several collaborations provide regular updates of
phenomenological PDF determinations, both
in the unpolarized~\cite{Ball:2014uwa,Ball:2017nwa,Harland-Lang:2014zoa,
Dulat:2015mca,Alekhin:2017kpj,Owens:2012bv} and in
the polarized~\cite{Nocera:2014gqa,deFlorian:2009vb,
  Sato:2016tuz,Hirai:2008aj} cases.
%
See Refs.~\cite{Rojo:2015acz,Butterworth:2015oua,Ball:2012wy,
Alekhin:2011sk,Forte:2013wc,Forte:2010dt,Perez:2012um,DeRoeck:2011na,
Accardi:2016ndt} for recent reviews of the global PDF fitting framework.

Alternatively, one can compute quantities related to PDFs directly 
from QCD, without resorting to
experimental data, by using the non-perturbative formulation known as
{\it lattice-QCD}~\cite{Olive:2016xmw,Gupta:1997nd}.
%
This method introduces an ultraviolet cutoff that allows the direct 
computation of the QCD path integral in discretised finite-volume 
Euclidean space-time.
%
In order to connect with experimental measurements, a 
continuum-limit extrapolation is necessary to remove the cutoff dependence 
and finite-volume effects can be removed by suitable extrapolation.

One of the main
conceptual advantages of lattice-QCD calculations is that
they require minimal external input: one needs only to 
set the hadronic scale $\Lambda_\text{QCD}$ and the values of the quark masses.
%
For calculations relevant to low-energy hadron structure, this means
setting the up, down and strange quark masses,
which is usually done using the pion and kaon masses as external inputs.
%
The overall hadronic scale can be set using well-determined baryon masses 
such as that of the $\Omega$ baryon.
%
Many QCD quantities can be then predicted 
using lattice-QCD, including quantities that are challenging to obtain from
experiments, providing thus a unique input to complement theoretical QCD
predictions.

From the point of
global PDF analyses, the availability of a wealth of high-precision collider measurements
from HERA, the Tevatron and the LHC, together with that of the corresponding
NNLO QCD and NLO electroweak perturbative calculations, are pushing the
accuracy frontier and leading to improved PDFs with reduced uncertainties,
in most cases at the few-percent level.
%
A paradigmatic illustration of this progress is provided by the gluon PDF, which has been
affected for a long time by rather large uncertainties due to the limited experimental
information.
%
However, in modern PDF fits the gluon is
constrained quite precisely from small to large-$x$,
thanks to the inclusion in the
global analysis of novel processes such as $D$-meson
production~\cite{Zenaiev:2015rfa,Gauld:2016kpd},
the transverse momentum of $Z$ bosons~\cite{Boughezal:2017nla},
inclusive jets production~\cite{Currie:2016bfm},
and top-quark pair differential
distributions~\cite{Czakon:2016olj,Guzzi:2014wia}, using state-of-the-art
perturbative QCD calculations.

In the case of lattice-QCD calculations, early attempts to determine PDFs were limited by the 
available computational resources and various technical challenges, with most results restricted to
the first couple moments of non-singlet PDFs at relatively heavy quark masses.
%
Recent progress has been driven both
by improved systematic control (physical pion mass, excited-state contamination, large volumes) 
for simple quantities such as nucleon matrix elements (which correspond to the low moments of PDFs) and the development of novel
strategies to overcome the limitations in computing the first few 
moments~\cite{Constantinou:2014tga,Syritsyn:2014saa,Lin:2012ev}, determine challenging quantities, 
such as gluon and flavour-singlet matrix elements, and directly calculate the Bjorken-$x$ dependence of PDFs \cite{Lin:2014zya,Alexandrou:2015rja,Chen:2016utp,Alexandrou:2016jqi}.
%
These developments have pushed lattice-QCD calculations
to the point where, for the first time, it is possible to provide information on the PDF shape
of specific flavour combinations, both for quarks and for antiquarks, and meaningful comparison with 
global fits can be made.
%
One of the main motivations for this effort is to reach a point where lattice-QCD
calculations can be used as external input to help constraining the
PDFs obtained from the global analysis framework.

Despite these rather important developments both from the lattice-QCD
and from the global fitting sides,
the communication between the
two communities has been so far very limited.
%
This situation led some of us
to organise the first edition of the
``Parton Distributions and Lattice Calculations in the LHC Era''
workshop (PDFLattice2017), which took
place in Balliol College (Oxford) in March
2017.\footnote{\url{http://www.physics.ox.ac.uk/confs/PDFlattice2017/index.asp}}
%
The aim of this workshop was to create a common ground for discussions
between the two communities, ensuring that we spoke the same language,
and start to consider in a quantitative way how lattice-QCD calculations could be used
to improve global PDF fits, and conversely, how PDF fits can be exploited to benchmark
exiting and future lattice calculations in a systematic way.

In this context, some of the questions that were addressed during this workshop
included:
\begin{itemize}
\item What information from PDF fits is relevant to constrain, test or validate lattice calculations?

\item What PDF-related quantities are most urgent
  to compute in lattice-QCD in terms of phenomenological relevance?

\item What information does lattice-QCD provide on the
  shape (Bjorken-$x$ dependence) of the PDFs? Which specific
  PDF moments,
  and up to which order can be computed?
  
\item How do we systematically and consistently quantify the systematic errors in these lattice calculations?

\item What accuracy do we need from lattice quantities 
  in order to have a significant impact on global PDF fits?

\item To what extent do available lattice results agree with the results of
  global PDF fits? Is there a tension between global PDF fits, PDF
  fits based on reduced datasets, and PDF calculations from the lattice?

  \item What is the ultimate accuracy that can be expected from lattice
calculations in the near and medium future? What is the ultimate
constraining power of lattice calculations on PDFs?

\end{itemize}

This white paper summarizes the joint
effort between the two communities to address these questions, 
following the very fruitful discussions and interactions that took place both during 
the PDFLattice2017 workshop and in the subsequent months.
%
While this document does not represent the final word on this topic,
it provides a solid starting point for subsequent collaborative efforts
and should guarantee a smooth 
interaction between the two communities in the near future.

The outline of this white paper is the following.
%
First of all, in Sect.~\ref{sec:theoryoverview} we present an overview of
the global QCD fit and lattice-QCD methods for the evaluation
of polarised and unpolarised parton distributions.
%
Then in Sect.~\ref{sec:benchmarking}
we summarise state-of-the-art benchmark
calculations between the most
recent available lattice-QCD calculations and global PDF fits
for the most important PDF moments.
%
In Sect.~\ref{sec:projections} we quantify the impact that
upcoming lattice calculations of PDF moments could have on unpolarised
and polarised global analysis, for different scenarios
about the total uncertainties that could be achieved with the former
method in the future.
%
Finally, in Sect.~\ref{sec:outlook} we summarise our studies
and discuss the outlook for future interactions between
the global QCD fit and lattice-QCD communities.
%
In addition, Appendix~\ref{app:notation} summarizes the conventions
that are adopted in this report for the definition of the PDF
moments, and then
Appendix~\ref{sec:LQCDtables} presents an exhaustive summary of existing calculations
of those moments in lattice-QCD.
