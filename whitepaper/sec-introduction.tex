%%%%%%%%%%%%%%%%%%%%%%%%%%%%%%%%%%%%%%%%%%%%%%%%%%%%%%%%%%%%%%%%%%%%%%%%%%%%%%%%
\section{Introduction and motivation}
%%%%%%%%%%%%%%%%%%%%%%%%%%%%%%%%%%%%%%%%%%%%%%%%%%%%%%%%%%%%%%%%%%%%%%%%%%%%%%%%

The detailed understanding of the inner structure of nucleons is an 
active research field with phenomenological implications in 
high-energy, hadron, nuclear and astroparticle physics.
%
Within quantum chromodynamics (QCD), information on this structure ---
specifically on how the nucleon's momentum and spin are divided among quarks 
and gluons --- is encoded in parton distribution functions (PDFs).

There exist two main methods to determine PDFs.\footnote{In this paper, we do
not discuss nonperturbative, QCD-based models of nucleon structure. 
We refer the reader to~\cite{Ball:2016spl,Nocera:2014uea} and references 
therein for details on unpolarized and polarized PDFs respectively.}

The first method is the {\it global QCD analysis}~\cite{Perez:2012um,
DeRoeck:2011na,Alekhin:2011sk,Ball:2012wy,Forte:2013wc,Jimenez-Delgado:2013sma,
Rojo:2015acz,Butterworth:2015oua,Accardi:2016ndt,Gao:2017yyd}.
%
It is based on QCD factorization of physical observables, {\it i.e.}
the fact that a class of hard-scattering cross-sections can be expressed as a 
convolution between short-distance, perturbative, matrix 
elements and long-distance, nonperturbative, PDFs.
%
By combining a variety of available hard-scattering experimental data with 
state-of-the-art perturbative calculations, complete PDF sets, including 
the gluon and various combinations of quark flavors, are currently determined
for protons, in both the unpolarized~\cite{Ball:2017nwa,Harland-Lang:2014zoa,
Dulat:2015mca,Alekhin:2017kpj,Accardi:2016qay} and 
polarized~\cite{Nocera:2014gqa,deFlorian:2009vb,Sato:2016tuz,Hirai:2008aj} case.

Recent progress in global QCD analyses has been driven, on the one hand, 
by the increasing availability of a wealth of high-precision measurements from 
Jefferson Lab, HERA, RHIC, the Tevatron and the LHC and, on the other hand, 
by the advancement in perturbative calculations of QCD and 
electroweak (EW) higher-order corrections.
%
Parton distributions are now determined with unprecedented precision, 
in many cases at the few-percent level.
%
A paradigmatic illustration of this progress is provided by both the 
unpolarized and polarized gluon PDFs, which were affected by rather large 
uncertainties until recently, due to the limited experimental information 
available.
%
In the unpolarized case, the gluon PDF is now constrained quite accurately from 
small to large $x$ thanks to the inclusion of processes such as 
inclusive deep-inelastic scattering (DIS)~\cite{Abramowicz:2015mha}, 
$D$-meson production~\cite{Zenaiev:2015rfa,Gauld:2016kpd},
the transverse momentum of $Z$ bosons~\cite{Boughezal:2017nla},
inclusive jet production~\cite{Currie:2016bfm}, and top-quark pair
distributions~\cite{Czakon:2016olj,Guzzi:2014wia}.
%
In the polarized case, the gluon PDF is now constrained from double 
spin-asymmetries for high-$p_T$ jet and pion production in proton-proton 
collisions~\cite{deFlorian:2014yva,Nocera:2014gqa}, 
although only in the medium-to-large $x$ region.

The second method is {\it lattice QCD}~\cite{Olive:2016xmw,Gupta:1997nd}.
%
It is based on the direct computation of the QCD path integral in a 
discretized finite-volume Euclidean space-time, providing a suitable 
ultraviolet cut-off.
%
To connect with experimental measurements, extrapolations to the 
continuum and infinite-volume limits are necessary so that any  
cut-off dependence and finite-volume effects, respectively, are removed.
%
Lattice-QCD calculations require minimal external input: one needs only to 
set the hadronic scale $\Lambda_\text{QCD}$ and the values of the quark masses.
%
For calculations relevant to low-energy hadron structure, this means
setting the up, down and strange quark masses,
which is usually done using the pion and kaon masses as external inputs.
%
The overall hadronic scale can be set using well-determined baryon masses 
such as that of the $\Omega$ baryon.
%
A variety of QCD quantities can then be computed using lattice QCD, including
moments of PDFs or of certain quark flavor PDF combinations.

Early lattice-QCD attempts to determine the proton PDFs were limited by the 
available computational resources and various technical challenges, with most 
results restricted to the first few moments of nonsinglet PDFs at relatively 
large (unphysical) quark masses.
%
Overcoming these limitations, recent progress has been mostly
driven by advances in two main areas. 
%
First, by improved systematic control (physical pion mass, excited-state 
contamination, large volumes) for quantities such as the nucleon matrix 
elements corresponding to the low moments of PDFs.
%
Second, by the  development of novel strategies
for the computation of the first few 
moments~\cite{Constantinou:2014tga,Syritsyn:2014saa,Lin:2012ev},
the determination of more challenging quantities 
such as gluon and flavor-singlet matrix elements, and
for the direct calculation of the 
Bjorken-$x$ dependence of PDFs~\cite{Lin:2014zya,Alexandrou:2015rja,
Chen:2016utp,Alexandrou:2016jqi}.

These developments have pushed lattice-QCD calculations to the point where, 
for the first time, it is possible to provide information on the PDF shape
of specific flavor combinations, both for quarks and for antiquarks, 
and where meaningful comparisons with global fits can be made.
%
Indeed, one of the main motivations for these lattice-QCD efforts is to 
achieve a sufficient accuracy to constrain the PDFs obtained from global 
analyses.

Despite these developments in both the global QCD analysis and lattice-QCD 
methods, interplay between the two --- and communication between the 
respective communities of physicists --- have been rather limited so far.
%
This situation led some of us to organize the first workshop on
{\it Parton Distributions and Lattice Calculations in the LHC Era}
(PDFLattice2017), which took place in Balliol College, University of 
Oxford, in March 2017.\footnote{\url{http://www.physics.ox.ac.uk/confs/PDFlattice2017/index.asp}}
%
The main goal of this workshop was to establish a common ground 
and language for discussions between the two communities.
%
In addition, we aimed to carry out a first quantitative exploration of how PDF 
fits can be exploited to benchmark existing and future lattice calculations,
and of how lattice-QCD calculations could be used to improve global PDF fits.
%
In this context, some of the questions that were addressed during this workshop
included the following.
\begin{itemize}
\item What information from PDF fits is relevant to constrain, 
  test, or validate lattice calculations?

\item What PDF-related quantities are most compelling
  to compute in lattice QCD in terms of phenomenological relevance?

\item What accuracy do we need from lattice quantities 
  in order to have a significant impact on global PDF fits?

\item What information does lattice QCD provide on the
  shape (Bjorken-$x$ dependence) of the PDFs? Which specific
  PDF moments can be computed?
  
\item How do we consistently quantify the systematic uncertainties 
  in lattice-QCD calculations?

\item To what extent do available lattice results agree with the results of
  global PDF fits? Is there a tension between global PDF fits, PDF
  fits based on reduced datasets, and PDF calculations from the lattice?

\item What is the accuracy that can be expected from lattice-QCD
  calculations in the near and medium future? What will be their
  constraining power on PDFs?

\end{itemize}

This white paper summarizes the joint effort between the two communities to 
address some of these questions, and follows up on the very fruitful 
discussions and interactions that took place both during 
the workshop and in the subsequent months.
%
While this document does not represent the final word on this topic, it 
provides a solid starting point for further collaborative efforts, and 
should facilitate smooth interactions between the two communities in the future.

The outline of this white paper is the following.
%
In Sec.~\ref{sec:theoryoverview} we review the global QCD analysis and 
lattice-QCD methods for the determination of polarized and unpolarized PDFs.
%
In Sec.~\ref{sec:benchmarking} we present state-of-the-art benchmarks 
for selected PDF moments between the most recent lattice-QCD calculations and 
global QCD analyses.
%
In Sec.~\ref{sec:projections} we quantitatively assess the impact that
lattice calculations of PDF-related quantities could have on unpolarized
and polarized global analyses, assuming different scenarios for the 
uncertainties in the lattice-QCD calculations.
%
In Sec.~\ref{sec:outlook} we conclude
and discuss future interactions between
the global-analysis and lattice-QCD communities.
%
In Appendix~\ref{app:notation} we summarize the conventional notation
adopted in this document for the definition of the PDF moments; 
in Appendix~\ref{sec:LQCDtables} we compile bibliographical tables for
existing lattice-QCD calculations of PDF moments;
and in Appendix~\ref{app:Hmoms} we collect some
additional results of PDF moments from global QCD analyses.

