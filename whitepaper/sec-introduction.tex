\section{Introduction and motivation}

The detailed understanding of the internal structure of protons is an active
research field with important phenomenological implications in a wide variety of fields
from high energy physics, nuclear physics and astroparticle physics.
%
There are two main methods to quantify how the momentum and the spin of nucleons
is divided among its constituents, the quarks and the gluons.
%
The first is the {\it global QCD analysis}, which exploits that fact that hard-scattering
cross-sections can be factorized into short-distance matrix elements, which can
be calculated in perturbation theory, and long-distance dynamics encoded in the
so-called {\it Parton Distribution Functions} (PDFs).
%
By combining therefore a variety of experimental data on hard-scattering processes
with state-of-the-art perturbative calculations by means of a robust statistical
methodology, it is possible to achieve a reasonably precise determination
of these non-perturbative PDFs for unpolarized and polarized nucleons.
%
In this context, several collaborations provide regular updates of
phenomenological PDF determinations, both
in the unpolarized~\cite{Ball:2012cx,Ball:2014uwa,Harland-Lang:2014zoa,
Dulat:2015mca,Alekhin:2017kpj,Owens:2012bv} and in
the polarized~\cite{Nocera:2014gqa,deFlorian:2009vb} cases.

The second approach is based on the direct computation of PDFs from first principles,
starting from the formal definition of PDFs and their moments as specific operators that
should be evaluated as the matrix elements of nucleon wave functions.
%
This method, known as {\it Lattice QCD}, is able to compute non-perturbative quantities
by means of a discretrization of the QCD action followed by a extrapolation
to the continuum limit.
%
The main advantage of lattice QCD calculations is that they require minimal external input
other than $m_p$, $\alpha_s(m_Z)$ and the values of the heavy quark masses, and in particular
they don't use the input from experimental measurements.
%
The drawback is that until recently the calculation of all except for the simplest
matrix elements, in particular the first moments of non-singlet PDF combinations,
was feasible due to CPU limitations.

In the recent times there has been very significant progress in our understanding
of the polarized and unpolarized partonic structure of the nucleon both
from the global PDF fitting and from the lattice QCD communities.
%
In the first case, the availability of a wealth of high-precision collider measurements
from HERA, the Tevatron and the LHC, together with that of the corresponding
NNLO QCD and NLO electroweak perturbative calculations, are pushing the
accuracy frontier and lead to improved PDFs with reduced uncertainties.

A paradigmatic illustration of this progress is provided by the gluon PDF, which has been
typically affected by rather large uncertainties due to the limited experimental
information, but is known nowadays rather more precisely from small to large-$x$
thanks to the inclusion in the
PDF fit of novel processes such as $D$ meson production~\cite{Gauld:2016kpd},
the transverse momentum of $Z$ bosons and top-quark pair differential
distributions~\cite{Czakon:2016olj}.
%
In the case of lattice QCD calculations, recent progress has been driven both
by improved numerical methods to calculate traditional quantities such as nucleon
matrix elements (moments of the PDFs) as well as the development of novel
strategies to compute directly the $x$ dependence of PDFs by means of the
introduction of {\it quasi-PDFs}.
%
These developments imply that, for the first time, it is possible to provide information on the PDF shape
of specific flavour combinations, both for quarks and for antiquarks.

Despite these rather important developments, the cross-talk between the
two communities has been so far very limited. This situation lead some of us
to organize the first edition of the {\it ``Parton Distributions and Lattice Calculations in the LHC Era''}
workshop (PDFLattice2017), which took
place in Balliol College (Oxford) during the 22nd, 23rd and 24th of March 2017.
%
The scope of this workshop was to create a common ground for the discussions
between the two communities, ensuring that we spoke the same language,
and starting to consider in a quantitative way how lattice QCD calculations could be used
to improve global PDF fits, and conversely, how PDF fits can be exploited to benchmark lattice calculations.
%
Some of the questions that we proposed ourselves to address during the workshop
included:
\begin{itemize}
\item What information from PDF fits is relevant to constrain/test/validate lattice calculations?

\item What PDF-related quantities are more urgent
  to compute lattice QCD, in terms of phenomenological relevance?

\item What information does lattice QCD provide on the
  shape (Bjorken-$x$ dependence) of the PDFs? Which specific
  PDF moments
  and up to which order can be computed?
  
\item Can we quantify the accuracy that lattice calculations should have
  in order to have a direct impact on global PDF fits?

\item To which extent available lattice results agree with the results of
  global PDF fits? Is there a tension between global PDF fits, PDF
  fits based on reduced datasets, and PDF calculations from the lattice?

  \item What is the ultimate accuracy that can be expected on lattice
calculations say within 5 years from now? What is the ultimate
constraining power of lattice calculations on PDFs?

\end{itemize}

This white-paper represents the effort of the two communities trying to address most
of the above questions, following a very fruitful cross-talks and interactions that took
place during the PDFLattice2017 workshop and the discussions in the subsequent weeks.
%
It does not represent our final word, but rather a motivation to encourage further
interactions between the two communities.

The outline of this documents is the following.
%
In Sect.~\ref{sec:theoryoverview} we present an overview of
the global QCD fit and lattice QCD methods for the evaluation
of polarized and unpolarized parton distributions.
%
Then in Sect.~\ref{sec:benchmarking}
we summarize  state-of-the-art benchmark
calculations between the most
recent available lattice QCD calculations and global PDF fits,
both in terms of moments and of $x$-space PDFs by means of
quasi-PDFs.
%
In Sect.~\ref{sec:projections} we quantify the impact that
upcoming lattice calculations could have in unpolarized
and polarized PDFs, and estimate the impact of these improvements
in collider phenomenology.
%
Finally, in Sect.~\ref{sec:outlook} we summarize our studies
and discuss the outlook for future interactions between
the global QCD fit and lattice QCD communities.
