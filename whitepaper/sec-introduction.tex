%%%%%%%%%%%%%%%%%%%%%%%%%%%%%%%%%%%%%%%%%%%%%%%%%%%%%%%%%%%%%%%%%%%%%%%%%%%%%%%%
\section{Introduction and motivation}
%%%%%%%%%%%%%%%%%%%%%%%%%%%%%%%%%%%%%%%%%%%%%%%%%%%%%%%%%%%%%%%%%%%%%%%%%%%%%%%%

The internal structure of nucleons is an active area of research
with important phenomenological implications for
high-energy, nuclear, and astroparticle physics.
%
There exist two main methods to quantify the parton distribution functions (PDFs)
of nucleons that determine how their momentum and spin 
are divided among their quarks and gluons constituents.
%
The first one
is the {\it global QCD analysis}, which exploits the fact that hard-scattering
cross sections can be factorised into the short-distance matrix elements
computable with perturbation theory and the long-distance
non-perturbative dynamics encoded in the PDFs.
%
By combining a wide variety of experimental data on hard-scattering processes
from lepton-proton and proton-proton collisions
with state-of-the-art perturbative calculations,
the PDFs of the proton can be determined
both in the unpolarized and polarized cases,
see~\cite{Perez:2012um,DeRoeck:2011na,Alekhin:2011sk,Ball:2012wy,
Forte:2013wc,Jimenez-Delgado:2013sma,Rojo:2015acz,Butterworth:2015oua,
Accardi:2016ndt,Gao:2017yyd}
for recent reviews.
%
Several collaborations provide regular updates of
their PDF determinations, both
within unpolarized~\cite{Ball:2017nwa,Harland-Lang:2014zoa,
  Dulat:2015mca,Alekhin:2017kpj,Accardi:2016qay} and polarized~\cite{Nocera:2014gqa,deFlorian:2009vb,
  Sato:2016tuz,Hirai:2008aj} global analyses.

Alternatively, one can compute quantities related to PDFs directly 
from QCD, without resorting to
experimental data, by using the non-perturbative formulation known as
{\it lattice-QCD}~\cite{Olive:2016xmw,Gupta:1997nd}.
%
This method introduces an ultraviolet cutoff that allows the direct 
computation of the QCD path integral in a discretised finite-volume 
Euclidean space-time.
%
In order to connect with experimental measurements, extrapolations to the 
continuum and infinite-volume limits are necessary in order to remove the  
cutoff dependence and finite-volume effects, respectively.

One of the main
conceptual advantages of lattice-QCD calculations is that
they require minimal external input: one needs only to 
set the hadronic scale $\Lambda_\text{QCD}$ and the values of the quark masses.
%
For calculations relevant to low-energy hadron structure, this means
setting the up, down and strange quark masses,
which is usually done using the pion and kaon masses as external inputs.
%
The overall hadronic scale can be set using well-determined baryon masses 
such as that of the $\Omega$ baryon.
%
A variety of important
QCD quantities can then be computing using lattice-QCD, including 
several ones that are challenging to obtain directly from experiment. 

From the perspective of global PDF analyses, the availability of a wealth of 
high-precision collider measurements from Jefferson Lab, HERA, RHIC, 
the Tevatron and the LHC, together with that of the corresponding
NNLO QCD and NLO electroweak perturbative calculations, are pushing the
accuracy frontier and leading to improved PDFs with reduced uncertainties,
in many cases at the few-percent level.
%
A paradigmatic illustration of this progress is provided by both the 
unpolarised and polarised gluon PDFs, which
until recently were affected by rather 
large uncertainties due to the limited experimental information
available.
%
In modern PDF fits, however, the unpolarised gluon is constrained quite 
accurately from small to large-$x$ thanks to the inclusion of processes such as $D$-meson
production~\cite{Zenaiev:2015rfa,Gauld:2016kpd},
the transverse momentum of $Z$ bosons~\cite{Boughezal:2017nla},
inclusive jet production~\cite{Currie:2016bfm}, and top-quark pair
distributions~\cite{Czakon:2016olj,Guzzi:2014wia}.
%
The polarised gluon is also constrained from double spin-asymmetries for 
high-$p_T$ jet and pion production in proton-proton 
collisions~\cite{deFlorian:2014yva,Nocera:2014gqa}, 
although only in the medium-to-large $x$ region.

In the case of lattice-QCD calculations, early attempts to determine
the nucleon PDFs were 
limited by the available computational resources and various technical 
challenges, with most results restricted to
the first few moments of non-singlet PDFs at relatively large (unphysical) quark masses.
%
Overcoming these limitations, recent progress has been mostly
driven by advances in two main areas. 
%
First, by improved systematic control (physical pion mass, excited-state 
contamination, large volumes) for quantities such as the nucleon matrix 
elements corresponding to the low moments of PDFs.
%
Second, by the  development of novel strategies
for the computation of the first few 
moments~\cite{Constantinou:2014tga,Syritsyn:2014saa,Lin:2012ev},
the determination of more challenging quantities 
such as gluon and flavour-singlet matrix elements, and
for the direct calculation of the 
Bjorken-$x$ dependence of PDFs~\cite{Lin:2014zya,Alexandrou:2015rja,
Chen:2016utp,Alexandrou:2016jqi}.

These developments have pushed lattice-QCD calculations to the point where, 
for the first time, it is possible to provide information on the PDF shape
of specific flavour combinations, both for quarks and for antiquarks, 
and where meaningful comparison with global fits can be made.
%
Indeed, one of the main motivations for these lattice-QCD efforts is to reach a point where 
the calculations can be used to constrain the PDFs obtained from global analyses.

Despite these exciting developments from both the lattice-QCD
and the global fitting approaches to
the nucleon structure, the communication between the
two communities has been so far rather limited.
%
This situation led some of us
to organise the first edition of the
``{\it Parton Distributions and Lattice Calculations in the LHC Era}''
workshop (PDFLattice2017), which took
place in the Balliol College of the University of Oxford in March
2017.\footnote{\url{http://www.physics.ox.ac.uk/confs/PDFlattice2017/index.asp}}
%
The main goal of this workshop was to create a common ground for discussions
between the two communities and ensure that we spoke the same language.
%
In addition, we aimed to carry out a first quantitative exploration
of how lattice-QCD calculations 
could be used to improve global PDF fits, and conversely, of how PDF fits 
can be exploited to benchmark existing and future lattice calculations 
in a systematic way.

In this context, some of the questions that were addressed during this workshop
included:
\begin{itemize}
\item What information from PDF fits is relevant to constrain, 
  test, or validate lattice calculations?

\item What PDF-related quantities are most urgent
  to compute in lattice-QCD in terms of phenomenological relevance?

\item What accuracy do we need from lattice quantities 
  in order to have a significant impact on global PDF fits?

\item What information does lattice-QCD provide on the
  shape (Bjorken-$x$ dependence) of the PDFs? Which specific
  PDF moments, and up to which order, can be computed?
  
\item How do we systematically and consistently quantify the systematic errors 
in  lattice-QCD calculations of the proton structure?

\item To what extent do available lattice results agree with the results of
  global PDF fits? Is there a tension between global PDF fits, PDF
  fits based on reduced datasets, and PDF calculations from the lattice?

\item What is the accuracy that can be expected from lattice-QCD
  calculations in the near and medium future? What will be their
  constraining power on PDFs?

\end{itemize}

This white paper summarises the joint
effort between the two communities to address some of these questions, 
and follows up on the very fruitful discussions and interactions that took place both during 
the Oxford workshop and in the subsequent months.
%
While this document does not represent the final word on this topic,
it provides a solid starting point for subsequent collaborative efforts,
and should facilitate smooth 
interactions between the two communities in the near future.

The outline of this white paper is the following.
%
First of all, in Sect.~\ref{sec:theoryoverview} we review
the global PDF analysis and lattice-QCD methods for the evaluation
of polarised and unpolarised parton distributions.
%
Then, in Sect.~\ref{sec:benchmarking}
we present state-of-the-art benchmarks for the most
recent lattice-QCD calculations and compare them with the results from global fits
for selected PDF moments.
%
Sect.~\ref{sec:projections} provides a first quantitative
exploration of the impact that
lattice calculations of PDF-related quantities could have on unpolarised
and polarised global analyses within different scenarios
for the uncertainties in the lattice-QCD calculations.
%
In Sect.~\ref{sec:outlook} we summarise our studies
and discuss the outlook for future interactions between
the global QCD fit and lattice-QCD communities.
%
Finally, appendix~\ref{app:notation} summarises the conventions
that are adopted in this report for the definition of the PDF
moments, a
Appendix~\ref{sec:LQCDtables} presents an exhaustive summary of existing calculations
of those moments in lattice-QCD,
and then Appendix~\ref{app:Hmoms} collects some
additional results for PDF moments from the global analysis.

%%%%%%%%%%%%%%%%%%%%%%%%%%%%%%%%%%%%%%%%%%%%%%%%%%%%%%%%%%%%%%%%%%%%%5
%%%%%%%%%%%%%%%%%%%%%%%%%%%%%%%%%%%%%%%%%%%%%%%%%%%%%%%%%%%%%%%%%%%%%5
