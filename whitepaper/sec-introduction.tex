%%%%%%%%%%%%%%%%%%%%%%%%%%%%%%%%%%%%%%%%%%%%%%%%%%%%%%%%%%%%%%%%%%%%%%%%%%%%%%%%
\section{Introduction and Motivation}
%%%%%%%%%%%%%%%%%%%%%%%%%%%%%%%%%%%%%%%%%%%%%%%%%%%%%%%%%%%%%%%%%%%%%%%%%%%%%%%%

The detailed understanding of the internal structure of protons is an active
research field with important phenomenological implications in a wide variety of fields
such as high-energy, nuclear physics and astroparticle physics.
%
There are two main methods to quantify how the momentum and spin of nucleons
are divided among its constituents, the quarks and gluons.
%
The first is global QCD analysis, which exploits that fact that hard-scattering
cross sections can be factorized into short-distance matrix elements, which can
be calculated in perturbation theory, and long-distance dynamics encoded in the
so-called parton distribution functions (PDFs).
%
By combining a variety of experimental data on hard-scattering processes
with state-of-the-art perturbative calculations by means of a robust statistical
methodology, it is possible to achieve a reasonably precise determination
of these nonperturbative PDFs for unpolarized and polarized nucleons.
%
In this context, several collaborations provide regular updates of
phenomenological PDF determinations, both
in the unpolarized~\cite{Ball:2012cx,Ball:2014uwa,Harland-Lang:2014zoa,
Dulat:2015mca,Alekhin:2017kpj,Owens:2012bv} and in
the polarized~\cite{Nocera:2014gqa,deFlorian:2009vb} cases.

Alternatively, once can compute quantities related to PDFs directly from the nonperturbative sector of the Standard Model, using the first-principles method known as lattice QCD. 
This method introduces an ultraviolet cutoff allowing 
the direct computation of the QCD path integral in discretized 
Euclidean spacetime. To connect to experimental observations, a continuum-limit extrapolation is necessary to remove the cutoff dependence and finite-volume effects.
%
The main advantage of lattice-QCD calculations is that they require minimal external input:
some way to set the quark masses, usually pion and kaon masses to set the up/down and strange,
and some way to set the lattice scale, sometimes by a well determined baryon mass. 
Many QCD quantities can be predicted by lattice QCD, including many quantities that are challenging to obtain from experiments. 
In its early days, lattice-QCD progress on PDFs was limited by available computational resources,
with most results concentrating on the first couple moments of non-singlet PDFs at relatively heavy quark masses. 
In recent years, multiple collaborations work on computations directly at the physical pion mass, which 
dramatically reduces one of the biggest systematic uncertainties, extrapolation in quark masses. In addition, notoriously difficult quantities, gluon contributions and flavor-singlet nucleon matrix elements, have now been calculated as well. Recently, there are also developments in overcoming the limitations in computing the first few moments~\cite{Constantinou:2014tga,Syritsyn:2014saa,Lin:2012ev} and even making direct calculations of the Bjorken-$x$ dependence of PDFs using large-momentum effective theory (LaMET)~\cite{Ji:2014gla}, computing quasi-PDFs within nucleons having finite momentum boost~\cite{Lin:2014zya,Alexandrou:2015rja,Chen:2016utp,Alexandrou:2016jqi}.

Recently, there has been very significant progress in our understanding
of the polarized and unpolarized partonic structure of the nucleon from both
the global PDF fitting and lattice-QCD communities.
%
In the first case, the availability of a wealth of high-precision collider measurements
from HERA, the Tevatron and the LHC, together with that of the corresponding
NNLO QCD and NLO electroweak perturbative calculations, are pushing the
accuracy frontier and leading to improved PDFs with reduced uncertainties.

A paradigmatic illustration of this progress is provided by the gluon PDF, which has been
affected by rather large uncertainties due to the limited experimental
information, but is known nowadays rather more precisely from small to large-$x$
thanks to the inclusion in the
PDF fit of novel processes such as $D$-meson production~\cite{Gauld:2016kpd},
the transverse momentum of $Z$ bosons and top-quark pair differential
distributions~\cite{Czakon:2016olj}.
%
In the case of lattice-QCD calculations, recent progress has been driven both
by better systematic control (physical pion mass, excited-state contamination) for the calculated traditional quantities such as nucleon
matrix elements (moments of the PDFs) as well as the development of novel
strategies to compute directly the $x$ dependence of PDFs by means of the
introduction of quasi-PDFs.
%
These developments imply that, for the first time, it is possible to provide information on the PDF shape
of specific flavour combinations, both for quarks and for antiquarks.

Despite these rather important developments, the cross-talk between the
two communities has been so far very limited. This situation led some of us
to organize the first edition of the ``Parton Distributions and Lattice Calculations in the LHC Era''
workshop (PDFLattice2017), which took
place in Balliol College (Oxford) during the 22nd, 23rd and 24th of March 2017.
%
The scope of this workshop was to create a common ground for the discussions
between the two communities, ensuring that we spoke the same language,
and starting to consider in a quantitative way how lattice-QCD calculations could be used
to improve global PDF fits, and conversely, how PDF fits can be exploited to benchmark lattice calculations.
%
Some of the questions that we proposed ourselves to address during the workshop
included:
\begin{itemize}
\item What information from PDF fits is relevant to constrain, test or validate lattice calculations?

\item What PDF-related quantities are most urgent
  to compute in lattice QCD in terms of phenomenological relevance?

\item What information does lattice QCD provide on the
  shape (Bjorken-$x$ dependence) of the PDFs? Which specific
  PDF moments,
  and up to which order can be computed?
  
\item How do we systematically and consistently quantify the systematic errors in these lattice calculations?

\item What accuracy do we need from lattice quantities 
  in order to have a significant impact on global PDF fits?

\item To what extent do available lattice results agree with the results of
  global PDF fits? Is there a tension between global PDF fits, PDF
  fits based on reduced datasets, and PDF calculations from the lattice?

  \item What is the ultimate accuracy that can be expected from lattice
calculations 5 years from now? What is the ultimate
constraining power of lattice calculations on PDFs?

\end{itemize}

This white paper represents the effort of the two communities to address most
of the above questions, following a very fruitful discussion and interactions that took
place during the PDFLattice2017 workshop and the discussions in the subsequent weeks.
%
It does not represent our final word, but rather a motivation to encourage further
interactions between the two communities.

The outline of this document is as follows:
%
In Sec.~\ref{sec:theoryoverview} we present an overview of
the global QCD fit and lattice-QCD methods for the evaluation
of polarized and unpolarized parton distributions.
%
Then in Sec.~\ref{sec:benchmarking}
we summarize state-of-the-art benchmark
calculations between the most
recent available lattice QCD calculations and global PDF fits,
both in terms of moments and of $x$-space PDFs by means of
quasi-PDFs.
%
In Sec.~\ref{sec:projections} we quantify the impact that
upcoming lattice calculations could have on unpolarized
and polarized PDFs and estimate the impact of these improvements
on collider phenomenology.
%
Finally, in Sec.~\ref{sec:outlook} we summarize our studies
and discuss the outlook for future interactions between
the global QCD fit and lattice-QCD communities.
