\subsection{Discussion}

We conclude this section with a brief discussion of the main lessons that
can be learned from this exercise, which provides the first quantitative 
estimate of the impact of present and future lattice-QCD calculations of PDF 
moments and $x$-space PDFs, for both polarized and unpolarized PDFs.

First, we have demonstrated that in the polarized case,
even with current uncertainties, lattice-QCD calculations of
selected PDF moments can impose sizable constraints on several
important polarized quark combinations.
%
This suggests that global polarized PDF analyses should consider
including existing lattice-QCD calculations in their fits to constrain some
of the least known quark combinations, such as the total strangeness.
%
The situation is rather different in the unpolarized case,
where a reduction of the current lattice-QCD uncertainties by a factor of 
between five and ten seems to be required to influence global fits.
%
This difference arises because unpolarized PDFs are known with much higher 
precision than polarized PDFs, thanks to the much wider amount of experimental 
data sensitive to unpolarized PDFs,
including the constraints from recent high-precision measurements at the LHC.
%
Thus, in addition to the differences highlighted  in Fig.~\ref{fig:Bmomsunp},
much more precise lattice-QCD calculations than in the polarized case 
need to be used to be competitive with current PDF fits.


Second, lattice-QCD calculations of the quark isotriplet combinations
$xu-xd$ and $x\bar{u}-x\bar{d}$ would be instrumental in constraining
quark PDFs at large $x$.
%
Even a calculation with $\delta_L\simeq 10\%$ uncertainties at large-$x$ would
start to provide useful constraints on global fits.
%
Moreover, we find that, in the unpolarized case, the information on the
PDFs that could be derived from a direct $x$-space calculation
from lattice-QCD is clearly superior to the information that can be obtained
from PDF moments alone, at least for the subset of PDFs and moments used in 
the present exercise.

The profiling studies presented in this section could be extended in
a number of directions.
%
In the polarized case, one could include the current lattice-QCD
values of the moments listed in Table~\ref{tab:BMpol} in global analyses: 
indeed, we have demonstrated that at the current level of uncertainties one 
expects to find some non-trivial constraints.
%
In this respect, a crucial topic to investigate is the compatibility 
(or lack thereof) of the existing lattice-QCD numbers compared to constraints 
from experimental data.
%
For both unpolarized and polarized PDFs, it would be interesting to include the 
effects of other moments and flavor combinations.
%
Higher moments, in particular, typically probe regions of higher $x$, compared
to lower moments, and in the large-$x$ regions uncertainties in the global-fit 
PDFs are more marked.
%
One could also consider the effects of the quark combinations for which 
$x$-space calculations might be available, for example those related to the 
proton strangeness.
%
Finally, a more refined analysis should include the theoretical correlations
expected in lattice-QCD calculations, for instance, in the case of $x$-space 
calculations, one expects neighboring points in $x$ to be highly correlated.

