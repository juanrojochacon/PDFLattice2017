\subsection{Discussion}

We conclude this section with a brief discussion of the main lessons that
can be derived from this exercise.

An important conclusion is that we have demonstrated that in the polarized case,
already at the current level of uncertainties lattice-QCD calculations of
selected moments can impose sizably constraints on several
important polarized quark combinations.
%
On the other hand, the situation is rather different in the unpolarized case,
where a reduction of the lattice-QCD uncertainties between a factor 5 and 10 as compared
to the current ones seems to be required in order to start to impact the global
fit.
%
The reason of this different behaviour can be straightforwardly explained from
the fact that unpolarized PDFs are known with much higher precision than the polarized
ones, thanks to the much wider amount of experimental data sensitive to them,
including the constraints from the recent high-precision measurements at the
LHC.

The performance studies presented in this section could be extended within
a number of directions.
%
In the polarized case, one could already start to include the current lattice-QCD
values of the moments listed in Table~\ref{tab:BMpol} in the global analysis: indeed,
we have demonstrated that at the current level of uncertainties one expects
to find some non-trivial constraints.
%
Both for the unpolarized and polarized cases, it would be interesting to include the effects
of other moments and flavour combinations.
%
Specifically, higher moments probe typically regions of higher $x$ as compared
to lower moments, and there the uncertainties in the phenomenological PDFs are
more marked.


Thirdly, here we have not considered lattice-QCD pseudo-data in terms of $x$-space
quantities, such as the one that could be provided by the quasi-PDF calculations
described in Sect.~\ref{sec:xdependence}.
%
Once calculations of $x$-space PDFs with a robust control over the systematic
uncertainties can be achieved, it would be straightforward to assess their
impact on the global fit by means of the same procedure adopted here.
%
Of particular interest here is the isovector combination $u(x)-d(x)$
(and the corresponding antiquark combination $\bar{u}(x)-\bar{d}(x)$), which
has been the focus of intense research recently.
%
If this combination can be calculated within lattice-QCD at large-$x$,
say for $x\simeq 0.5$, it would
provide significant constraints in a region where the PDF uncertainties are large,
leading to another important ingredient to add to the toolbox
of global PDF fits.


