\subsection{Discussion}

We conclude this section with a brief discussion of the main lessons that
can be learned from this exercise, aiming to provide a first quantitative estimate
of the impact of present and future lattice-QCD calculations for PDF moments
and $x$-space PDFs both in the polarized and unpolarized cases.

A first important conclusion that can be derived from these
results is that we have demonstrated that in the polarized case,
already at the current level of uncertainties, the lattice-QCD calculations of
selected PDF moments can impose sizably constraints on several
important polarized quark combinations.
%
This suggest that global polarized PDF analysis should start to consider
including existing calculations in their fits to constrain some
of the worse known quark combinations such as the total strangeness.

On the other hand, the situation is rather different in the unpolarized case,
where a reduction of the lattice-QCD uncertainties between a factor 5 and 10 as compared
to the current ones seems to be required in order to start to impact the global
fit.
%
The reason of this different behaviour can be straightforwardly explained from
the fact that unpolarized PDFs are known with much higher precision than the polarized
ones, thanks to the much wider amount of experimental data sensitive to them,
including the constraints from the recent high-precision measurements at the
LHC.
%
Thus much more precise lattice-QCD calculations than in the polarized case
need to be used to be competitive with current PDF fits.

Another important conclusion of these initial studies is that
a lattice-QCD calculation of the quark isotriplet combinations
$xu-xd$ and $x\bar{u}-x\bar{d}$ would be instrumental to constrain
the quark PDFs at large-$x$.
%
Already a calculation with $\delta_L\simeq 10\%$ uncertainties at large-$x$ would
start to provide useful constrains on the global fit.
%
Moreover, we also find that in the unpolarized case the information on the
PDFs that could be derived from a direct $x$-space calculation
from lattice-QCD is clearly superior to the one that can be obtained
from PDF moments, at least based on the subset used in the present
exercise.

The performance studies presented in this section could be extended within
a number of directions.
%
In the polarized case, one could already start to include the current lattice-QCD
values of the moments listed in Table~\ref{tab:BMpol} in the global analysis: indeed,
we have demonstrated that at the current level of uncertainties one expects
to find some non-trivial constraints.
%
In this respect, a crucial topic to investigate will be the compatibility (or lack thereof)
of the existing lattice numbers as compared to the constrains from experimental data.
%
Both for the unpolarized and polarized cases, it would be interesting to include the effects
of other moments and flavour combinations.
%
Specifically, higher moments probe typically regions of higher $x$ as compared
to lower moments, and there the uncertainties in the phenomenological PDFs are
more marked.
%
Thirdly, one could consider the effects of ther quark combinations for which $x$-space
calculations might be available, for example those related to the proton strangeness.
%
Finally, future more refined analysis should include the theoretical correlations
expected from a lattice calculation, for instance in the case of the $x$-space calculation,
where neighboring points in $x$ are expected to be highly correlated.

%%%%%%%%%%%%%%%%%%%%%%%%%%%%%%%%%%%%%%%%%%%%%%%%%%%




