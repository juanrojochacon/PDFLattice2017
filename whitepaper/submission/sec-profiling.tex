\subsubsection{Hessian profiling analysis}
\label{sec:hessianprofiling}

To complement the results obtained
with the Bayesian reweighting approach,
we use a profiling method, suitable
for Hessian PDF sets, to estimate the effect of including
lattice-QCD pseudo-data into the fit~\cite{Paukkunen:2014zia,Camarda:2015zba}.
%
We choose HERAPDF2.0~\cite{Abramowicz:2015mha}
as a representative set of Hessian PDFs.
%
As in the case of the Bayesian reweighting
exercise presented in the previous section
we consistently use the same lattice-QCD
pseudo-data on PDF moments to estimate the impact on HERAPDF2.0.
%
An additional advantage of the HERAPDF2.0 set is
the use
of a standard tolerance
$\Delta\chi^2=1$ for defining the 68\%-CL PDF
uncertainties,
%%%% RST , which provides a correspondence between the profiling and reweighting methods.
which enables a robust framework for applying the profiling method. 


For Hessian PDF sets, the Hessian profiling method
can be used to both check the compatibility of new data with a given PDF set,
and also  estimate the impact these data will have on the PDFs. 
In the following we describe the essential components of the profiling method, 
and assume  that the  Hessian PDF set uses a tolerance of $\Delta\chi^2=1$, 
which corresponds to 68\%~CL uncertainties,
as is the case with the HERAPDF2.0 set.\footnote{In this exercise
we consider only the {\it experimental} HERAPDF2.0
uncertainties, but not the {\it model} and {\it parametrization}
variations, which are not suited for profiling.}
%
The central values of the considered moments are obtained using the central PDFs and the corresponding
errors are calculated according to:
\begin{equation}
\delta\mathcal{F}_i = \frac{1}{2} \sqrt{\sum_{k}\left(\mathcal{F}_i(f_k^+)-\mathcal{F}_i(f_k^-)\right)^2}\, ,
\quad i=1,\ldots,N_\text{mom} \, ,
\end{equation}
where $k$ labels the number of error PDFs (Hessian eigenvectors)
which have both a positive and negative direction.
%
In the profiling method, one considers a $\chi^2$ function in which the $\chi^2$ of the new
data has been added to the initial $\chi^2_0$, namely
\begin{equation}
\label{eq:newchi2}
\chi^2_{\text{new}} = \chi_0^2 + \sum_{k}^{N_{\text{eig}}} z_k^2
                    + \sum_{i=1}^{N_{\text{data}}}
                      \frac{\lp \mathcal{F}_i - \mathcal{F}_i^{\rm(exp)}\rp^2}
                           {\lp\delta\mathcal{F}_i^{\rm(exp)}\rp^2}\,,
\end{equation}
where $\chi^2_0$ is the value of the $\chi^2$ function in the minimum of the initial PDF set,
$z_k$ are the parameters diagonalizing the Hessian matrix of the initial PDF set,
$N_{\text{eig}}$ is the dimension of the eigenvector space in which initial Hessian errors are defined
(half of the number of error PDFs), $\mathcal{F}_i^{\rm(exp)}$ is the new
\hbox{(pseudo-)data},
and $\mathcal{F}_i$ the corresponding theory prediction.

In the spirit of the Hessian method, the new theory predictions $\mathcal{F}_i$ can be expanded
using a linear approximation:
\begin{equation}
\mathcal{F}_i \simeq \mathcal{F}_i[S_0] + \sum_k \frac{\partial\mathcal{F}_i[S]}{\partial z_k}\bigg|_{S=S_0} z_k \quad
              \simeq \mathcal{F}_i[S_0] + \sum_k D_{ik} w_k \ ,
\end{equation}
where $S_0$ represents the central PDF and we have defined
\be
D_{ik}=\frac{1}{2}(\mathcal{F}_i[S_k^+]-\mathcal{F}_i[S_k^-]) \, ;
\ee
here the  derivative has been approximated by a finite difference of the 
Hessian PDF error sets $S_k^{\pm}$.
%
The new $\chi^2$ of Eq.~\eqref{eq:newchi2} can now be minimized with respect to the parameters $w_k$,
which results in:
\begin{equation}
%\boldsymbol{\vec{w}_{\text{{\bf min}}}} = \boldsymbol{-B^{-1}} \boldsymbol{\vec{a}},
%
w_k^{min}  = \sum_n \ -B_{kn}^{-1} \, a_n \quad ,
\end{equation}
where we have introduced
\begin{equation}
%\begin{split}
B_{kn} = \sum_i \frac{D_{ik}D_{in}}{\lp\delta\mathcal{F}_i^{\rm(exp)}\rp^2} + \delta_{kn},
%\\
\qquad
\qquad
a_k = \sum_i \frac{D_{ik}(\mathcal{F}_i[S_0] - \mathcal{F}_i^{\rm(exp)})}{\lp\delta\mathcal{F}_i^{\rm(exp)}\rp^2} \, . 
%\end{split}
\end{equation}

The key result of the Hessian profiling method
is that now the components of the solution 
%$\boldsymbol{\vec{w}_{\text{{\bf min}}}}$ 
$w_k^{min}$
define a new set
of PDFs representing a global minimum after including the new data:
\begin{equation}
f_{\text{new}} = f_{S_0} + \sum_{k=1}^{N_{\text{eig}}} \frac{f_{S_k^+}-f_{S_k^-}}{2} w_k^{\text{min}} \ .
\end{equation}
%At the same time 
%%$\boldsymbol{\vec{w}_{\text{{\bf min}}}}$ 
%$w_k^{min}$
%also  defines  a penalty term 
%\begin{equation}
%P = \sum_{k=1}^{N_{\text{eig}}} \lp w_k^{\text{min}} \rp^2 \, ,
%\end{equation}
%which can be used to estimate whether the new data is consistent with the initi%al set of PDFs.
%%
%Specifically, a value of
%the penalty term of $P\ll1$ means that the new data is consistent
%with that included in the original fit.
%%
A set of new error PDFs can be also defined; in this case the matrix $B_{kn}$ plays the role of
the Hessian matrix from which the PDF uncertainties
can be obtained. 

We performed this study using the xFitter program~\cite{Alekhin:2014irh}
assuming the same three scenarios for the lattice-QCD pseudo-data as 
in Table~\ref{tab:scenarios}. 
%
The results are shown in Table~\ref{tab:unpolmomentsProf}, where we tabulate 
the uncertainties of the input HERAPDF2.0 PDF in column two and the 
corresponding uncertainties for each scenario in columns three to five. 
%
The analogous results from the reweighting method, applied to the 
NNPDF3.1 data set, were listed in Table~\ref{tab:unpolmomentsrw}.

%-------------------------------------------------------------------------------
\begin{table}[!t]
\centering
\footnotesize
\renewcommand{\arraystretch}{1.3} 
\begin{tabular}{lcccc}
\toprule 
&  Original  & Scenario A  &  Scenario B  &  Scenario C  \\
\midrule
  $\la x\ra_{u^+}$     
&  $0.3720\pm 0.0036$  
&  $0.3720\pm 0.0030$  
&  $0.3720\pm 0.0027$  
&  $0.3720\pm 0.0020$ \\
  $\la x\ra_{d^+}$     
&  $0.1845\pm 0.0053$  
&  $0.1845\pm 0.0028$  
&  $0.1845\pm 0.0023$  
&  $0.1845\pm 0.0015$ \\
  $\la x\ra_{s^+}$     
&  $0.0346\pm 0.0037$  
&  $0.0346\pm 0.0015$  
&  $0.0346\pm 0.0012$  
&  $0.0346\pm 0.0009$ \\
  $\la x\ra_{g}$       
&  $0.4006\pm 0.0078$  
&  $0.4006\pm 0.0042$  
&  $0.4006\pm 0.0035$  
&  $0.4006\pm 0.0024$ \\
  $\la x\ra_{u^+-d^+}$ 
&  $0.1875\pm 0.0074$  
&  $0.1875\pm 0.0045$  
&  $0.1875\pm 0.0039$  
&  $0.1875\pm 0.0027$ \\
\bottomrule
\end{tabular}
\caption{\small Values of the unpolarized PDF moments
  used as lattice-QCD pseudo-data, as well as the corresponding results
  after the profiling  for the
three scenarios summarized in Table~\ref{tab:scenarios}.
%
The HERAPDF2.0 PDFs were used, and the PDF uncertainties quoted correspond in all cases to 68\%~CL intervals.
%
The corresponding results of applying the reweighting method
to NNPDF3.1 were listed in Table~\ref{tab:unpolmomentsrw}.
\label{tab:unpolmomentsProf}
}
\end{table}
%-------------------------------------------------------------------------------

From a comparison of the constraining power of the lattice-QCD pseudo-data  
displayed in Table~\ref{tab:unpolmomentsProf} to Table~\ref{tab:unpolmomentsrw},
we observe a consistent trend between Bayesian reweighting of NNDPF3.1 and 
Hessian profiling of HERAPDF2.0.
%
The PDF uncertainties for $\la x\ra_{d^+}$ ($\la x\ra_{s^+}$
and  $\la x\ra_g$) reduce by a factor of roughly
four (four and three, respectively) compared to the original
HERAPDF2.0 uncertainties.
%
When comparing with Sec.~\ref{sec:projections:rw},
the initial uncertainties of the HERAPDF2.0  analysis 
are affected by the choice of data (DIS data only), and 
the number and form of the parametrization (14 parameter HERAPDF form);
the final uncertainties are determined by the profiling procedure. 
%
In particular the profiling for the HERAPDF2.0 study assigns an effective 
uncertainty on the pseudodata corresponding to $\Delta\chi^2=1$, whereas the 
constraint in the NNPDF study is weaker, as it would be for a PDF set with 
eigenvectors, but which applied a tolerance criterion. 
%
While these initial studies are instructive, 
further comparisons of these analyses would be valuable. 

In Fig.~\ref{fig:pdfsProf} we present a comparison of the
$u^+$, $d^+$, $g$, and $s^+$ PDFs at the scale of $Q^2=4\text{ GeV}^2$
between the original  HERAPDF2.0 set and the results of the profiling
exercise for Scenarios~A, B and C.
%
Only the {\it experimental} PDF uncertainties are shown in this comparison,
but not the {\it model} and {\it parametrization} variations.
%
The corresponding results based on the reweighting
of NNPDF3.1 were shown in Figs.~\ref{fig:impactUnpol}
and~\ref{fig:impactUnpollargex}.

%-------------------------------------------------------------------------------
\begin{figure}[!t]
\centering
\includegraphics[width=0.45\textwidth]{plots/ratio_uPubar_Q2.pdf}
\includegraphics[width=0.45\textwidth]{plots/ratio_dPdbar_Q2.pdf}\\
\includegraphics[width=0.45\textwidth]{plots/ratio_g_Q2.pdf}
\includegraphics[width=0.45\textwidth]{plots/ratio_sPsbar_Q2.pdf}
\caption{\small Comparison
of the $u^+$, $d^+$, $g$, and $s^+$ PDFs at the scale of $Q^2=4\text{ GeV}^2$
between the original  HERAPDF2.0 set and the results of the profiling
exercise accounting for the constraints of
the lattice-QCD moments
pseudo-data in Scenarios~A, B and C.
%
Only the {\it experimental} PDF uncertainties are shown,
but not the {\it model} and {\it parametrization} variations.
}
\label{fig:pdfsProf}
\end{figure}
%-------------------------------------------------------------------------------

From Fig.~\ref{fig:pdfsProf} we see that, as expected, the impact of the 
lattice pseudo-data is greatest in the medium and large-$x$ regions.
%
The precise impact on the PDFs is rather
similar for the three scenarios, with the most optimistic
Scenario~C leading to the largest reduction in uncertainties.
%
The quark flavor combinations that are most affected by the
lattice-QCD pseudo-data are the $d^{+}$ and $s^{+}$ PDFs,
and, to a lesser extent, the gluon PDF.
%
The improvement in the PDF uncertainties for $d^{+}$ and $s^{+}$
occurs because the DIS data
used in HERAPDF2.0 include only limited constraints
on quark flavor separation, and, for these PDFs, the lattice-QCD 
pseudo-data add important new information.

