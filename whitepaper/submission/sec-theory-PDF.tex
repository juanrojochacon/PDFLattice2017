\subsection{Parton distribution functions}
\label{Sec:IntroPDFs}

Quantum chromodynamics is the non-abelian quantum field 
theory that describes the strong interaction.
%
It provides the theoretical foundation for the phenomenological ideas of 
quark model, color charge, and partons as hadron constituents.
%
The power of QCD to describe physics from the pion mass scale all the way up 
to the scale of high-energy colliders, such as the LHC, relies on the 
remarkable properties of asymptotic freedom~\cite{Gross:1973ju,Gross:1973id,
Gross:1974cs,Politzer:1974fr} and 
factorization~\cite{Collins:1987pm,Collins:1989gx}.

At high energies, or short distances, the QCD coupling is small 
and perturbation theory can accurately characterize the relevant scattering 
processes~\cite{Campbell:2006wx}.
%
At low energies, or larger distances, nonperturbative effects give rise to 
quark confinement and spontaneous chiral symmetry breaking~\cite{Gasser:1983yg}.
%
The connection between low- and high-energy dynamics is provided by QCD 
factorization theorems~\cite{Collins:1987pm,Collins:1989gx}: 
short-distance physics above the factorization scale $\mu$ is captured by 
partonic hard-scattering cross-sections calculated perturbatively as a 
power series expansion in the QCD coupling, while the 
long-distance physics below the factorization scale $\mu$ is described by 
nonperturbative quantities.
%
In a collinear, leading-twist factorization framework, these quantities are
universal ({\it i.e.} process-independent) PDFs.
%
Depending on the helicity state of the parent hadron, one usually 
distinguishes between helicity-averaged (unpolarized, henceforth)
and helicity-dependent (polarized, henceforth) PDFs.

Unpolarized PDFs are denoted as 
\begin{equation}
f(x,\mu^2)\equiv f^{\rightarrow}(x,\mu^2) + f^{\leftarrow}(x,\mu^2)\mbox{,}\qquad 
f=\{g,u,\bar{u},d,\bar{d},s,\bar{s},...\}
\,\mbox{,}
\label{eq:unpPDFs}
\end{equation}
where $x$ is the fraction
of the hadron longitudinal momentum carried by the parton,
and the sum over parton's helicities aligned along ($\rightarrow$) and 
opposite ($\leftarrow$) the parent's nucleon helicity is made explicit.
%
An additional index could be used to denote the hadronic species (proton,
neutron, pion, \dots).
%
However, we omit such a designation, as we only refer to the proton
in this paper.

At leading order (LO) in the QCD coupling series, unpolarized PDFs 
describe the probability distribution of a parton with a specified 
momentum fraction $x$.
%
The total momentum carried by each parton flavor is then given by 
the first moment of the corresponding PDF, for instance
%
\begin{align}
\int_{0}^{1}dx\ x\ \left[u(x,\mu^2)\right] 
= & {}  
\left\langle x\right\rangle _{u}(\mu^2)\,, \label{eq:umoment1}\\
\int_{0}^{1}dx\ x\ \left[u(x,\mu^2)+\bar{u}(x,\mu^2)\right] 
= & {} 
\left\langle x\right\rangle _{u^{+}}(\mu^2)\,. \label{eq:uplusmoment1}
\end{align}
%
Here, $\left\langle x\right\rangle _{u}$ is the momentum
carried by the up-quark, and $\left\langle x\right\rangle _{u^{+}}$ is
the momentum carried by the sum of up and anti-up quarks~\footnote{We always 
 refer to $q^+$ to indicate the sum of the quark and anti-quark PDFs of the 
 same flavor.},
see Appendix~\ref{app:notation} for our notational conventions.

Polarized PDFs describe the extent to which quarks and gluons 
with a given momentum fraction $x$ have their spins aligned with the spin 
direction of a fast moving nucleon in a helicity eigenstate. 
%
They are denoted as 
\begin{equation}
\Delta f(x,\mu^2) \equiv f^{\rightarrow}(x,\mu^2) - f^{\leftarrow}(x,\mu^2)
\mbox{,}\qquad f=\{g,u,\bar{u},d,\bar{d},s,\bar{s},...\}
\,\mbox{,}
\label{eq:polPDFs}
\end{equation}
where, as in Eq.~\eqref{eq:unpPDFs}, $x$ is the fractional 
momentum carried by the parton,
and the parton's spin alignment along ($\rightarrow$) or opposite 
($\leftarrow$) the polarization direction of its parent nucleon
is made explicit.

Much of the interest in polarized PDFs is related to the fact that 
their zeroth moments can be interpreted as the fractions of the proton's 
spin carried by the corresponding partons.
%
They are therefore the key to one of the most fundamental, 
but not yet satisfactorily answered questions in hadronic physics,
{\it i.e.}, how the spin of the proton is distributed among its constituents.
%
Specifically, the zeroth moments of the singlet and the gluon polarized PDFs,
\begin{align}
\Delta\Sigma(\mu^2)
& =
\sum_{q}^{N_f}\int_0^1 dx 
\left[\Delta q(x, \mu^2) + \Delta\bar{q}(x, \mu^2)\right]
\equiv
\sum_q^{N_f}\langle 1 \rangle_{\Delta q^+}(\mu^2)\,,
\label{eq:singletmom}
\\
\Delta G(\mu^2)
& =
\int_0^1 dx \Delta g(x,\mu^2)
\equiv
\langle 1 \rangle_{\Delta g}(\mu^2)
\,,
\label{eq:moments}
\end{align}
where $N_f$ is the number of active flavors,
directly contribute to the proton spin sum rule~\cite{Leader:2013jra}.

Beyond LO, PDFs are renormalization scheme-dependent 
quantities, typically worked out in the $\overline{\rm MS}$ 
scheme~\cite{tHooft:1973mfk,Weinberg:1951ss}.
%
When PDFs are convolved with the appropriate partonic hard-scattering 
cross-sections, computed in the same scheme, the corresponding physical 
observables are scheme-independent, up to subleading spurious 
terms in the perturbative expansion. 

Both unpolarized and polarized PDFs are accessible, theoretically and 
experimentally, through the forward Compton scattering amplitude
\begin{equation}
\label{eq:Compton}
T_{\mu\nu}(p,q,s) 
= 
\int {\rm d}^4\!z\, e^{iqz}  \langle p,s |T J_\mu(z) J_\nu(0)|p,s\rangle
\end{equation}
at large virtual photon momenta $q^2=-Q^2$. 
%
Here $T$ is the time-ordering operator, $J_\mu(z)$ and $J_\nu(0)$ are vector
currents at space-time points $z$ and $0$ respectively, and the 
external states are hadronic states with momentum $p$ and spin $s$.

The most general form of the Compton amplitude $T_{\mu\nu}(p,q)$ 
reads~\cite{Manohar:1992tz}
\begin{align}
T_{\mu\nu}(p,q,s) 
= {} & 
  \left(-g_{\mu\nu}+\frac{q_\mu q_\nu}{q^2}\right)\mathcal{F}_1(\omega,Q^2) 
+ \left(p_\mu-\frac{p\cdot q}{q^2}q_\mu\right) \left(p_\nu-\frac{p\cdot q}{q^2}q_\nu\right) \frac{1}{p\cdot q} \mathcal{F}_2(\omega,Q^2)
\nonumber\\ 
& {} \quad  
+ i\,\epsilon_{\mu\nu\lambda\sigma}q^\lambda s^\sigma \frac{1}{p\cdot q}\mathcal{G}_1(\omega,Q^2)
+ i\,\epsilon_{\mu\nu\lambda\sigma}q^\lambda \left(p\cdot q\, s^\sigma - s\cdot q\, p^\sigma\right) \frac{1}{(p\cdot q)^2}\mathcal{G}_2(\omega,Q^2)\,,
\label{eq:Comptampl}
\end{align}
where $\omega=2p\cdot q/q^2$ and $\mathcal{F}_1$, $\mathcal{F}_2$, 
$\mathcal{G}_1$ and $\mathcal{G}_2$ are the Compton amplitude structure 
functions.
%
They can be related to the electromagnetic structure functions
$F_1$, $F_2$, $g_1$ and $g_2$, used to parametrize the deep-inelastic 
scattering (DIS) hadronic tensor\footnote{A more
 general expression of the DIS hadronic tensor including electroweak currents
 can be worked out, see~\cite{Anselmino:1993tc,Anselmino:1992rn}.}
\begin{align}
W_{\mu\nu}(p,q,s)
= {} &
\frac{1}{4\pi}\int d^4z e^{iqz}\langle p,s |[J_\mu(z),J_\nu(0)]|p,s\rangle
\nonumber
\\
= {} &
\left(-g_{\mu\nu} +  \frac{q_\mu q_\nu}{q^2}\right) F_1(x,Q^2)
+\left( p_\mu - \frac{p\cdot q}{q^2}q_\mu \right)
 \left(p_\nu - \frac{p\cdot q}{q^2}q_\nu \right) \frac{1}{p\cdot q}
F_2(x, Q^2)
\nonumber
\\
& +i\,\epsilon_{\mu\nu\lambda\sigma}q^\lambda s^\sigma
\frac{1}{p\cdot q} g_1(x,Q^2)
+ i\,\epsilon_{\mu\nu\lambda\sigma}q^\lambda(p\cdot q\, s^\sigma - s\cdot q\, p^\sigma)
\frac{1}{(p\cdot q)^2}g_2(x,Q^2)\,,
\label{eq:hadtensor}
\end{align}
where $x=1/\omega$ is the Bjorken variable identified with the parton
fractional momentum at Born level; 
see~\cite{Anselmino:1992rn,Manohar:1992tz} for details.
%
Specifically, given the definitions in \eqref{eq:Comptampl} and 
\eqref{eq:hadtensor}, the optical theorem implies that twice the imaginary 
part of $T_{\mu\nu}$ is equal to $W_{\mu\nu}$ times $4\pi$.
%
Neglecting target mass corrections, one has
\begin{align}
\mathcal{F}_1(\omega,Q^2) 
= {} & 2 \omega^2 \int_0^1 dx\,  \frac{xF_1(x,Q^2)}{1-(\omega x)^2} 
= \sum_{n=2,4,\cdots}^\infty 2\omega^n \int_0^1 dx\, x^{n-1} F_1(x,Q^2) \,, \\
\mathcal{G}_1(\omega,Q^2) 
= {} & 2 \omega \int_0^1 dx\, \frac{g_1(x,Q^2)}{1-(\omega x)^2} 
= \sum_{n=1,3,\cdots}^\infty 2\omega^n \int_0^1 dx\, x^{n-1} g_1(x,Q^2)\,.
\end{align}

At a sufficiently high momentum transfer $Q^2$, power corrections can be 
neglected and QCD factorization allows one to write the structure functions 
$F_1(x,Q^2)$ and $g_1(x,Q^2)$ as a convolution between perturbatively-computable
hard-scattering cross-sections and nonperturbative parton distributions:
\begin{align}
F_1(x,Q^2) 
= {} 
& x\sum_f \int_x^1 \frac{{\rm d}z}{z}\,C_{1,f}\left(\frac{x}{z},\alpha_s(Q^2)\right)f(z,Q^2) \,, \label{eq:Fi}\\
g_1(x,Q^2) 
= {} 
& \sum_f \int_x^1\frac{dz}{z}\, \Delta C_{1,f}\left(\frac{x}{z},\alpha_s(Q^2)\right) \Delta f(z,Q^2) \,.
\label{pdf}
\end{align}
%
Here, the sums run over the number of active
flavors at the scale $Q^2$ (including the gluon), $C_{1,f}$ and 
$\Delta C_{1,f}$ are the perturbative partonic hard-scattering cross-sections,
$\alpha_s$ is the QCD strong coupling, and $f(x,Q^2)$ and $\Delta f(x,Q^2)$ 
are the unpolarized and polarized PDFs.

Parton distributions allow for a proper field-theoretic definition as matrix 
elements in a hadron state of bilocal operators that act to count the number 
of quarks and gluons carrying a fraction $x$ of the hadron's momentum.
%
The definitions are usually stated in the light-cone frame, where 
the hadron carries momentum $p$ with plus/minus components
$p^\pm=(p^0\pm p^3)/\sqrt{2}$, and transverse components equal to zero.
%
For example, in the case of unpolarized and polarized quark PDFs, one has
\begin{align}
q(x) & = \frac{1}{4\pi}
\int dy^-e^{-iy^-xp^+}\langle p|\bar{\psi}(0,y^-,\mathbf{0}_\perp)
\gamma^+\mathcal{G}\psi(0,0,\mathbf{0})|p\rangle\,,
\label{eq:LCdefunp}\\
\Delta q(x) & = \frac{1}{4\pi}
\int dy^-e^{-iy^-xp^+}\langle p, s|\bar{\psi}(0,y^-,\mathbf{0}_\perp)
\gamma^+\gamma^5\mathcal{G}\psi(0,0,\mathbf{0})|p, s\rangle\,,
\label{eq:LCdefpol}
\end{align}
where $\psi$ is the quark field and $\mathcal{G}$ is an appropriate gauge link
required to make Eqs.~\eqref{eq:LCdefunp}--\eqref{eq:LCdefpol} gauge invariant.
%
See Refs.~\cite{Collins:1981uw,Curci:1980uw,Baulieu:1979mr,Collins:1989gx} 
for the definition of $\mathcal{G}$ and for
explicit light-cone formul{\ae} of unpolarized and polarized gluon PDFs.

While PDFs cannot be calculated perturbatively, their dependence on the scale 
$\mu$ resulting from factorization can be.
%
This is done by means of the
DGLAP (Dokshitzer-Gribov-Lipatov-Altarelli-Parisi) 
evolution equations~\cite{Dokshitzer:1977sg,Gribov:1972ri,Altarelli:1977zs},
a set of integro-differential coupled equations of the form
\begin{equation}
  \label{eq:dglapunp}
\frac{\partial f^\prime(x,\mu^2)}{\partial \ln \mu^2}
=
\sum_{f=g,q,\bar{q}}\int_x^1 
\frac{{\rm d}z}{z}P_{f^\prime f}\left(\frac{x}{z},\alpha_s(\mu^2)\right)f(z,\mu^2)\, ,
\end{equation}
%
\begin{equation}
  \label{eq:dglappol}
\frac{\partial \Delta f^\prime(x,\mu^2)}{\partial \ln \mu^2}
=
\sum_{f=g,q,\bar{q}}\int_x^1 
\frac{{\rm d}z}{z}\Delta P_{f^\prime f}\left(\frac{x}{z},\alpha_s(\mu^2)\right)\Delta f(z,\mu^2)\, .
\end{equation}
%
In short, the logarithmic derivative of the PDF is determined by a convolution
of the PDFs with the unpolarized (polarized) DGLAP kernels $P_{f^\prime f}$
($\Delta P_{f^\prime f}$), which can be 
computed perturbatively in powers of $\alpha_{s}$.
%
The unpolarized splitting functions $P_{f^\prime f}$ are currently completely 
known up to NNLO~\cite{Moch:2004pa,Vogt:2004mw} in the $\overline{\rm MS}$ 
renormalization scheme.
%
Results for the unpolarized nonsinglet splitting functions have appeared 
recently at N$^3$LO~\cite{Davies:2016jie,Moch:2017uml}.
%
The polarized splitting functions $\Delta P_{f^\prime f}$ are currently known 
up to  NNLO~\cite{Moch:2014sna} in the $\overline{\rm MS}$ scheme.
%
The DGLAP evolution equations can be solved numerically using
either $x$-space or Mellin $N$-space techniques that are widely available 
in various public codes~\cite{Vogt:2004ns,Salam:2008qg,Botje:2010ay,
Bertone:2013vaa,Bertone:2015cwa}.
%
The typical level of agreement for the results of the PDF evolution 
has been demonstrated to be of 
$\mathcal{O}(10^{-5})$~\cite{Giele:2002hx,Dittmar:2005ed}.

%The power series expansion of both the unpolarized and polarized 
%splitting functions contains terms proportional to $\ln\mu^2$, 
%$\ln(1/x)$ and $\ln(1-x)$.
%
%While DGLAP evolution sums power of $\alpha_s\ln\mu^2$ contributions,
%in the small-$x$ kinematic region one may also sum power of $\ln(1/x)$
%contributions, independent of the value of $\mu^2$.
%
%This is achieved by the BFKL (Balitsky-Fadin-Kuraev-Lipatov) equations,
%which resum $\ln(1/x)$ terms to all orders.
%
%They were originally formulated in the unpolarized case at leading logarithm 
%(LL) order~\cite{Fadin:1975cb,Kuraev:1976ge,Kuraev:1977fs,Balitsky:1978ic},
%and later extended to next-to-leading logarithmic (NLL) 
%order~\cite{Fadin:1998py,Camici:1997ij,Ciafaloni:1998gs}
%
%The DGLAP and BFKL frameworks can be matched into a single set of evolution 
%equations by means of the small-$x$ resummation
%formalism~\cite{Ciafaloni:1999yw,Ciafaloni:2000cb,Ciafaloni:2003ek,
%Ciafaloni:2003rd,Altarelli:2005ni,White:2006yh,Altarelli:2008aj,
%Ciafaloni:2007gf,Bonvini:2016wki}.

%In the polarized case, the use of the small-$x$ formalism was pioneered 
%decades ago by Kirschner and Lipatov~\cite{Kirschner:1983di} 
%(see also~\cite{Kirschner:1994rq,Kirschner:1994vc,Griffiths:1999dj}) 
%and later by Bartels, Ermolaev, and Ryskin in the context of the structure 
%function $g_1$~\cite{Bartels:1995iu,Bartels:1996wc}.
%
%Small-$x$ evolution equations which resum powers of $\alpha_s\ln^2(1/x)$ 
%in the polarization-dependent evolution along with powers of $\alpha_s\ln(1/x)$
%in the unpolarized evolution have been recently reconsidered using the 
%formalism of Wilson line-like operators~\cite{Kovchegov:2015pbl}.
%
%Both numerical~\cite{Kovchegov:2016weo}
%and analytical~\cite{Kovchegov:2016zex,Kovchegov:2017jxc}
%solutions to these small-$x$ evolution equations have been derived
%for the flavor singlet combination of polarized PDFs.
%
%Results have been extended also to the case of the polarized gluon 
%PDF~\cite{Kovchegov:2017lsr}.
