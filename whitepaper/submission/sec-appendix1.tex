\section{Definition of the PDF moments}
\label{app:notation}

In this appendix, we summarize the conventions adopted in this paper to denote 
the moments of relevant unpolarized and polarized PDF combinations.
%
We focus on the quantities which can be presently computed in lattice QCD,
although those used for benchmarks in Sec.~\ref{sec:benchmarking} are only
a subset of them.
%
In the equations below, we use the shorthand notation
\begin{equation}
q^\pm \equiv q\pm\bar{q}\, 
\quad\text{ and }\quad
\Delta q^\pm \equiv \Delta q\pm\Delta\bar{q}\, 
,\qquad q=u,d,s,c \,,
\end{equation}
%
for unpolarized and polarized PDFs respectively.
%
We identify $\mu$ with the QCD factorization scale and $Q$ with the 
characteristic scale of a given hard-scattering process.
%
The use of the following notation is strongly recommended for any comparison 
between lattice-QCD computations and global-fit determinations of 
PDF moments.

\begin{itemize}

\item Unpolarized moments.

\begin{enumerate}

\item The first moment of the total $u^+-d^+$ PDF combination
\begin{equation}
\left.\langle x\rangle_{u^+-d^+}(\mu^2)\right|_{\mu^2=Q^2}
=
\int_0^1 dx\, x\left\{u(x,Q^2)+\bar{u}(x,Q^2)-d(x,Q^2)-\bar{d}(x,Q^2)\right\} \, .
\label{eq:unpfmumdtot}
\end{equation}

\item The second moment of the valence $u^--d^-$ PDF combination
\begin{equation}
\left.\langle x^2\rangle_{u^--d^-}(\mu^2)\right|_{\mu^2=Q^2}
=
\int_0^1 dx\, x^2\left\{u(x,Q^2)-\bar{u}(x,Q^2)-d(x,Q^2)+\bar{d}(x,Q^2)\right\} \, .
\label{eq:unpsmumdval}  
\end{equation}

\item The first moment of the individual quark $q^+$ total PDF combination
\begin{equation}
\left.\langle x\rangle_{q^+=u^+,d^+,s^+,c^+}(\mu^2)\right|_{\mu^2=Q^2}
=
\int_0^1 dx\, x\left\{q(x,Q^2)+\bar{q}(x,Q^2)\right\} \, .
\label{eq:unpfmiqtot}
\end{equation}

\item The second moment of the individual quark $q^-$ valence PDF combination
\begin{equation}
\left.\langle x^2\rangle_{q^-=u^-,d^-,s^-,c^-}(\mu^2)\right|_{\mu^2=Q^2}
=
\int_0^1 dx\, x^2\left\{q(x,Q^2)-\bar{q}(x,Q^2)\right\} \, .
\label{eq:unpsmiqval}
\end{equation}

\item The first moment of the gluon PDF
\begin{equation}
\left.\langle x \rangle_g(\mu^2)\right|_{\mu^2=Q^2}
=
\int_0^1 dx\, x\, g(x,Q^2) \, .
\label{eq:unpfmg}
\end{equation}

\end{enumerate}

\item Polarized moments.

\begin{enumerate}

\item The zeroth moment of the total $u^+-d^+$ PDF combination
\begin{equation}
\left.\langle 1 \rangle_{\Delta u^+-\Delta d^+}(\mu^2)\right|_{\mu^2=Q^2}
=
\int_0^1 dx \left\{\Delta u(x,Q^2)+\Delta\bar{u}(x,Q^2)
-\Delta d(x,Q^2)-\Delta\bar{d}(x,Q^2)\right\} \, .
\label{eq:polzmumdtot}
\end{equation}

\item The first moment of the valence $u^--d^-$ PDF combination
\begin{equation}
\left.\langle x\rangle_{\Delta u^--\Delta d^-}(\mu^2)\right|_{\mu^2=Q^2}
=
\int_0^1 dx\, x\left\{\Delta u(x,Q^2)-\Delta\bar{u}(x,Q^2)-\Delta d(x,Q^2)+\Delta \bar{d}(x,Q^2)\right\}
\label{eq:polfmumdval}  
\end{equation}

\item The zeroth moment of the individual quark $q^+$ total PDF combination
\begin{equation}
\left.\langle 1\rangle_{q^+=\Delta u^+,\Delta d^+,\Delta s^+,\Delta c^+}(\mu^2)\right|_{\mu^2=Q^2}
=
\int_0^1 dx \left\{\Delta q(x,Q^2)+\Delta\bar{q}(x,Q^2)\right\} \, .
\label{eq:polzmiqtot}
\end{equation}

\item The first moment of the individual quark $q^-$ valence PDF combination
\begin{equation}
\left.\langle x\rangle_{\Delta q^-=\Delta u^-,\Delta d^-,\Delta s^-,\Delta c^-}(\mu^2)\right|_{\mu^2=Q^2}
=
\int_0^1 dx\, x\left\{\Delta q(x,Q^2)-\Delta\bar{q}(x,Q^2)\right\} \, .
\label{eq:polfmiqval}
\end{equation}

\end{enumerate}

\end{itemize}

Some of these moments have a direct physical interpretation, see 
Sec.~\ref{Sec:IntroPDFs}.
%
For instance, Eq.~\eqref{eq:unpfmiqtot} and Eq.~\eqref{eq:polzmiqtot}
correspond respectively to the proton's momentum and spin fractions carried
by a given quark flavor (and its corresponding antiquark) at the scale 
$\mu^2=Q^2$.
%
Higher moments and/or moments of other flavor combinations are readily
computable from any phenomenological PDF set.
%
We do not consider them though, as the corresponding lattice-QCD
computations are outside the current reach.
