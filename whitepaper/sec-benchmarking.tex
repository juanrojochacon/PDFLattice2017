%%%%%%%%%%%%%%%%%%%%%%%%%%%%%%%%%%%%%%%%%%%%%%%%%%%%%%%%%%%%%%%%%%%%%%%%%%%%%%%%
\section{Benchmarks}
\label{sec:benchmarking}
%%%%%%%%%%%%%%%%%%%%%%%%%%%%%%%%%%%%%%%%%%%%%%%%%%%%%%%%%%%%%%%%%%%%%%%%%%%%%%%%

The aim of this section is to provide a quantitative comparison between 
current lattice-QCD and global-fit results.
%
To this purpose, we shall identify the appropriate benchmark quantities, 
and define the criteria to carefully appraise the corresponding determinations
available in the literature.
%
For each benchmark quantity, we shall specify the prescription adopted to 
select and combine, on the one hand, lattice-QCD calculations, and, on the 
other hand, global PDF-fit determinations.
%
We finally present our benchmark numbers from each side and compare them.

%%%%%%%%%%%%%%%%%%%%%%%%%%%%%%%%%%%%%%%%%%%%%%%%%%%%%%%%%%%%%%%%%%%%%%%%%%%%%%%%
\subsection{Benchmark criteria}
\label{subsec:BC}

We start by describing our benchmark criteria, which include the definition
of the benchmark quantities and the determination of their reference values,
based on a careful assessment of the lattice-QCD and global-PDF results 
available in the literature.

\subsubsection{Benchmark quantities}
\label{subsubsec:BQ}

We identify our benchmark quantities with the lowest moments of unpolarised 
and polarised PDFs, or of some of their combinations, specifically: 
$\langle x\rangle_{u^+-d^+}$, $\langle x \rangle_{u^+}$, $\langle x \rangle_{d^+}$, 
$\langle x \rangle_{s^+}$ and $\langle x \rangle_{g}$ (in the unpolarised case); 
$g_A\equiv\langle 1 \rangle_{\Delta u^+ - \Delta d ^+}$, 
$\langle 1 \rangle_{\Delta u^+}$, $\langle 1 \rangle_{\Delta d^+}$,  
$\langle 1 \rangle_{\Delta s^+}$ and $\langle x \rangle_{\Delta u^- - \Delta d^-}$ 
(in the polarised case).
%
We use the notation outlined in Appendix~\ref{app:notation}.
%
The focus on these quantities is motivated by the fact that
current lattice calculations of higher moments and/or of other PDFs or PDF 
combinations are not sufficiently controlled to allow for a meaningful 
comparison between lattice-QCD and global-fit results.
%
Therefore, these will not be considered here. 

\subsubsection{Appraising lattice-QCD calculations}
\label{subsubsec:BClQCD}

To accurately assess the current state-of-the-art for lattice calculations
available in the literature, we follow a procedure inspired by the review of 
low-energy mesons undertaken by the Flavor Lattice Averaging Group 
(FLAG)~\cite{Aoki:2016frl}. 
%
For each lattice calculation, we assess the status of each source of 
uncertainty outlined in Sec.~\ref{Sec:IntroLQCD}. 
%
We use a rating system inspired by FLAG, awarding a blue star (\bstar) for 
sources of uncertainty that are well-controlled or very conservatively 
estimated, a blue circle (\bcirc) for sources of uncertainty that have been 
controlled or estimated to some extent, and a red square (\rsquare) for 
uncertainties that have not met our criteria or for which no estimate is given.
%
Specifically, the rating system works as follows.

\begin{itemize}
\item {\bfseries Discretisation effects and the continuum limit.} 
We assume that the lattice actions are ${\cal O}(a)$-improved, {\it i.e.}, 
that the discretization errors vanish quadratically with the lattice spacing. 
%
For unimproved actions an additional lattice spacing is required. 
%
We require that these criteria are satisfied in each case for at least one 
pion mass below 300 MeV.
%
\begin{itemize}
%
\item[\bstar] At least three lattice spacings with at least two lattice 
spacings below 0.1 fm and a range of lattice spacings that satisfies 
$[a_{\mathrm{max}}/a_{\mathrm{min}}]^2 \geq 2$.
%
\item[\bcirc] At least two lattice spacings with at least one point below 
0.1 fm and a range of lattice spacings that satisfies 
$[a_{\mathrm{max}}/a_{\mathrm{min}}]^2 \geq 1.4$.
%
\end{itemize}
%
To receive a \bstar~or \bcirc~either a continuum extrapolation must be 
performed or the results must demonstrate no significant discretisation 
effects over the appropriate range of lattice spacings.

\item {\bfseries Unphysical pion masses.} 
For the following criteria, we define a physical pion mass ensemble 
to be one with $M_\pi=135\pm 10$~MeV.
%
\begin{itemize}
\item[\bstar] One ensemble with a physical pion mass \emph{or} a chiral 
extrapolation with three or more pion masses, with at least two pion masses 
below 250 MeV and at least one below 200~MeV.
%
\item[\bcirc] A chiral extrapolation with three or more pion mass, two of 
which are below 300~MeV.
%
\end{itemize}

\item {\bfseries Finite volume effects.} 
%
For calculations that use a mixed action approach, {\it i.e.},
with different lattice actions for the valence and sea quarks, 
we apply the criteria for $M_\pi L$ to the valence quarks.
%
\begin{itemize}
%
\item[\bstar] Ensembles with $M_{\pi,\mathrm{min}}L\geq 4$, \emph{or} at least 
three volumes with spatial extent $L>2.5$~fm.
\item[\bcirc] Ensembles with $M_{\pi,\mathrm{min}}L \geq 3.4$, \emph{or} at least 
two volumes with spatial extent $L>2.5$~fm.
\end{itemize}

\item {\bfseries Excited state contamination.} 
The following criteria must be satisfied for every pion mass and lattice 
spacing.
%
\begin{itemize}
%
\item[\bstar] At least three source-sink separations or a variational method 
to optimise the operator derived from at least a $3\times 3$ correlator matrix.
% 
\item[\bcirc] Two source-sink separations at every pion mass and lattice 
spacing, or three or more source-sink separations at one pion mass below 
300~MeV. 
%
For the variational method, an optimised operator derived from a $2\times 2$ 
correlator matrix at every pion mass and lattice spacing, or a $3\times 3$ 
correlator matrix for two pion masses below 300 MeV.
%
\end{itemize}

\item {\bfseries Renormalization.} 
\begin{itemize}
%
\item[\bstar] Nonperturbative renormalisation.
%
\item[\bcirc] Perturbative renormalisation.
%
\end{itemize}
%
For $g_A$ we also award a \bstar~for calculations that use fermion actions 
for which $Z_A/Z_V=1$ or employ combinations of quantities for which the 
renormalisation is unity by construction.

\item {\bfseries Lattice-spacing determination.} 
For lattice-QCD calculations of nucleons, the lattice-spacing determination is 
generally sufficiently precise that it is a very small or negligible source
of systematic uncertainty. 
%
Therefore we do not include an assessment of the lattice-spacing
determination in our criteria.

\end{itemize}

We now turn to summarise the current status of lattice-QCD calculations of
our benchmark moments of unpolarised and polarised PDFs respectively.
%
Following FLAG, we consider only those results that are published in 
peer-reviewed journals or that have appeared as preprints. 
%
Where recent results are a clear update of previously published work, we do 
not include the earlier results.
%
A bibliographical compilation of the results available in the literature 
is given for completeness in 
Tabs.~\ref{tab:latticebibfirst}-\ref{tab:latticebiblast} 
of Appendix~\ref{sec:LQCDtables}.
%
We characterise the state-of-the-art results according to the criteria 
described above, and provide a prescription to combine those which satisfy 
them into a single benchmark value.

We remark that our criteria, and the corresponding ratings, are chosen 
not only to provide as fair an assessment of the relative merits of various 
calculations as possible, but to be aspirational. 
%
Where lattice-QCD results do not meet these standards, we hope that the lattice 
community will work towards improved calculations and greater precision as 
part of a concerted and inter-disciplinary effort to better understand
nucleon structure.

\paragraph{Unpolarised parton distributions.}
We summarise the current status of lattice-QCD calculations of the benchmark 
moments of unpolarised PDFs listed in Sec.~\ref{subsubsec:BQ} in 
Tab.~\ref{tab:unpolLQCDstatus1}. 
%
In the first column we indicate the computed moment, in the second column
the collaboration who performed the computation and in the third column
the corresponding reference.
%
In the fourth column we indicate the number of sea quark flavours, $N_f$, 
and in the fifth column we show whether the calculation has been published (P) 
or has appeared as a preprint (PreP).
%
In the following five columns we assess each source of systematic uncertainty
according to the criteria listed above. 
%
In the last column, we report the computed value at $\mu^2=Q^2=4$ GeV$^2$.
%
We indicate in parentheses the 68\% confidence level (CL) statistic uncertainty 
and the total systematic uncertainty respectively.
%
We do not list results that have not been extrapolated to the physical pion 
mass, nor do we include quenched results in Tab.~\ref{tab:unpolLQCDstatus1}. 
%
For completeness, we report these results in Tab.~\ref{tab:unpolLQCDstatus1B} 
of Appendix~\ref{sec:LQCDtables}.

%-------------------------------------------------------------------------------
\begin{table}[!t] 
\renewcommand{\arraystretch}{1.2} 
\centering 
\begin{threeparttable}
\begin{tabular}{llcllccccccl}
Mom. & Collab. & Ref. & $N_f$ & Status & 
\begin{rotate}{70}{discretisation}\end{rotate} &
\begin{rotate}{70}{quark mass}\end{rotate} &
\begin{rotate}{70}{finite volume}\end{rotate} &
\begin{rotate}{70}{renormalisation}\end{rotate} &
\begin{rotate}{70}{excited states}\end{rotate}&
& Value\\
\toprule
$\langle x\rangle_{u^+-d^+}$ 
& LHPC\,14  
  & \cite{Green:2012ud} 
  & 2+1 
  & P  
  & \rsquare 
  & \bstar   
  & \bstar   
  & \bstar 
  & \bstar 
  & 
  & 0.140(21)\\
& ETMC 17  
  & \cite{Alexandrou:2017oeh} 
  & 2   
  & PreP 
  & \rsquare 
  & \bstar   
  & \rsquare 
  & \bstar 
  & \bstar 
  & $^*$ 
  & 0.194(9)(11)\\
& RQCD 14  
  & \cite{Bali:2014gha} 
  & 2   
  & P  
  & \rsquare 
  & \rsquare 
  & \bcirc   
  & \bstar 
  & \bstar 
  & $^{**}$ 
  & 0.217(9)\\
\midrule
$\langle x\rangle_{u^+}$
&  ETMC 17  
  & \cite{Alexandrou:2017oeh} 
  & 2 
  & PreP 
  & \rsquare 
  & \bstar   
  & \rsquare 
  & \bstar 
  & \bstar 
  & $^{*\triangleright}$ 
  & $0.453(57)(48)$\\
\midrule
$\langle x\rangle_{d^+}$
& ETMC 17  
  & \cite{Alexandrou:2017oeh} 
  & 2 
  & PreP 
  & \rsquare 
  & \bstar   
  & \rsquare 
  & \bstar 
  & \bstar 
  & $^{*\triangleright}$ 
  & $0.259(57)(47)$\\
\midrule
$\langle x\rangle_{s^+}$
& ETMC 17  
  & \cite{Alexandrou:2017oeh} 
  & 2 
  & PreP 
  & \rsquare  
  & \bstar   
  & \rsquare 
  & \bstar 
  & \bstar 
  & $^{*\triangleright}$ & $0.092(41)(0)$\\
\midrule
$\langle x\rangle_{g}$
& ETMC 17  
  & \cite{Alexandrou:2017oeh} 
  & 2 
  & PreP 
  & \rsquare 
  & \bstar   
  & \rsquare 
  & \bcirc 
  & \bstar 
  & $^*$ 
  & 0.267(22)(27)\\
%\midrule
%$\frac{\langle x\rangle_{u^+-d^+}}{\langle x\rangle_{u^++d^+-2s^+}}$
%&  ETMC 15 
%   & \cite{Alexandrou:2015qia} 
%   & 2+1+1 
%   & P 
%   & \rsquare 
%   & \rsquare 
%   & \bstar 
%   & \bstar 
%   & \rsquare 
%   & 
%   & $0.39(1)(4)$\\
\bottomrule
\end{tabular}
\begin{tablenotes}
\footnotesize
\item[$\ \,*$] Study employing a single physical pion mass ensemble.
\item[$**$] Study employing a single ensemble with $m_\pi=150$~MeV.
\item[$\ \,\triangleright$] The mixing with $\langle x\rangle_{g}$ is computed.
\end{tablenotes}
\end{threeparttable}
\caption{\small Status of current lattice-QCD calculations of the benchmark 
first moments of unpolarised PDFs listed in Sec.~\ref{subsubsec:BQ}.
%
A detailed description of each entry, including the symbols used to 
characterise the various sources of systematics are described in the text.
%
Values are shown at $\mu^2=Q^2=4$ GeV$^2$.
%
Uncertainties are (68\% CL) statistics and systematics respectively.}
\label{tab:unpolLQCDstatus1}
\end{table}
%-------------------------------------------------------------------------------

As apparent from Tab.~\ref{tab:unpolLQCDstatus1}, there are no lattice 
calculations of the considered first moments for which all systematics 
have been fully explored and controlled. 
%
In the case of $\langle x\rangle_{u^+-d^+}$ three different results are available 
in the literature.
%
We present the lattice-QCD benchmark value for this quantity 
as a best-estimate band.
% 
This band extends from the mean minus the error of the smallest result to the 
mean plus the error of the largest, and includes all results listed in 
Tab.~\ref{tab:unpolLQCDstatus1} with two or more flavours of sea quarks.
%
Current studies are indeed not sufficiently precise to distinguish between 
results with different numbers of sea quark flavours.
%
In the case of $\langle x \rangle_{u^+}$, $\langle x \rangle_{d^+}$, 
$\langle x \rangle_{s^+}$ and $\langle x \rangle_g$ there is only one
lattice result available in the literature. 
%
For these quantities, our lattice-QCD benchmark value is the single result.
% 
However, it should be noted that these results may underestimate some sources 
of uncertainty. 

Finally, we summarise the current status of lattice-QCD calculations of the 
second moment of the unpolarised valence quark PDFs, specifically 
$\langle x^2 \rangle_{u^-}$ and $\langle x^2\rangle_{u^--d^-}$
in Tab.~\ref{tab:unpolLQCDstatus2B} of Appendix~\ref{sec:LQCDtables}.
% 
The study of these moments is not sufficiently mature to provide benchmark 
values and we only list the results for completeness.

\paragraph{Polarised parton distributions.}
The zeroth moment of the isotriplet polarised PDF combination is related to the 
axial charge of the nucleon, $g_A\equiv \langle 1\rangle_{\Delta u^+-\Delta d^+}$.
%
This quantity is of central importance to nucleon physics and has long been 
considered an important benchmark for lattice calculations. 
%
Historically, lattice-QCD calculations of the axial charge have underestimated 
the experimental value $g_A^{\mathrm{exp}} = 1.2723(23)$~\cite{Olive:2016xmw}, 
which is most precisely determined from neutron decays. 
%
Thus the axial charge has been the single most-studied moment in lattice QCD.
%
We summarise the current status of these calculations in 
Tab.~\ref{tab:gAstatus} in the same format as in 
Tab.~\ref{tab:unpolLQCDstatus1}.
%
All results are quoted at $\mu^2=Q^2=4$ GeV$^2$.

%-------------------------------------------------------------------------------
\begin{table}[!t]
\renewcommand{\arraystretch}{1.2} 
\centering
\begin{threeparttable}
\begin{tabular}{llcllccccccl}
Mom. & Collab. & Ref. & $N_f$ & Status &  
\begin{rotate}{70}{discretisation}\end{rotate} &
\begin{rotate}{70}{quark mass}\end{rotate} &
\begin{rotate}{70}{finite volume}\end{rotate} &
\begin{rotate}{70}{renormalisation}\end{rotate} &
\begin{rotate}{70}{excited states}\end{rotate}&
& Value \\
\toprule
$g_A$
& CalLat\,17 
  & \cite{Berkowitz:2017gql} 
  & 2+1+1 
  & PreP 
  & \rsquare 
  & \bstar  
  & \rsquare 
  & \bstar 
  & \bstar 
  & $^\diamond$ 
  & 1.278(21)(26) \\
& PNDME\,16  
  & \cite{Bhattacharya:2016zcn} 
  & 2+1+1 
  & P    
  & \bcirc   
  & \bstar  
  & \bcirc   
  & \bstar 
  & \bstar 
  & 
  & 1.195(33)(20)\\
& LHPC\,14    
  & \cite{Green:2012ud} 
  & 2+1 
  & P 
  & \rsquare 
  & \bstar 
  & \bstar 
  & \bstar  
  & \bstar & & 0.97(8)\\
& Mainz\,17   
  & \cite{Capitani:2017qpc} 
  & 2 
  & PreP 
  & \bstar 
  & \bcirc 
  & \bstar 
  & \bstar  
  & \bstar 
  & 
  & $1.278(68)({}^{+0}_{-0.087})$\\
& ETMC\,17    
  & \cite{Alexandrou:2017hac} 
  & 2 
  & PreP 
  & \rsquare  
  & \bstar 
  & \rsquare  
  & \bstar  
  & \bstar 
  & $^*$ 
  & 1.212(33)(22)\\
& RQCD\,15    
  & \cite{Bali:2014nma} 
  & 2 
  & P 
  & \bcirc 
  & \bcirc  
  & \bcirc  
  & \bstar   
  & \bcirc 
  & $^\ddag$
  & 1.280(44)(46) \\
  & QCDSF\,14   
  & \cite{Horsley:2013ayv} 
  & 2 
  & P 
  & \bcirc 
  & \bcirc  
  & \bcirc  
  & \bstar  
  & \rsquare 
  & $^\ddag$
  & 1.29(5)(3) \\
\bottomrule
\end{tabular}
\begin{tablenotes}
\footnotesize
\item[$*$] Study employing a single physical pion mass ensemble.
\item[$^\ddag$] $g_A$ is determined via the ratio $g_A/f_\pi$ employing the 
physical value for $f_\pi$.
\item[$\diamond$] Approach inspired by the Feynman-Hellmann method is employed.
\end{tablenotes}
\end{threeparttable}
\caption{\small Same as Tab.~\ref{tab:unpolLQCDstatus1}, but for the axial 
coupling, $g_A\equiv \langle 1\rangle_{\Delta u^+-\Delta d^+}$. 
%
Studies with three or more red squares are omitted.
%
Values are shown at $\mu^2=Q^2=4$ GeV$^2$.}
\label{tab:gAstatus}
\end{table}
%-------------------------------------------------------------------------------

As apparent from Tab.~\ref{tab:gAstatus}, all systematics are controlled for
only three calculations of $g_A$.
%
One of them~\cite{Bhattacharya:2016zcn} is for $N_f=2+1+1$, while two of them
are for $N_f=2$~\cite{Capitani:2017qpc,Bali:2014nma}.
%
In the former case, our benchmark value corresponds to the single calculation;
in the latter case, our benchmark value corresponds to a weighted average 
of~\cite{Capitani:2017qpc} and~\cite{Bali:2014nma}, assuming correlations
between the results, and applying the procedure of \cite{XXX}.
%
In summary, our benchmark values are
\begin{equation}\label{eq:gAcriteria}
g_A^{N_f=2+1+1} = 1.195(33)(20)
\,,\qquad \mathrm{and}\qquad 
g_A^{N_f=2} = 1.279(50)\,.
\end{equation}

%To provide a benchmark value for the axial coupling, we use an approach based 
%on the FLAG procedure. 
%%
%We distinguish between calculations that use different numbers of sea quarks 
%and exclude results that do not meet the criteria listed above. 
%%
%For $N_f = 2+1+1$, there is only a single 
%calculation~\cite{Bhattacharya:2016zcn} that satisfies these criteria. 
%%
%For $N_f = 2$, we apply a weighted average of~\cite{Capitani:2017qpc} 
%and~\cite{Bali:2014nma}, assuming correlations between the results and 
%applying the procedure of \cite{XXX}. 
%%
%Thus we obtain
%\begin{equation}\label{eq:gAcriteria}
%g_A^{N_f=2+1+1} = 1.195(33)(20)
%\,,\qquad \mathrm{and}\qquad 
%g_A^{N_f=2+1+1} = 1.279(50)\,.
%\end{equation}
%%
%For the $N_f=2+1+1$ result, there is only a single calculation, and this 
%value may underestimate some sources of systematic uncertainty. 
%
%To incorporate information from the result of~\cite{Berkowitz:2017gql}, 
%which uses the same gauge configurations, we also carry out a simultaneous fit 
%to the two sets of data. 
%

We observe that the result of~\cite{Berkowitz:2017gql}, despite it does
not fulfill all our requirements on systematic uncertainties, uses the same 
gauge configurations as those of~\cite{Bhattacharya:2016zcn}.
%
Therefore, we also carry out a simultaneous fit to the two results for
completeness.
%
We use a fit function of the form
\begin{equation}
g_A^{\mathrm{fit}} 
= 
c_0 + 
c_1a + 
c_2a^2 + 
c_3M_\pi^2 + 
c_4M_\pi^2 \exp(-M_\pi L) +
c_5M_\pi^2 \log\left(\frac{M_\pi^2}{\Lambda_{\chi \mathrm{PT}}^2}\right)\,.
\end{equation}
%
The coefficient $c_1$ captures ${\cal O}(a)$ effects present in the valence 
quark action of~\cite{Bhattacharya:2016zcn}, while~\cite{Berkowitz:2017gql} 
has discretisation effects starting at ${\cal O}(a^2)$. 
%
The term proportional to $c_4$ captures the leading finite-volume effects and 
$c_3$ and $c_5$ represent chiral extrapolation terms. 
%
Modifications to this fit form, including  setting $c_5=0$, have a negligible 
effect on the fit results, within extrapolation uncertainties, and the final 
result is in very good agreement with a weighted average of the two 
calculations, assuming 100\% correlations, which is 
$g_A^{N_f=2+1+1,\mathrm{avg}} = 1.243(36)$. 
%
Based on this fit, we find a best-estimate band of
\begin{equation}\label{eq:gAfit}
g_A^{N_f=2+1+1,\mathrm{fit}} = \numrange{1.22}{1.28}\,.
\end{equation}

We plot all lattice results for the axial coupling, listed in 
Tab.~\ref{tab:gAstatus}, in Fig.~\ref{fig:gaLQCDstatus}. 
%
We show the world average experimental value as a vertical black line. 
%
The light gray bands for $N_f=2+1+1$ and $N_f=2$ represent the best-estimate 
results of Eq.~\eqref{eq:gAcriteria} and the dashed gray band for
$N_f=2+1+1$ is the combined fit band given in Eq.~\eqref{eq:gAfit}. 

%-------------------------------------------------------------------------------
\begin{figure}[!t]
\centering
\includegraphics[scale=0.7]{plots/ga_summary.pdf}\\
\caption{\small Summary of the current status of lattice-QCD calculations of 
the axial charge, $g_A\equiv \langle 1\rangle_{\Delta u^+-\Delta d^+}$.
%
The vertical black line represents the current experimental world average 
$g_A^{\mathrm{exp}} = 1.2723(23)$~\cite{Olive:2016xmw}. 
%
The light gray bands for $N_f=2+1+1$ and $N_f=2$ represent the best-estimate 
results of Eq.~\eqref{eq:gAcriteria} and the dashed gray band for
$N_f=2+1+1$ is the fit band of Eq.~\eqref{eq:gAfit}.}    
\label{fig:gaLQCDstatus}
\end{figure}
%-------------------------------------------------------------------------------

In addition to the axial charge, we summarise the zeroth moments of the 
individual light quark total helicity distributions in 
Tab.~\ref{tab:polLQCDstatus0}. 
%
We summarise the status of lattice-QCD calculations of the
first moments of the polarised PDF combination 
$\langle x \rangle_{\Delta u^- - \Delta d^-}$ in Tab.~\ref{tab:polLQCDstatus1}. 
%
We use the same format as in Tab.~\ref{tab:unpolLQCDstatus1}.
%
All values are at $\mu^2=Q^2=4$ GeV$^2$.
%
Available results that have not been extrapolated to the physical pion mass
or quenched results are not reported here, but in 
Tabs.~\ref{tab:polLQCDstatus1B}-\ref{tab:polLQCDstatus2B} of
Appendix~\ref{sec:LQCDtables} for completeness.

In the case of $\langle 1 \rangle_{\Delta u^+}$ and $\langle 1 \rangle_{\Delta d^+}$,
there is only one result for each quantity available in the literature.
%
Therefore, despite the corresponding systematic uncertainties are not 
completely under control and possibly underestimated, we take the individual 
results as our benchmark values.
%
In the case of $\langle 1 \rangle_{\Delta s^+}$ and 
$\langle x \rangle_{\Delta u^- - \Delta d^-}$, instead, several results are available
in the literature, although without a full characterisation of
their systematic uncertainties.
%
We present our lattice-QCD benchmark value for these quantitites as
a best-estimate band extending from the mean minus the error of the 
smallest result to the mean plus the error of the largest. 
%
We include all results with two or more flavours of sea quarks listed in 
Tabs.~\ref{tab:polLQCDstatus0}-\ref{tab:polLQCDstatus1} respectively.

%-------------------------------------------------------------------------------
\begin{table}[!t]
\renewcommand{\arraystretch}{1.2} 
\centering
\begin{threeparttable}
\begin{tabular}{llcllccccccl}
Mom. & Collab. & Ref. & $N_f$ & Status &
\begin{rotate}{70}{discretisation}\end{rotate} &
\begin{rotate}{70}{quark mass}\end{rotate} &
\begin{rotate}{70}{finite volume}\end{rotate} &
\begin{rotate}{70}{renormalisation}\end{rotate} &
\begin{rotate}{70}{excited states}\end{rotate}&
& Value \\
\midrule
$\langle 1\rangle_{\Delta u^+}$
& ETMC\,17 
  & \cite{Alexandrou:2017oeh} 
  & 2 
  & PreP 
  & \rsquare 
  & \bstar 
  & \rsquare 
  & \bstar 
  & \bstar 
  & $^*$ 
  & $0.830(26)(4)$\\
\midrule
$\langle 1\rangle_{\Delta d^+}$
& ETMC\,17  
  & \cite{Alexandrou:2017oeh} 
  & 2 
  & PreP 
  & \rsquare 
  & \bstar 
  & \rsquare  
  & \bstar 
  & \bstar 
  & $^*$ 
  & $-0.386(16)(6)$\\
\midrule
$\langle 1\rangle_{\Delta s^+}$
& $\chi$QCD\,17 
  & \cite{Gong:2015iir} 
  & 2+1 
  & P 
  & \rsquare  
  & \bcirc 
  & \bcirc  
  & \bstar 
  & \bstar
  & $^{\dagger,\triangleleft}$ 
  & -0.0403(44)(78)\\
& Engelhardt\,12 
  & \cite{Engelhardt:2012gd} 
  & 2+1 
  & P 
  & \rsquare  
  & \rsquare 
  & \bcirc  
  & \bstar  
  & \bstar  
  & $^\triangleleft$ 
  & -0.031(17)\\
& ETMC\,17 
  & \cite{Alexandrou:2017oeh} 
  & 2 
  & PreP 
  & \rsquare  
  & \bstar 
  & \rsquare  
  & \bstar  
  & \bstar 
  & $^*$ 
  & -0.042(10)(2)\\
\bottomrule
\end{tabular}
\begin{tablenotes}
\footnotesize
\item[$*$] Study employing a single physical pion mass ensemble.
\item[$\dagger$] Partially quenched simulation with $m_\pi=330$~MeV. 
Criteria applied to the valence quarks. 
\item[$\triangleleft$] Some parts of the renormalisation are estimated, 
see references for details.
\end{tablenotes}
\end{threeparttable}
\caption{\small Summary of the current status of lattice-QCD calculations 
of the zeroth moments of longitudinally polarised PDFs.}
\label{tab:polLQCDstatus0}
\end{table}
%-------------------------------------------------------------------------------

%-------------------------------------------------------------------------------
\begin{table}[!t] 
\renewcommand{\arraystretch}{1.2}
\centering
\vspace{2cm}
\begin{threeparttable}
\begin{tabular}{llcllccccccl}
Mom. & Collab. & Ref. & $N_f$ & Status &
\begin{rotate}{70}{discretisation}\end{rotate}  &
\begin{rotate}{70}{quark mass}\end{rotate}      &
\begin{rotate}{70}{finite volume}\end{rotate}   &
\begin{rotate}{70}{renormalisation}\end{rotate} &
\begin{rotate}{70}{excited states}\end{rotate}  &
& Value \\
\toprule
$\langle x\rangle_{\Delta u^--\Delta d^-}$
& RBC/ 
  & \multirow{2}{*}{\cite{Aoki:2010xg}} 
  & \multirow{2}{*}{2+1} 
  & \multirow{2}{*}{P} 
  & \multirow{2}{*}{\rsquare}  
  & \multirow{2}{*}{\rsquare} 
  & \multirow{2}{*}{\bstar}  
  & \multirow{2}{*}{\bstar}  
  & \multirow{2}{*}{\rsquare} 
  &  
  & 0.256(23)/\\
& UKQCD\,10 
  &  
  &  
  &  
  &   
  &  
  &   
  &   
  &  
  &  
  & 0.205(59)\\
& LHPC\,10 
  & \cite{Bratt:2010jn} 
  & 2+1 
  & P 
  & \rsquare  
  & \rsquare 
  & \bcirc  
  & \bcirc  
  & \rsquare 
  &  
  & 0.1972(55)\\
& ETMC\,15 
  & \cite{Abdel-Rehim:2015owa} 
  & 2 
  & P 
  & \rsquare  
  & \bstar 
  & \rsquare  
  & \bstar  
  & \bstar 
  & $^*$ 
  & 0.229(33)\\
\bottomrule
\end{tabular}
\begin{tablenotes}
\footnotesize
\item[$*$] Study employing a single physical pion mass ensemble.
\end{tablenotes}
\end{threeparttable}
\caption{\small Summary of the current status of lattice-QCD calculations of 
moments of longitudinally polarised PDFs.}
\label{tab:polLQCDstatus1}
\end{table}
%-------------------------------------------------------------------------------

\subsubsection{Appraising global-fit results}
\label{subsubsec:GPDFfits}

The current state-of-the-art for global PDF fit determinations and their 
uncertainties has been carefully assessed in dedicated reviews
recently~\cite{Forte:2013wc,Jimenez-Delgado:2013sma} and summarised in
Sec.~\ref{sec:unpPDFs}. 
%
It is now recognised that the PDF uncertainty receives various contributions: 
the measurement uncertainty propagated from the
experimental data, uncertainties associated with incompatibility of the 
fitted experiments, procedural uncertainties such as those related to the
functional form of PDFs, and the handling of systematic errors.
%
As outlined in Sec.~\ref{sec:unpPDFs}, in principle all these uncertainties 
can be accounted for through suitable methods either in the Hessian or the 
MC frameworks.
%
In practice, there is a significant spread in the sophistication 
of those methods between unpolarised and polarised PDF fits, as we will 
delineate in the following.

In Sec.~\ref{sec:unpPDFs}, it was also emphasised that there are additional 
theoretical uncertainties on PDFs coming from the uncertainties associated to 
the input values of the physical parameters used in the fit (such as the 
reference value of the strong coupling) and to missing higher order
uncertainties (given that fits are usually performed with fixed-order
perturbation theory).
%
The size of the former can be accounted for by studying the stability of the 
results upon variation of the input parameters; the size of the latter are 
currently unknown, although it is supposed to be subdominant.
%
Therefore, theoretical uncertainties will not be considered in the following.

We now turn to summarise the results for our benchmark moments listed in 
Sec.~\ref{subsubsec:BQ} based on current global-fit determinations of
unpolarised and polarised PDFs respectively.
%
We critically assess the corresponding uncertainties and specify how the
available results are combined into a single benchmark value.

\paragraph{Unpolarised parton distributions.}

We summarise the current status of global-fit results of the benchmark
moments of unpolarised PDFs listed in Sec.~\ref{subsubsec:BQ} 
in Tab.~\ref{tab:unpPDFmoms}.
%
In the first column we indicate the computed momentum, and in the subsequent 
six columns its value as obtained from the most recent available PDF 
determinations: NNPDF3.1~\cite{Ball:2017nwa},
CT14~\cite{Dulat:2015mca}, MMHT2014~\cite{Harland-Lang:2014zoa},
ABMP16~\cite{Alekhin:2017kpj} (with $n_f=4$ flavours), 
CJ15~\cite{Accardi:2016qay} and 
HERAPDF2.0~\cite{Abramowicz:2015mha} respectively.
%
The most relevant features of these PDF sets have been presented in 
Sec.~\ref{sec:unpPDFs}.
%
All values in Tab.~\ref{tab:unpPDFmoms} are displayed
at $\mu^2=Q^2=4$ GeV$^2$. 
%
They have been obtained from the default PDF sets at the highest available 
perturbative order, which is NNLO for all of them, except for CJ15, 
for which it is NLO.
%
The uncertainties correspond to 68\% CL bands, except for the CT14 PDF set,
where they denote the 90\% CL band.
%
Note however that no tolerance is used for the CJ15 PDF set, hence the 
smaller uncertainties w.r.t. all the other PDF sets.
%
In the case of the HERAPDF2.0 set, the PDF error band is the sum in quadrature 
of the statistical, model and parametrisation uncertainties.

%-------------------------------------------------------------------------------
\begin{table}[!t]
\centering
\renewcommand{\arraystretch}{1.2}
\begin{tabular}{lcccccc}
\toprule
Mom. 
& NNPDF3.1 & CT14 & MMHT2014 & ABMP2016 & CJ15 & HERAPDF2.0 \\
\midrule
$\langle x \rangle_{u^+-d^+}$ 
& 0.152(3) & 0.158(6) & 0.151(4) & 0.167(4) & 0.152(2) & 0.188(3)\ \,\\
$\langle x \rangle_{u^+}$    
& 0.348(4) & 0.348(5) & 0.348(5) & 0.353(3) & 0.348(1) & 0.372(4)\ \,\\
$\langle x \rangle_{d^+}$    
& 0.196(3) & 0.190(5) & 0.197(5) & 0.186(3) & 0.196(1) & 0.185(7)\ \,\\
$\langle x \rangle_{s^+}$    
& 0.039(3) & 0.035(9) & 0.035(9) & 0.041(2) & 0.031(1) & 0.035(11)\\
$\langle x \rangle_{g}$     
& 0.410(4) & 0.416(9) & 0.411(9) & 0.412(4) & 0.416(1) & 0.401(10)\\
\bottomrule
\end{tabular}
\caption{\small Status of current global-fit determinations of the 
benchmark moments of unpolarised PDFs listed in Sec.~\ref{subsubsec:BQ}.
All values are shown at $\mu^2=Q^2=4$ GeV$^2$.}
\label{tab:unpPDFmoms}
\end{table}
%-------------------------------------------------------------------------------

In order to provide a benchmark value for the first moments of unpolarised PDFs
listed in Tab.~\ref{tab:unpPDFmoms}, we follow the latest PDF4LHC 2015 
recommendations~\cite{Butterworth:2015oua}.
%
Even though the recommendations were primarily formulated for the usage of PDFs
in LHC-related physics, and alternative recommendations have been 
suggested~\cite{Accardi:2016ndt}, we find it useful to apply them here as well.
%
The reason is twofold.
%
First, this benchmark exercise aims at accuracy and precision,  
two of the guiding principles underlying the recommendations.
%
Second, they led to the release of specific PDF sets which can be easily 
used to provide all the needed benchmark values.

While we refer the reader to Ref.~\cite{Butterworth:2015oua} for details,
here we only mention that the PDF4LHC15 PDF sets were constructed by means of
a statistical combination (an unweighted average) of the 
NNPDF3.0~\cite{Ball:2014uwa}, CT14 and MMHT2014 PDF sets.\footnote{The 
NNPDF3.1 PDF set was not available when the recommendations were formulated.}
%
The three PDF sets were selected among all the publicly available PDF sets
based on four criteria~\cite{Butterworth:2015oua}.
%
\begin{itemize}
%
\item A global data set, from a wide variety of observables and processes, 
should be included in the analysis.
%
\item Theoretical hard cross sections should be evaluated up to NNLO in a
general-mass variable-flavour number scheme with up to $n_f^{\rm max}=5$ 
active quark flavours.
%
\item The central value of the strong coupling at the $Z$-boson mass,
$\alpha_S(M_Z^2)$ should be fixed at an agreed common value, consistent 
with the PDF world-average~\cite{Olive:2016xmw} ($\alpha_s(M_Z)=0.118$).
%
\item All known experimental and procedural sources of uncertainty should be 
properly accounted for.
%
\end{itemize}
%
The ABMP2016 set (as well as its previous versions) does not fulfill the second 
and third criteria; the CJ15 set does not fulfill the first, second and fourth
criteria, while the HERAPDF2.0 set does not fulfill the first criterion.
%
Hence they were not included in the PDF4LHC2015 PDF sets.

\paragraph{Polarised parton distributions.}
%
We summarise the current status of global-fit results of the benchmark
moments of polarised PDFs listed in Sec.~\ref{subsubsec:BQ} in 
Tab.~\ref{tab:polPDFmoms}.
%
In the first column we indicate the computed momentum, and in the subsequent 
two columns its value as obtained from the most recent available PDF 
determinations: NNPDFpol1.1~\cite{Nocera:2014gqa}, 
DSSV08~\cite{deFlorian:2009vb}~\footnote{The DSSV08 analysis has been updated
by the DSSV14 analysis~\cite{deFlorian:2014yva} only in the determination
of the gluon PDF. The moments in Tab.~\ref{tab:polPDFmoms} are therefore  
unchanged in the two analyses.} and JAM15~\cite{Sato:2016tuz}.
%
The most relevant features of these PDF sets have been presented in
Sec.~\ref{sec:polPDFs}.
%
All values in Tab.~\ref{tab:unpPDFmoms} are displayed
at $\mu^2=Q^2=4$ GeV$^2$ at NLO.
%
The uncertainties correspond to 68\% CL bands, with no tolerance used for the 
DSS08 PDF set.
%
In the case of the JAM15 fit, we do not provide a value for 
$\langle x \rangle _{\Delta u^--\Delta d^-}$.
%
The fit is based on inclusive DIS data sets only, which are not sensitive to 
the valence distribution $\Delta u^- - \Delta d^-$.

%-------------------------------------------------------------------------------
\begin{table}[!t]
\centering
\renewcommand{\arraystretch}{1.2}
\begin{tabular}{lccc}
\toprule
Mom. 
& NNPDFpol1.1 & DSSV08 & JAM15 \\
\midrule
$\langle 1 \rangle_{\Delta u^+-\Delta d^+}$ &
\ 1.250(16) & \ 1.260(18) & 1.314(6)\, \\
$\langle 1 \rangle_{\Delta u^+}$ &
\ 0.794(46) & \ 0.814(12) & \ 0.831(21)\\
$\langle 1 \rangle_{\Delta d^+}$ &  
-0.453(52)  &  -0.456(11) &  -0.476(22)\\
$\langle 1 \rangle_{\Delta s^+}$ &  
-0.120(81)  &  -0.112(23) &  -0.109(20)\\
$\langle x \rangle_{\Delta u^- - \Delta d^-}$ &     
\ 0.195(14) &  0.203(9)\, &  --- \\
\bottomrule
\end{tabular}
\caption{\small Status of current global-fit determinations of the 
benchmark moments of polarised PDFs listed in Sec.~\ref{subsubsec:BQ}.
All values are shown at $\mu^2=Q^2=4$ GeV$^2$.}
\label{tab:polPDFmoms}
\end{table}
%-------------------------------------------------------------------------------

As outlined in Sec.~\ref{sec:polPDFs}, polarised PDFs cannot be determined in a 
global QCD analysis with the same accuracy as their unpolarised counterparts.
%
Also, because they do not enter precision physics studies at the LHC, the
selection and combination of different PDF sets have received much less
attention.
%
No recommendations analogous to those from the PDF4LHC working group
exist for polarised PDFs.
%
Therefore, in order to provide a benchmark value for the relevant moments of 
polarised PDFs listed in Tab.~\ref{tab:polPDFmoms}, we apply an unweighted 
average of the NNPDFpol1.1, DSSV08 and JAM15 PDF sets.

The rationale for this choice is twofold.
%
On the one hand, we maximise the amount of experimental information 
that can determine the central value of our benchmark moments.
%
As explained in sec.~\ref{sec:polPDFs}, the NNPDFpol1.1 and the DSSV08 PDF 
sets are based on a very similar set of inclusive DIS data, while the JAM15 
PDF set is based on a much wider number of data points.
%
This wider set can help constraining the moments of the total quark 
distributions.
%
The NNPDFpol1.1 and the DSSV08 PDF sets are based respectively on $pp$ and 
SIDIS data to disentangle the quark and antiquark distributions.
%
This can help constraining the moments of the valence distributions.
%
On the other hand, we balance the rather different uncertainties among the 
three PDF sets, specifically the very conservative NNPDFpol1.1 estimate
against the smaller DSSV08 and JAM15 values.
%
This way we avoid in particular a possible underestimation of the procedural
uncertainties induced by the choice of a simple PDF parametrisation 
adopted in the DSSV08 and JAM15 fits.

%%%%%%%%%%%%%%%%%%%%%%%%%%%%%%%%%%%%%%%%%%%%%%%%%%%%%%%%%%%%%%%%%%%%%%%%%%%%%%%%
\subsection{Comparing lattice-QCD and global-fit benchmark moments}
\label{subsec:BN}

We can now compare the lattice QCD and global fit results illustrated in 
Secs.~\ref{subsubsec:BClQCD}-\ref{subsubsec:GPDFfits} for the unpolarised
and polarised PDF moments respectively.

\paragraph{Unpolarised parton distributions.}
%
The benchmark values of the first moments of the unpolarised PDFs, obtained
as described in Secs.~\ref{subsubsec:BClQCD}-\ref{subsubsec:GPDFfits}, 
are summarised in Tab.~\ref{tab:BMunp}.
%
Results from a single lattice calculation, which might underestimate some 
sources of uncertainty, are denote with superscript~$\dagger$.
%
All values are at $\mu^2=Q^2=2$ GeV$^2$.
%
For ease of comparison, they are also displayed in 
the left panel of Fig.~\ref{fig:Bmoms}.

%-------------------------------------------------------------------------------
\begin{table}[!t]
\centering
\renewcommand{\arraystretch}{1.2}
\begin{tabular}{lcc}
\toprule
Moment & Lattice QCD & Global Fit\\
\midrule
$\langle x \rangle_{u^+ -d^+}$ 
& \numrange{0.119}{0.226} 
& 0.155(5)\\
$\langle x \rangle_{u^+}$     
& 0.453(75)$^\dagger$ 
& 0.347(5)\\
$\langle x \rangle_{d^+}$     
& 0.259(74)$^\dagger$ 
& 0.193(6)\\
$\langle x \rangle_{s^+}$     
& 0.092(41)$^\dagger$ 
& 0.036(6)\\
$\langle x\rangle_{g}$       
& 0.267(35)$^\dagger$ 
& 0.414(9)\\
\bottomrule
\end{tabular}
\caption{\small Benchmark values for lattice-QCD calculations and global-fit 
determinations of the benchmark moments of unpolarised PDFs.
%
All values are shown at $\mu^2=Q^2=4$ GeV$^2$.
%
Results with a superscript~$\dagger$ are from a single lattice 
calculation; they may underestimate some sources of uncertainty.}
\label{tab:BMunp}
\end{table}
%-------------------------------------------------------------------------------

%-------------------------------------------------------------------------------
\begin{figure}[!t]
\centering
\includegraphics[scale=0.44,angle=270]{plots/unpmoms}
\includegraphics[scale=0.44,angle=270]{plots/polmoms}\\
\caption{\small A comparison of the unpolarised (left) and polarised (right) 
PDF benchmark moments between lattice QCD computations and global fit results.
%
all values are shown at $\mu^2=Q^2=4$ GeV$^2$.}
\label{fig:Bmoms}
\end{figure}
%-------------------------------------------------------------------------------

As apparent from Tab.~\ref{tab:BMunp} and Fig.~\ref{fig:Bmoms}, there is a 
significant difference in the uncertainties between the lattice QCD and 
global fit results, with the latter always about one order of magnitude 
smaller than the former.
%
Moreover, even within their large uncertainties, the lattice QCD results are 
not compatible with the global fit results for the first moments of the 
total up and strange quark and the gluon PDFs.
%
In the case of quarks, the discrepancy is below 2$\sigma$ (in units of the 
lattice QCD uncertainty), while in the case of the gluon the discrepancy is
slightly larger than 3$\sigma$.

We note that the amount of experimental information that constrains the
total up quark distribution is the largest among all distributions.
%
Therefore it seems unlikely that its global-fit central value could vary 
significantly in the future and become compatible with the current
lattice result.
%
Conversely, the amount of experimental information that constrains the
total strange distribution in a global fit is less abundant and accurate.
%
A slight shift in its central value, towards the current lattice QCD results,
might be observed in the future, as soon as new data sensitive to the strange 
quark will become available.
%
Finally, in order to reconcile the lattice-QCD and the global-fit results
of the first moment of the gluon PDF, one should assume a completely
different behaviour of the gluon PDF below the HERA kinematic
coverage, $x\sim10^{-5}$ (see Fig.~\ref{fig:kinplot-report}).
%
Such a kinematic region remains completely unexplored.

All these remarks still hold if individual lattice-QCD and/or global-fit
results in Tabs.~\ref{tab:unpolLQCDstatus1}-\ref{tab:unpPDFmoms} are 
compared instead of their benchmark values in Tab.~\ref{tab:BMunp}. 
%
These results suggest that both greater accuracy and greater precision are
required to lattice QCD calculations in order to match the accuracy and 
precision of the first moments of unpolarised PDFs determined from a global
fit to the data.

\paragraph{Polarised parton distributions.}
%
%
The benchmark values of the first moments of the unpolarised PDFs, obtained
as described in Secs.~\ref{subsubsec:BClQCD}-\ref{subsubsec:GPDFfits}, 
are summarised in Tab.~\ref{tab:BMpol}.
%
Results from a single lattice calculation, which might underestimate some 
sources of uncertainty, are denote with superscript~$\dagger$.
%
All values are at $\mu^2=Q^2=2$ GeV$^2$.
%
For ease of comparison, they are also displayed in 
the right panel of Fig.~\ref{fig:Bmoms}.

%-------------------------------------------------------------------------------
\begin{table}[!t]
\centering
\renewcommand{\arraystretch}{1.2}
\begin{tabular}{lcc}
\toprule
Moment & Lattice QCD & Global Fit\\
\midrule
\multirow{2}{*}{$g_A\equiv\langle 1\rangle_{\Delta u^+ - \Delta d^+}$} 
& 1.195(39) ($N_f=2+1+1$) 
& \multirow{2}{*}{\ 1.275(12)} \\
& 1.279(50) ($N_f=2$) \\
$\langle 1 \rangle_{\Delta u^+}$     
& 0.830(26)$^\dagger$ 
& \ 0.813(25)\\
$\langle 1 \rangle_{\Delta d^+}$     
& -0.386(17)$^\dagger$ 
& -0.462(29)\\
$\langle 1 \rangle_{\Delta s^+}$     
& -0.052\,--\,-0.014
& -0.114(43)\\
$\langle x\rangle_{\Delta u^- - \Delta d^-}$       
& \numrange{0.146}{0.279} 
& \ 0.199(16)\\
\bottomrule
\end{tabular}
\caption{\small Same as Tab.~\ref{tab:BMunp}, but for the polarised benchmark 
moments.}
\label{tab:BMpol}
\end{table}
%-------------------------------------------------------------------------------











%We denote these cases in Tab.\ref{tab:LQCDunpol} with a superscript dagger.
%We provide a systematic comparison between existing lattice-QCD computations
%and global PDF fit results for the lowest two moments of unpolarized
%and longitudinally polarized PDFs. 
%
%Current lattice calculations of higher moments are not sufficiently controlled 
%to allow for a meaningful comparison between lattice and global fit results.
%
%With the exception of the axial coupling, $g_A$, there are no lattice calculations
%for which all systematics have been fully explored and controlled. Therefore, 
%for quantities other than $g_A$, we present a best-estimate band, rather than a global average. 
%This band extends from the mean minus the error of the
%smallest result to the mean plus the error of the largest, and includes all results
%with two or more flavours of sea quarks, because current studies are not sufficiently
%precise to distinguish between results with different numbers of sea quark flavours.
%In three cases, $\langle x \rangle_{u,d,s}$, $\langle x \rangle_g$, $\langle 1 \rangle_{\Delta u,\Delta d}$, there is only one
%lattice result available. For these quantities, our best-estimate band is given by this single result, but 
%it should be noted that these results may underestimate some sources of uncertainty. We denote
%these cases in Tables \ref{tab:LQCDunpol} and \ref{tab:LQCDpol0} with a superscript dagger.
%In all benchmark tables, the moments are understood to be evaluated at a scale $\mu^2 = Q^2=4$ GeV$^2$. 
%
%We provide complete listings of available results in Appendix \ref{sec:LQCDtables}.
