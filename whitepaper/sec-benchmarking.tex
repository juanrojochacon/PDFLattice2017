%%%%%%%%%%%%%%%%%%%%%%%%%%%%%%%%%%%%%%%%%%%%%%%%%%%%%%%%%%%%%%%%%%%%%%%%%%%%%%%%
\section{Benchmarks}
\label{sec:benchmarking}
%%%%%%%%%%%%%%%%%%%%%%%%%%%%%%%%%%%%%%%%%%%%%%%%%%%%%%%%%%%%%%%%%%%%%%%%%%%%%%%%

In this section we present benchmarks for calculations of the low 
Mellin moments of PDFs and of the $x$-dependence of PDFs. We provide a systematic
comparison between lattice-QCD results and determinations from global fitting
calculations.
%
%%%%%%%%%%%%%%%%%%%%%%%%%%%%%%%%%%%%%%%%%%%%%%%%%%%%%%%%%%%%%%%%%%%%%%%%%%%%%%%%
\subsection{Benchmark criteria for lattice QCD}
To accurately assess the current state-of-the-art for lattice calculations, we
follow a procedure inspired by the review of low-energy mesons undertaken 
by the Flavor Lattice Averaging Group (FLAG) \cite{Aoki:2016frl}. For each
lattice calculation, we assess the status of each source of uncertainty outlined
in Sec.~\ref{Sec:IntroLQCD}. We use a rating system similar to FLAG, awarding a
blue star (\bstar) for sources of uncertainty that are well-controlled or very conservatively
estimated, a blue circle (\bcirc) for sources of uncertainty that have been controlled or estimated to some extent,
and a red square (\rsquare) for uncertainties that have not met our criteria or for which no estimate is given.

We present our determination of the current, state-of-the-art
``world average'' for lattice-QCD results of the benchmark quantities. and 
provide both summary tables and overview comparison plots. Following FLAG, we include only those results that
are published in peer-reviewed journals or that have appeared as preprints. Where
recent results are a clear update of previously published work, we do not 
include the earlier results in the global average, but include them in the bibliography. Finally,
we remark that our criteria, and the corresponding ratings, are chosen not only
to provide as fair an assessment of the relative merits of various calculations as
possible, but to be aspirational. Where lattice-QCD results do meet these
standards, we hope that the lattice community will work towards improved calculations
and greater precision as part of a concerted and inter-disciplinary effort to better understand
nucleon structure.
%
We apply the following criteria and award a red square (\rsquare) wherever the criteria listed are not met.
\paragraph{Discretisation effects and the continuum limit} We assume that the lattice actions are
${\cal O}(a)$-improved, that is, that the discretization errors vanish quadratically with the lattice spacing. 
For unimproved actions an additional lattice spacing is required. We require that these
criteria are satisfied in each case for at least one pion mass below 300 MeV.
\begin{itemize}
\item[\bstar] At least three lattice spacings with at least two lattice spacings below 0.1 fm and
a range of lattice spacings that satisfies $[a_{\mathrm{max}}/a_{\mathrm{min}}]^2 \geq 2$.
\item[\bcirc] At least two lattice spacings with at least one point below 0.1 fm and a range of lattice spacings 
satisfies $[a_{\mathrm{max}}/a_{\mathrm{min}}]^2 \geq 1.4$.
\end{itemize}
To receive a \bstar~or \bcirc~either a continuum extrapolation must be performed or the results must
demonstrate no significant discretisation effects over the appropriate range of lattice spacings.

\paragraph{Unphysical pion masses} For the following criteria, we define a physical pion mass ensemble 
to be one with $M_\pi=135\pm 10$~MeV.
\begin{itemize}
\item[\bstar] One ensemble with a physical pion mass \emph{or} a chiral extrapolation with three or more pion masses,
with at least two pion masses below 250 MeV and at least one below 200~MeV.
\item[\bcirc] A chiral extrapolation with three or more pion mass, two of which are below 300~MeV.
\end{itemize}

\paragraph{Finite volume effects} For calculations that use a mixed action approach, that is,
with different lattice actions for the valence and sea quarks, we apply the criteria for $M_\pi L$ to the valence quarks.
\begin{itemize}
\item[\bstar] Ensembles with $M_{\pi,\mathrm{min}}L\geq 4$, \emph{or} at least three volumes with spatial extent
$L>2.5$~fm.
\item[\bcirc] Ensembles with $M_{\pi,\mathrm{min}}L \geq 3.4$, \emph{or} at least two volumes with spatial extent
$L>2.5$~fm.
\end{itemize}

\paragraph{Excited state contamination} The following criteria must be satisfied for every pion mass and lattice spacing.
\begin{itemize}
\item[\bstar] At least three source-sink  separations or a variational method to optimise the
operator derived from at least a $3\times 3$ correlator matrix. 
\item[\bcirc] Two source-sink separations at every pion mass and lattice spacing, or three or more source-sink separations at one pion mass
below 300~MeV. For the variational method, an optimised operator derived from a $2\times 2$ correlator matrix at every pion mass and
lattice spacing, or a $3\times 3$ correlator matrix for two pion masses below 300 MeV.
\end{itemize}

\paragraph{Renormalization} 
\begin{itemize}
\item[\bstar] Nonperturbative renormalisation.
\item[\bcirc] Perturbative renormalisation.
\end{itemize}
For the axial coupling, $g_A$, we also award a \bstar~for calculations that use fermion actions for which $Z_A/Z_V=1$ or employ combinations of quantities for which the renormalisation is unity by construction.

\paragraph{Lattice-spacing determination} For lattice-QCD calculations of nucleons, the lattice-spacing determination is generally 
sufficiently precise that it is a very small or negligible source
of systematic uncertainty. Therefore we do not include an assessment of the lattice-spacing
determination in our criteria.

%%%%%%%%%%%%%%%%%%%%%%%%%%%%%%%%%%%%%%%%%%%%%%%%%%%%%%%%%%%%%%%%%%%%%%%%%%%%%%%%
\subsection{Moments}

Here, we provide a systematic comparison between existing
lattice-QCD results and PDF fitting calculations
for a number of moments both for unpolarized
and for polarized PDFs.
%


%%%%%%%%%%%%%%%%%%%%%%%%%%%%%%%%%%%%%%%
\subsubsection{Unpolarized parton distributions}

The moments that we will compare are the following:

\begin{itemize}

\item Second moment of $u^+-d^+$, defined as:
  \be
  \la x \ra_{u^+-d^+}\equiv \int_0^1 dx\,x\,\lc u(x,\mu)
  +\bar{u}(x,\mu)
-d(x,\mu) - \bar{d}(x,\mu) \, .
  \rc
  \ee

\item Third moment of $u^--d^-$, defined as:
  \be
  \la x^2 \ra_{u^--d^-}\equiv \int_0^1 dx\,x^2\,\lc u(x,\mu)
  -\bar{u}(x,\mu)
-d(x,\mu) + \bar{d}(x,\mu) \, .
  \rc
  \ee

  \end{itemize}

In all the cases, $\mu$ should be identified with the QCD
factorization scale, the scale that separates
long-distance from short-distance dynamics in perturbative
factorization.
%
Note that it is customary in PDF fits to assume
$\mu=\mu_F=\mu_R$, though in principle the two scales
could have different values.


\textcolor{blue}{FIO: added sample tables copied from presentations; 
if the format is OK we can replace/update with the real benchmarks. }

%%%%%%%%%%%%%%%%%%%%%%%%%%%%%%%%%%%%%%%%%%%%%%%%%%%%%%%%%%%%%
%%%%%%%%%%%%%%%%%%%%%%%%%%%%%%%%%%%%%%%%%%%%%%%%%%%%%%%%%%%%%
\begin{table}[t]
\renewcommand{\arraystretch}{1.2} 
\centering
\begin{tabular}{@{}lccc@{}}
\hline 
\rule[-3 ex]{0pt}{7 ex}  %% add some extra space
$\left\langle x\right\rangle _{u}-\left\langle x\right\rangle _{d}$ 
   & Central Value & PDF error (\%) & Shift (\%) \\
\hline 
NNPDF3.0 & 0.136 & 2.4 & -\\
CT14 & 0.140 & 3.4 & +2.5 \\
MMHT14 & 0.134 & 2.6 & -1.5 \\
ABMP16 & 0.150 & 1.9 & +10 \\
... &  &  & \\
\hline 
\end{tabular}

\caption{Benchmark for $\left\langle x\right\rangle _{u}-\left\langle x\right\rangle _{d}$
... at $Q=5$ GeV}

\end{table}
%%%%%%%%%%%%%%%%%%%%%%%%%%%%%%%%%%%%%%%%%%%%%%%%%%%%%%%%%%%%%
%%%%%%%%%%%%%%%%%%%%%%%%%%%%%%%%%%%%%%%%%%%%%%%%%%%%%%%%%%%%%


%%%%%%%%%%%%%%%%%%%%%%%%%%%%%%%%%%%%%%%%%%%%%%%%%%%%%%%%%%%%%
%%%%%%%%%%%%%%%%%%%%%%%%%%%%%%%%%%%%%%%%%%%%%%%%%%%%%%%%%%%%%
\begin{table}[t]
\renewcommand{\arraystretch}{1.2} 
\centering
\begin{tabular}{@{}lccc@{}}
\hline 
\rule[-3 ex]{0pt}{7 ex}  %% add some extra space
$\left\langle x\right\rangle _{u}-\left\langle x\right\rangle _{d}$ 
   & Central Value & PDF error (\%) & Shift (\%) \\
\hline 
NNPDF3.0 & 0.102 & 2.4 & -\\
CT14 & 0.104 & 3.2 & +2.4\\
MMHT14 & 0.101 & 2.6 & -1.5\\
ABMP16 & 0.113 & 1.9 & +11\\
... &  &  & \\
\hline 
\end{tabular}
\caption{Benchmark for $\left\langle x\right\rangle _{u}-\left\langle x\right\rangle _{d}$
... at $Q=100$ GeV}
\end{table}
%%%%%%%%%%%%%%%%%%%%%%%%%%%%%%%%%%%%%%%%%%%%%%%%%%%%%%%%%%%%%
%%%%%%%%%%%%%%%%%%%%%%%%%%%%%%%%%%%%%%%%%%%%%%%%%%%%%%%%%%%%%

%%%%%%%%%%%%%%%%%%%%%%%%%%%%%%%%%%%%%%%%%%%%%%%%%%%%%%%%%%%%%
%%%%%%%%%%%%%%%%%%%%%%%%%%%%%%%%%%%%%%%%%%%%%%%%%%%%%%%%%%%%%
\begin{table}[t]
\renewcommand{\arraystretch}{1.2} 
\centering
\begin{tabular}{@{}lccc@{}}
\hline 
\rule[-3 ex]{0pt}{7 ex}  %% add some extra space
$\left\langle x\right\rangle _{\bar{u}}-\left\langle x\right\rangle _{\bar{d}}$ 
   & Central Value & PDF error (\%) & Shift (\%)\\
\hline 
NNPDF3.0 & -0.0038 & 51 & -\\
CT14 & -0.0055 & 25 & +43\\
MMHT14 & -0.0060 & 14 & +57\\
ABMP16 & -0.0059 & 11 & +54\\
... &  &  & \\
\hline 
\end{tabular}
\caption{Benchmark for $\left\langle x\right\rangle _{\bar{u}}-\left\langle x\right\rangle _{\bar{d}}$
... at $Q=5$ GeV}
\end{table}
%%%%%%%%%%%%%%%%%%%%%%%%%%%%%%%%%%%%%%%%%%%%%%%%%%%%%%%%%%%%%
%%%%%%%%%%%%%%%%%%%%%%%%%%%%%%%%%%%%%%%%%%%%%%%%%%%%%%%%%%%%%

%%%%%%%%%%%%%%%%%%%%%%%%%%%%%%%%%%%%%%%%%%%%%%%%%%%%%%%%%%%%%
%%%%%%%%%%%%%%%%%%%%%%%%%%%%%%%%%%%%%%%%%%%%%%%%%%%%%%%%%%%%%
\begin{table}[t]
\renewcommand{\arraystretch}{1.2} 
\centering
\begin{tabular}{@{}lccc@{}} 
\hline 
\rule[-3 ex]{0pt}{7 ex}  %% add some extra space
$\left\langle x\right\rangle _{s^{+}}$ 
   & Central Value & PDF error (\%) & Shift (\%) \\
\hline 
NNPDF3.0 & 0.46 & 6 & -\\
CT14 & 0.43 & 18 & -7\\
MMHT14 & 0.43 & 16 & -7\\
ABMP16 & 0.47 & 3 & +2\\
... &  &  & \\
\hline 
\end{tabular}

\caption{Benchmark for $\left\langle x\right\rangle _{s^{+}}$ ... at $Q=5$
GeV}
\end{table}
%%%%%%%%%%%%%%%%%%%%%%%%%%%%%%%%%%%%%%%%%%%%%%%%%%%%%%%%%%%%%
%%%%%%%%%%%%%%%%%%%%%%%%%%%%%%%%%%%%%%%%%%%%%%%%%%%%%%%%%%%%%

%%%%%%%%%%%%%%%%%%%%%%%%%%%%%%%%%%%%%%%%%%%%%%%%%%%%%%%%%%%%%
%%%%%%%%%%%%%%%%%%%%%%%%%%%%%%%%%%%%%%%%%%%%%%%%%%%%%%%%%%%%%
\begin{table}[t]
\renewcommand{\arraystretch}{1.2} 
\centering
\begin{tabular}{@{}lccc@{}}
\hline 
\rule[-3 ex]{0pt}{7 ex}  %% add some extra space
$\dfrac{\left\langle x\right\rangle _{s^{+}}}{\left\langle x\right\rangle _{\bar{u}}-\left\langle x\right\rangle _{\bar{d}}}$ 
   & Central Value & PDF error (\%) & Shift (\%)\\
\hline 
NNPDF3.0 & 0.64 & 8 & -\\
CT14 & 0.62 & 21 & -3\\
MMHT14 & 0.59 & 19 & -7\\
ABMP16 & 0.66 & 4 & 4\\
... &  &  & \\
\hline 
\end{tabular}
\caption{Benchmark for $\left\langle x\right\rangle _{s^{+}}/\left[\left\langle x\right\rangle _{\bar{u}}-\left\langle x\right\rangle _{\bar{d}}\right]$
... at $Q=5$ GeV}
\end{table}
%%%%%%%%%%%%%%%%%%%%%%%%%%%%%%%%%%%%%%%%%%%%%%%%%%%%%%%%%%%%%
%%%%%%%%%%%%%%%%%%%%%%%%%%%%%%%%%%%%%%%%%%%%%%%%%%%%%%%%%%%%%



%%%%%%%%%%%%%%%%%%%%%%%%%%%%%%%%%%%%%%%
\subsubsection{Polarized parton distributions}

Next we repeat the same exercise, now for the moments
of polarized parton distributions.

%%%%%%%%%%%%%%%%%%%%%%%%%%%%%%%%%%%%%%%
\subsection{$x$-space dependent PDFs}

Next we present a number of benchmark comparisons for
PDFs at the level of their Bjorken-$x$
dependence.
