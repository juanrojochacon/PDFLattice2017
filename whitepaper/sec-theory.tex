
\section{Theory overview}
\label{sec:theoryoverview}

It is specially important that we make sure that we use a consistent
notation both in the lattice sections and in the global
PDF fitting versions.

\subsection{Lattice QCD}
\label{Sec:IntroLQCD}
\input{IntroLQCD}

\subsubsection{Mellin moments of PDFs from lattice QCD}
\label{Sec:MomentsLQCD}
\input{MomentsLQCD}

\subsubsection{The $x$-dependence of PDFs from lattice QCD}

\paragraph{Inversion method}
%\label{Sec:InversionMethod}
\input{InversionMethod}

\paragraph{RQCD Approach}
%Gunnar Bali (0.5 page)

\paragraph{PDFs from the Hadronic Tensor}
%\label{sec:HadronicTensorMethod}
\input{HadronicTensorMethod}

\paragraph{Quasi-PDFs}
%\label{Sec:QuasiPDFMethod}
\input{QuasiPDFMethod}




\subsection{Global PDF fits}

Collinear unpolarized and polarized PDF analysis.

%%%%%%%%%%%%%%%%%%%%%%%%%%%%%%%%%%%%%%%%%%%%%%%%%%%%%%%
\subsubsection{Unpolarized PDFs}




\textcolor{blue}{HEADINGS: this is just for editing; we'll remove in the final. }

\textcolor{blue}{INTRO:} 
We express the collinear unpolarized PDFs as $f_{i}(x,\mu)$
where the index $i$ represents the parton flavor $i=\{g,u,\bar{u},d,\bar{d},s,\bar{s},...\}$,
$x$ is the fractional momentum carried by the parton, and $\mu$
is the factorization (energy) scale.\footnote{We could add an additional index to specify the particular hadron
(proton, neutron, pion, nuclei, ...); as we mainly refer to the proton
in this work, we will omit such a designation unless necessary.} The PDF is a scheme-dependent quantity, and we typically work in
the $\overline{MS}$ scheme; when this is convoluted with an appropriate
hard cross section (Wilson coefficient), we obtain a scheme-independent
physical observable. 

\textcolor{blue}{X-DEPENDENCE:} 
The $x$-dependence of the PDFs must
be deduced by comparing with experimental data in a global fit. For
this purpose, we often parameterize the $x$-dependence of the PDFs
in the generic form: 
\begin{equation}
f_{i}(x,\mu)\sim x^{a}(1-x)^{b}\:C(x)\quad.
\end{equation}
Here, the $x^{a}$ term controls the small-$x$ behavior, the $(1-x)^{b}$
term controls the large-$x$ behavior, and $C(x)$ represents the
remaining $x$-dependence. The $a$ exponent is negative and generally
in the range $-1$ to $-2$; thus, the PDFs diverge as $1/x^{\sim1.5}$
for small $x$ and the number of soft partons is infinite. The $a$
exponent must be larger than $-2$ or the momentum sum rule will diverge.
The $b$ exponent is positive, and this ensures the PDF goes to zero
as $x\to1$.


\textcolor{blue}{NUMBER SUM RULES:} 
There are a few constraints we
can impose on the $x$-dependence of the PDFs at this point. Since
the proton has the quantum numbers of two up quarks and one down quark,
we have the following quark number sum rules given in terms of first
moments: 
%
\begin{eqnarray*}
\int_{0}^{1}dx\ \left[u(x,\mu)-\bar{u}(x,\mu)\right] & =\left\langle 1\right\rangle _{u^{-}}= & 2\\
\int_{0}^{1}dx\ \left[d(x,\mu)-\bar{d}(x,\mu)\right] & =\left\langle 1\right\rangle _{d^{-}}= & 1\\
\int_{0}^{1}dx\ \left[s(x,\mu)-\bar{s}(x,\mu)\right] & =\left\langle 1\right\rangle _{s^{-}}= & 0
\end{eqnarray*}
with similar results for the heavy quarks: $\left\langle 1\right\rangle _{c^{-}}=\left\langle 1\right\rangle _{b^{-}}=\left\langle 1\right\rangle _{t^{-}}=0$.

\textcolor{blue}{MOMENTUM SUM RULE:} The fractional momentum carried
by each parton flavor is given by the second moments: 
%
\begin{eqnarray*}
\int_{0}^{1}dx\ x\ \left[u(x,\mu)\right] & = & \left\langle x\right\rangle _{u}\\
\int_{0}^{1}dx\ x\ \left[u(x,\mu)+\bar{u}(x,\mu)\right] & = & \left\langle x\right\rangle _{u^{+}}\ .
\end{eqnarray*}
%
Here, $\left\langle x\right\rangle _{u}$ is the fractional momentum
carried by the up-quark, $\left\langle x\right\rangle _{u^{+}}$ is
the fractional momentum carried by the up-quark and anti-up-quark.
Since the total momentum of the proton must equal the momentum of
its constituents, we have the momentum sum rule constraint: 
%
\begin{eqnarray*}
1 & = & \left\langle x\right\rangle _{g}+\left\langle x\right\rangle _{u^{+}}+\left\langle x\right\rangle _{d^{+}}+\left\langle x\right\rangle _{s^{+}}+\left\langle x\right\rangle _{c^{+}}+\left\langle x\right\rangle _{b^{+}}+\left\langle x\right\rangle _{t^{+}}+...
\end{eqnarray*}
%
where the ``...'' represents any other partonic components (such
as a photon). 

\textcolor{blue}{$\mu$-DEPENDENCE:} 
The $\mu$-dependence of the
PDFs is given by the DGLAP evolution equation\cite{Dokshitzer:1977sg,Gribov:1972ri,Altarelli:1977zs}
\begin{eqnarray*}
\frac{\partial}{\partial\ln\mu^{2}}\:f_{i}(x,\mu) & = & \sum_{j=g,q,\bar{q}}\ P_{ij}(x)\otimes f_{j}(x,\mu)\quad.
\end{eqnarray*}
Here, the logarithmic derivative of the PDF is determined by a convolution
of the PDFs with the DGLAP kernel $P_{ij}(x)$ which can be computed
perturbatively in powers of $\alpha_{s}(\mu)$; $P_{ij}(x)$ is known
to NNLO.\footnote{The DGLAP (Dokshitzer\textendash Gribov\textendash Lipatov\textendash Altarelli\textendash Parisi)
evolution can be modified by $\ln(1/x)$ terms at small-$x$, and
this is characterized by the BFLK (Balitsky-FadinKuraev-Lipatov) equations.\cite{Kuraev:1976ge,Kuraev:1977fs,Balitsky:1978ic}
Additionally, at large-$x$ and small $\mu$ scale the above framework
can receive corrections from non-factorizeable higher-twist (HT) corrections.} When performing the global fit to the data, we use the DGLAP equations
to combine data from different $\mu$ scales when constraining the
PDFs. 




%%%%%%%%%%%%%%%%%%%%%%%%%%%%%%%%%%%%%%%%%%%%%%%%%%%%%%%
\subsubsection{Polarized PDFs}
