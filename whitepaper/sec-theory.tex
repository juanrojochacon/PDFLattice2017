
\section{Theory overview}
\label{sec:theoryoverview}

It is specially important that we make sure that we use a consistent
notation both in the lattice sections and in the global
PDF fitting versions.

\subsection{Lattice QCD}
\label{Sec:IntroLQCD}
Quantum Chromodynamics (QCD) draws together the ideas of the quark model, the concept of color, and the notion of constituent partons into a single gauge theory that captures our current understanding of the strong nuclear force. The viability of QCD at hadronic energy scales relies on two rather remarkable features, quark confinement and spontaneous chiral symmetry breaking. The study of these phenomena requires a tool that goes beyond perturbation theory, a tool that can handle both strongly-coupled fields at long distances and the ultraviolet divergences of the theory at short distances. This tool is Lattice Quantum Chromodynamics (LQCD).

LQCD is QCD formulated on a finite Euclidean lattice and is generally studied by numerical computation of QCD correlation functions in the path-integral formalism, using methods adapted from statistical mechanics. To make contact with the physical world and experimental data, the numerical results are extrapolated to the infinite-volume and continuum limits. The past decade has seen significant progress in the development of efficient algorithms for the generation of gauge field configurations, which represent the QCD vacuum, and tools for extracting the relevant information from LQCD correlation functions. LQCD calculations have reached a level where they not only complement, but also guide current and forthcoming experimental programs.

LQCD calculations suffer from several sources of systematic uncertainty, each of which must be controlled to make meaningful contact with experimental data: discretisation effects and the continuum limit; unphysical pion masses; finite volume effects; renormalisation; and lattice-spacing determination. We briefly review these main sources of systematic uncertainty here; for a fuller account see, for example, \cite{Aoki:2016frl}.

\paragraph{Discretisation effects and the continuum limit} There is a fair degree of flexibility in discretising the QCD action. This has led to a variety of formulations, which differ mainly in the choice of fermion action. In the continuum limit, which corresponds to taking the lattice spacing $a$ to zero with all physical quantities fixed, the simplest discretisations differ from continuum QCD at ${\cal O}(a)$. In practice, one cannot afford to perform numerical simulations at arbitrarily small lattice spacings, because the cost of computation increases with a large inverse power of the lattice spacing, and ${\cal O}(a)$ effects can therefore be significant, even with current lattice spacings ranging from $0.3 \,\mbox{fm}$ to $0.05 \,\mbox{fm}$. To accelerate the convergence to the continuum limit, improved quark and gluon actions are widely used, which include higher-dimension operators to reduce the discretisation errors to $O(a^2)$ or better.

\paragraph{Unphysical pion masses} The computational cost of the fermion contribution to the path integral increases with a large inverse power of the bare quark mass (or, equivalently, the pion mass). LQCD calculations are therefore often performed at unphysically heavy pion masses, although results at the physical point have become increasingly common, albeit with large errors. To obtain results at the physical pion mass, lattice data are generated at a sequence of pion masses and then extrapolated to the physical pion mass. To control the associated systematic uncertainties, these extrapolations are guided by effective theories. In particular, the pion-mass dependence can be parameterised with chiral perturbation theory ($\chi$PT), which accounts for the Nambu-Goldstone nature of the lowest excitations that occur in the presence of light quarks.

\paragraph{Finite volume effects} Numerical LQCD calculations are necessarily restricted to a finite spacetime volume. For most simple quantities, these effects decay exponentially with the size of the lattice, and therefore the easiest way to minimise or eliminate finite volume effects is to choose the volume sufficiently large in physical units. Unfortunately, this can be prohibitively expensive as one approaches the continuum limit and the number of lattice sites grows. Finite volume $\chi$PT is the preferred tool to develop systematic expansions that provide quantitative information on finite-volume effects. In general, finite volume effects of hadrons are dominated by their interactions with pions, which can travel around the (periodic) lattice many times. Numerical evidence suggests that lattice sizes of $m_\pi L \geq 4$, where $m_\pi$ is the pion mass, are generally sufficiently large that finite volume effects are negligible, with the current precision of LQCD calculations.

\paragraph{Renormalisation} The matrix elements extracted from a LQCD calculation at a given lattice spacing are bare matrix elements, rendered finite by the presence of the lattice spacing, which serves as a gauge-invariant UV regulator. To take the continuum limit, i.e. remove the regulator, one must renormalise the corresponding operators and fields. Although renormalisation is traditionally discussed in the framework of perturbation theory, at hadronic energy scales the renormalisation constants should be computed non-perturbatively to avoid uncontrolled perturbative truncation uncertainties. In QCD with only light quarks it is technically advantageous to employ so-called mass-independent renormalisation schemes. This requires a renormalisation condition that can be implemented on the lattice as well as in continuum perturbation theory, for example, the RI-MOM scheme~\cite{Martinelli:1994ty}. 

In addition, on a hypercubic lattice, the orthogonal group $O(4)$ of continuum Euclidean spacetime is reduced to the hypercubic group $H(4)$. Thus, operators are classified according to irreducible representations of $H(4)$~\cite{Gockeler:1996mu}. Different irreducible representations belonging to the same $O(4)$ multiplet will, in general, give different answers at finite lattice spacing, an effect that can be reduced by improving the operators~\cite{Gockeler:2004wp}. Conversely, operators that lie in different irreducible representations $O(4)$, but the irreducible representations of $H(4)$, will mix at finite lattice spacing but not in the continuum. When these operators have differing mass dimension, the mixing coefficients scale with the inverse lattice spacing to some power, and diverge in the continuum limit. This power-divergent mixing must be removed non-perturbatively, and is a particular challenge for lattice calculations of the Mellin moments of PDFs (see Section \ref{Sec:MomentsLQCD}).

Finally, it is worth noting that factorisation, the key assumption of the operator product expansion (OPE), demands that the non-perturbatively renormalised hadron matrix elements match the perturbatively renormalised Wilson coefficients. This appears to be the case for scales $\mu^2 \gtrsim 10 \, \mbox{GeV}^2$ at least~\cite{Gockeler:2010yr}. This, however, is a fundamental aspect of QCD, and is not restricted to LQCD. The DGLAP evolution equations, for example, work best for $q^2_{\rm min} \approx 15 \, \mbox{GeV}^2$~\cite{Abramowicz:2015mha}, which should be kept in mind when comparing lattice results with phenomenology.

\paragraph{lattice-spacing determination} Numerical LQCD calculations naturally determine dimensionless quantities. Extracting physical values requires the introduction of a scale, taken from experiment or through a well-defined theoretical procedure, that is referred to as ``scale-setting''. Popular reference scales include light decay constants, hadron masses, scales defined in terms of the heavy quark potential or, most recently, the Wilson flow time $\sqrt{t_0}$~\cite{Luscher:2010iy}. The Wilson flow scale has become increasingly popular, because it can be computed rather cheaply and with high precision, unlike hadron masses, for example.

These sources of systematic uncertainty all need to be under control when confronting experimental data with lattice results, or vice versa. For a coherent assessment of the present state of LQCD calculations of various quantities, the degree to which each systematic has been controlled in a given calculation is an important consideration. Therefore, in the following sections, we indicate the quality of the lattice calculations, based on criteria inspired by the FLAG analysis of flavour physics on the lattice~\cite{Aoki:2016frl}.

\subsubsection{Mellin moments of PDFs from lattice QCD}
\label{Sec:MomentsLQCD}
Parton distribution functions (PDFs) are defined as matrix elements of fields at light-like separations, which cannot be directly determined in Euclidean LQCD. Instead, the traditional approach for LQCD calculations is to determine the matrix elements of local twist-two operators, where twist is the dimension minus the spin, that can be related to the Mellin moments of PDFs. In principle, given a sufficient number of Mellin moments, PDFs can be reconstructed from the inverse Mellin transform. In practice, however, the calculation is limited to the lowest three moments, because power-divergent mixing occurs between twist-two operators on the lattice. Three moments is insufficient to reconstruct the momentum dependence of the PDFs without significant model dependence~\cite{Detmold:2003rq}.

The lowest three moments do provide, however, useful information, both as benchmarks of LQCD calculations and as constraints in global extractions of PDFs. Here we briefly review the determination of Mellin moments of PDFs from LQCD. 

PDFs are accessible, experimentally and theoretically, through the Compton amplitude
\begin{equation}
T_{\mu\nu}(p,q) = \int {\rm d}^4\!x\, {\rm e}^{iqx}  \langle p,s |T J_\mu(x) J_\nu(0)|p,s\rangle   
\end{equation}
at large photon momenta $q^2=-Q^2$. Here $T$ is the time-ordering operator, the $J_\mu(x)$ are local vector bilinears at spacetime point $x$, and the external states are hadronic states with momentum $p$ and spin $s$. The most general form of the Compton amplitude $T_{\mu\nu}(p,q)$ for polarised deep-inelastic scattering from a proton targets reads
\begin{align}
T_{\mu\nu}(p,q) = {} & \left(\delta_{\mu\nu}-\frac{q_\mu q_\nu}{q^2}\right) \mathcal{F}_1(\omega,Q^2) + \left(p_\mu-\frac{pq}{q^2}q_\mu\right) \left(p_\nu-\frac{pq}{q^2}q_\nu\right) \frac{1}{pq} \mathcal{F}_2(\omega,Q^2)\\ 
& {} \quad  + \epsilon_{\mu\nu\lambda\sigma}q_\lambda s_\sigma \frac{1}{pq}\mathcal{G}_1(\omega,Q^2) + \epsilon_{\mu\nu\lambda\sigma}q_\lambda \left(pq s_\sigma - sq p_\sigma\right) \frac{1}{(pq)^2}\mathcal{G}_2(\omega,Q^2)
\end{align}
where $\mathcal{F}_1$, $\mathcal{F}_2$, $\mathcal{G}_1$ and $\mathcal{G}_2$ have different physical interpretations [\textbf{Can we be more specific here?}]. The unpolarised and polarised PDFs can be deduced from $\mathcal{F}_1$ and $\mathcal{G}_1$ by first rewriting them in terms of the structure functions spin-independent and spin-dependent structure functions, $F_1(x,Q^2)$ and $g_1(x,Q^2)$,
\begin{align}
\mathcal{F}_1(\omega,Q^2) = {} & 4 \omega^2 \int_0^1 dx\,  \frac{xF_1(x,Q^2)}{1-(\omega x)^2} = \sum_{n=2,4,\cdots}^\infty 4\omega^n \int_0^1 dx\, x^{n-1} F_1(x,Q^2) \,, \\
\mathcal{G}_1(\omega,Q^2) = {} & 4 \omega \int_0^1 dx\, \frac{g_1(x,Q^2)}{1-(\omega x)^2} = \sum_{n=1,3,\cdots}^\infty 4\omega^n \int_0^1 dx\, x^{n-1} g_1(x,Q^2).
\end{align}
Here $\omega=2pq/q^2$ and target mass corrections have been discarded. Odd moments may be obtained from the mixture of currents $J_\mu J_\nu^5$.

At sufficiently high momentum transfer that power corrections can be neglected, factorisation~\cite{Sterman:1995fz} at the scale $\mu$ lets us write the structure functions $F_1(x,Q^2)$ and $g_1(x)$ of the proton as 
\begin{align}
F_1(x,Q^2) = {} & \sum_{i=u,d, \cdots, g} \int_x^1\frac{dy}{y}\, c_{1,i}(x/y,\mu^2)|_{\mu^2=Q^2}\, f_i(y,Q^2) \,,\\
g_1(x,Q^2) = {} & \sum_{i=u,d, \cdots, g} \int_x^1\frac{dy}{y}\, e_{1,i}(x/y,\mu^2)|_{\mu^2=Q^2}\, \Delta f_i(y,Q^2) \,,
\label{pdf}
\end{align}
where $c_{1,i}$ and $e_{1,i}$ are splitting functions and the $f_i(x,Q^2)$ and $\Delta f_i(x,Q^2)$ are the unpolarised and polarised PDFs, respectively. Here the sum runs over the quark and antiquark flavours, $u$, $d$, $\bar{u}$ and $\bar{d}$, and the gluon $g$.

Using the operator product expansion (OPE), the Mellin moments of structure functions, and the corresponding PDFs, can be expressed in terms of matrix elements of local operators:
\begin{align}
2 \int_0^1 dx\, x^{n-1} F_1(x,Q^2) &= \sum_{i=u,d, \cdots, g} c_{1,i}^n(\mu^2)\, v_i^n(\mu^2)|_{\mu^2=Q^2} = \sum_{i=u,d, \cdots, g} c_{1,i}^n(\mu^2)\, \int_0^1 dx\, x^{n-1} f_i(x,Q^2)\,,\\
4 \int_0^1 dx\, x^n g_1(x,Q^2) &= \sum_{i=u,d, \cdots, g} e_{1,i}^n(\mu^2)\, a_i^n(\mu^2)|_{\mu^2=Q^2} = \sum_{i=u,d, \cdots, g} e_{1,i}^n(\mu^2)\, \int_0^1 dx\, x^n\, 2 \Delta f_i(x,Q^2) \,,
\end{align}
where $v_i^n(\mu^2)$ and $a_i^n(\mu^2)$ are reduced matrix elements of the appropriate twist-two operators~\cite{Gockeler:1995wg},
\begin{equation}
\begin{split}
\frac{1}{2} \sum_s \langle p,s|\mathcal{O}^i_{\{\mu_1,\cdots,\mu_n\}}|p,s\rangle &= 2 v_i^n\, [p_{\mu_1}\cdots p_{\mu_n} - {\rm traces}] \,, \\
\langle p,s|\mathcal{O}^{5\,i}_{\{\sigma \mu_1,\cdots,\mu_n\}}|p,s\rangle &= \frac{1}{n+1} a_i^n\, [s_\sigma p_{\mu_1}\cdots p_{\mu_n} - {\rm traces}] \\
\end{split}
\end{equation}
and $c_{1,i}^n(\mu^2)$ and $e_{1,i}^n(\mu^2)$ are the corresponding Wilson coefficients
\begin{equation}
c_{1,i}^n(\mu^2) = \int_0^1 dy\, y^{n-1} c_{1,i}(y,\mu^2)\,, \quad
e_{1,i}^n(\mu^2) = \int_0^1 dy\, y^n e_{1,i}(y,\mu^2)\,.
\end{equation}
The operators relevant for the lowest two moments are listed in Table \ref{Tab:twist2}. The operator $\mathcal{O}^q_{\mu_1\mu_2}$ decomposes into two different representations of $H(4)$~\cite{Gockeler:1996mu}, each with different lattice artifacts and renormalisation factors. In the continuum limit, however, both operators should lead to the same result. In contrast, the operator $\mathcal{O}^q_{\mu_1\mu_2\mu_3}$ splits into several representations transforming identically under $H(4)$, causing the corresponding operators to mix under renormalisation on the lattice.
\begin{table}
\caption{\label{Tab:twist2}
Operators relevant to the lowest two Mellin moments of polarised and unpolarised PDFs.
}
\begin{ruledtabular}%Note this requires RevTeX, but makes the spacing look more professionally typeset.
\begin{tabular}{ccc}
Matrix element & Operator & Observable \\ 
\vspace*{-10pt}\\
\hline 
\vspace*{-10pt}\\
$v_q^2$\,, $v_{\bar{q}}^2$  & $\displaystyle \left(\frac{{\rm i}}{2}\right) \bar{q}(x)\gamma_{\mu_1} \overleftrightarrow{D}_{\mu_2} q(x)$ & $\langle x \rangle_q$\,, $\langle x \rangle_{\bar{q}}$   \\
$v_q^3$\,, $v_{\bar{q}}^3$  & $\displaystyle \left(\frac{{\rm i}}{2}\right)^2 \bar{q}(x)\gamma_{\mu_1} \overleftrightarrow{D}_{\mu_2} \overleftrightarrow{D}_{\mu_3} q(x)$ & $\langle x^2 \rangle_q$\,, $\langle x^2 \rangle_{\bar{q}}$ \\
$a_q^0$ & $\displaystyle \bar{q}(x)\gamma_{\sigma} \gamma_5 q(x)$ & $2\, \langle 1 \rangle_{\Delta q}$ \\
$a_q^1$ & $\displaystyle \left(\frac{{\rm i}}{2}\right) \bar{q}(x)\gamma_{\sigma} \gamma_5 \overleftrightarrow{D}_{\mu_1} q(x)$ & $2\, \langle x \rangle_{\Delta q}$ \\
$v_g^2$ & $\displaystyle - {\rm Tr}\, F_{\mu_1\alpha}F_{\mu_2\alpha}$ & $\langle x \rangle_g$ \\
\end{tabular}
\end{ruledtabular}
\end{table}

\paragraph*{Higher-twist contributions} The discussion so far has focussed on the limit in which higher twist contributions, suppressed by powers of the momentum-transfer, have been ignored. In fact, higher twist contributions to the lowest moment of the structure function $F_1(x,Q^2)$ are found to be of ${\cal O}(1\, \mbox{GeV}^2/Q^2)$ \cite{Blumlein:2008kz}. For LQCD, typically $Q^2 \simeq 1/a^2$, and at present lattice spacings this corresponds to $Q^2 = O(10\,\mbox{GeV}^2)$ or a higher-twist contribution of $5 - 10\, \%$. With contributions of higher-twist included, the OPE reads
\begin{equation}
2 \int_0^1 dx\, x F_1^q(x,Q^2) = c_{1,q}^2(\mu^2)\, v_q^2(\mu^2)|_{\mu^2=Q^2} + \frac{\bar{c}_{1,q}^2(\mu^2)}{Q^2}\, \bar{v}_q^2(\mu^2)|_{\mu^2=Q^2} + \cdots \,,
\label{tex}
\end{equation}
where $\bar{c}_{1,q}^2$ and $\bar{v}_q^2(\mu^2)$ are Wilson coefficient and reduced matrix element of a generic twist-four operator. Both twist-two and four contributions mix under renormalisation, to the extent that the perturbative series for the Wilson coefficients $c_{1,q}^2(\mu^2)$ diverges due to the presence of IR renormalon singularities. This ambiguity is cancelled by that in the twist-four matrix element $\bar{v}_q^2(\mu^2)$ that arises as a result of an UV renormalon singularity~\cite{Martinelli:1996pk}. If mixing effects are ignored, the uncertainties will be, at least, comparable to the power corrections themselves. Power corrections can be assessed most efficiently, and the twist expansion tested, by a direct LQCD evaluation of the Compton amplitude, which we discuss in Section \ref{Sec:InversionMethod}.

\paragraph*{Beyond the first three moments} Moving beyond the lowest three moments requires overcoming the challenge of power-divergent mixing for LQCD twist-two operators. One novel approach to this problem~\cite{Davoudi:2012ya} builds upon the physical intuition that as long as the scale associated with the operator (for the twist-two operators, this is the renormalisation scale $\mu$) is taken to be much smaller than the hadronic scale but much larger than the inverse lattice spacing, no singularity necessarily arises as one takes the continuum limit. The operator can still probe the correct hadron structure at the scale $\mu$, but should be insensitive to the details of the discretisation of the operator at shorter distances. A simple way to incorporate an intrinsic ``smearing” scale for an operator is to sum over bilinears of quark fields that are displaced over many lattice sites in a small (compared to the scale $1/\mu$) region of Euclidean spacetime (an alternative approach appears in~\cite{Monahan:2015lha}). To ensure that the correct $SO(4)$ transformation properties of the matrix elements are recovered in the continuum limit, one must project the sum using hyperspherical harmonics. The properties of these operators, such as their mixing patterns and scaling properties, are discussed in detail in
Ref.~\cite{Davoudi:2012ya}. In particular, while the classical mixing with lower and higher spin operators are both suppressed by $\sim a^2$ for spatially improved operators, the mixing at one-loop in lattice perturbation theory is suppressed by ${\cal O}(\ alpha_s a)$ or ${\cal O}(\ alpha_s a^2)$, depending on the lattice action used and provided that the gauge links used in constructing the gauge invariant bilinears are tadpole-improved and smeared over a region whose physical size is held fixed as the continuum limit is taken. In principle, this allows higher moments of PDFs to be obtained from LQCD, without power-divergences. Numerical investigations of this approach, which requires gauge configurations with very fine lattice spacings, are underway. 

\subsubsection{The $x$-dependence of PDFs from lattice QCD}

\paragraph{Inversion method}
%\label{Sec:InversionMethod}
\input{InversionMethod}

\paragraph{RQCD Approach}
%Gunnar Bali (0.5 page)

\paragraph{PDFs from the Hadronic Tensor}
%\label{sec:HadronicTensorMethod}
\input{HadronicTensorMethod}

\paragraph{Quasi-PDFs}
%\label{Sec:QuasiPDFMethod}
\input{QuasiPDFMethod}




\subsection{Global PDF fits}

Collinear unpolarized and polarized PDF analysis.

%%%%%%%%%%%%%%%%%%%%%%%%%%%%%%%%%%%%%%%%%%%%%%%%%%%%%%%
\subsubsection{Unpolarized PDFs}




\textcolor{blue}{HEADINGS: this is just for editing; we'll remove in the final. }

\textcolor{blue}{INTRO:} 
We express the collinear unpolarized PDFs as $f_{i}(x,\mu)$
where the index $i$ represents the parton flavor $i=\{g,u,\bar{u},d,\bar{d},s,\bar{s},...\}$,
$x$ is the fractional momentum carried by the parton, and $\mu$
is the factorization (energy) scale.\footnote{We could add an additional index to specify the particular hadron
(proton, neutron, pion, nuclei, ...); as we mainly refer to the proton
in this work, we will omit such a designation unless necessary.} The PDF is a scheme-dependent quantity, and we typically work in
the $\overline{MS}$ scheme; when this is convoluted with an appropriate
hard cross section (Wilson coefficient), we obtain a scheme-independent
physical observable. 

\textcolor{blue}{X-DEPENDENCE:} 
The $x$-dependence of the PDFs must
be deduced by comparing with experimental data in a global fit. For
this purpose, we often parameterize the $x$-dependence of the PDFs
in the generic form: 
\begin{equation}
f_{i}(x,\mu)\sim x^{a}(1-x)^{b}\:C(x)\quad.
\end{equation}
Here, the $x^{a}$ term controls the small-$x$ behavior, the $(1-x)^{b}$
term controls the large-$x$ behavior, and $C(x)$ represents the
remaining $x$-dependence. The $a$ exponent is negative and generally
in the range $-1$ to $-2$; thus, the PDFs diverge as $1/x^{\sim1.5}$
for small $x$ and the number of soft partons is infinite. The $a$
exponent must be larger than $-2$ or the momentum sum rule will diverge.
The $b$ exponent is positive, and this ensures the PDF goes to zero
as $x\to1$.


\textcolor{blue}{NUMBER SUM RULES:} 
There are a few constraints we
can impose on the $x$-dependence of the PDFs at this point. Since
the proton has the quantum numbers of two up quarks and one down quark,
we have the following quark number sum rules given in terms of first
moments: 
%
\begin{eqnarray*}
\int_{0}^{1}dx\ \left[u(x,\mu)-\bar{u}(x,\mu)\right] & =\left\langle 1\right\rangle _{u^{-}}= & 2\\
\int_{0}^{1}dx\ \left[d(x,\mu)-\bar{d}(x,\mu)\right] & =\left\langle 1\right\rangle _{d^{-}}= & 1\\
\int_{0}^{1}dx\ \left[s(x,\mu)-\bar{s}(x,\mu)\right] & =\left\langle 1\right\rangle _{s^{-}}= & 0
\end{eqnarray*}
with similar results for the heavy quarks: $\left\langle 1\right\rangle _{c^{-}}=\left\langle 1\right\rangle _{b^{-}}=\left\langle 1\right\rangle _{t^{-}}=0$.

\textcolor{blue}{MOMENTUM SUM RULE:} The fractional momentum carried
by each parton flavor is given by the second moments: 
%
\begin{eqnarray*}
\int_{0}^{1}dx\ x\ \left[u(x,\mu)\right] & = & \left\langle x\right\rangle _{u}\\
\int_{0}^{1}dx\ x\ \left[u(x,\mu)+\bar{u}(x,\mu)\right] & = & \left\langle x\right\rangle _{u^{+}}\ .
\end{eqnarray*}
%
Here, $\left\langle x\right\rangle _{u}$ is the fractional momentum
carried by the up-quark, $\left\langle x\right\rangle _{u^{+}}$ is
the fractional momentum carried by the up-quark and anti-up-quark.
Since the total momentum of the proton must equal the momentum of
its constituents, we have the momentum sum rule constraint: 
%
\begin{eqnarray*}
1 & = & \left\langle x\right\rangle _{g}+\left\langle x\right\rangle _{u^{+}}+\left\langle x\right\rangle _{d^{+}}+\left\langle x\right\rangle _{s^{+}}+\left\langle x\right\rangle _{c^{+}}+\left\langle x\right\rangle _{b^{+}}+\left\langle x\right\rangle _{t^{+}}+...
\end{eqnarray*}
%
where the ``...'' represents any other partonic components (such
as a photon). 

\textcolor{blue}{$\mu$-DEPENDENCE:} 
The $\mu$-dependence of the
PDFs is given by the DGLAP evolution equation\cite{Dokshitzer:1977sg,Gribov:1972ri,Altarelli:1977zs}
\begin{eqnarray*}
\frac{\partial}{\partial\ln\mu^{2}}\:f_{i}(x,\mu) & = & \sum_{j=g,q,\bar{q}}\ P_{ij}(x)\otimes f_{j}(x,\mu)\quad.
\end{eqnarray*}
Here, the logarithmic derivative of the PDF is determined by a convolution
of the PDFs with the DGLAP kernel $P_{ij}(x)$ which can be computed
perturbatively in powers of $\alpha_{s}(\mu)$; $P_{ij}(x)$ is known
to NNLO.\footnote{The DGLAP (Dokshitzer\textendash Gribov\textendash Lipatov\textendash Altarelli\textendash Parisi)
evolution can be modified by $\ln(1/x)$ terms at small-$x$, and
this is characterized by the BFLK (Balitsky-FadinKuraev-Lipatov) equations.\cite{Kuraev:1976ge,Kuraev:1977fs,Balitsky:1978ic}
Additionally, at large-$x$ and small $\mu$ scale the above framework
can receive corrections from non-factorizeable higher-twist (HT) corrections.} When performing the global fit to the data, we use the DGLAP equations
to combine data from different $\mu$ scales when constraining the
PDFs. 




%%%%%%%%%%%%%%%%%%%%%%%%%%%%%%%%%%%%%%%%%%%%%%%%%%%%%%%
\subsubsection{Polarized PDFs}
