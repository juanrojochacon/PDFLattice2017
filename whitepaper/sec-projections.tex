%%%%%%%%%%%%%%%%%%%%%%%%%%%%%%%%%%%%%%%%%%%%%%%%%%%%%%%%%%%%%%%%%%%%%%%%%%%%%%%%
\section{Improving PDF fits with lattice QCD calculations}
\label{sec:projections}
%%%%%%%%%%%%%%%%%%%%%%%%%%%%%%%%%%%%%%%%%%%%%%%%%%%%%%%%%%%%%%%%%%%%%%%%%%%%%%%%

In this section we provide an initial estimate of the potential
impact of present and future lattice-QCD calculations
in global unpolarised and polarised PDF fits.
%
This study is carried out with two publicly available
tools: the
Bayesian reweighting
method~\cite{Ball:2011gg,Ball:2010gb} applied to the
NNPDF3.1 and NNPDFpol1.1 sets; and the Hessian
profiling method~\cite{Camarda:2015zba} applied to
HERAPDF2.0~\cite{Abramowicz:2015mha}.
%
Both approximate methods allow us to quantify the impact of new measurements
(or of future measurements, if pseudo-data is used) on a PDF set without
having to repeat the global analysis.
%
The main limitation of these methods is that they are only reliable
in those cases for which the impact of the new (pseudo-)data
is not too strong.

For simplicity, we restrict  ourselves here to study the
impact of a subset of the moments that can be
evaluated using lattice QCD.
%
In particular,
we focus on those that can be currently calculated
with the smallest uncertainties, as discussed in
Sect.~\ref{sec:benchmarking}.
%
We do not consider here
the rest of the PDF moments discussed in
Sect.~\ref{Sec:IntroLQCD} and Appendix~\ref{sec:LQCDtables},
since these are typically affected by larger systematic uncertainties.
%
In addition, we consider in this exercise pseudo-data based on $x$-space
lattice QCD calculations, such as those from the quasi-PDF approach
discussed in Sect.~\ref{sec:xdependence}.
%
As we show, particularly in the unpolarised case, the
constraining power of direct $x$-space calculations is
superior to that of the PDF moments.

\subsection{Impact of lattice calculations of PDF moments}
In this first part of this exercise, we quantify the
constraining power of existing and future calculations
of PDF moments on both unpolarised and polarised
global analysis.
%
This analysis is carried out with two complementary methods:
the Bayesian reweighting method suitable
for Monte Carlo sets such as NNPDF3.1 and NNPDFpol1.1; 
and the profiling method suitable for Hessian sets such as HERAPDF2.0.
%
%To begin with, we present next the settings of this exercise,
%including the choice of PDF moments, and then we follow
%by presenting the reweighting and profiling results.

% General projections for moments
\subsubsection{Settings}


Taking into account
these considerations, we will consider in this analysis the following
moments.
%
For the unpolarized case, we will use
\be
  \la x\ra_{u^+}\, , \quad
\la x\ra_{d^+}\, , \quad
\la x\ra_{s^+}\, , \quad
\la x\ra_{g}\, , \quad {\rm and} \quad
\la x\ra_{u^+-d^+} \, ,
\ee
while for the polarized side, we will include instead
the following five moments:
\be
\la 1\ra_{\Delta u^+}\, , \quad
\la 1\ra_{\Delta d^+}\, , \quad
\la 1\ra_{\Delta s^+}\, , \quad
\la x\ra_{\Delta u^--\Delta d^-}\, , \quad {\rm and} \quad
\la 1\ra_{\Delta u^+ - \Delta d^+} \, .
\ee
Recall that Appendix~\ref{app:notation} contains the
explicit definitions and conventions used for these moments.
%
Therefore, we see that for the unpolarized case we include
the second moments (momentum fractions) of $q^+$ (with $q=u,d,s$),
of the gluon, and of the isoscalar combination $u^+-d^+$.
%
In the polarized case instead, we include the first moments (which
contribute to the proton spin content) of $\Delta q^+$ (with $q=u,d,s$)
and of the isoscalar combination $\Delta u^+-\Delta d^+$, as well as
the second moment of $\Delta u^- - \Delta d^-$.

In the present exercise we will consider three
different scenarios, which we denote
as Scenario A, B, and C respectively, for the total systematic
uncertainty than we associate to lattice
QCD calculations of PDF moments.
%
In Table~\ref{tab:scenarios} we summarize the
values assumed for this total uncertainty
    of the lattice QCD calculation, denoted by $\delta_L$, for each
    of the various unpolarized and polarized PDF moments that enter
    this analysis.
    %
    We emphasize that here, while trying to be reasonably
    realistic, we do not aim to associate a given scenario
    within a specific time-scale for the calculation.
    %
    Our results  merely provide an illustrative guidance about the potential
    constraining power of existing and future lattice QCD calculations
    of PDF  moments in the context
    of a global analysis.
    
    The motivation for our choice of the scenarios
    in Table~\ref{tab:scenarios}
    is rather different from the unpolarized and polarized cases.
    %
    For the polarized fits,
    scenario A assumes that the uncertainties $\delta_L$
    for the lattice QCD calculations
    are on same the ball-park of the current ones, taking as
    representative values for the latter  those from the
    state-of-the-art lattice QCD calculations
    selected for the benchmarking exercise of Sect.~\ref{sec:benchmarking},
    and summarized in Table~\ref{tab:BMpol}.
    %
    Then scenarios B and C represent two possible optimistic scenarios for the
    future improvement of these systematic uncertainties, where these are decreased
    by roughly a factor 2 and a factor 4 with respect current values.
    
    On the other hand, for the unpolarized case scenario A is based on values
    of $\delta_L^{(i)}$ already rather smaller than the typical
    uncertainties that affect state-of-the-art calculations, see Table~\ref{tab:BMunp}.
    %
    The reason is that we have verified that including the pseudo-data $\mathcal{F}_i^{(\rm)}$
    assuming lattice-QCD uncertainties of similar size as those of Table~\ref{tab:BMunp}
    leaves the PDFs essentially unchanged, and only once the uncertainties
    $\delta_L^{(i)}$ are significantly reduced that we start to obtain a reduction
    of the uncertainties from the global fit.
    %
    The only connection with the uncertainties of the calculations in Table~\ref{tab:BMunp}
    is that we assume that $\delta_L^{(i)}$ is typically larger for $\la x\ra_{s^+}$
    and $\la x\ra_{u^+-d^+}$ as compared to the other moments.
    %
    A total systematic error of $\delta_L^{(i)}$ is probably the best that one can achieve
    within a lattice-QCD calculation even in principle, since at that level many other
    effects such as QED corrections become relevant and these are much more difficult
    to deal with.
    %
    For both
    the polarized and
    the unpolarized case,
    we emphasize that the generalization of these projections to other conceivable scenarios
    is straightforward and can be obtained from the authors upon request.
 
%%%%%%%%%%%%%%%%%%%%%%%%%%%%%%%%%%%%%%%
\begin{table}[t]
  \centering
  \renewcommand{\arraystretch}{1.3} 
  \begin{tabular}{c||ccccc}
    \hline
    Scenario &  \multicolumn{5}{c}{$\delta_L^{(i)}$ for unpolarized moments}   \\
&    $\la x\ra_{u^+}$  &   $\la x\ra_{d^+}$   &  $\la x\ra_{s^+}$  &
$\la x\ra_{g}$  &   $\la x\ra_{u^+-d^+}$  \\
    \hline
    Current  & $\sim 16\%$  &  $\sim 30\%$
    & $\sim 45\%$  & $\sim 13\%$  &  $\sim 60\%$ \\
    A   & 3\%  & 3\% &  5\% &  3\% &  5\% \\
 B   & 2\%  & 2\% &  4\% &  2\% &  4\%  \\
  C   & 1\%  & 1\% &  3\% &  1\% &  3\%  \\
    \hline
  \end{tabular}\vspace{0.7cm}
   \begin{tabular}{c||ccccc}
    \hline
    Scenario   &
    \multicolumn{5}{c}{$\delta_L^{(i)}$ for polarized moments} \\ 
& $\la 1\ra_{\Delta u^+}$  & $\la 1\ra_{\Delta d^+}$  & $\la 1\ra_{\Delta s^+}$
&  $\la x\ra_{\Delta u^--\Delta d^-}$  &  $\la 1\ra_{\Delta u^+ - \Delta d^+}$\\
    \hline
    Current  &
    $\sim 3\%$  & $\sim 5\%$ & $\sim 70\%$ & $\sim 65\%$ & $\sim 3\%$ \\
    \hline
    A   & 
    5\% &    10\%  &   100\% &    70\%  &    5\% \\
 B   &
 3\% &    5\%  &   50\% &    30\%  &    3\% \\
  C   & 1\% &    2\%  &   20\% &    15\%  &    1\% \\
    \hline
  \end{tabular}
   \caption{\small The three scenarios assumed here
     for the total percentage
     systematic uncertainty
    in future lattice QCD calculation $\delta_L$ for each
    of the unpolarized (upper) and polarized (lower table) PDF
    moments that are included
    in the present reweighting analysis.
    %
    In addition, the first line indicates the current systematic
    uncertainties of the state-of-the-art lattice QCD calculations
    selected for the benchmarking exercise of Sect.~\ref{sec:benchmarking},
    and summarized in Tables~\ref{tab:BMunp} and~\ref{tab:BMpol}
    for the unpolarized and polarized cases, respectively.
    %
    See text for more details.
\label{tab:scenarios}
  }
\end{table}
%%%%%%%%%%%%%%%%%%%%%%%%%%%%%%%%%%%%%%%




% PDF moment analysis using reweighting
\subsubsection{Bayesian reweighting analysis}

The procedure followed to quantify the impact of future
lattice QCD calculations
in PDF fits  (for the three scenarios of Table~\ref{tab:scenarios})
is common for unpolarized
and polarized global analyses.
%
We briefly describe this procedure here,
and refer to~\cite{Ball:2011gg,Ball:2010gb} for
additional details.
\begin{itemize}
\item First of all, we generate pseudo-data for the lattice QCD calculation
  of the PDF moments used in this exercise, namely $\la x\ra_{u^+}$,
$\la x\ra_{d^+}$,
$\la x\ra_{s^+}$,
$\la x\ra_{g}$, and
  $\la x\ra_{u^+-d^+}$ for the unpolarized case, and
  $\la 1\ra_{\Delta u^+}$,
$\la 1\ra_{\Delta d^+}$,
$\la 1\ra_{\Delta s^+}$,
$\la x\ra_{\Delta u^--\Delta d^-}$, and
  $\la 1\ra_{\Delta u^+ - \Delta d^+}$ for the polarized case.
  %
  We denote generically these moments by $\mathcal{F}_i$.
\item This pseudo-data, denoted by $\mathcal{F}_i^{\rm (exp)}$,
  is constructed by taking the central values from
  the corresponding NNPDF fits, NNPDF3.1 NNLO for the unpolarized case and NNPDFpol1.1 NLO
  for the polarized one.
  %
  That is, we {\it assume} for simplicity that the central value
  of such future lattice calculations would coincide with the current ones
  from the global fit.\footnote{Repeating the exercise with the actual lattice QCD
    central values would be straightforward, but
    lies beyond the scope of the present studies.}
  %
  As discussed in Sect.~\ref{sec:unpPDFs}, this corresponds to computing
  the mean over the Monte Carlo replica sample,
  \be
  \label{eq:pseudodatadef}
  \mathcal{F}_i^{\rm (exp)} \equiv \frac{1}{N_{\rm rep}}\sum_{k=1}^{N_{\rm rep}}
  \mathcal{F}_i^{\rm (k)} \, , \quad i=1,\ldots,N_{\rm mom} \, ,
  \ee
  where $N_{\rm mom}$ are the number of PDF moments that will be included
  in the reweighting, in this case $N_{\rm mom}=5$ both for the unpolarized
  and polarized cases.
  %
  To be consistent with the calculations in Sect.~\ref{sec:benchmarking},
  here the central values of the pseudo-data Eq.~(\ref{eq:pseudodatadef})
  are also evaluated at $Q^2=4$ GeV$^2$ (see Tables~\ref{tab:unpPDFmoms} and~\ref{tab:polPDFmoms}).
\item The uncertainty in the pseudo data, denoted by $\delta\mathcal{F}_i^{\rm (exp)} $,
  is taken for each moment to be the value indicated in
  Table~\ref{tab:scenarios} for each of the three scenarios.%
  That is, we have that the absolute uncertainty on the $i$-th moment
  will be given by $\delta\mathcal{F}_i^{\rm (exp)}=\delta_L^{(i)}\mathcal{F}_i^{\rm (exp)} $.
\item Using the pseudo-data (central values and total uncertainties)
  as defined above, we next need to compute
  the weights  $\omega_k$.
  %
  These weights
  quantify the agreement between each of $N_{\rm rep}$ replicas
  of the input PDF set and the corresponding lattice pseudo-data.
  %
  Specifically, first of all we compute the $\chi^2$ between each of the Monte Carlo
  replicas and the lattice pseudo-data as follows,
  \be
  \chi^{2(k)}= \sum_{i=1}^{\rm N_{\rm mom}} \frac{\lp
    \mathcal{F}_i^{\rm (k)} -\mathcal{F}_i^{\rm (exp)} \rp^2}{
    \lp \delta\mathcal{F}_i^{\rm (exp)}\rp^2} \, , \quad k=1,\ldots,N_{\rm rep} \, ,
  \ee
  assuming that there are no correlations between the different $N_{\rm mom}$ moments.
  %
  This assumption in general might not be a good approximation, since most lattice
  QCD systematic errors are correlated among the different moments, and should be
  avoided provided the full breakdown of systematic error of each quantity is available.

  
  Once the values of the $\chi^2$ have been evaluated,
  we can compute the corresponding weights for each replica.
  %
  The relation between the weights $w_k$  and the values of
  the $\chi^{2(k)}$ of each replica is the following
  \be
  \omega_k =\frac{\lp \chi^{2(k)} \rp^{(N_{\rm mom}-1)/2}\exp(-\chi^{2(k)}/2)}{
  \sum_{k=1}^{N_{\rm rep}} \lc \lp \chi^{2(k)} \rp^{(N_{\rm mom}-1)/2}\exp(-\chi^{2(k)}/2)\rc} \, ,
  \ee
  where the denominator ensures that the weight admit
  a probabilistic interpretation, that is, $\sum_k w_k=1$.
  %
  These weights represent a measure of the agreement of the individual replicas with the new pseudo-data.
  %
  For instance, replicas which have associated values
  of the moments far from the pseudo-data (within uncertainties) will
  have associated a very large weight, being thus effectively discarded.
\item These weights can be use to now recompute the PDFs, their moments,
  as well as any generic cross-section.
  %
  This procedure emulates the
  impact that adding these lattice-QCD pseudo-data in a complete PDF fit would have.
  %
  For instance, after the reweighting the mean value of
  the PDF moments should be computed as
   \be
  \label{eq:pseudodatadef1}
  \mathcal{F}_i^{\rm (rw)} \equiv \frac{1}{N_{\rm rep}}\sum_{k=1}^{N_{\rm rep}}\omega_k
  \mathcal{F}_i^{\rm (k)} \, , \quad i=1,\ldots,N_{\rm mom} \, ,
  \ee
  and same for the associated uncertainties.
\end{itemize}

One limitation of the reweighting procedure just
outlined is that it can be considered as fully 
 reliable provided only the 
  effective number of replicas $N_{\rm eff}$ that survive the reweighting
  procedure (which is a measure of the amount
  of information left) is not too small.
  %
  This effective number of replicas
    is quantified in terms of the Shannon entropy from information
    theory, namely
    \be
    \label{eq:effnrep}
    N_{\rm eff}\equiv \exp\lc \frac{1}{N_{\rm rep}}\omega_k
    \log \lp N_{\rm rep}/\omega_k\rp\rc \, .
    \ee
    Finding that $N_{\rm eff}\ll N_{\rm rep}$ means that the pseudo-data
    has a large impact on the fit, potentially leading to a large
    reduction of the PDF uncertainties.
    %
    But if the effective number of replicas becomes too
    small, say $N_{\rm eff}\lsim 25$, then the results
    become unreliable since they are affected by large
    statistical fluctuations.

    Therefore, before considering the effects
    of the lattice-QCD pseudo-data at the PDF
    level, we need to ensure that the
    three scenarios defined
    in Table~\ref{tab:scenarios} still lead
    to values of $N_{\rm eff}$ large enough for
    the reweighting procedure to be reliable.
  %
In Table~\ref{tab:neff} we indicate the effective number of replicas
    $N_{\rm eff}$, Eq.~(\ref{eq:effnrep}), remaining when the pseudo-data
    on the PDF moments is included in the global
    fit according to the 
    %
    For completeness, we also indicate here the original number
    of replicas $N_{\rm rep}$ for the original
    PDF sets, NNPDF3.1 NNLO and NNPDFpol1.1 respectively.
    %
    As we can see, there is a marked decrease of $N_{\rm rep}$
    for the three scenarios, indicating that adding the
    PDF moments leads to non-trivial constraints on the global
    fit.
    %
    For instance, in the most optimistic scenario,
    Scenario A, the effective number of replicas is around five (three)
    smaller than the starting number of replicas.

%%%%%%%%%%%%%%%%%%%%%%%%%%%%%%%%%%%%%%%%%%%%%%
\begin{table}[t]
  \centering
  \renewcommand{\arraystretch}{1.3} 
  \begin{tabular}{c|c|c}
    \hline
    &  NNPDF3.1  &  NNPDFpol1.1 \\
    \hline
    \hline
    $N_{\rm rep}$ original   &   1000 &  100   \\
    \hline
     $N_{\rm eff}$ Scenario A    &   740  &  72   \\
     $N_{\rm eff}$ Scenario B    &   750   &   59  \\
     $N_{\rm eff}$ Scenario C   &   510  &   20  \\
    \hline
  \end{tabular}
  \caption{\small The effective number of replicas
    $N_{\rm eff}$, Eq.~(\ref{eq:effnrep}), remaining when the pseudo-data
    on the PDF moments is included in the global
    fit according to the scenarios outlined
    in Table~\ref{tab:scenarios}.
    %
    For completeness, we also indicate the original number
    of replicas $N_{\rm rep}$ for the original
    PDF sets, NNPDF3.1 NNLO and NNPDFpol1.1 respectively.
    \label{tab:neff}
  }
\end{table}
%%%%%%%%%%%%%%%%%%%%%%%%%%%%%%%%%%%%%%%%%%%%%%

\subsubsection*{Impact on unpolarized global fits}
\label{subsec:upolfits}
%
We start by discussing the results of applying the reweighting procedure
outlined above to a representative unpolarized
global fit, in this case the NNPDF3.1 NNLO analysis.
%
To begin with, in Table~\ref{tab:unpolmomentsrw} we summarize
the values of the unpolarized PDF moments
used as pseudo-data $\mathcal{F}_i^{(\rm exp)}$,
as well as the corresponding results
  after the reweighting has been performed for the
three scenarios summarized in 
in Table~\ref{tab:scenarios}.
%
The PDF uncertainties quoted there correspond to 68\% confidence level intervals.
%
We recall that, as explained above, the three scenarios considered here exhibit
uncertainties $\delta_L^{(i)}$ on the lattice-QCD pseudo-data rather smaller
than those of current calculations (see Table~\ref{tab:BMunp}).

%%%%%%%%%%%%%%%%%%%%%%%%%%%%%%%%%%%%%%%%%%%%%%%%%%%%%%%%
\begin{table}[t]
  \centering
  \renewcommand{\arraystretch}{1.3} 
\begin{tabular}{c||c|c|c|c}
  \hline &  Original  & Scen A  &  Scen B  &  Scen C  \\
  \hline
  \hline
  $\la x\ra_{u^+}$     &   $0.348 \pm  0.005$    &  $ 0.349 \pm 0.004$     &
  $ 0.349 \pm 0.004$   &  $ 0.349 \pm 0.003$   \\
  $\la x\ra_{d^+}$     &   $0.196\pm  0.004$     & $0.196 \pm0.004$       &
  $0.196 \pm0.003$ &   $0.196 \pm0.002$ \\
  $\la x\ra_{s^+}$     &   $0.0393 \pm 0.0036$   &  $0.0389\pm 0.0030$   &
 $0.0389\pm 0.0024$   &   $0.0389\pm 0.0014$  \\
  $\la x\ra_{g}$       &   $0.4097\pm 0.0042$    &  $0.4097 \pm 0.0043$    &
   $0.4097 \pm 0.0040$  &    $0.4097 \pm 0.0029$  \\
  $\la x\ra_{u^+-d^+}$  &   $0.1522 \pm 0.0033$   &  $0.1521 \pm 0.0037$   &
   $0.1521 \pm 0.0035$ &    $0.1521 \pm 0.0029$ \\
  \hline
\end{tabular}
\caption{\small Values of the unpolarized PDF moments
  used as pseudo-data, as well as the corresponding results
  after the reweighting has been performed for the
three scenarios summarized in 
in Table~\ref{tab:scenarios}.
%
The PDF uncertainties quoted correspond in all cases to 68\%
CL intervals.
\label{tab:unpolmomentsrw}
}
\end{table}
%%%%%%%%%%%%%%%%%%%%%%%%%%%%%%%%%%%%%%%%%%%%%%%%%%%%%%%%

From Table~\ref{tab:unpolmomentsrw} we see that a significant
reduction of the uncertainties in the unpolarized PDF moments is challenging to achieve
unless we go to the most aggressive scenarios.
%
For instance, in scenario C, which is about the best precision that
can achieved from lattice-QCD, the PDF uncertainties on the second moments
(that is, the momentum fractions) for $u^+,d^+,s^+$ and $g$ decrease by around
between 30\% and 60\%.
%
The most marked decrease is for the strange momentum fraction, since this is the
one affected by the largest PDF errors to begin with.
%
On the other hand, the non-singlet combination $\la x\ra_{u^+-d^+}$ is essentially
unchanged in all three scenarios.
%
Note that the reason why in Table~\ref{tab:unpolmomentsrw} the central values
of the PDF moments are very stable is that by construction we assume that the
lattice-QCD central value corresponds to the PDF one.
%
Of course in a realistic situation, such as that summarized in the
comparisons of Fig.~\ref{fig:Bmoms}, this will not be necessarily the case
and there also the central value of the PDF moments expected to vary
upon reweighting.


%------------------------------------------------------
\begin{figure}[!t]
\centering
\includegraphics[scale=0.45]{plots/xg-unpol-lattice-relerr.pdf}
\includegraphics[scale=0.45]{plots/xup-unpol-lattice-relerr.pdf}
\includegraphics[scale=0.45]{plots/xdp-unpol-lattice-relerr.pdf}
\includegraphics[scale=0.45]{plots/xsp-unpol-lattice-relerr.pdf}
\caption{\small The percentage PDF uncertainty in NNPDF3.1 NNLO
  for the gluon and the $u^+$, $d^+$ and $s^+$ quark PDFs at
  $Q^2=4$ GeV$^2$,
  compared to the results of including the five lattice
  QCD moments as pseudo-data points in the fit using the three
  different scenarios in  Table~\ref{tab:scenarios}.
  %
See text for more details.
}    
\label{fig:impactUnpol}
\end{figure}
%----------------------------------------------------------

The result that, at least with the specific moments that we have used
in this exercise, it will be challenging to reduce the
PDF uncertainties is also illustrated by the comparisons of 
 Fig.~\ref{fig:impactUnpol}, where we show
percentage PDF uncertainty in NNPDF3.1 NNLO
for the gluon and the $u^+$, $d^+$ and $s^+$ quark PDFs in NNPDF3.1 NNLO
at $Q^2=4$ GeV$^2$,
compared to the corresponding results once
the lattice-QCD pseudo-data points of Table~\ref{tab:scenarios} have
been added by reweighting.
  %
In the case of the $u^+,d^+$ and $s^+$, we observe a slight reduction
of the PDF uncertainties, which is more marked as the move
from scenario A to C.
%
For instance, in the latter case the percentage PDF
uncertainty on $u^+$ ($d^+$ and $s^+$) at $x\simeq 0.1$
decreases from 1.8\% to 1.2\% (from 2.2\% to 1.7\% and from 13\% to 10\% respectively).
%
The PDF uncertainties of the gluon PDF, however,
are essentially unchanged even in the most optimistic scenario.

Focusing on the large-$x$ region, where the sensitivity to the
results of the second moments is more marked, in
Fig.~\ref{fig:impactUnpollargex} we show a similar comparison
as in Fig.~\ref{fig:impactUnpo} but now showing the ratio of the
  PDF uncertainties in the fits based on the three scenarios
  as a ratio of the original
  PDF uncertainty of the NNPDF3.1 NNLO set, for the $d^+$
  and $s^+$ total quark PDFs.
  %
  This comparison illustrates better that the relative reduction
  of the PDF uncertainties upon the addition of the lattice-QCD
  pseudo-data is not completely flat, and that it exhibits some
  non-trivial structure.

%------------------------------------------------------
\begin{figure}[!t]
\centering
\includegraphics[scale=0.45]{plots/xdp-unpol-lattice-relerr-largex.pdf}
\includegraphics[scale=0.45]{plots/xsp-unpol-lattice-relerr-largex.pdf}
\caption{\small Same as Fig.~\ref{fig:impactUnpol}, now focusing
  on the large-$x$ region, and showing the ratio of the
  PDF uncertainties in the fits based on the three scenarios
  as a ratio of the original
  PDF uncertainty of the NNPDF3.1 NNLO set, for the $d^+$
  and $s^+$ total quark PDFs.
}    
\label{fig:impactUnpollargex}
\end{figure}
%----------------------------------------------------------

\subsubsection*{Impact on polarized global fits}

Next, we move to discuss the results of applying the
reweighting procedure this time to a representative polarized
global fit, specifically the NNPDFpol1.1 NLO set.
%
First of all, in Table~\ref{tab:polmomentsrw}
we list the values of the polarized PDF moments
  used as pseudo-data, as well as the corresponding results
  after the reweighting has been performed for the
three scenarios summarized in 
in Table~\ref{tab:scenarios}.
%
The PDF uncertainties quoted correspond in all cases to 68\%
CL intervals.
%
As we can see from this comparison, already in the first scenario
(which assumes lattice-QCD pseudo-data with similar uncertainties
as existing calculations) there is a marked impact on the
polarized PDF moments.
%
Specifically, for both $\la 1\ra_{\Delta u^+}$ and $\la 1\ra_{\Delta d^+}$
the PDF uncertainties are roughly halved, with the same
trend but less marked for $\la 1\ra_{\Delta s^+}$.
%
At this level, there is no impact for the non-singlet
combinations $\la 1\ra_{\Delta u^+ - \Delta d^+}$
and $\la x\ra_{\Delta u^--\Delta d^-}$.

%%%%%%%%%%%%%%%%%%%%%%%%%%%%%%%%%%%%%%%%%%%%%%%%%%%%%%%%
\begin{table}[t]
  \centering
  \renewcommand{\arraystretch}{1.3} 
\begin{tabular}{c||c||c|c|c}
  \hline &  Original  & Scen A  &  Scen B  & Scen C  \\
  \hline
  $\la 1\ra_{\Delta u^+}$    &  $+0.788\pm  0.079$   & $+0.798\pm  0.039$     &
  $+0.797\pm  0.023$ &   $+0.790\pm  0.009$ \\
  $\la 1\ra_{\Delta d^+}$   &  $-0.450 \pm 0.083$  &  $-0.450 \pm 0.042$  &
  $-0.456 \pm 0.026$    &  $-0.465 \pm 0.012$   \\
  $\la 1\ra_{\Delta s^+}$    &  $-0.124\pm   0.108 $  & $-0.120\pm   0.070 $  &
  $-0.121\pm   0.076 $    &   $-0.111\pm   0.029 $  \\
  $\la 1\ra_{\Delta u^+ - \Delta d^+}$  & $+1.250 \pm 0.024$   & $+1.250 \pm 0.022$  &
  $+1.253 \pm 0.016$ &    $+1.256 \pm 0.012$  \\
  $\la x\ra_{\Delta u^--\Delta d^-}$     & $+0.196 \pm 0.014$    & $+0.195 \pm 0.014$
  & $+0.196 \pm 0.016$     &  $+0.198 \pm 0.012$    \\
  \hline
\end{tabular}
\caption{\small Same as Table~\ref{tab:unpolmomentsrw}, now for
  the polarized PDF moments computed with NNPDFpol1.1.
  %
  The corresponding impact at the PDF level is shown in
  Fig.~\ref{fig:impactPol}.
\label{tab:polmomentsrw}
}
\end{table}
%%%%%%%%%%%%%%%%%%%%%%%%%%%%%%%%%%%%%%%%%%%%%%%%%%%%%%%%

As we further decrease the assumed uncertainties in the lattice-QCD pseudo-data
for the PDF moments, we observe a consequent reduction of the uncertainties
in the global fit.
%
In the most optimistic scenario C, we find that for both
$\la 1\ra_{\Delta u^+}$ and $\la 1\ra_{\Delta d^+}$ there is an uncertainty
reduction by about an order of magnitude as compared to the current values,
and about a factor 5 for $\la 1\ra_{\Delta s^+}$.
%
Therefore, we demonstrate that future lattice-QCD calculations of
polarized PDF moments can potentially lead to a much more
precise understanding of the spin structure of the proton.
%
The other quark combinations exhibit less sensitivity to the inclusion
of the PDF moments in the global fit, given that to begin with
they are already quite well constrained by available experimental
data.
%
Specifically, the PDF uncertainties for  $\la 1\ra_{\Delta u^+ - \Delta d^+}$
are reduced by a factor 2 in this quite optimistic scenario, while
those of $\la x\ra_{\Delta u^--\Delta d^-}$ are essentially unaffected even
in this case.

Next we move to illustrate the impact of the lattice-QCD pseudo-data
on the polarized PDFs themselves, rather than on their
moments.
%
With this motivation,
in Fig.~\ref{fig:impactPol} we present a 
similar comparison as that of Fig.~\ref{fig:impactUnpol}, now
  showing the absolute PDF uncertainties of the NNPDFpol1.1 fit,
  compared to the corresponding results once the lattice pseudo-data
  on polarized moments in included in the analysis by means of the
  reweighting.
  %
  The reason to show absolute rather than relative uncertainties
  is that, unlike unpolarized PDFs, polarized PDFs often exhibit nodes
  (in particular for strangeness and the gluon) and in the nearby regions
  the concept of relative uncertainty becomes ill-defined.
  
%-------------------------------------------------------------------
\begin{figure}[!t]
\centering
\includegraphics[scale=0.45]{plots/xg-pol-lattice-relerr.pdf}
\includegraphics[scale=0.45]{plots/xup-pol-lattice-relerr.pdf}
\includegraphics[scale=0.45]{plots/xdp-pol-lattice-relerr.pdf}
\includegraphics[scale=0.45]{plots/xsp-pol-lattice-relerr.pdf}
\caption{\small Same as Fig.~\ref{fig:impactUnpol}, now
  showing the absolute PDF uncertainties of the NNPDFpol1.1 fit
   $Q^2=4$ GeV$^2$,
  compared to the corresponding results once the lattice pseudo-data
  on polarized moments in included in the analysis via
  reweighting.
}    
\label{fig:impactPol}
\end{figure}
%---------------------------------------------------------------------

From Fig.~\ref{fig:impactPol} we find that for scenarios
A and B there only a very moderate reduction (or even a slight increase)
of the PDF uncertainties, seemingly at odds with the results
for their moments in Table~\ref{tab:polmomentsrw}.
%
The reason is that the first PDF moments alone provide only limited
information on the shape of the PDFs itself, and therefore in some
cases one will find a large error reduction on the moments (since these
are the fitted quantified) than on the PDFs themselves (which are
only directly constrained).
%
On the other hand, once the lattice-QCD pseudo-data uncertainties
decrease beyond a certain level, the start to impact the PDF shape
as well, as we can see from the results of scenario C.
%
In that case we find that the PDF uncertainties can decreases by up to a factor
2 (3) for $\Delta d^+(x,Q)$ ($\Delta s^+(x,Q)$).
%
Interestingly, we see that the relative reduction of PDF uncertainties is more
or less constant along the whole range of $x$, consistent with the fact that
the lattice-QCD pseudo-data is only sensitive to the total integral
of the $x$ distribution.



% PDF moment analysis using Hessian profiling 
\subsubsection{Hessian profiling analysis}
\label{sec:hessianprofiling}

To complement the results obtained
with the Bayesian reweighting approach,
we use a profiling method, suitable
for Hessian PDF sets, to estimate the effect of including
lattice-QCD pseudo-data into the fit~\cite{Paukkunen:2014zia,Camarda:2015zba}.
%
We choose HERAPDF2.0~\cite{Abramowicz:2015mha}
as a representative set of Hessian PDFs.
%
As in the case of the Bayesian reweighting
exercise presented in the previous section
we consistently use the same lattice-QCD
pseudo-data on PDF moments to estimate the impact on HERAPDF2.0.
%
An additional advantage of the HERAPDF2.0 set is
the use
of a standard tolerance
$\Delta\chi^2=1$ for defining the 68\%-CL PDF
uncertainties,
%%%% RST , which provides a correspondence between the profiling and reweighting methods.
which enables a robust framework for applying the profiling method. 


For Hessian PDF sets, the Hessian profiling method
can be used to both check the compatibility of new data with a given PDF set,
and also  estimate the impact these data will have on the PDFs. 
In the following we describe the essential components of the profiling method, 
and assume  that the  Hessian PDF set uses a tolerance of $\Delta\chi^2=1$, 
which corresponds to 68\%~CL uncertainties,
as is the case with the HERAPDF2.0 set.\footnote{In this exercise
we consider only the {\it experimental} HERAPDF2.0
uncertainties, but not the {\it model} and {\it parametrization}
variations, which are not suited for profiling.}
%
The central values of the considered moments are obtained using the central PDFs and the corresponding
errors are calculated according to:
\begin{equation}
\delta\mathcal{F}_i = \frac{1}{2} \sqrt{\sum_{k}\left(\mathcal{F}_i(f_k^+)-\mathcal{F}_i(f_k^-)\right)^2}\, ,
\quad i=1,\ldots,N_\text{mom} \, ,
\end{equation}
where $k$ labels the number of error PDFs (Hessian eigenvectors)
which have both a positive and negative direction.
%
In the profiling method, one considers a $\chi^2$ function in which the $\chi^2$ of the new
data has been added to the initial $\chi^2_0$, namely
\begin{equation}
\label{eq:newchi2}
\chi^2_{\text{new}} = \chi_0^2 + \sum_{k}^{N_{\text{eig}}} z_k^2
                    + \sum_{i=1}^{N_{\text{data}}}
                      \frac{\lp \mathcal{F}_i - \mathcal{F}_i^{\rm(exp)}\rp^2}
                           {\lp\delta\mathcal{F}_i^{\rm(exp)}\rp^2}\,,
\end{equation}
where $\chi^2_0$ is the value of the $\chi^2$ function in the minimum of the initial PDF set,
$z_k$ are the parameters diagonalizing the Hessian matrix of the initial PDF set,
$N_{\text{eig}}$ is the dimension of the eigenvector space in which initial Hessian errors are defined
(half of the number of error PDFs), $\mathcal{F}_i^{\rm(exp)}$ is the new
\hbox{(pseudo-)data},
and $\mathcal{F}_i$ the corresponding theory prediction.

In the spirit of the Hessian method, the new theory predictions $\mathcal{F}_i$ can be expanded
using a linear approximation:
\begin{equation}
\mathcal{F}_i \simeq \mathcal{F}_i[S_0] + \sum_k \frac{\partial\mathcal{F}_i[S]}{\partial z_k}\bigg|_{S=S_0} z_k \quad
              \simeq \mathcal{F}_i[S_0] + \sum_k D_{ik} w_k \ ,
\end{equation}
where $S_0$ represents the central PDF and we have defined
\be
D_{ik}=\frac{1}{2}(\mathcal{F}_i[S_k^+]-\mathcal{F}_i[S_k^-]) \, ;
\ee
here the  derivative has been approximated by a finite difference of the 
Hessian PDF error sets $S_k^{\pm}$.
%
The new $\chi^2$ of Eq.~\eqref{eq:newchi2} can now be minimized with respect to the parameters $w_k$,
which results in:
\begin{equation}
%\boldsymbol{\vec{w}_{\text{{\bf min}}}} = \boldsymbol{-B^{-1}} \boldsymbol{\vec{a}},
%
w_k^{min}  = \sum_n \ -B_{kn}^{-1} \, a_n \quad ,
\end{equation}
where we have introduced
\begin{equation}
%\begin{split}
B_{kn} = \sum_i \frac{D_{ik}D_{in}}{\lp\delta\mathcal{F}_i^{\rm(exp)}\rp^2} + \delta_{kn},
%\\
\qquad
\qquad
a_k = \sum_i \frac{D_{ik}(\mathcal{F}_i[S_0] - \mathcal{F}_i^{\rm(exp)})}{\lp\delta\mathcal{F}_i^{\rm(exp)}\rp^2} \, . 
%\end{split}
\end{equation}

The key result of the Hessian profiling method
is that now the components of the solution 
%$\boldsymbol{\vec{w}_{\text{{\bf min}}}}$ 
$w_k^{min}$
define a new set
of PDFs representing a global minimum after including the new data:
\begin{equation}
f_{\text{new}} = f_{S_0} + \sum_{k=1}^{N_{\text{eig}}} \frac{f_{S_k^+}-f_{S_k^-}}{2} w_k^{\text{min}} \ .
\end{equation}
%At the same time 
%%$\boldsymbol{\vec{w}_{\text{{\bf min}}}}$ 
%$w_k^{min}$
%also  defines  a penalty term 
%\begin{equation}
%P = \sum_{k=1}^{N_{\text{eig}}} \lp w_k^{\text{min}} \rp^2 \, ,
%\end{equation}
%which can be used to estimate whether the new data is consistent with the initi%al set of PDFs.
%%
%Specifically, a value of
%the penalty term of $P\ll1$ means that the new data is consistent
%with that included in the original fit.
%%
A set of new error PDFs can be also defined; in this case the matrix $B_{kn}$ plays the role of
the Hessian matrix from which the PDF uncertainties
can be obtained. 

We performed this study using the xFitter program~\cite{Alekhin:2014irh}
assuming the same three scenarios for the lattice-QCD pseudo-data as 
in Table~\ref{tab:scenarios}. 
%
The results are shown in Table~\ref{tab:unpolmomentsProf}, where we tabulate 
the uncertainties of the input HERAPDF2.0 PDF in column two and the 
corresponding uncertainties for each scenario in columns three to five. 
%
The analogous results from the reweighting method, applied to the 
NNPDF3.1 data set, were listed in Table~\ref{tab:unpolmomentsrw}.

%-------------------------------------------------------------------------------
\begin{table}[!t]
\centering
\footnotesize
\renewcommand{\arraystretch}{1.3} 
\begin{tabular}{lcccc}
\toprule 
&  Original  & Scenario A  &  Scenario B  &  Scenario C  \\
\midrule
  $\la x\ra_{u^+}$     
&  $0.3720\pm 0.0036$  
&  $0.3720\pm 0.0030$  
&  $0.3720\pm 0.0027$  
&  $0.3720\pm 0.0020$ \\
  $\la x\ra_{d^+}$     
&  $0.1845\pm 0.0053$  
&  $0.1845\pm 0.0028$  
&  $0.1845\pm 0.0023$  
&  $0.1845\pm 0.0015$ \\
  $\la x\ra_{s^+}$     
&  $0.0346\pm 0.0037$  
&  $0.0346\pm 0.0015$  
&  $0.0346\pm 0.0012$  
&  $0.0346\pm 0.0009$ \\
  $\la x\ra_{g}$       
&  $0.4006\pm 0.0078$  
&  $0.4006\pm 0.0042$  
&  $0.4006\pm 0.0035$  
&  $0.4006\pm 0.0024$ \\
  $\la x\ra_{u^+-d^+}$ 
&  $0.1875\pm 0.0074$  
&  $0.1875\pm 0.0045$  
&  $0.1875\pm 0.0039$  
&  $0.1875\pm 0.0027$ \\
\bottomrule
\end{tabular}
\caption{\small Values of the unpolarized PDF moments
  used as lattice-QCD pseudo-data, as well as the corresponding results
  after the profiling  for the
three scenarios summarized in Table~\ref{tab:scenarios}.
%
The HERAPDF2.0 PDFs were used, and the PDF uncertainties quoted correspond in all cases to 68\%~CL intervals.
%
The corresponding results of applying the reweighting method
to NNPDF3.1 were listed in Table~\ref{tab:unpolmomentsrw}.
\label{tab:unpolmomentsProf}
}
\end{table}
%-------------------------------------------------------------------------------

From a comparison of the constraining power of the lattice-QCD pseudo-data  
displayed in Table~\ref{tab:unpolmomentsProf} to Table~\ref{tab:unpolmomentsrw},
we observe a consistent trend between Bayesian reweighting of NNDPF3.1 and 
Hessian profiling of HERAPDF2.0.
%
The PDF uncertainties for $\la x\ra_{d^+}$ ($\la x\ra_{s^+}$
and  $\la x\ra_g$) reduce by a factor of roughly
four (four and three, respectively) compared to the original
HERAPDF2.0 uncertainties.
%
When comparing with Sec.~\ref{sec:projections:rw},
the initial uncertainties of the HERAPDF2.0  analysis 
are affected by the choice of data (DIS data only), and 
the number and form of the parametrization (14 parameter HERAPDF form);
the final uncertainties are determined by the profiling procedure. 
%
In particular the profiling for the HERAPDF2.0 study assigns an effective 
uncertainty on the pseudodata corresponding to $\Delta\chi^2=1$, whereas the 
constraint in the NNPDF study is weaker, as it would be for a PDF set with 
eigenvectors, but which applied a tolerance criterion. 
%
While these initial studies are instructive, 
further comparisons of these analyses would be valuable. 

In Fig.~\ref{fig:pdfsProf} we present a comparison of the
$u^+$, $d^+$, $g$, and $s^+$ PDFs at the scale of $Q^2=4\text{ GeV}^2$
between the original  HERAPDF2.0 set and the results of the profiling
exercise for Scenarios~A, B and C.
%
Only the {\it experimental} PDF uncertainties are shown in this comparison,
but not the {\it model} and {\it parametrization} variations.
%
The corresponding results based on the reweighting
of NNPDF3.1 were shown in Figs.~\ref{fig:impactUnpol}
and~\ref{fig:impactUnpollargex}.

%-------------------------------------------------------------------------------
\begin{figure}[!t]
\centering
\includegraphics[width=0.45\textwidth]{plots/ratio_uPubar_Q2.pdf}
\includegraphics[width=0.45\textwidth]{plots/ratio_dPdbar_Q2.pdf}\\
\includegraphics[width=0.45\textwidth]{plots/ratio_g_Q2.pdf}
\includegraphics[width=0.45\textwidth]{plots/ratio_sPsbar_Q2.pdf}
\caption{\small Comparison
of the $u^+$, $d^+$, $g$, and $s^+$ PDFs at the scale of $Q^2=4\text{ GeV}^2$
between the original  HERAPDF2.0 set and the results of the profiling
exercise accounting for the constraints of
the lattice-QCD moments
pseudo-data in Scenarios~A, B and C.
%
Only the {\it experimental} PDF uncertainties are shown,
but not the {\it model} and {\it parametrization} variations.
}
\label{fig:pdfsProf}
\end{figure}
%-------------------------------------------------------------------------------

From Fig.~\ref{fig:pdfsProf} we see that, as expected, the impact of the 
lattice pseudo-data is greatest in the medium and large-$x$ regions.
%
The precise impact on the PDFs is rather
similar for the three scenarios, with the most optimistic
Scenario~C leading to the largest reduction in uncertainties.
%
The quark flavor combinations that are most affected by the
lattice-QCD pseudo-data are the $d^{+}$ and $s^{+}$ PDFs,
and, to a lesser extent, the gluon PDF.
%
The improvement in the PDF uncertainties for $d^{+}$ and $s^{+}$
occurs because the DIS data
used in HERAPDF2.0 include only limited constraints
on quark flavor separation, and, for these PDFs, the lattice-QCD 
pseudo-data add important new information.



% x-space analysis using reweighting
\subsection{Impact of $x$-space lattice QCD calculations of PDFs}
\label{sec:projectionsxspace}

In this section we perform an initial exploration of the
potential impact that future lattice QCD calculations
of $x$-space PDFs can have on the global analysis.
%
Specifically, we will focus on the isotriplet
combination $x u-x d$ ($x\Delta u - x\Delta d$), which is the
one for which more progress has been performed recently.
%
Following the same Bayesian reweighting procedure that
has been employed to quantify the impact of the PDF moments,
we have generated pseudo-data for the isotriplet
combinations
\be
\label{eq:isotriplet_unpol}
u(x_i,Q^2)-d(x_i,Q^2) \, \quad{\rm and} \, \quad
\bar{u}(x_i,Q^2)-\bar{d}(x_i,Q^2) \, , \quad i=1,\ldots,N_x \, ,
\ee
for the unpolarized case, and for
\be
\label{eq:isotriplet_pol}
\Delta u(x_i,Q^2)-\Delta d(x_i,Q^2) \, \quad{\rm and} \, \quad
\Delta\bar{u}(x_i,Q^2)-\Delta\bar{d}(x_i,Q^2) \, , \quad i=1,\ldots,N_x \, ,
\ee
for the polarized case, with $N_x$ being the number of points
in $x$-space that are being sampled and we
take $Q^2=4$ GeV$^2$, consistently with the choice used
in the case of the PDF moments.

We consider here three scenarios, denoted by scenario D, E, and F,
for the total uncertainty that will be assigned to
the lattice-QCD calculations of the specific quark
combinations listed in Eqns.~(\ref{eq:isotriplet_unpol})
and~(\ref{eq:isotriplet_pol}).
%
First of all we indicate the values of $\left\{ x_i \right\}$
that have been chosen for this exercise.
%
With the motivation that the lattice-QCD computation is expected to have the
smallest systematic uncertainties at large $x$,
we have chosen the $N_x=5$ points to be
\be
x_i = 0.70\, ,0.75,\, 0.80,\, 0.85, \, 0.90 \, .
\ee
For each scenario we assume that that the total error of the lattice calculation
is the same for each value of $\left\{ x_i \right\}$, and
moreover we neglect the possible correlations between neighboring $x$-points.
%
We have assumed then that we have $\delta_{L}=12\%, 6\%$ and 3\% for scenarios
D, E, and F, respectively.
%
Note that here we assume the same values of $\delta_{L}$ for the polarized
and unpolarized cases, as well as for both the quark
and the antiquark isotriplet combinations Eqns.~(\ref{eq:isotriplet_unpol})
and~(\ref{eq:isotriplet_pol}).

The results of this exercise are summarized
in Fig.~\ref{fig:impactxspace}, where we show the
ratio of PDF uncertainties to the original
  NNPDF3.1 (NNPDFpol1.1) in the fits where lattice-QCD pseudo-data
  on $x$-space PDFs has been added to the global unpolarized
  (polarized) analysis.
  %
  Specifically, we show the impact on the PDF uncertainties
  in $\bar{u}$ and $\bar{d}$ at large-$x$ in the upper
  plots, with the corresponding comparison for $\Delta\bar{u}$
  and $\Delta\bar{d}$ in the lower plots.
  %
  From this comparison, we find that
  indeed these lattice-QCD calculations can reduce significantly
  the errors of the large-$x$ unpolarized and polarized
  antiquarks.
  %
  Taking into account that the PDF uncertainties on the large-$x$
  antiquarks to begin with are rather large, and that they
  enter a number of important BSM search channels
  (such as for instance for new heavy gauge bosons $W'$ and $Z'$)
  is clear that such calculations would have direct
  phenomenological implications.

%-------------------------------------------------------------------
\begin{figure}[!t]
\centering
\includegraphics[scale=0.45]{plots/xubar-unpol-lattice-relerr-xdata-xspace.pdf}
\includegraphics[scale=0.45]{plots/xdbar-unpol-lattice-relerr-xdata-xspace.pdf}
\includegraphics[scale=0.45]{plots/xubar-pol-lattice-relerr-xdata-xspace.pdf}
\includegraphics[scale=0.45]{plots/xdbar-pol-lattice-relerr-xdata-xspace.pdf}
\caption{\small The ratio of PDF uncertainties to the original
  NNPDF3.1 (NNPDFpol1.1) in the fits where lattice-QCD pseudo-data
  on $x$-space PDFs has been added to the global unpolarized
  (polarized) analysis.
  %
  Specifically, we show the impact on the PDF uncertainties
  in $\bar{u}$ and $\bar{d}$ at large-$x$ in the upper
  plots, with the corresponding comparison for $\Delta\bar{u}$
  and $\Delta\bar{d}$ in the lower plots.
}    
\label{fig:impactxspace}
\end{figure}
%---------------------------------------------------------------------

Specifically, from Fig.~\ref{fig:impactxspace} we find that
in unpolarized case the large-$x$ PDF uncertainties can be reduced
down to $60\%$ of its original value.
%
We also find that there are no big differences between the three
scenarios, suggesting that a direct lattice-QCD calculation
of $x u-x d$ does not need to reach uncertainties
at the few-percent level in order to impact positively
the PDFs.
%
In the polarized case, we find a similar result but the reduction
of PDF uncertainties can be more marked.
%
For instance in the case of $\Delta \bar{d}$, at $x\simeq 0.8$
the resulting PDF uncertainty from scenario C is less than 50\%
of the original one.
%
Note that the in a Monte Carlo approach such as NNPDF, the
PDF uncertainties fluctuate themselves, specially at low-scales,
explaining the wiggles in these plots.

Of course the results of this exercise have to be interpreted
with some care.
%
First of all, the results depend sensitively on the specific values of
$\left\{ x_i \right\}$
that we have assumed for the lattice-QCD calculation,
as well as of the associated uncertainties.
%
Also, the quantitative results would vary if a different input PDF set
was used, for example the HERAPDF2.0 set that was used for
the Hessian profiling exercise of Sect.~\ref{sec:hessianprofiling}.
%
But even accounting for these caveats, is clear that a direct
computation of the isotriplet combination $x u-x d$ on the lattice
has the potential to constrain the large-$x$ PDFs in
a more significant way that the corresponding PDF moments calculation,
specially in the unpolarized case.
%
And given the importance of large-$x$ antiquarks for LHC phenomenology,
pursuing this approach is thus most promising.


% General discussion
\subsection{Discussion}

We conclude this section with a brief discussion of the main lessons that
can be learned from this exercise, which provides the first quantitative estimate
of the impact of present and future lattice-QCD calculations of PDF moments
and $x$-space PDFs, for both polarised and unpolarised PDFs.

First, we have demonstrated that in the polarised case,
even with current uncertainties, lattice-QCD calculations of
selected PDF moments can impose sizable constraints on several
important polarised quark combinations.
%
This suggests that global polarised PDF analyses should consider
including existing lattice-QCD calculations in their fits to constrain some
of the least known quark combinations, such as the total strangeness.
%
The situation is rather different in the unpolarised case,
where a reduction of the current lattice-QCD uncertainties by a factor of between five 
and ten seems to be required to influence global fits.
%
This difference arises because unpolarised PDFs are known with much higher precision than polarised
PDFs, thanks to the much wider amount of experimental data sensitive to unpolarised PDFs,
including the constraints from recent high-precision measurements at the
LHC.
%
Thus, in addition to the differences highlighted  in Fig.~\ref{fig:Bmomsunp},
much more precise lattice-QCD calculations than in the polarised case 
need to be used to be competitive with current PDF fits.


Second, lattice-QCD calculations of the quark isotriplet combinations
$xu-xd$ and $x\bar{u}-x\bar{d}$ would be instrumental in constraining
quark PDFs at large-$x$.
%
Even a calculation with $\delta_L\simeq 10\%$ uncertainties at large-$x$ would
start to provide useful constraints on global fits.
%
Moreover, we find that, in the unpolarised case, the information on the
PDFs that could be derived from a direct $x$-space calculation
from lattice-QCD is clearly superior to the information that can be obtained
from PDF moments alone, at least for the subset of PDFs and moments used in the present
exercise.

The profiling studies presented in this section could be extended in
a number of directions.
%
In the polarised case, one could include the current lattice-QCD
values of the moments listed in Tab.~\ref{tab:BMpol} in global analyses: 
indeed, we have demonstrated that at the current level of uncertainties one 
expects to find some non-trivial constraints.
%
In this respect, a crucial topic to investigate is the compatibility 
(or lack thereof) of the existing lattice-QCD numbers compared to constraints 
from experimental data.
%
For both unpolarised and polarised PDFs, it would be interesting to include the 
effects of other moments and flavour combinations.
%
Higher moments, in particular, typically probe regions of higher $x$, compared
to lower moments, and in the large-$x$ regions uncertainties in the phenomenological PDFs are
more marked.
%
One could also consider the effects of the quark combinations for which $x$-space
calculations might be available, for example those related to the proton strangeness.
%
Finally, a more refined analysis should include the theoretical correlations
expected in lattice-QCD calculations, for instance, in the case of $x$-space calculations,
one expects neighbouring points in $x$ to be highly correlated.

%%%%%%%%%%%%%%%%%%%%%%%%%%%%%%%%%%%%%%%%%%%%%%%%%%%


