The lattice-QCD method is based on regularizing QCD on a finite Euclidean 
lattice and is generally studied by numerical computation of QCD correlation 
functions in the path-integral formalism~\cite{DiPierro:2000nt,Lepage:1998dt,
Luscher:1998pe,Gupta:1997nd}, using methods adapted from statistical 
mechanics~\cite{Binder:2015klx,Newman:1999mng}.
%
To make contact with experimental data, the numerical results are extrapolated 
to the continuum and infinite-volume limits.
%
The past decade has seen significant progress in the development of efficient 
algorithms for the generation of ensembles of gauge field configurations and 
tools for extracting the relevant information from lattice-QCD
correlation functions.
%
In this respect, lattice-QCD calculations have reached a level where
they not only complement, but also guide current and forthcoming
experimental programs~\cite{Brodsky:2015aia,Aschenauer:2014twa}.

In this section, we discuss the sources of systematic uncertainties
that affect current lattice QCD calculations, and we present 
lattice-QCD methods to determine either the Mellin moments of PDFs
or the complete PDF $x$-dependence.

\subsubsection{Systematic uncertainties}
Lattice-QCD calculations must demonstrate control over all sources of
systematic uncertainty introduced by the discretization of QCD on the
lattice to make meaningful contact with experimental data.
%
These
include discretization effects that vanish in the continuum limit;
extrapolation from unphysically heavy pion masses; finite volume
effects; and renormalization of composite operators.
%
To take the continuum limit requires accurate determinations of the 
lattice spacing.
% 
We briefly review these main sources of systematic uncertainty here; for a 
fuller account see, for example, Ref.~\cite{Aoki:2016frl}.

\begin{itemize}

\item {\bfseries Discretization effects and the continuum limit.} 
There is a fair degree of flexibility in discretizing the QCD action. 
%
This has led to a variety of formulations, which differ mainly in the choice of
the action for quarks.
%
In the continuum limit, which corresponds to taking
the lattice spacing $a$ to zero with all physical quantities fixed,
the simplest discretizations differ from continuum QCD at ${\mathcal
O}(a)$.
%
In practice, one cannot afford to perform numerical
simulations at arbitrarily small lattice spacings, because the cost of
computation increases with a large inverse power of the lattice
spacing, and ${\cal O}(a)$ effects can be significant even
with current lattice spacings ranging from $0.15 \,\mbox{fm}$ to
$0.05 \,\mbox{fm}$.
%
To accelerate the convergence to the continuum
limit, improved quark and gluon actions are widely used, which include
higher-dimension operators to reduce the discretization errors to
${\cal O}(a^2)$ or better.
%
Chiral fermions with automatic $\mathcal{O}(a)$ improvement and small 
$\mathcal{O}(a^2)$ discretization errors are also adopted to admit 
calculations on coarser lattice spacings~\cite{Creutz:2011hy,Vladikas:2011bp,
Chandrasekharan:2004cn}. 

\item {\bfseries Pion mass dependence.} 
The computational cost of the fermion contribution to the path
integral increases with a large inverse power of the bare quark mass
(or, equivalently, the pion mass).
%
Lattice-QCD calculations are therefore
often performed at unphysically heavy pion masses, although results calculated
directly with physical pion masses have become increasingly common, albeit with larger
errors.
%
To obtain results at the physical pion mass, lattice data are
generated at a sequence of pion masses and then extrapolated to the
physical pion mass.
%
To control the associated systematic
uncertainties, these extrapolations are guided by effective
theories.
%
In particular, the pion-mass dependence can be parametrized
using chiral perturbation theory ($\chi$PT)~\cite{Golterman:2009kw}, 
which accounts for the
Nambu-Goldstone nature of the lowest excitations that occur in the
presence of light quarks. 
%
%With multiple lattices at different sea quark masses including some
%at physical pion mass, several lattice spacings and volumes, one can  
%perform global fits with judicious functional forms to determine
%the final result at the physical pion mass and extrapolated continuum and 
%infinite volumes.

\item {\bfseries Finite volume effects.} Numerical lattice-QCD 
calculations are necessarily restricted to a finite space-time
volume, {\it e.g.}, a hypercube of side $L$.
%
For most simple quantities, these effects decay exponentially
with the size of the lattice~\cite{Luscher:1985dn,Luscher:1986pf}, and 
therefore the easiest way to
minimize or eliminate finite volume effects is to choose the volume
sufficiently large in physical units.
%
Unfortunately, this can be
prohibitively expensive as one approaches the continuum limit, requiring the
number of lattice sites to grow as $L/a$ in all four directions. 
%
Finite volume $\chi$PT is the preferred
tool to develop systematic expansions that provide quantitative
information on finite-volume effects.
%
In general, finite volume
effects of hadrons are dominated by their interactions with pions,
which can travel around the (periodic) lattice many times.
%
Numerical evidence suggests that lattice sizes of $m_\pi L \geq 4$, where
$m_\pi$ is the pion mass, are generally sufficiently large that finite
volume effects are negligible for mesons, within the current precision 
of lattice-QCD calculations.
%
From the studies of the pseudoscalar and electromagnetic form factors of the 
nucleon, it is evident that larger physical volumes are needed for the 
baryons.

\item {\bfseries Excited state contamination.} 
At small Euclidean times, a lattice-QCD correlation function
is a sum over a tower of states that behave as $e^{-m_it}$, where $m_i$ is the 
energy of the state and $t$ is the Euclidean time. 
%
Thus, at large Euclidean times,
ground-state quantities can be extracted by fitting to the dominant 
exponential behavior.
%
Unfortunately, the signal-to-noise ratio is exponentially suppressed 
as $e^{-(E_N-3m_\pi/2)t}$, where $E_N$ is the nucleon energy~\cite{Lepage:1989hd}.
%
Thus, lattice-QCD results
are extracted from an intermediate region in which excited state contributions 
are either small or well-controlled and the signal-to-noise ratio is 
sufficiently large that the signal can be reliably extracted. 
%
This is a particular challenge for baryons and is one of the largest 
sources of systematic uncertainties for nucleon matrix elements.

\item {\bfseries Renormalization.} The matrix elements extracted from a 
lattice-QCD calculation at a given lattice spacing are bare matrix elements,
rendered finite by the presence of the lattice spacing, which serves
as a gauge-invariant UV regulator. 
%
To take the continuum limit, {\it i.e.}, remove the regulator, one must 
renormalize the corresponding operators and fields and match them to some 
common scheme and scale used by phenomenologists. 
%
Although renormalization is traditionally
discussed in the framework of perturbation theory, at hadronic energy
scales the renormalization constants should be computed
nonperturbatively to avoid uncontrolled uncertainties due to 
truncated perturbative results.
%
To compare with phenomenology, which uses the $\overline{\rm MS}$ scheme, 
a conversion factor from the nonperturbative scheme must be computed 
perturbatively. 
%
This requires a renormalization condition that can be implemented on the 
lattice and in continuum perturbation theory. 
%
In QCD with only light quarks it is technically advantageous to employ 
so-called mass-independent renormalization schemes. 
%
A common choice is the regularization-independent/momentum (RI/MOM) 
scheme~\cite{Martinelli:1994ty}.

In addition, on a hypercubic lattice, the orthogonal group $O(4)$ of
continuum Euclidean space-time is reduced to the hypercubic group
$H(4)$.
%
Thus, operators are classified according to irreducible
representations of $H(4)$~\cite{Gockeler:1996mu}.
%
Different irreducible representations belonging to the same $O(4)$ multiplet
will, in general, give different answers at finite lattice spacing, an effect 
that can be reduced by improving the operators~\cite{Gockeler:2004wp}.
%
Conversely, operators that lie in different irreducible representations of 
$O(4)$, but the same irreducible representations of $H(4)$, will mix at finite 
lattice spacing but not in the continuum. 
%
When these operators have lower mass dimensions,
the mixing coefficients scale with the inverse lattice spacing to some
power, and diverge in the continuum limit.
%
This power-divergent mixing
must be removed nonperturbatively, and is a particular challenge for
lattice calculations of the Mellin moments of PDFs (see
Sec.~\ref{Sec:MomentsLQCD}).

%Finally, it is worth noting that factorization, the key assumption of
%the operator product expansion (OPE), demands that the nonperturbatively 
%renormalized hadron matrix elements are matched to the perturbatively 
%renormalized Wilson coefficients at a scale where the perturbative 
%expressions show convergence. This appears to be
%the case for scales $\mu^2 \gtrsim 10\mbox{ GeV}^2$ at
%least~\cite{Gockeler:2010yr}. This, however, is a fundamental aspect
%of QCD, and is not restricted to lattice QCD. The DGLAP evolution equations,
%for example, work best for $q^2_{\rm min} \approx
%15\mbox{ GeV}^2$~\cite{Abramowicz:2015mha}, which should be kept in
%mind when comparing lattice results with phenomenology.

\item {\bfseries Lattice-spacing determination.} 
Numerical lattice-QCD calculations naturally determine all dimensionful 
quantities in units of the lattice spacing. 
%
Thus, extracting physical values requires the determination of the lattice 
scale. 
%
This is achieved by matching a quantity with mass dimension to its experimental 
value or through a well-defined theoretical procedure, that is referred to as
{\it scale-setting}. 
%
Popular reference scales include light decay constants, hadron masses, 
scales defined in terms of the heavy quark potential or, most recently, 
the length scales $\sqrt{t_0}$~\cite{Luscher:2010iy} and 
$w_0$~\cite{Borsanyi:2012zs} defined via the Wilson  gradient 
flow~\cite{Luscher:2010iy}. 
%
These scales can be computed cheaply and can be used to 
match scales between different gauge ensembles very accurately.  
%
However, a hadron mass or a decay constant --- which are known accurately 
from experiment and can be computed precisely in lattice-QCD --- 
have to be used for absolute scale setting. 
%
A popular hadronic mass for this purpose is the mass of the triply strange 
$\Omega$ baryon~\cite{Durr:2008zz} or the 2S-1S splitting in the Upsilon 
spectrum~\cite{Kendall:2008zz}.

\end{itemize}

These sources of systematic uncertainty all need to be under control
when confronting experimental data with lattice results, or vice
versa.
%
For a coherent assessment of the present state of lattice-QCD
calculations of various quantities, the degree to which each
systematic has been controlled in a given calculation is an important
consideration.
%
In Sec.~\ref{subsubsec:BClQCD}, we characterize the
quality of the lattice calculations, based on criteria inspired by the
FLAG analysis of flavor physics on the lattice~\cite{Aoki:2016frl}.

\subsubsection{Mellin moments of PDFs from lattice QCD}
\label{Sec:MomentsLQCD}

Parton distributions cannot be directly determined in Euclidean lattice QCD, 
because their field-theoretic definition involves fields at light-like 
separations.
%
Instead, the traditional approach for lattice-QCD calculations has been to 
determine the matrix elements of local twist-two operators, where twist is the 
dimension minus the spin, that can be related to the Mellin moments of PDFs.
%
In principle, given a sufficient number of Mellin moments, PDFs can be 
reconstructed from the inverse Mellin transform. In practice, however, 
the calculation is limited to the lowest three moments, because power-divergent 
mixing occurs between twist-two operators on the lattice.
%
Three moments are insufficient to fully reconstruct the momentum dependence of 
the PDFs without significant model dependence~\cite{Detmold:2003rq}.
%
The lowest three moments do provide, however, useful information, both as 
benchmarks of lattice-QCD calculations and as constraints in global extractions 
of PDFs. 
%
Here we briefly review the determination of Mellin moments of PDFs from lattice 
QCD. 

Using the operator product expansion (OPE)~\cite{Zimmermann:1972tv}, the Mellin 
moments of the structure functions and the corresponding PDFs can be expressed, 
up to higher-twist effects, in terms of matrix elements of local operators:
\begin{align}
\!\!\!2 \int_0^1 dx\, x^{n-1} F_1(x,Q^2) &= \sum_a C_{1,a}^n(\mu^2)\, v_a^n(\mu^2)|_{\mu^2=Q^2} = \sum_a C_{1,a}^n(Q^2)\, \int_0^1 dx\, x^{n-1} f_a(x,Q^2)\,,\\
4 \int_0^1 dx\, x^n g_1(x,Q^2) &= \sum_a \Delta C_{1,a}^n(\mu^2)\, a_a^n(\mu^2)|_{\mu^2=Q^2} = \sum_a \Delta C_{1,a}^n(Q^2)\, \int_0^1 dx\, x^n\, 2 \Delta f_a(x,Q^2)\,,
\end{align}
where $v_i^n(\mu^2)$ and $a_i^n(\mu^2)$ are reduced matrix elements of the appropriate twist-two operators~\cite{Gockeler:1995wg},
\begin{align}
\frac{1}{2} \sum_s \langle p,s|\mathcal{O}^i_{\{\mu_1,\cdots,\mu_n\}}|p,s\rangle = {} & 2 v_i^n\, [p_{\mu_1}\cdots p_{\mu_n} - {\rm traces}] \,, \label{eq:twist2me}\\
\langle p,s|\mathcal{O}^{5\,i}_{\{\sigma \mu_1,\cdots,\mu_n\}}|p,s\rangle = {} & \frac{1}{n+1} a_i^n\, [s_\sigma p_{\mu_1}\cdots p_{\mu_n} - {\rm traces}]\,,
\end{align}
and $C_{1,i}^n(\mu^2)$ and $\Delta C_{1,i}^n(\mu^2)$ are the Mellin moments of the corresponding Wilson coefficients
\begin{equation}
C_{1,i}^n(\mu^2) = \int_0^1 dy\, y^{n-1} C_{1,i}(y,\mu^2)\,, \quad
\Delta C_{1,i}^n(\mu^2) = \int_0^1 dy\, y^n \Delta C_{1,i}(y,\mu^2)\,.
\end{equation}
The trace terms include operators with at least one factor of the metric 
tensor $g^{\mu_i \mu_j}$ multiplied by operators of dimension $(n+2)$ with 
$n-2$ Lorentz indices. 
%
The operators relevant for the lowest two moments are listed in 
Table~\ref{Tab:twist2}. 
%
The operator $\mathcal{O}^q_{\mu_1\mu_2}$ decomposes into two different 
representations of $H(4)$~\cite{Gockeler:1996mu}, each with different 
lattice artifacts and renormalization factors. 
%
In the continuum limit, however, both operators should lead to the same result. 
%
In contrast, the operator $\mathcal{O}^q_{\mu_1\mu_2\mu_3}$ splits into several 
representations of $O(4)$ transforming identically under $H(4)$ and causing 
the corresponding operators to mix under renormalization on the lattice.

%-------------------------------------------------------------------------------
\begin{table}
\renewcommand{\arraystretch}{1.6} 
\centering
\begin{tabular}{@{}ccc@{}}
\toprule
Matrix element & Operator & PDF moment \\ 
\midrule
$v_q^2$\,, $v_{\bar{q}}^2$  & 
$\displaystyle \left({\rm i}/2\right) \bar{q}(x)\gamma_{\mu_1} \overleftrightarrow{D}_{\mu_2} q(x)$ & 
$\langle x \rangle_{q^+}$\\
$v_q^3$\,, $v_{\bar{q}}^3$  & $\displaystyle \left({\rm i}/2\right)^2 \bar{q}(x)\gamma_{\mu_1} \overleftrightarrow{D}_{\mu_2} \overleftrightarrow{D}_{\mu_3} q(x)$ & $\langle x^2 \rangle_{q^-}$\\
$a_q^0$ & $\displaystyle \bar{q}(x)\gamma_{\sigma} \gamma_5 q(x)$ & 
$2\, \langle 1 \rangle_{\Delta q^+}$ \\
$a_q^1$ & $\displaystyle \left({\rm i}/2\right) \bar{q}(x)\gamma_{\sigma} \gamma_5 \overleftrightarrow{D}_{\mu_1} q(x)$ & $2\, \langle x \rangle_{\Delta q^-}$ \\
$v_g^2$ & $\displaystyle - {\rm Tr}\, F_{\mu_1\alpha}F_{\mu_2\alpha}$ & $\langle x \rangle_g$ \\
\bottomrule
\end{tabular}
\caption{\label{Tab:twist2}
\small List of operators relevant for the computation of the lowest two 
Mellin moments of polarized and unpolarized PDFs.
%
Here we indicate, for each operator, the corresponding matrix element and
the specific PDF moment that can be evaluated (see
Appendix~\ref{app:notation} for the notation used).
}
\end{table}
%-------------------------------------------------------------------------------

\paragraph*{Higher-twist contributions.}
%
The discussion so far has focused on the limit in which higher-twist 
contributions, suppressed by powers of the momentum transfer, have been ignored.
%
In fact, higher-twist contributions to the lowest moment of the structure 
function $F_1(x,Q^2)$ are found to be of 
${\cal O}(1\mbox{ GeV}^2/Q^2)$~\cite{Blumlein:2008kz}.
%
For lattice QCD, typically $Q^2 \simeq 1/a^2$, and at present lattice spacings 
this corresponds to $Q^2 = {\cal O}(10\mbox{ GeV}^2)$ or a higher-twist 
contribution of 5--$10\%$. 
%
With contributions of higher-twist included, the OPE for $F_1$ reads
\begin{equation}
2 \int_0^1 dx\, x F_1^q(x,Q^2) = C_{1,q}^2(\mu^2)\, v_q^2(\mu^2)|_{\mu^2=Q^2} + \frac{\bar{C}_{1,q}^2(\mu^2)}{Q^2}\, \bar{v}_q^2(\mu^2)|_{\mu^2=Q^2} + \cdots \,,
\label{tex}
\end{equation}
where $\bar{C}_{1,q}^2$ and $\bar{v}_q^2(\mu^2)$ are the Wilson coefficient and 
reduced matrix element of a generic twist-four operator. 
%
Both twist-two and four contributions mix under renormalization, to the extent 
that the perturbative series for the Wilson coefficients $C_{1,q}^2(\mu^2)$ 
diverges due to the presence of infrared (IR) renormalon singularities.
%
This ambiguity is canceled by that in the twist-four matrix element 
$\bar{v}_q^2(\mu^2)$ that arises as a result of an ultraviolet (UV) 
renormalon singularity~\cite{Martinelli:1996pk}. 
%
If mixing effects are ignored, the uncertainties will be, at least, comparable 
to the power corrections themselves.
%
Power corrections can be assessed most efficiently, and the twist expansion 
tested, by a direct lattice-QCD evaluation of the Compton amplitude, which we 
discuss in Sec.~\ref{Sec:InversionMethod}.

\paragraph*{Beyond the first three moments.}
%
Moving beyond the lowest three moments requires overcoming the challenge of 
power-divergent mixing for lattice-QCD twist-two operators.
%
One novel approach to this problem~\cite{Davoudi:2012ya} builds upon the 
physical intuition that as long as the scale associated with the operator 
(for the twist-two operators, this is the renormalization scale $\mu$) is taken 
to be much larger than the hadronic scale but much smaller than the inverse 
lattice spacing, no singularity necessarily arises as one takes the continuum 
limit.
%
The operator can still probe the correct hadron structure at the scale $\mu$, 
but should be insensitive to the details of the discretization of the operator 
at shorter distances.
%
A simple way to incorporate an intrinsic {\it smearing} scale for an operator 
is to sum over bilinears of quark fields that are displaced over many lattice 
sites in a small (compared to the scale $1/\mu$) region of Euclidean space-time 
(an alternative approach appears in Ref.~\cite{Monahan:2015lha}).

To ensure that the correct $SO(4)$ transformation properties of the matrix 
elements are recovered in the continuum limit, one must project the sum using 
hyper-spherical harmonics.
%
The properties of these operators, such as their mixing patterns and scaling 
properties, are discussed in detail in Ref.~\cite{Davoudi:2012ya}.
%
In particular, while the classical mixing with lower and higher spin operators 
are both suppressed by $\sim a^2$ for spatially improved operators, the mixing 
at one-loop in lattice perturbation theory is suppressed 
by ${\cal O}(\alpha_s a)$ or ${\cal O}(\alpha_s a^2)$. 
%
The suppression depends on the lattice action used, provided that the gauge 
action adopted to construct the gauge-invariant bilinears is tadpole-improved 
and smeared over a region whose physical size is held fixed as the continuum 
limit is taken. 
%
In principle, this allows higher moments of PDFs to be obtained from lattice 
QCD, without power divergences. Numerical investigations of this approach,
which requires gauge configurations with very fine lattice spacings, are 
underway.
%
Other approaches that avoid power-divergent mixing have also been suggested, 
including coupling fictitious heavy quarks to light-quark 
currents~\cite{Detmold:2005gg}, and calculating current correlators in 
position space~\cite{Braun:2007wv}. 
%
The practical application of these ideas is yet to be studied nonperturbatively.

\subsubsection{The $x$-dependence of PDFs from lattice QCD}
\label{sec:xdependence}

While the lowest three moments of PDFs can provide important benchmarks for 
lattice-QCD calculations of nucleon structure, and useful constraints in global 
extractions of PDFs, they are not in themselves sufficient to determine the 
$x$-dependence of PDFs.
%
In the following section we summarize recent approaches to determining the 
$x$-dependence of PDFs directly from lattice QCD.

\paragraph*{Hadronic tensor.} 
In principle, PDFs can be determined from hadronic tensors provided the 
higher-twist contributions, which have different $Q^2$ dependence than the 
leading-twist, can be subtracted. 
%
Calculating the hadronic tensor in the Euclidean path-integral approach
has the advantage that no renormalization is required if conserved vector      
currents are used in the current-current correlation and only finite 
renormalizations are needed for the local currents.
%
Furthermore, since the structure functions are frame-independent, they 
can be calculated in any momentum frame of the nucleon. 
%
One can choose the nucleon momenta and momentum transfers judiciously 
to have a desirable coverage of $x$ for a given $Q^2$. 
%
However, the inverse Laplace transform that is needed to convert the hadronic tensor from Euclidean space to Minkowski space can be a 
challenge~\cite{Liu:1993cv,Liu:1999ak}. 
%
Three numerical approaches, the Backus-Gilbert method~\cite{Hansen:2017mnd}, 
improved maximum entropy, and fitting with model spectral functions, 
are suggested to tackle this inverse Laplace-transform 
problem~\cite{Liu:2016djw}. 
%
In Ref.~\cite{Liu:1993cv} sea partons are separated into {\it connected sea} 
and {\it disconnected sea} contributions, based on the distinct topologies of 
the diagrams in a lattice computation. 
%
This distinction can help identify the impact on PDF uncertainties of 
improving the uncertainties associated with disconnected diagrams determined 
using lattice-QCD.
%
The extended evolution equations to accommodate both the connected sea 
(CS) and disconnected sea (DS) partons are derived in Ref.~\cite{Liu:2017lpe}.

\paragraph*{The inversion method.} 
\label{Sec:InversionMethod}

The Compton amplitude $T_{\mu\nu}(p,q)$, Eq.~\eqref{eq:Compton}, can be
directly obtained in lattice QCD, including disconnected contributions,  
by a simple extension~\cite{Chambers:2017dov} of existing implementations of 
the Feynman-Hellmann technique to lattice QCD~\cite{Horsley:2012pz,
Chambers:2014qaa,Chambers:2015bka}.
%
Provided one works at sufficiently large $Q^2$, the Compton amplitude will be 
dominated by twist-two contributions.
%
Varying $Q^2$ allows one to test the twist expansion and, in particular, 
isolate twist-four contributions. Moreover, one can distinguish between 
contributions from up, down and strange quarks, connected and disconnected, 
by appropriate insertions of the electromagnetic current.

To compute the Compton amplitude from the Feynman-Hellmann relation, a 
perturbation to the QCD Lagrangian is introduced, for example,
\begin{equation}
\mathcal{L}(x) 
\rightarrow 
\mathcal{L}(x) + \lambda \mathcal{J}_3(x)\,, 
\quad 
\mathcal{J}_3(x)
=
Z_V\cos(\vec{q} \cdot \vec{x})\; 
e_q \,\bar{q}(x)\gamma_3 q(x) 
\label{in}
\end{equation}
where $q$ is the quark field to which the photon is attached, and $e_q$ its 
electric charge. 
%
For simplicity, we consider the local vector current only, so that the 
renormalization factor $Z_V$ is known and no further renormalization is needed. 
%
Taking the second derivative of the nucleon two-point function 
\begin{equation}
\langle N(\vec{p},t) \bar{N}(\vec{p},0)\rangle_\lambda 
\simeq 
C_\lambda\, e^{-E_\lambda(p,q)\,t}
\end{equation}
with respect to $\lambda$ on both sides, gives
\begin{equation}
-2 E_\lambda(p,q)\, 
\frac{\partial^2}{\partial\lambda^2}  E_\lambda(p,q)\,\big|_{\lambda=0} 
= 
T_{33}(p,q) \,.
\end{equation}
For $p_3=q_3=q_4=0$ this leaves us with
\begin{equation}
T_{33}(p,q) 
= 
4 \omega^2 \int_0^1 dx\,  \frac{xF_1(x,Q^2)}{1-(\omega x)^2} \,.
\label{ff}
\end{equation}
%
Extracting the polarized structure functions requires insertions of two 
different currents with $\mu\neq \nu$. 
%
The idea is then to solve Eq.~\eqref{ff} for $F_1(x,Q^2)$ numerically.
%
In Refs.~\cite{Ji:2001wha,Chambers:2017dov} it was shown that the unpolarized 
structure function $F_1(x,Q^2)$ can be computed from a lattice calculation 
of the Compton amplitude, devoid of any renormalization and mixing issues. 
%Furthermore, by extending the calculation to values $\omega > 1$ 
%it becomes possible to compute the structure functions down to fractional 
%momenta $x = {\cal O}(0.001)$. 
With the same method, PDFs can be computed directly without the need to go 
through the structure functions, provided $Q^2$ is sufficiently large that 
power corrections can be neglected. 

\paragraph*{Quasi-PDFs.}
Quasi-PDFs provide an alternative approach to determining the $x$-dependence 
of PDFs directly from lattice QCD~\cite{Ji:2013dva,Ji:2014gla}. 
%
In the following discussion, we focus on the flavor-nonsinglet quasi-PDF, 
for which we can ignore mixing with the gluon quasi-PDF. 
%
The unpolarized quark quasi-PDF is defined as the momentum-dependent
nonlocal forward matrix element
\begin{align}\label{eq:qPDF}
\widetilde{q}(x,\Lambda,p_z)  
= {} &  \int \frac{dz}{2\pi} e^{-i x z p_z} p_z h(z,p_z), \nonumber \\
h(z,p_z) 
= {} &
\frac{1}{4 p_{\alpha}}\sum_{s=1}^2\left\langle p,s\right\vert \bar{\psi}(z)\gamma_\alpha e^{ig\int_0^z
A_z(z^\prime) dz^\prime} \psi(0) \left\vert p,s\right\rangle,
\end{align}
%
where $\Lambda$ is an UV cut-off scale, such as the inverse lattice spacing 
$1/a$. 
%
The Lorentz index $\alpha$ of the matrix $\gamma_\alpha$ is generally chosen 
to be spatial, $\alpha = z$, but the alternative choice $\alpha = 4$ is also 
possible and removes part of the leading order twist-4 
contamination~\cite{Xiong:2013bka,Radyushkin:2016hsy}. 
%
Because $p$ is finite, the momentum fraction $x$ can be larger than unity.

The quasi-PDF is defined for nucleon states at finite momentum and must be 
related to the corresponding light-front PDF\footnote{In this context the term 
 light-front PDF is used to distinguish ordinary PDFs, 
 Eqs.~\eqref{eq:unpPDFs}--\eqref{eq:polPDFs} from quasi-PDFs, 
 Eq.~\eqref{eq:qPDF}.}, 
for which the nucleon momentum is taken to infinity.
%
In the  large-momentum  effective field theory (LaMET) approach, the
quasi-PDF $\widetilde{q}(x,\Lambda,p_z)$ can be related to the $p_z$-independent
light-front PDF $q(x,Q^2)$ through~\cite{Ji:2013dva,Ji:2014gla}
\begin{equation} \label{eq:qPDFmatching}
\widetilde{q}(x,\Lambda ,p_z) = 
  \int_{-1}^1 \frac{dy}{\left\vert y\right\vert} 
    Z\left( \frac{x}{y}, \frac{\mu}{p_z}, \frac{\Lambda}{p_z}\right)_{\mu^2 = Q^2} q(y,Q^2) +
  \mathcal{O}\left( \frac{\Lambda_\text{QCD}^2}{p_z^2},\frac{M^2}{p_z^2}\right), 
\end{equation}
where $\mu$ is the renormalization scale,
$Z$ is a matching kernel and $M$ is the nucleon mass.
%
Here the $\mathcal{O}\left(M^2/p_z^2\right)$ terms are target-mass corrections 
and the $\mathcal{O}\left(\Lambda_\text{QCD}^2/p_z^2\right)$ terms are 
higher-twist effects, both of which are suppressed at large nucleon momentum. 
%
%A complementary approach to LaMET views the quasi-PDF as a 
%{\it lattice cross-section} from which the light-front PDF can be 
%factorized~\cite{Ma:2014jla,Ma:2014jga,Ma:2017pxb}.
%
An alternative, but related, construction is proposed in 
Refs.~\cite{Radyushkin:2016hsy,Radyushkin:2017cyf} and explored in 
Ref.~\cite{Orginos:2017kos}.

Preliminary results from lattice calculations of quasi-PDFs have been 
encouraging~\cite{Lin:2014zya,Alexandrou:2015rja,Chen:2016utp,
Alexandrou:2016jqi}. 
%
However, there are a number of remaining challenges that must be overcome for 
an {\it ab initio} determination of the $x$-dependence of PDFs directly from 
lattice QCD that incorporates complete control over systematic uncertainties. 
%
Lattice calculations of quasi-PDFs are subject to the same sources of 
systematic uncertainty that affect all lattice calculations, see 
Sec.~\ref{Sec:IntroLQCD}. 
%
Here we focus on systematic uncertainties that are more specific to quasi-PDFs.
%
These are uncertainties associated with the finite nucleon momentum of the 
lattice calculations and to the renormalization of quasi-PDFs.

\begin{itemize}

\item Preliminary nonperturbative studies of the quasi-PDF used nucleon 
momenta in the range $p_z = 2\pi/L$ to $10\pi/L$, where $L$ is the physical 
extent of the lattice, corresponding to $p_z = 0.5$ to 
$2.5$~GeV~\cite{Lin:2014zya,Alexandrou:2015rja,Chen:2016utp,Alexandrou:2016jqi}.
%
At such low momenta, higher-twist and target mass corrections are likely to be 
considerable.

Target mass corrections can be removed to all orders~\cite{Chen:2016utp}, and 
twist-4 contributions can be removed in 
principle~\cite{Chen:2016utp,Radyushkin:2016hsy}, leaving higher-twist 
contamination. 
%
To reduce these remaining effects starting at $O(\Lambda_{\rm QCD}^2/p_z^2)$, 
the authors of Refs.~\cite{Lin:2014zya,Chen:2016utp} extrapolated to infinite 
nucleon momentum using the fit ansatz $a + b/p_z^2$ for each value of $x$. 
%
Although the effects of finite nucleon momentum can be mitigated, a quark-model 
study asserts that reducing systematic uncertainties to less than 20\% at 
moderate values of $x$ requires significantly larger values of nucleon 
momentum~\cite{Gamberg:2014zwa}, and at larger values of $x$ 
(roughly $x\simeq 1$) requires nucleon momentum as large as $p_z > 4$~GeV.

The size of the nucleon momentum is currently limited by the decreasing 
signal-to-noise ratio at large momenta, which requires very high statistics 
to extract a signal. 
%
New approaches to high-momentum nucleons are being investigated, with the most 
promising an approach that employs momentum smearing~\cite{Bali:2016lva}. 
%
This method has been applied to quasi-PDFs in 
Refs.~\cite{Alexandrou:2016jqi,Green:2017xeu}, demonstrating a large 
improvement in the signal-to-noise ratio by reaching momenta of about $2.5$~GeV.

\item The leading-twist quasi-PDFs and light-front PDFs are connected through 
the matching (or {\it factorization}) relation, Eq.~\eqref{eq:qPDFmatching}. 
%
Provided the quasi and light-front PDFs share the same IR behavior, the 
matching kernel can be determined in perturbation theory~\cite{Xiong:2013bka}. 
%
The one-loop matching kernel including gluon channel has been recently 
reported~\cite{Wang:2017qyg}.
%
The factorization of the IR structure of quasi-PDFs into light-front PDFs and an IR-safe matching kernel was claimed to hold to all orders in Refs.~\cite{Ma:2014jla,Ma:2014jga,Ma:2017pxb}.
%
More specifically, Refs.~\cite{Ma:2014jla,Ma:2014jga} claim that the 
factorization holds to all orders provided that UV divergences 
are properly renormalized.
%
However, Ref.~\cite{Li:2016amo} asserted that there might be subtleties beyond 
leading order in perturbation theory. 
%
A distinct, but similar, issue is the IR structure of extended operators in 
Euclidean and Minkowski space-time. 
%
There are again subtleties in perturbation theory~\cite{Carlson:2017gpk}, 
but arguments based on general field-theoretic grounds demonstrate that the 
quasi-PDF extracted from an Euclidean correlation function is exactly the 
same matrix element as that determined from the LSZ reduction formula in 
Minkowski space-time~\cite{Briceno:2017cpo}.

In contrast to the IR structure, the UV structure of the quasi-PDF is quite 
different from the UV structure of the light-front PDF: the former has both 
linear and logarithmic divergences, while the latter contains only logarithmic 
divergences. 
%
Although there are no power-divergences in dimensional regularization, 
quasi-PDFs determined on the lattice are regulated by the inverse lattice 
spacing. 
%
In the continuum limit (for which $a\to 0$, with all physical quantities held 
fixed) there is a divergence, associated with the length of the Wilson line $z$, 
that scales as $z/a$. This divergence must be removed nonperturbatively.

For a general nonlocal bilinear operator with Lorentz structure $\Gamma$, 
the renormalized operator $O_{\Gamma}^{\rm (ren)}(z,\mu)$ is related to its bare 
operator $O^{(0)}_{\Gamma}(z)$ by~\cite{Dotsenko:1979wb,Arefeva:1980zd, 
Craigie:1980qs,Stefanis:1983ke,Dorn:1986dt}
\begin{equation}\label{eq:renorm_non-local}
O_{\Gamma}^{\rm (ren)}(z,\mu)=e^{\delta m(\mu)|z|}Z_{\psi, z}(\mu,z)O^{(0)}_{\Gamma}(z),
\end{equation}
where $\delta m$ is the mass renormalization of a test particle moving along 
the Wilson line of length $z$ and $Z_{\psi, z}(\mu,z)$ removes the remaining 
logarithmic divergences associated with the Wilson line endpoints 
(the quark fields). 
%
This result holds to all orders in perturbation theory: the exponentiated 
counterterm $\delta m(\mu)$ completely removes the linear divergence and the 
quasi-PDF can be renormalized 
multiplicatively~\cite{Ji:2017oey,Ishikawa:2017faj}. 
%
The exponentiated counterterm can be determined using a static heavy quark 
potential, which shares the same power-law divergence as the nonlocal quark 
bilinear~\cite{Musch:2010ka,Ishikawa:2016znu,Chen:2016fxx,Green:2017xeu}. 
%
An alternative approach for controlling the power divergence has been proposed 
in Ref.~\cite{Monahan:2016bvm}.

Once the linear divergence has been removed nonperturbatively, lattice 
perturbation theory can be used to renormalize the remaining logarithmic
divergences in the quasi-PDF~\cite{Ishikawa:2016znu,Chen:2016fxx,
Carlson:2017gpk,Xiong:2017jtn}. 
%
A delicate point regarding the renormalization is the mixing among certain 
subsets of these nonlocal operators. 
%
Such a mixing has been identified at 
one-loop in perturbation theory in Ref.~\cite{Constantinou:2017sej} 
for a variety of fermion/gluon actions or nonperturbatively based on 
symmetries~\cite{Chen:2017mzz,Chen:2017mie}. 
%
The mixing coefficients are necessary to disentangle the individual matrix 
elements for each quasi-PDF from lattice calculation data. 
%
Of particular interest is the case of the unpolarized quasi-PDF, which mixes 
with the scalar quasi-PDF if the Lorentz index of Eq.~\eqref{eq:qPDF} is in the 
same direction as the Wilson line. 
%
In contrast, the polarized and transversity PDFs with a Lorentz index in the 
Wilson line direction do not exhibit any mixing (to one-loop in 
perturbation theory). 

In addition, nonperturbative schemes, such as the 
RI/MOM scheme~\cite{Martinelli:1994ty}, 
can be used to renormalize matrix elements determined on the lattice. 
%
Nonperturbative schemes avoid the use of lattice perturbation theory at 
low energy scales (usually chosen to be $\mu = \pi/a$), although perturbative 
matching between renormalization schemes is still necessary for PDFs expressed 
in the $\overline{\rm MS}$ scheme. 
%
Combining a nonperturbative renormalization scheme with a step-scaling 
procedure~\cite{Luscher:1991wu} significantly reduces perturbative truncation 
uncertainties by providing a nonperturbative method for reaching high energy 
scales.
% 
Nonperturbative renormalization methods for quasi-PDFs have recently been 
constructed and applied in 
Refs.~\cite{Alexandrou:2017huk,Chen:2017mzz,Green:2017xeu}.

These nonperturbative procedures also remove the mixing between the unpolarized 
quasi-PDF and the twist-3 scalar operator, which occurs for lattice 
regularization that break chiral symmetry, through the construction 
of a $2\times2$ mixing matrix. 
%
The mixing coefficients do not contain any divergences. 
%
Further details can be found in Refs.~\cite{Alexandrou:2017huk,Chen:2017mzz}. 

It was recently observed that potential problems with the power divergent 
mixing patterns of DIS operators may arise when lattice regularization
is used~\cite{Rossi:2017muf}.
%
Further investigations into this issue would be interesting.

\end{itemize}

Lattice calculations of the $x$-dependence of PDFs have not matured 
up to the point to control all these sources of systematic uncertainty.       
%
Recent progress, however, has led to preliminary results that are encouraging. 
%
Here we highlight these results for the $x$-dependence
of the unpolarized and polarized PDFs extracted from lattice QCD. 

% FIXME: will merge these 2 paragraphs when the new figure is done! 
Fig.~\ref{fig:qPDF-demo} shows example results for the renormalized unpolarized 
PDFs from Ref.~\cite{Chen:2017mzz} and polarized PDF from 
Ref.~\cite{Alexandrou:2017huk}.
%The negative-$x$ part of PDF is related to the antiquark distribution via
%\be
%\bar{u}(x)-\bar{d}(x) = - u(-x)+ d(-x)\, , \qquad \text{for} \quad x>0 \, ,
%\ee
%for the unpolarized case, and
%\be
%\Delta\bar{u}(x)-\Delta\bar{d}(x) =  \Delta u(-x)- \Delta d(-x)\, , \qquad \tex%t{for} \quad x>0 \, ,
%\ee
%for the polarized distribution. 
In both cases, a nonperturbative renormalization procedure is applied to the 
bare matrix elements that appeared in earlier work~\cite{Lin:2014zya,
Alexandrou:2015rja,Chen:2016utp,Alexandrou:2016jqi,Alexandrou:2016eyt}.
%
For the unpolarized PDF, the calculation is carried out at a pion mass of 
310~MeV, includes one-loop matching and target mass corrections at the 
renormalization scale $\mu^2=4$~GeV$^2$, and the leading higher-twist 
$O(\Lambda_\text{QCD}^2/p_z^2)$ contributions have been 
removed~\cite{Chen:2016utp}. 
%
Multiple source-sink separations are used to take into account the effects of 
excited-state contamination, which become more important at large momentum. 
%
Mixing under renormalization has been estimated to be a small effect but is not 
yet computed explicitly. 
%
More recent work at the physical pion mass~\cite{Lin:2017ani} uses a different 
operator to avoid mixing effects. 
%
The polarized PDF has the advantage that is free from mixing, and is computed in
Ref.~\cite{Alexandrou:2017huk}  with fully renormalized matrix element, 
at a pion mass of 375 MeV. 
%
The matching to $\overline{\rm MS}$ at $\mu^2=4$~GeV$^2$ does not include any 
linearly divergent term, as the matrix element in 
coordinate space is renormalized.
% 
Note that in both cases, the antiquark asymmetry is compatible with zero 
within current uncertainties, contrary to earlier unrenormalized 
results~\cite{Lin:2014zya,Alexandrou:2015rja,Chen:2016utp,Alexandrou:2016eyt}.
%
This is mainly due to the rapid increase of the renormalization factor with 
Wilson-line length, which amplifies the finite-volume effect from truncating 
long-range correlations. 
%
Ref.~\cite{Lin:2017ani} showed that this truncation causes unphysical 
oscillations in the sea-flavor asymmetry and proposed that the oscillations 
can be removed by either imposing a filter to reduce the weighting of 
long-range correlations or by taking the derivative of the matrix element in 
coordinate space. 
%
The effectiveness of both these two methods is 
demonstrated in Refs.~\cite{Alexandrou:2017dzj,Lin:2017ani}. 

%-------------------------------------------------------------------------------
\begin{figure}[!t]
\centering
\includegraphics[scale=0.22,angle=270]{plots/unpxq}
\includegraphics[scale=0.22,angle=270]{plots/unpxqbar}\\
\includegraphics[scale=0.22,angle=270]{plots/polxq}
\includegraphics[scale=0.22,angle=270]{plots/polxqbar}\\
\caption{\small LP3's renormalized unpolarized isovector quark (top left) and 
  antiquark (top right) PDF combinations at physical pion mass with the renormalization scale 
  $\mu=2$~GeV~\cite{Lin:2017ani}. 
  %
  ETMC's renormalized polarized isovector quark (bottom left) and antiquark
  (bottom right) PDF combinations at pion mass of 
  375~MeV~\cite{Alexandrou:2017huk}.
  %
  Note that only statistical errors are shown here; the systematics are yet to be addressed. The small-$x$ region ($x< 0.2$) can suffer larger systematics than the rest of the distribution due to the limited nucleon boost momentum.
  } 
\label{fig:qPDF-demo}
\end{figure}
%-------------------------------------------------------------------------------

\paragraph*{Pseudo-PDFs.} 
The general dependence of the  matrix element $h(z,p_z)$ of Eq.~\eqref{eq:qPDF} 
on the hadron momentum $p$ and the displacement of the quark and antiquark 
fields $z$ can be expressed as a function of the Lorentz invariants 
$\nu=z\cdot p$ (Ioffe time~\cite{Ioffe:1969kf,Braun:1994jq}) 
and $z^2$, where $z$ and $p$ are general 4-vectors.  
%
We can thus introduce
\begin{equation}
\overline{h}(\nu,z^2) \equiv h(z,p_z)\,.
\end{equation}

The pseudo-PDF is then defined by the Fourier transform
%
\begin{equation}
{\mathcal P}(x,z^2)=\int \frac{d\nu}{2\pi} e^{-ix\nu} \overline{h}(\nu,z^2),
\end{equation}
which has support only in the physical range 
$x=[-1,1]$ \cite{Radyushkin:2016hsy,Radyushkin:2017cyf}. 
%
As discussed in Refs.~\cite{Radyushkin:2016hsy,Radyushkin:2017cyf}, the pseudo-PDF 
is directly related to both the PDFs and the 
transverse-momentum--dependent PDFs (TMDs).
%
In Ref.~\cite{Radyushkin:2017cyf}, using the temporal gamma matrix in the matrix 
element, a possible factorization of the TMD and PDF was conjectured which 
implies that the ratio
%
\begin{equation}
{\mathcal M}(\nu,z^2) =\frac{\overline h(\nu,z^2)}{\overline h(0,z^2)}
\label{eq:RatioPseudo}
\end{equation}
is directly related to the PDFs as 
\begin{equation}
{\mathcal M}(\nu,z^2) =Q(\nu,z^2) + {\cal O}(z^2).
\label{eq:IoffePDF}
\end{equation}
%
Here $Q(\nu,z^2)$ is the Ioffe time PDF~\cite{Ioffe:1969kf,Braun:1994jq}, 
which is just the Fourier transform of the PDFs,
\begin{equation}
{q}(x,1/z^2)=\int \frac{d\nu}{2\pi} e^{-ix\nu} Q(\nu,z^2).
\end{equation}
%
The ratio in Eq.~\eqref{eq:RatioPseudo} has a well-defined continuum 
limit and requires no renormalization. 
%
The polynomial corrections in Eq.~\eqref{eq:IoffePDF} are due to violations of 
the factorization conjecture, while the PDF ${q}(x,1/z^2)$ is the PDF in a 
particular scheme defined at scale $1/z^2$. Matching to $\overline{\rm MS}$ 
can be performed in perturbation theory following standard methodology. 
%
One loop results can be found in Refs.~\cite{Ji:2017rah,Radyushkin:2017lvu}.
%
A preliminary study was presented in Refs.~\cite{Orginos:2017kos,Karpie:2017bzm}, 
where it was shown that indeed the conjectured factorization is observed and 
the residual corrections are small. 
%
Further  evidence of the expected 
perturbative evolution of the Ioffe time PDFs was also observed. 
%
This methodology  is currently under study and results from realistic calculations are soon to be expected.


{\bf Lattice cross sections.}  Like extracting PDFs from QCD global fits of high energy scattering data, PDFs can also be extracted from analyzing ``data'' generated by lattice-QCD calculation of good {\it lattice cross sections} \cite{Ma:2014jla,Ma:2014jga}. A {\it lattice cross section} is defined as a single-hadron matrix element of a time-ordered, renormalized nonlocal operator ${\cal O}_n(z)$: ${\sigma}_{n}(\nu,z^2,p^2)=\langle p| {T}\{{\cal O}_n({z})\}|p\rangle$ with four-vector, $p$, $z$ and $\nu$ defined above and renormalization scale suppressed. The $p$ and $z$ effectively define the ``collision'' kinematics, and the choice of ${\cal O}_n$ determines the dynamical features of the lattice cross section. A good lattice cross section should have the following three key properties: (1) calculable in lattice-QCD with an Euclidean time, (2) has a well-defined continuum limit as the lattice spacing $a\to 0$, and (3) has the same and factorizable logarithmic collinear (CO) divergences as that of PDFs, which connects the good lattice cross sections to PDFs, just like how high energy hadronic cross sections are related to PDFs in terms of QCD factorization.  

A class of {\it good} lattice cross sections was constructed in terms of a correlation of two {\it renormalizable} currents, ${\cal O}_{j_1j_2}(z)\equiv z^{d_{j_1}+d_{j_2}-2} Z_{j_1} Z_{j_2}\, j_1(z) j_2(0)$, with dimension ($d_j$) and renormalization constant ($Z_j$) of the current $j$.  There could be many choices for the current, such as a vector quark current, $j_q^V(z) = \overline{\psi}_q(z)\gamma\cdot{z}\, {\psi}_{q}(z)$, or a tensor gluonic current, $j_g^{\mu\nu}(z)\propto F^{\mu\rho}(z){F_{\rho}}^\nu(z)$ \cite{Ma:2017pxb}.  Different combinations of the two currents could help enhance the lattice cross sections' flavor dependence.  If $z^2$ is sufficiently small, the lattice cross section constructed from two renormalizable currents could be factorized into PDFs \cite{Ma:2017pxb},
\begin{equation}\label{eq:fac}
{\sigma}_{n}(\nu,z^2,p^2)=\sum_{a}\int_{-1}^1 \frac{dx}{x}\, f_{a}(x,\mu^2) 
K_{{n}}^{a}(x\nu,z^2,x^2p^2,\mu^2) +O(z^2\Lambda_{\rm QCD}^2)\, ,
\end{equation}
where $\mu$ is the factorization scale, $K_n^{a}$ are perturbatively calculable hard coefficients, and $f_{a}$ is PDF of flavor $a=q,g$ with anti-quark PDFs expressed by quark PDFs using the relation $f_{\bar{a}}(x,\mu^2)=-f_{{a}}(-x,\mu^2)$.  PDFs could be extracted from global fits of lattice-QCD generated data for various lattice cross sections $\sigma_{n}(\nu,z^2,p^2)$ with corresponding perturbatively calculated coefficients $K_n^{a}$ in Eq.~(\ref{eq:fac}).

The quasi-PDFs and pseudo-PDFs introduced above could be derived by choosing 
\begin{equation}
{\cal O}_{q}(z)=Z_q(z^2)\overline{\psi}_q(z)\gamma\cdot {z}\, \Phi(z,0){\psi}_q(0)\,.
\end{equation}
 with the renormalization constant $Z_q(z^2)$ and the path ordered gauge link $\Phi(z,0)={\cal P}e^{-ig\int_0^{1} z\cdot A(\lambda z)\,d\lambda}$ \cite{Ma:2017pxb}.  With $K^{q(0)}_{q}(x \nu,z^2,0,\mu)= 2 x \nu  e^{i x \nu}$, one finds,
\begin{equation}\label{eq:lcsQuasi}
\int \frac{d \nu}{\nu}\, \frac{e^{-i x \nu}}{4\pi} \sigma_{q}(\nu,z^2,p^2)\approx f_{q}(x,\mu)\, ,
\end{equation}
modulo $O(\alpha_s)$ and higher twist corrections.  By choosing $z_0=0$ and both $\vec{p}$ and $\vec{z}$ along the ``3"-direction, one finds that $\nu=-z_3\, p_3$ and the left hand side of Eq.~(\ref{eq:lcsQuasi}) is the quasi-quark distribution introduced in Ref.~\cite{Ji:2013dva} if the integral is performed by fixing $p_3$, while it is effectively the pseudo-quark distribution used in Ref.~\cite{Orginos:2017kos} if the integral is performed by fixing $z_3$. %That is, these two approaches for extracting PDFs are equivalent if matching coefficients are calculated to the lowest order in $\alpha_s$, but different if higher order contributions need to be considered. 
Eq.~(\ref{eq:lcsQuasi}) indicates that the quasi-PDFs and pseudo-PDFs are two special cases of good lattice cross sections. 
%=======================================================================

