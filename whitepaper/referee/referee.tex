\documentclass[11pt]{article} 
\usepackage{graphicx}
\usepackage{afterpage}
\usepackage{epsfig,cite}
\usepackage{amssymb}
\usepackage{amsmath}
\usepackage{dsfont}
\usepackage{multirow}
\usepackage{url,hyperref}
\textwidth=17cm \textheight=22.5cm   
\topmargin -1.5cm \oddsidemargin -0.3cm %\evensidemargin -0.8cm  
\begin{document}

We thank the referee for his/her careful reading of the manuscript and for 
his/her detailed criticism. We have addressed all the issues raised by 
him/her, which we now discuss in turn.

\begin{enumerate}

\item We added a sentence at the end of the first paragraph of Sect.~2.2.2
to clarify how we count the order of each moment. We also included a 
cross-reference to Appendix~A, where we give explicit definitions for each
moment. This should avoid any confusion.

\item We added a paragraph at the end of the Pseudo-PDFs section in which
we discuss the fiducial region for quasi- and pseudo-PDF methods.
We included the reference suggested by the referee. 

\item We modified the discussion at the end of page~17 to take into 
account referee's suggestions. We specified that
the studies in Refs.~[120,156] are performed in quenched QCD, and that the
same conclusions are obtained in unquenched studies of TMDs in Ref.~[138].
We also added a sentence pointing the reader to references discussing the
relationship between TMDs and pseudo- and quasi-PDFs.
The factorization conjecture was spelled out more explicitly around Eq.~(2.33).

\item We clarified the merits of lattice-QCD calculations {\it vs} those
of global QCD fits in an extra sentence at the end of the first paragraph
in Sect.~5. We extended the discussion on ways to improve benchmarks 
by adding two sentences in Sect.~5 (first paragraph on page~57).

\item The relevant results for the second moments of unpolarized PDFs
were already included in the original version of the manuscript. 
Specifically, their definitions are given in Appendix~A, Eqs.~(A.3) and (A.5),
results from lattice QCD computations are given in Table~B.2, and results from
global QCD fits are given in Table~C.1.

\item We understand the rationale behind the referee's request. However, the 
emphasis is on the comparison between lattice QCD and global fit results in 
their combined form, as given, already side-by-side, in Tables~3.7-3.8. 
Therefore, we refrain from including an additional table in which individual 
values presented in Tables~3.1-3.6 are collected altogether.

\item Figures 4.1 and 4.3 were redone so that the
caption does not cover the curves anymore.

\item We added a footnote at the end of the first paragraph of Sect.~1,
in which we mention nonperturbative, QCD-based models as a 
further method to determine PDFs. Appropriate references were included.
We added representative references to TMDs and pion PDFs in the 
penultimate paragraph of Sect.~5. They include the reference suggested by 
the referee.

\end{enumerate}

On top of referee's comments, we modified the original version
of the paper in two respects. 

\begin{enumerate}

\item We added a discussion on lattice cross-sections in a new dedicated
paragraph at the end of Sect.~2.2.

\item We emphasized that the benchmark criteria for lattice QCD 
calculations defined in Sect.~3.1.2 are {\it aspirational}, and that
modifications to the rating system adopted in the paper will naturally
occur as the lattice-QCD results evolve. This is stated in a couple of 
additional sentences at the end of Sect.~3.1.2, and in an extra sentence 
in Sect.~5 (at the end of the first paragraph on page~57).

\end{enumerate}

These modifications take into account some suggestions and comments that we
have received after the preprint has circulated on the arXiv.

\end{document}
