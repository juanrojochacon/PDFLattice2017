Quantum Chromodynamics (QCD) is the regnant theory of strong interactions. It provides the 
theoretical foundation 
for the phenomenological ideas of the quark model, the concept of color, and partons. 
The power of QCD to explain physics from hadronic scales all the way up to high-energy collider
experiments relies on the remarkable features of asymptotic freedom and factorisation.
At high energies, or short distances, the QCD coupling constant is small and perturbative
techniques can accurately characterise the relevant physics. At low energies, or larger
distances, nonperturbative effects give rise to quark confinement and spontaneous chiral symmetry breaking. 
The study of these phenomena requires tools that go beyond perturbation theory, tools that can handle both strongly
coupled fields at long distances and the ultraviolet divergences of
the theory at short distances. This tool is lattice QCD. Factorisation provides the key to relate these
approaches: short-distance physics above the factorisation scale $\mu$ is captured by hard-scattering
coefficients calculated perturbatively, and the long-distance physics below this scale is described
by universal PDFs (or equivalent quantities). %In the following we restrict our overview to the
%elements of both global fits and lattice QCD that are relevant to the determination of PDFs (for more complete
%reviews of both topics, see \textbf{CITE\cite{???}}) and discuss both unpolarised and polarised PDFs. %% This sentence repeats comments in the Introduction.

At leading order in the strong-coupling expansion, 
collinear unpolarised PDFs describe the probability of finding a parton with specified longitudinal momentum fraction 
of the nucleon's momentum. 
We express the collinear unpolarised PDFs as $f_{i}(x,\mu)$,
where the index $i$ represents the parton flavour $i=\{g,u,\bar{u},d,\bar{d},s,\bar{s},...\}$ and
$x$ is the fractional momentum carried by the parton.\footnote{We could add an additional index to specify the particular hadron
(proton, neutron, pion, nuclei, ...); as we mainly refer to the proton
  in this paper, we will omit such a designation unless necessary.}
%
The value of $\mu$ is usually chosen to be the typical scale of the hard scattering
process involving the PDFs.
%
Beyond leading order, the PDFs themselves
are scheme-dependent quantities, and one typically works in
the $\overline{\rm MS}$ scheme; when this is convoluted with an appropriate
hard cross section (Wilson coefficient), we obtain a scheme-independent
physical observable (up to subleading terms in the perturbative expansion). 
%
The fractional momentum carried
by each parton flavour is given by the second moments of these PDFs, for instance
%
\begin{align}
\int_{0}^{1}dx\ x\ \left[u(x,\mu)\right] = & {}  \left\langle x\right\rangle _{u}\,, \label{eq:umoment1}\\
\int_{0}^{1}dx\ x\ \left[u(x,\mu)+\bar{u}(x,\mu)\right] = & {} \left\langle x\right\rangle _{u^{+}}\,. \label{eq:uplusmoment1}
\end{align}
%
Here, $\left\langle x\right\rangle _{u}$ is the fractional momentum
carried by the up-quark, $\left\langle x\right\rangle _{u^{+}}$ is
the fractional momentum carried by the up-quark and anti-up-quark.
%
The reader should notice that we are being particularly careful
with the notation, and that we always refer to $q^+$ to indicate
the sum of the quark and anti-quark PDF of the same flavour.


Helicity parton distributions describe the extent to which quarks and gluons 
with a given momentum fraction $x$ have their spins aligned with the spin 
direction of a fast moving nucleon in a helicity eigenstate. 
%
The corresponding integrals over all $x$ relate to one of the most fundamental, 
but not yet satisfactorily answered questions in hadronic physics: 
how is the spin of the proton distributed among its constituents? 
%
Helicity PDFs may be probed in spin asymmetries for reactions at large momentum 
transfer; the probes used so far are inclusive and semi-inclusive 
deep-inelastic lepton scattering, as well as polarised $pp$ scattering. We express the
longitudinally-polarised PDFs as
\begin{equation}
\Delta f_i(x,\mu^2) \equiv f_i^{\uparrow}(x,\mu^2) - f_i^{\downarrow}(x,\mu^2)\,.
\label{eq:polPDFs}
\end{equation}
These PDFs define the net amount of parton 
densities with spin aligned along ($\uparrow$) or opposite ($\downarrow$)
the polarisation direction of their parent nucleon.
%
Much of the interest in the $\Delta f_i$ originates from the fact that their 
first moments can be interpreted as the fractions of proton spin 
carried by the corresponding quarks/antiquarks and gluons respectively. In particular,
the first moments of the singlet and the gluon polarised PDFs,
\begin{align}
\Delta\Sigma(\mu^2)
& =
\sum_{q}^{n_f}\int_0^1 dx 
\left[\Delta q(x, \mu^2) + \Delta\bar{q}(x, \mu^2)\right]
\equiv
\sum_q^{n_f}\langle 1 \rangle_{\Delta q^+}\,,
\\
\Delta G(\mu^2)
& =
\int_0^1 dx \Delta g(x,\mu^2)
\equiv
\langle 1 \rangle_{\Delta g}
\,,
\label{eq:moments}
\end{align}
contribute to the proton spin sum rule~\cite{Leader:2013jra}.
%

Both unpolarised and polarised PDFs are accessible, theoretically and experimentally, through the Compton amplitude
\begin{equation}
T_{\mu\nu}(p,q) = \int {\rm d}^4\!x\, {\rm e}^{iqx}  \langle p,s |T J_\mu(x) J_\nu(0)|p,s\rangle   
\end{equation}
at large virtual photon momenta $q^2=-Q^2$. Here $T$ is the time-ordering operator, the $J_\mu(x)$ are local vector bilinears at space-time point $x$, and the external states are hadronic states with momentum $p$ and spin $s$. The most general form of the Compton amplitude $T_{\mu\nu}(p,q)$ for polarised deep-inelastic scattering from a proton targets reads
\begin{align}
T_{\mu\nu}(p,q) = {} & \left(\delta_{\mu\nu}-\frac{q_\mu q_\nu}{q^2}\right) \mathcal{F}_1(\omega,Q^2) + \left(p_\mu-\frac{pq}{q^2}q_\mu\right) \left(p_\nu-\frac{pq}{q^2}q_\nu\right) \frac{1}{pq} \mathcal{F}_2(\omega,Q^2)\,,\\ 
& {} \quad  + \epsilon_{\mu\nu\lambda\sigma}q_\lambda s_\sigma \frac{1}{pq}\mathcal{G}_1(\omega,Q^2) + \epsilon_{\mu\nu\lambda\sigma}q_\lambda \left(pq s_\sigma - sq p_\sigma\right) \frac{1}{(pq)^2}\mathcal{G}_2(\omega,Q^2)\,.
\end{align}
The imaginary parts of the Compton amplitude structure functions, $\mathcal{F}_1$, $\mathcal{F}_2$, $\mathcal{G}_1$ and $\mathcal{G}_2$, are related to the usual structure functions of the hadronic tensor, $F_1(x,Q^2)$, $F_2(x,Q^2)$, $g_1(x,Q^2)$ and $g_2(x,Q^2)$, through the optical theorem. The unpolarised and polarised PDFs can be deduced from $\mathcal{F}_1$ and $\mathcal{G}_1$ by first rewriting them in terms of the spin-independent and spin-dependent structure functions, $F_1(x,Q^2)$ and $g_1(x,Q^2)$,
\begin{align}
\mathcal{F}_1(\omega,Q^2) = {} & 4 \omega^2 \int_0^1 dx\,  \frac{xF_1(x,Q^2)}{1-(\omega x)^2} = \sum_{n=2,4,\cdots}^\infty 4\omega^n \int_0^1 dx\, x^{n-1} F_1(x,Q^2) \,, \\
\mathcal{G}_1(\omega,Q^2) = {} & 4 \omega \int_0^1 dx\, \frac{g_1(x,Q^2)}{1-(\omega x)^2} = \sum_{n=1,3,\cdots}^\infty 4\omega^n \int_0^1 dx\, x^{n-1} g_1(x,Q^2)\,.
\end{align}
Here $\omega=2pq/q^2$ and target mass corrections have been discarded. Odd moments may be obtained from the mixture of currents $J_\mu J_\nu^5$.

At sufficiently high momentum transfer that power corrections can be neglected, factorisation~\cite{Sterman:1995fz} enables us to write the structure functions $F_1(x,Q^2)$ and $g_1(x)$ of the proton as 
\begin{align}
F_1(x,Q^2) = {} & x\sum_a \int_x^1 \frac{{\rm d}z}{z}\,C_{1,a}\left(\frac{x}{z},\alpha_S(Q^2)\right)f_a(z,Q^2) \,, \label{eq:Fi}\\
g_1(x,Q^2) = {} & \sum_a \int_x^1\frac{dy}{y}\, e_{1,a}\left(\frac{x}{z},\alpha_S(Q^2)\right) \Delta f_a(z,Q^2) \,,
\label{pdf}
\end{align}
where $x=Q^2/2p\cdot q$ is the standard Bjorken variable and the sum over $a$ accounts for the effects of all active partons and runs over the quark and antiquark flavours, $u$, $d$, $\bar{u}$ and $\bar{d}$, and the gluon $g$. The $C_{1,a}$ and $e_{1,a}$ are hard coefficient functions and splitting functions that can be computed perturbatively as an expansion in $\alpha_S(Q^2)$ and the $f_i(x,Q^2)$ and $\Delta f_i(x,Q^2)$ are the unpolarised and polarised PDFs, respectively.

While the PDFs themselves cannot be calculated using perturbative methods, their dependence on the scale $\mu$ can be, and is given by the  
DGLAP evolution equations~\cite{Dokshitzer:1977sg,Gribov:1972ri,Altarelli:1977zs},
a set of integro-differential coupled equations of the form
\begin{equation}
  \label{eq:dglap}
\frac{\partial f_i(x,\mu^2)}{\partial \ln \mu^2}=\sum_{j=g,q,\bar{q}}\int_x^1 \frac{{\rm d}z}{z}P_{ij}(x/z)f_j(z,\mu^2)\,.
\end{equation}
Here, the logarithmic derivative of the PDF is
determined\footnote{The DGLAP (Dokshitzer\textendash Gribov\textendash Lipatov\textendash Altarelli\textendash Parisi)
evolution can be modified by $\ln(1/x)$ terms at small-$x$, and
this is characterised by the BFLK (Balitsky-Fadin-Kuraev-Lipatov) equations~\cite{Kuraev:1976ge,Kuraev:1977fs,Balitsky:1978ic} by a convolution
of the PDFs with the DGLAP kernel $P_{ij}$ which can be computed
perturbatively in powers of $\alpha_{s}(\mu)$.
%
The splitting functions $P_{ij}$ are currently known to NNLO.
%
Additionally, at large-$x$ and small $\mu$ scale the above framework
can receive corrections from non-factorisable higher-twist corrections.}
%
The DGLAP evolution equations can be solved numerically using
either $x$-space or Mellin $N$-space techniques by means of various
publicly available
codes~\cite{Bertone:2013vaa,Salam:2008qg,Botje:2010ay}, where the typical level of agreement
for the results of the PDF evolution is $\mathcal{O}(10^{-5})$.

The dependence of the helicity PDFs on the 
scale $\mu$ can also be computed perturbatively by means of DGLAP evolution 
equations, Eq.~\eqref{eq:dglap}.
%
In this case, the unpolarised splitting kernels $P_{ij}$ should be replaced with their
polarised counterparts, $\Delta P_{ij}$, which have been computed to 
NLO~\cite{Mertig:1995ny,Vogelsang:1995vh,Vogelsang:1996im}
and recently to NNLO~\cite{Moch:2014sna} in the $\overline{\rm MS}$ 
renormalisation scheme.
%
Small-$x$ evolution equations which resum powers of $\alpha_s\ln^2(1/x)$
in the polarisation-dependent evolution along with powers of $\alpha_s\ln(1/x)$
in the unpolarised evolution have been also proposed using the formalism
of Wilson line-like operators~\cite{Kovchegov:2015pbl}.
%
Both numerical~\cite{Kovchegov:2016weo}
and analytical~\cite{Kovchegov:2016zex,Kovchegov:2017jxc}
solutions to these small-$x$ evolution equations have been derived
for the flavour singlet combination of polarised PDFs.
%
Results have been extended also to the case of the polarised gluon 
PDF~\cite{Kovchegov:2017lsr}.
