\subsubsection{Analysis settings}
\label{sec:projections:settings}

In the unpolarized case,  we consider
the first moments (momentum fractions) of $q^+=q+\bar{q}$ (with $q=u,d,s$),
of the gluon, and of the isoscalar combination $u^+-d^+$.
%
In the polarized case, we consider the zeroth moments (spin fractions) 
of $\Delta q^+$ (with $q=u,d,s$) and of the isoscalar combination 
$g_A=\Delta u^+-\Delta d^+$, and the first moment of the 
$\Delta u^- - \Delta d^-$ combination.
%
Using the notation outlined in Appendix~\ref{app:notation}, we have
\begin{align}
\la x\ra_{u^+}\, , \
\la x\ra_{d^+}\, , \
\la x\ra_{s^+}\, , \
\la x\ra_{g}\, ,  \
\la x\ra_{u^+-d^+} & \qquad\text{for the unpolarized case} \, ,
\label{eq:UM}\\
\la 1\ra_{\Delta u^+}\, , \
\la 1\ra_{\Delta d^+}\, , \
\la 1\ra_{\Delta s^+}\, , \
\la x\ra_{\Delta u^--\Delta d^-}\, ,\
\la 1\ra_{\Delta u^+ - \Delta d^+} & \qquad\text{for the polarized case}\, .
\label{eq:PM}
\end{align}

We look at three different scenarios, which we denote
as Scenario A, B, and C, for the projected total systematic
uncertainty associated with lattice-QCD calculations.
%
Our choice for this uncertainty is denoted by $\delta_L^{(i)}$.
%
It is summarized in Table~\ref{tab:scenarios} for each PDF moment $i$ in 
Eqs.~\eqref{eq:UM}-\eqref{eq:PM} and for each scenario.
%
Current uncertainties on lattice-QCD results 
(see Sects.~\ref{subsubsec:BClQCD}-\ref{subsec:BN})
are also quoted for comparison.
%
We emphasize that, while trying to be reasonably
realistic, we do not associate a given scenario
with a specific time scale for the calculation.
%
Our results provide a guide to the potential
constraining power of future lattice-QCD calculations
of PDF moments once included in global analyses. 
 
%-------------------------------------------------------------------------------
\begin{table}[!t]
\centering
\footnotesize
\renewcommand{\arraystretch}{1.3} 
\begin{tabular}{cccccc}
\toprule
Scenario &  \multicolumn{5}{c}{$\delta_L^{(i)}$ for unpolarized moments} \\
& $\la x\ra_{u^+}$ 
& $\la x\ra_{d^+}$ 
& $\la x\ra_{s^+}$  
& $\la x\ra_{g}$  
&   $\la x\ra_{u^+-d^+}$  \\
\midrule
Current  
& $\sim 16\%$  
& $\sim 30\%$ 
& $\sim 45\%$  
& $\sim 13\%$  
& $\sim 60\%$ \\
A   & 3\%  & 3\% &  5\% &  3\% &  5\% \\
B   & 2\%  & 2\% &  4\% &  2\% &  4\%  \\
C   & 1\%  & 1\% &  3\% &  1\% &  3\%  \\
\bottomrule
\\
\toprule
Scenario & \multicolumn{5}{c}{$\delta_L^{(i)}$ for polarized moments} \\ 
& $\la 1\ra_{\Delta u^+}$  
& $\la 1\ra_{\Delta d^+}$  
& $\la 1\ra_{\Delta s^+}$
& $\la x\ra_{\Delta u^--\Delta d^-}$  
& $\la 1\ra_{\Delta u^+ - \Delta d^+}$\\
\midrule
Current  
& $\sim 3\%$  
& $\sim 5\%$ 
& $\sim 70\%$ 
& $\sim 65\%$ 
& $\sim 3\%$ \\
A   & 5\% & 10\%  & 100\% & 70\%  & 5\% \\
B   & 3\% &  5\%  &  50\% & 30\%  & 3\% \\
C   & 1\% &  2\%  &  20\% & 15\%  & 1\% \\
\bottomrule
\end{tabular}
\caption{\small The three scenarios assumed for the total percentage
systematic uncertainty $\delta_L^{(i)}$ in future lattice-QCD calculations.
%
The unpolarized (upper table) and polarized (lower table) PDF moments
included in the analysis are shown.
%
Current systematic uncertainties in state-of-the-art lattice-QCD calculations
are also displayed according to the benchmark exercise performed in
Sect.~\ref{sec:benchmarking} (see also Tables~\ref{tab:BMunp} 
and~\ref{tab:BMpol} for the unpolarized and polarized cases, respectively).
\label{tab:scenarios}
}
\end{table}
%-------------------------------------------------------------------------------

   
Our choice of uncertainties in Tab.~\ref{tab:scenarios}
is rather different for the unpolarized and polarized cases.
%
For the unpolarized case, scenario A is based on values
of $\delta_L^{(i)}$ rather smaller than the typical uncertainties that affect 
state-of-the-art lattice-QCD calculations, see Table~\ref{tab:BMunp}.
%
As expected from Fig.~\ref{fig:Bmomsunp}, and as we have explicitly verified,
including lattice-QCD pseudo-data with uncertainties of similar size as those 
of Table~\ref{tab:BMunp} leaves unpolarized PDFs essentially unchanged.
%
Significantly reduced uncertainties $\delta_L^{(i)}$ must be assumed to 
demonstrate any impact on global fits.
%
We assume that $\delta_L^{(i)}$ is typically larger for $\la x\ra_{s^+}$
and $\la x\ra_{u^+-d^+}$, compared to the other moments, in line with 
what is observed from Table~\ref{tab:BMunp}.
%
Scenarios B and C are rather optimistic, in that they require systematic 
uncertainties to decrease by roughly a factor of two and a factor of four 
with respect to scenario A.
%    
For the polarized case, scenario A assumes that the uncertainties 
$\delta_L^{(i)}$ are similar to current uncertainties in
state-of-the-art lattice-QCD calculations, see Sect.~\ref{sec:benchmarking},
and Table~\ref{tab:BMpol}.

We note that a total systematic error of $\delta_L^{(i)}\sim 1\%$
is probably the best that one can achieve within a lattice-QCD calculation 
in the near future, since at that level several other effects, such as QED 
corrections, become relevant. These are much more difficult to deal with.
%
For both the polarized and the unpolarized case, the generalization of these 
projections to other conceivable scenarios
is straightforward and can be obtained from the authors upon request.
 


