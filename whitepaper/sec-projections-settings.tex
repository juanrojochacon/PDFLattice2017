\subsubsection{Analysis settings}
\label{sec:projections:settings}

The following projection studies, which quantify
the impact of present and future lattice-QCD calculations
of PDF moments, are based on the following specific choice
of moments.
%
For the unpolarised case, we use
\be
  \la x\ra_{u^+}\, , \quad
\la x\ra_{d^+}\, , \quad
\la x\ra_{s^+}\, , \quad
\la x\ra_{g}\, , \quad {\rm and} \quad
\la x\ra_{u^+-d^+} \, ,
\ee
while for polarised PDFs, we include
the following five moments:
\be
\la 1\ra_{\Delta u^+}\, , \quad
\la 1\ra_{\Delta d^+}\, , \quad
\la 1\ra_{\Delta s^+}\, , \quad
\la x\ra_{\Delta u^--\Delta d^-}\, , \quad {\rm and} \quad
\la 1\ra_{\Delta u^+ - \Delta d^+} \, .
\ee
Appendix~\ref{app:notation} contains the
explicit definitions and conventions used for these moments.
%
That is, for the unpolarised case we include
the second moments (momentum fractions) of $q^+$ (with $q=u,d,s$),
of the gluon, and of the isoscalar combination $u^+-d^+$.
%
In the polarised case, we include the first moments (which
contribute to the proton spin sum rule) of $\Delta q^+$ (with $q=u,d,s$)
and of the isoscalar combination $g_A=\Delta u^+-\Delta d^+$, as well as
the second moment of the $\Delta u^- - \Delta d^-$ combination.

We consider three
different scenarios, which we denote
as Scenario A, B, and C respectively, for the total systematic
uncertainty than we associate with lattice-QCD calculations of PDF moments.
%
In Tab.~\ref{tab:scenarios} we summarise our choice of projected
total uncertainty from lattice-QCD calculations, denoted by $\delta_L$, for each
PDF moment.
    %
    We emphasise that, while trying to be reasonably
    realistic, we do not associate a given scenario
    within a specific time-scale for the calculation.
    %
    Our results provide a guide to the potential
    constraining power of existing and future lattice QCD calculations
    of PDF  moments in the context
    of a global analysis.
    
    The motivation for our choice of the scenarios
    in Tab.~\ref{tab:scenarios}
    is rather different for the unpolarised and polarised cases.
    %
    For the unpolarised case, scenario A is based on values
    of $\delta_L^{(i)}$ rather smaller than the typical
    uncertainties that affect state-of-the-art lattice calculations, see Tab.~\ref{tab:BMunp}.
    %
    Scenarios B and C represent two optimistic scenarios for the
    future improvement of these systematic uncertainties, where these are decreased
    by roughly a factor of two and a factor of four with respect to scenario A.
    
    In contrast, for the polarised fits,
    scenario A assumes that the uncertainties $\delta_L$
    are similar to current uncertainties in
    state-of-the-art lattice QCD calculations
    selected for the benchmarking exercise of Sect.~\ref{sec:benchmarking},
    and summarised in Tab.~\ref{tab:BMpol}.
    %
    We have verified that, for unpolarised PDFs,
    including lattice-QCD pseudo-data with uncertainties of similar size as those of Tab.~\ref{tab:BMunp}
    leaves the PDFs essentially unchanged, and it is only once the uncertainties
    $\delta_L^{(i)}$ are significantly reduced that uncertainties from the global fit
    are reduced.
    %
    The only connection with the uncertainties of the calculations in Tab.~\ref{tab:BMunp}
    is that we assume that $\delta_L^{(i)}$ is typically larger for $\la x\ra_{s^+}$
        and $\la x\ra_{u^+-d^+}$, compared to the other moments.

    We also note that
    a total systematic error of $\delta_L^{(i)}\sim 1\%$
    is probably the best that one can achieve
    within a lattice-QCD calculation in the near future, since at that level several other
    effects, such as QED corrections,
    become relevant and these are much more difficult
    to deal with.
    %
    For both
    the polarised and
    the unpolarised case,
    we emphasise that the generalization of these projections to other conceivable scenarios
    is straightforward and can be obtained from the authors upon request.
 
%%%%%%%%%%%%%%%%%%%%%%%%%%%%%%%%%%%%%%%
\begin{table}[t]
  \centering
  \renewcommand{\arraystretch}{1.3} 
  \begin{tabular}{c||ccccc}
    \hline
    Scenario &  \multicolumn{5}{c}{$\delta_L^{(i)}$ for unpolarised moments}   \\
&    $\la x\ra_{u^+}$  &   $\la x\ra_{d^+}$   &  $\la x\ra_{s^+}$  &
$\la x\ra_{g}$  &   $\la x\ra_{u^+-d^+}$  \\
    \hline
    \hline
    Current  & $\sim 16\%$  &  $\sim 30\%$ 
    & $\sim 45\%$  & $\sim 13\%$  &  $\sim 60\%$ \\
    A   & 3\%  & 3\% &  5\% &  3\% &  5\% \\
 B   & 2\%  & 2\% &  4\% &  2\% &  4\%  \\
  C   & 1\%  & 1\% &  3\% &  1\% &  3\%  \\
    \hline
  \end{tabular}\vspace{0.7cm}
   \begin{tabular}{c||ccccc}
    \hline
    Scenario   &
    \multicolumn{5}{c}{$\delta_L^{(i)}$ for polarised moments} \\ 
& $\la 1\ra_{\Delta u^+}$  & $\la 1\ra_{\Delta d^+}$  & $\la 1\ra_{\Delta s^+}$
&  $\la x\ra_{\Delta u^--\Delta d^-}$  &  $\la 1\ra_{\Delta u^+ - \Delta d^+}$\\
    \hline
    \hline
    Current  &
    $\sim 3\%$  & $\sim 5\%$ & $\sim 70\%$ & $\sim 65\%$ & $\sim 3\%$ \\
    \hline
    A   & 
    5\% &    10\%  &   100\% &    70\%  &    5\% \\
 B   &
 3\% &    5\%  &   50\% &    30\%  &    3\% \\
  C   & 1\% &    2\%  &   20\% &    15\%  &    1\% \\
    \hline
  \end{tabular}
   \caption{\small The three scenarios assumed here
     for the total percentage
     systematic uncertainty
    in future lattice QCD calculation $\delta_L$ for each
    of the unpolarised (upper) and polarised (lower table) PDF
    moments that are included
    in this projection analysis.
    %
    In addition, the first line indicates the current systematic
    uncertainties of the state-of-the-art lattice QCD calculations
    selected for the benchmarking exercise of Sect.~\ref{sec:benchmarking},
    and summarised in Tables~\ref{tab:BMunp} and~\ref{tab:BMpol}
    for the unpolarised and polarised cases, respectively.
    %
    See text for more details.
\label{tab:scenarios}
  }
\end{table}
%%%%%%%%%%%%%%%%%%%%%%%%%%%%%%%%%%%%%%%
