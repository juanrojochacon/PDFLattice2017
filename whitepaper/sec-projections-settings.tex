\subsubsection{Settings}


Taking into account
these considerations, we will consider in this analysis the following
moments.
%
For the unpolarized case, we will use
\be
  \la x\ra_{u^+}\, , \quad
\la x\ra_{d^+}\, , \quad
\la x\ra_{s^+}\, , \quad
\la x\ra_{g}\, , \quad {\rm and} \quad
\la x\ra_{u^+-d^+} \, ,
\ee
while for the polarized side, we will include instead
the following five moments:
\be
\la 1\ra_{\Delta u^+}\, , \quad
\la 1\ra_{\Delta d^+}\, , \quad
\la 1\ra_{\Delta s^+}\, , \quad
\la x\ra_{\Delta u^--\Delta d^-}\, , \quad {\rm and} \quad
\la 1\ra_{\Delta u^+ - \Delta d^+} \, .
\ee
Recall that Appendix~\ref{app:notation} contains the
explicit definitions and conventions used for these moments.
%
Therefore, we see that for the unpolarized case we include
the second moments (momentum fractions) of $q^+$ (with $q=u,d,s$),
of the gluon, and of the isoscalar combination $u^+-d^+$.
%
In the polarized case instead, we include the first moments (which
contribute to the proton spin content) of $\Delta q^+$ (with $q=u,d,s$)
and of the isoscalar combination $\Delta u^+-\Delta d^+$, as well as
the second moment of $\Delta u^- - \Delta d^-$.

In the present exercise we will consider three
different scenarios, which we denote
as Scenario A, B, and C respectively, for the total systematic
uncertainty than we associate to lattice
QCD calculations of PDF moments.
%
In Table~\ref{tab:scenarios} we summarize the
values assumed for this total uncertainty
    of the lattice QCD calculation, denoted by $\delta_L$, for each
    of the various unpolarized and polarized PDF moments that enter
    this analysis.
    %
    We emphasize that here, while trying to be reasonably
    realistic, we do not aim to associate a given scenario
    within a specific time-scale for the calculation.
    %
    Our results  merely provide an illustrative guidance about the potential
    constraining power of existing and future lattice QCD calculations
    of PDF  moments in the context
    of a global analysis.
    
    The motivation for our choice of the scenarios
    in Table~\ref{tab:scenarios}
    is rather different from the unpolarized and polarized cases.
    %
    For the polarized fits,
    scenario A assumes that the uncertainties $\delta_L$
    for the lattice QCD calculations
    are on same the ball-park of the current ones, taking as
    representative values for the latter  those from the
    state-of-the-art lattice QCD calculations
    selected for the benchmarking exercise of Sect.~\ref{sec:benchmarking},
    and summarized in Table~\ref{tab:BMpol}.
    %
    Then scenarios B and C represent two possible optimistic scenarios for the
    future improvement of these systematic uncertainties, where these are decreased
    by roughly a factor 2 and a factor 4 with respect current values.
    
    On the other hand, for the unpolarized case scenario A is based on values
    of $\delta_L^{(i)}$ already rather smaller than the typical
    uncertainties that affect state-of-the-art calculations, see Table~\ref{tab:BMunp}.
    %
    The reason is that we have verified that including the pseudo-data $\mathcal{F}_i^{(\rm)}$
    assuming lattice-QCD uncertainties of similar size as those of Table~\ref{tab:BMunp}
    leaves the PDFs essentially unchanged, and only once the uncertainties
    $\delta_L^{(i)}$ are significantly reduced that we start to obtain a reduction
    of the uncertainties from the global fit.
    %
    The only connection with the uncertainties of the calculations in Table~\ref{tab:BMunp}
    is that we assume that $\delta_L^{(i)}$ is typically larger for $\la x\ra_{s^+}$
    and $\la x\ra_{u^+-d^+}$ as compared to the other moments.
    %
    A total systematic error of $\delta_L^{(i)}$ is probably the best that one can achieve
    within a lattice-QCD calculation even in principle, since at that level many other
    effects such as QED corrections become relevant and these are much more difficult
    to deal with.
    %
    For both
    the polarized and
    the unpolarized case,
    we emphasize that the generalization of these projections to other conceivable scenarios
    is straightforward and can be obtained from the authors upon request.
 
%%%%%%%%%%%%%%%%%%%%%%%%%%%%%%%%%%%%%%%
\begin{table}[t]
  \centering
  \renewcommand{\arraystretch}{1.3} 
  \begin{tabular}{c||ccccc}
    \hline
    Scenario &  \multicolumn{5}{c}{$\delta_L^{(i)}$ for unpolarized moments}   \\
&    $\la x\ra_{u^+}$  &   $\la x\ra_{d^+}$   &  $\la x\ra_{s^+}$  &
$\la x\ra_{g}$  &   $\la x\ra_{u^+-d^+}$  \\
    \hline
    Current  & $\sim 16\%$  &  $\sim 30\%$
    & $\sim 45\%$  & $\sim 13\%$  &  $\sim 60\%$ \\
    A   & 3\%  & 3\% &  5\% &  3\% &  5\% \\
 B   & 2\%  & 2\% &  4\% &  2\% &  4\%  \\
  C   & 1\%  & 1\% &  3\% &  1\% &  3\%  \\
    \hline
  \end{tabular}\vspace{0.7cm}
   \begin{tabular}{c||ccccc}
    \hline
    Scenario   &
    \multicolumn{5}{c}{$\delta_L^{(i)}$ for polarized moments} \\ 
& $\la 1\ra_{\Delta u^+}$  & $\la 1\ra_{\Delta d^+}$  & $\la 1\ra_{\Delta s^+}$
&  $\la x\ra_{\Delta u^--\Delta d^-}$  &  $\la 1\ra_{\Delta u^+ - \Delta d^+}$\\
    \hline
    Current  &
    $\sim 3\%$  & $\sim 5\%$ & $\sim 70\%$ & $\sim 65\%$ & $\sim 3\%$ \\
    \hline
    A   & 
    5\% &    10\%  &   100\% &    70\%  &    5\% \\
 B   &
 3\% &    5\%  &   50\% &    30\%  &    3\% \\
  C   & 1\% &    2\%  &   20\% &    15\%  &    1\% \\
    \hline
  \end{tabular}
   \caption{\small The three scenarios assumed here
     for the total percentage
     systematic uncertainty
    in future lattice QCD calculation $\delta_L$ for each
    of the unpolarized (upper) and polarized (lower table) PDF
    moments that are included
    in the present reweighting analysis.
    %
    In addition, the first line indicates the current systematic
    uncertainties of the state-of-the-art lattice QCD calculations
    selected for the benchmarking exercise of Sect.~\ref{sec:benchmarking},
    and summarized in Tables~\ref{tab:BMunp} and~\ref{tab:BMpol}
    for the unpolarized and polarized cases, respectively.
    %
    See text for more details.
\label{tab:scenarios}
  }
\end{table}
%%%%%%%%%%%%%%%%%%%%%%%%%%%%%%%%%%%%%%%


